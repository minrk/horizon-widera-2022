
\TODO{In this abstract we want to summarize the vision of the project;
  - Context: Open Science; what is Jupyter; Jupyter is big
  - Who are we
  - What is our goal
  - What is our strategy / concept
Points to hit:
- Open Science should be practical, not just available
- Jupyter is part of the solution
- *brief* highlight of how/why Jupyter and Binder make sense:
  - Jupyter is widely adopted
  - notebook encapsulate computation
  - Binder builds on Jupyter to enable shareable reproducible environments
  - Jupyter is web-based, enabling building services
- What we plan to do
  - improve Jupyter/Binder toward open science
  - operate Jupyter-based services on EOSC
  - Open Science training (skip?)
- Who we are
  - Core Jupyter experts
  - Domain experts motivating/validating Jupyter improvements
}


% \begin{verbatim}
% Call: Increasing the service offer of the EOSC Portal

% Title: Open Science Publication of Research Environments (OSPRE)

% Points to hit:

% - Open Science should be practical, not just available
% - Jupyter is part of the solution
% - *brief* highlight of how/why Jupyter and Binder make sense:
%   - Jupyter is widely adopted
%   - notebook encapsulate computation
%   - Binder builds on Jupyter to enable shareable reproducible environments
%   - Jupyter is web-based, enabling building services
% - What we plan to do
%   - improve Binder toward open science
%   - operate Jupyter-based services on EOSC
%   - Open Science training (skip?)
% - Who we are
%   - Core Jupyter experts
%   - Domain experts motivating/validating Jupyter improvements
% \end{verbatim}



% % (a3) prompt: Services supporting scholarly communication and open access (4M): based on existing initiatives across Europe (institutional and thematic repositories, aggregators, etc.), the services should empower researchers and research communities and initiatives with the necessary tools and functionalities for systematic publishing, analysing and re-using of scientific results beyond publications (data, software and other artefacts), as well as supporting long-term preservation and curation. The services should also enable scientific workflows with adequate metrics and monitoring mechanisms supporting career development and the monitoring of funding and research impact. Support to a catch-all repository for open research should be provided.
%
% %  To truly achieve the societal goals of Open Science,
% %  we must make progress beyond the `mere availability' of scientific results,
% %  to the practical usability and exploitation of such data once it is made available,
% %  an area where there is much room for improvement.
% %  The Jupyter project and its ecosystem show great promise
% %  as tools for bridging this gap; for making Open Science
% %  useful and accessible to all,
% %  from researchers to educators to public citizens.
% %  The Jupyter Notebook and Jupyter ecosystem are of increasing
% %  importance in computational science, data science, academia,
% %  industry, governments, and service providers,
% %  used by millions worldwide.
% %  Jupyter notebooks have great potential to push Open Science
% %  forward because they provide a complete description of a
% %  computational study that can be turned into a publication
% %  or produce part of a publication, such as a figure,
% %  making complex tasks reproducible.
% %  The Jupyter-based Binder project adds a means to execute notebooks
% %  in specified computational environments, an aspect of reproducibility
% %  not yet widely supported.
% %  In \TheProject, we will extend the capabilities of the Jupyter
% %  tools and ecosystem to add functionality that we view as having great
% %  importance for EOSC and Open Science more
% %  widely and operate services on EOSC as a demonstration.
% %
% %  Many \TheProject partners have longstanding experience and
% %  leadership roles in the Jupyter ecosystem,
% %  and in deploying services built on Jupyter to many users across the globe.
% %  Complementary to this core expertise,
% %  we integrate partners focussing on the application of these tools from a wide range of disciplines,
% %  both to demonstrate and ensure that our developments serve
% %  real-world Open Science use cases.
% %
%
%   To truly achieve the societal goals of Open Science, we must make progress
%   beyond the `mere availability' of scientific results as Open Access, to the
%   practical usability and exploitation of such artefacts once they are made
%   available, an area where there is much room for improvement. The Jupyter
%   ecosystem shows great promise as a collection of tools for bridging this gap;
%   for making Open Science useful and accessible to all, empowering researchers,
%   educators, and public citizens. Jupyter is of increasing importance in
%   computational science, data science, academia, industry, governments, and
%   service providers, and used by millions worldwide. Jupyter notebooks have
%   great potential to push Open Science beyond publications because they
%   encapsulate a computational study that may be part of a publication, such as
%   the creation of a figure, a major part of making complex tasks reproducible.
%   The Jupyter-based Binder project adds a means to execute notebooks in
%   specified computational environments, an aspect of reproducibility not yet
%   widely supported, and of great falue to re-using scientific results.
%
%   Tools such as Jupyter and Binder increase the value of all existing Open
%   Access initiatives by adding the axis of interactive computability, empowering
%   researchers to produce derivative and validating (or refuting) work. Services
%   such as Binder also xpose public metrics and monitoring, supporting the
%   monitoring of research impact and career development for any users of the
%   system.
%
%   We will (i) extend the capabilities of the Jupyter tools and ecosystem to add
%   functionality that we view as essential to and providing great value for EOSC
%   and Open Science, focused on accessibility, interactive publications, and
%   reproducibility. Based on this framework of improved Jupyter tools, it will be
%   possible to build Open Science Publication of Research Environments
%   (\TheProject), and (ii) build a range of diverse innovative open services on
%   EOSC as part of this project, both to demonstrate and ensure that our
%   developments serve real-world Open Science use cases.
%
%   Many \TheProject partners have longstanding experience and leadership roles in
%   the Jupyter ecosystem, and in deploying services built on Jupyter to many
%   users across the globe. Complementary to this core expertise, we integrate
%   partners focusing on the application of these tools to a wide range of
%   scientific disciplines and communities, for which EOSC-hosted demonstrator
%   services are developed.

\begin{abstract}
  \TODO{Comment on Binder being a generic tool -- not limited to Jupyter}

  \TODO{Comment on Best practice guidelines and Training we deliver}

  \begin{draft}
  Clearer problem statement: shift emphasis from work description to problem/impact

  Note from review:
doesn't provide a clear justification for the work, nor says much in terms of impact (eg, `scale', match with expected outcomes...). It mostly talks about what the project would do, as technical/research work, but doesn't say much about most of the whys. Why should the EC fund your project, and fund it now – ie, why is reproducibility with existing tools an important issue that must be fixed now to achieve the expected outcomes and Destination impacts? Here you could reuse some of the things you say eg in 1.1.4 Motivation.

- Also, I’m not sure it clearly says what is the problem you are trying to solve (it talks about the work to be done, which is not quite the same). Also, make clear that the scope of the work goes beyond Jupyter.

- You should make it clear here that this is about *computational* reproducibility. You say so in the 2st par of 1.1.1 Ambition, but it bears repeating here. Remember that that the abstract is used to select reviewers, and that the call topic is about reproducibility in general, not just about computational tools.
  \end{draft}

The societal goals of Open Science -- particularly improved validation and reuse of research outputs --
can only be achieved if reproducibility is an integral part of the process.
\emph{Merely available} code is of drastically less value than code
that can be run easily by anyone.
As institutions and policies increasingly adopt Open and Reproducible policies,
researchers urgently need practical tools to fulfill these requirements.
Further, policy makers can benefit from tools for evaluation and enforcement.
In \TheProject, we focus on \emph{computational} reproducibility.

To reproduce computational results, the software environment in
which to execute the code must be created.
This project automates and improves ways to do this by developing and extending the
Binder project.

Binder aims to \myemph{automate existing practices}
for reproducible computational environments,
solving a key piece of reproducibility without dictating the execution workflow itself.
Binder is a subproject of Jupyter,
providing open source tools such as the widely used Jupyter notebook.

Binder has demonstrated potential in the Jupyter community by serving over ten million user sessions in 2021 at mybinder.org,
building over fifty thousand unique computational environments.
We aim to bring this potential to researchers not yet served by the tools,
by extending the robustness and the applicability of the
Binder tools.

We motivate and validate our efforts with selected use cases, such as
data publishing and in HPC contexts where Binder tools are not currently practical,
and invite the global community of researchers to contribute to the open source project.

Project members have longstanding experience and leadership roles in the
Jupyter ecosystem, and in deploying services built on
Jupyter to millions of users across the globe.
Complementary to this core expertise,
we integrate partners focusing on the application of these tools to a range of scientific disciplines and communities.

\end{abstract}

%%% Local Variables:
%%% mode: latex
%%% TeX-master: "proposal"
%%% End:
