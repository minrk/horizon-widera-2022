\paragraph*{Project Jupyter}
\label{sec:project-jupyter}
Project Jupyter \cite{Jupyter}, which has grown increasingly popular in the scientific
computing community, has become the \emph{lingua franca} of interactive
computing in both academia and industry \cite{Perkel2018}. The main goal of Project Jupyter
is to provide a consistent set of tools to improve researchers'
workflows from the exploratory phase of the analysis to the communication
of the results \cite{Kluyver2016,Granger2021}.
Jupyter has grown rapidly in popularity and adoption both in the industry and academia.
We estimate the user base of the Jupyter notebook to be in the millions
\cite{jupyter-grant}. Users range from data scientists to researchers,
educators, and students from many fields, including journalists and librarians.
The number of publicly hosted Notebook documents is exceeding 8~million~\cite{notebookcount}.

The \textbf{Jupyter Notebook} is the flagship application of Project Jupyter.
It allows the creation of notebook documents, containing a mixture of text and
\textbf{interactively executable code}, along with rich output from running that code.
Figure~\ref{fig:notebook-screenshot} shows an open notebook including graphs
from an audio processing example. Notebook documents are readily shareable,
providing a popular way to describe and illustrate computational methods and
tools. \TODO{We should update the notebook: (i) point to a GitHub repo with it,
  and (ii) binder-enable it. Sources are in data/notebook-figure}


\paragraph*{Jupyter notebook and reproducibility}
While Jupyter notebooks can make computational and data-driven research more effective
\cite{Perkel2018,Fangohr2020,Granger2021}, they also have great potential to push
open and reproducible science forward \cite{Beg2021}. The notebook provides a complete
description of a computational study (Step 1 in
figure~\ref{fig:use-cases-binder}), and the notebook can -- in principle -- be
turned into a publication, or can be used to provide the required computation
for a part of a publication, such as a figure (Step 2 in
figure~\ref{fig:use-cases-binder}). Once the researcher has specified what
software is required to execute the notebook (Step 3 in
figure~\ref{fig:use-cases-binder}), the study is completely reproducible by
anyone (Step 4 in figure~\ref{fig:use-cases-binder}).

In this way, the notebook \emph{enables reproducibility} of complex workflows
with minimal additional effort on the user side. This approach is used by a
substantial number of scientists for publications already (for example
\cite{GitHubRepoExampleAlbert2016,GitHubRepoExampleCortes2018,Beg2019-blochpoint-data-repository, }
\TODO{insert publications with reproducible repositories - the LIGO paper?
  Anything else}): it is hard to prove
but it seems plausible that a significant fraction of the 30,000 sessions
triggered on the mybinder.org service every day are used for reproducible repositories
(Section~\ref{sec:mybinder}).

% HF: this is a reference to the Joel Grus criticism. Not sure if we need it.
% The paper by Beg2021 addresses that in section 9.
(We note in passing that the use of Jupyter notebooks alone does not guarantee
reproducibility: it requires some training and/or experience to correctly specify a
computational environment and to capture that information in a machine readable
way. It is also important to include all computational steps in the notebook in
order~\cite{Beg2021}.)

\begin{terminology}{(Jupyter)}
\begin{description}
\item[Project Jupyter] The over-arching term used to refer to the large collaboration of open source
  tools for developing and sharing computational ideas

\item[Notebook] A Jupyter Notebook. An executable document that can
  combine text, code, and outputs, such as figures.
  A Jupyter notebook can serve the same function as the \textbf{Script}.

\item[JupyterLab] The flagship application of Project Jupyter.
  Allows \textbf{interactive execution} and creation of
  \textbf{Notebook} documents.
  Built on web technology,
  can be run on any computer resources on a network,
  from laptops to cloud to supercomputers.

\item[JupyterHub] The multi-user extension of the Jupyter Servers, such as JupyterLab.
  JupyterHub runs one or more JupyterLab instance for each user,
  and can run on any \emph{shared} computational infrastructure,
  from home servers to the cloud to supercomputers~\cite{Fangohr2020}.

\end{description}
\end{terminology}
