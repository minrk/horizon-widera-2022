\subsubsection{Project Jupyter and the surrounding ecosystem}
\label{sec:project-jupyter}

\begin{figure}[htb]\centering
  \includegraphics[width=0.9\textwidth]{use-cases-binder-logbook-solution.png}
  \caption{A typical use case for Jupyter notebooks in research.
            Image by Juliette Belin for the OpenDreamKit project, used under
            CC-BY-SA.}\label{fig:use-cases-binder}
\end{figure}

\noindent\textbf{Jupyter ecosystem as the root of \TheProject}

\TheProject has chosen to centre its efforts on the Jupyter software
ecosystem, in particular Binder and repo2docker.
Figure~\ref{fig:use-cases-binder} summarises a typical use
case of Jupyter Notebook and Binder;
both are described in more detail below.

The Jupyter notebook and Jupyter ecosystem are of increasing
importance in computational science and data science, in academia,
industry, and services. In addition to supporting high productivity of
researchers, they have great potential to push Open and Reproducible Science forward:
the notebook provides a complete description of a computational and
data science study (Step 1 in figure~\ref{fig:use-cases-binder}), and the notebook can -- in principle -- be turned
into a publication, or can be used to provide the required computation
for a part of a publication, such as a figure
(Step 2 in figure~\ref{fig:use-cases-binder}). Once the researcher has
specified what software is required to execute the notebook (Step 3
in figure~\ref{fig:use-cases-binder}), the study is completely
reproducible by anyone (Step 4 in figure~\ref{fig:use-cases-binder}).

In this way, the notebook \textbf{enables reproducibility} of complex tasks
with minimal additional effort on the user side.
The Binder project allows to execute such notebooks in
tailored computational environments; an aspect of reproducibility that
is not widely supported yet,
and a great opportunity for improving best practices in Open Science.

Furthermore, for users wanting to connect
to a local Jupyter notebook server on their machine, or to connect to
a server somewhere else on the Internet, the users only need a
web-browser to display and use the notebook regardless of the location
of the notebook server,
allowing computation to run anywhere from a local laptop to a remote supercomputer or in the cloud.
Because of these characteristics,
the Notebook is already planned to become an
important service on the European Open Science Cloud (EOSC) (for
example in \cite{panosc}),
and is an ideal component to use when building Open Science Services.

\medskip\noindent\textbf{Project Jupyter}

\emph{Project Jupyter} \cite{Jupyter}, which has grown increasingly popular in the scientific
computing community, has become the \emph{lingua franca} of interactive
computing in both academia and industry. The main goal of Project Jupyter
is to provide a consistent set of tools to improve researchers'
workflows from the exploratory phase of the analysis to the communication
of the results \cite{Kluyver2016}.

Split in 2014 from the \emph{IPython Project} \cite{IPython}, Jupyter has grown rapidly in
popularity and adoption both in the industry and academia. We estimate the user
base of the Jupyter notebook to be in the millions \cite{jupyter-grant}. Users range from data
scientists to researchers, educators, and students from many fields,
including journalists and librarians. In 2017, the Jupyter
team was awarded the \emph{ACM Software System Award}, an annual award that
honors people or an organization \emph{"for developing a software system that had a
lasting influence"}. Prior recipients include \emph{Unix}, \emph{TCP/IP}, and
the \emph{World Wide Web} \cite{acm-award}.

A large number of discrete software components make up Project Jupyter.
While these interact with one another, many can be installed separately
to serve various use cases. For this proposal, we loosely divide the
software involved into \emph{Jupyter core} developed under the guidance
of the developers who started the project, and the broader \emph{Jupyter
ecosystem} including software developed by third parties,
which may interact or build upon core Jupyter components.
Some of the components and concepts important to \TheProject are detailed below.

\begin{figure}[ht]\centering
  \centering
  \includegraphics[width=0.9\textwidth]{spectrogram_smaller.png}
  \caption{A notebook document in the Jupyter Notebook interface.}\label{fig:notebook-screenshot}
\end{figure}

\medskip\noindent\emph{Jupyter core}
\begin{itemize}
  \item \label{sec:jupyter-notebook} The \textbf{Jupyter Notebook} is the flagship application of Project Jupyter.
  It allows the creation of notebook documents, containing a mixture of text and
  interactively executable code, along with rich output from running that code.
  Figure \ref{fig:notebook-screenshot} shows an open notebook including graphs
  from an audio processing example. Notebook documents are readily shareable,
  providing a popular way to describe and illustrate computational methods and
  tools. \TODO{We should update the notebook: (i) point to a github repo with
    it, and (ii) binder-enable it, and (iii) increase the pixel resolution of
    the screen shot. Madison?}
  \textbf{JupyterLab} is the new, modular, extensible client application
  for Jupyter notebooks, but the document format, server, and user model are the same.

  \item \textbf{Jupyter kernels} are the backend software which allow Jupyter to execute
  code in many different programming languages. The \textbf{IPython} kernel is
  the reference kernel, supporting the Python programming language, and is
  developed by the Jupyter core team. Kernels for other languages are maintained
  by third parties

  \item \textbf{JupyterHub} is a multi-user extension of the Jupyter Notebook.
  It runs on one or more notebook servers, for example at a research institution.
  Users can log in to author and run notebooks securely through their web
  browser, without needing to install any special software on their own
  computer.

\end{itemize}

\medskip\noindent\emph{Jupyter ecosystem}\label{jupyter-ecosystem}

While Jupyter is a large, distributed, coordinated project,
the wider community of Jupyter users develops a great deal of
software with Jupyter integration,
providing increased or domain-specific functionality,
building on top of Jupyter, or integrating core Jupyter components in some aspect.
We call this the \textbf{Jupyter ecosystem}.
The broader Jupyter ecosystem includes many more projects than we will describe
here, but a selection of projects which are relevant to
\TheProject includes:

\begin{itemize}
  \item \textbf{Binder} builds on JupyterHub to allow sharing executable
  environments along with data files and a description of the software components
  required to run the notebooks. When someone accesses a Binder repository,
  the service builds the computational environment on demand, allowing them to
  execute and modify a copy of the notebooks.
  \textbf{repo2docker} \cite{repo2docker} and \textbf{BinderHub} \cite{binder} are components of the Binder
  software. \TOWRITE{}{More here, as repo2docker is key}
\end{itemize}

\begin{figure}[ht]\centering
  \includegraphics[width=0.5\textwidth]{ipywidgets_example.png}
  \caption{An example of using two simple slider widgets to explore the
  parameter space of a function. The \texttt{@interact} decorator creates
  the widgets and connects them to the function.}
  \label{fig:ipywidgets-example}
\end{figure}

