% ---------------------------------------------------------------------------
%  Section 2: Impact
% ---------------------------------------------------------------------------

\clearpage
\section{Impact}
\label{sec:impact}

\subsection{Pathway toward impact}

\eucommentary{
e.g. 4 pages
Impact: Logical steps towards the achievement of the expected impacts of the project over time, in particular beyond the duration of a project. A pathway begins with the projects' results, to their dissemination, exploitation and communication, contributing to the expected outcomes in the work programme topic, and ultimately to the wider scientific, economic and societal impacts of the work programme destination.
}

\begin{draft}
{
\textcolor{red}
%\eucommentary{
Provide a narrative explaining how the project's results are expected to make a
difference in terms of impact, beyond the immediate scope and duration of the
project. The narrative should include the components below, tailored to your
project.

(a)	Describe the unique contribution your project results would make towards (1) the outcomes specified in this topic, and (2) the wider impacts, in the longer term, specified in the respective destinations in the work programme.
	Be specific, referring to the effects of your project, and not R\&I in general in this field.
	State the target groups that would benefit. Even if target groups are mentioned in general terms in the work programme, you should be specific here, breaking target groups into particular interest groups or segments of society relevant to this project.
	The outcomes and impacts of your project may:
  \begin{compactitem}


\item	Scientific, e.g. contributing to specific scientific advances, across and within disciplines, creating new knowledge, reinforcing scientific equipment and instruments,  computing systems (i.e. research infrastructures);
\item	Economic/technological, e.g. bringing new products, services, business processes to the market, increasing efficiency, decreasing costs, increasing profits, contributing to standards’ setting,  etc.
\item	Societal , e.g. decreasing CO2 emissions, decreasing avoidable mortality,
improving policies and decision making, raising consumer awareness.
\end{compactitem}
Only include such outcomes and impacts where your project would make a
significant and direct contribution. Avoid describing very tenuous links to
wider impacts. However, include any potential negative environmental outcome or
impact of the project including when expected results are brought at scale (such
as at commercial level). Where relevant, explain how the potential harm can be
managed.

(b)	Describe any requirements and potential barriers - arising from factors beyond the scope and duration of the project - that may determine whether the desired outcomes and impacts are achieved. These may include, for example, other R\&I work within and beyond Horizon Europe; regulatory environment; targeted markets; user behaviour. Indicate if these factors might evolve over time. Describe any mitigating measures you propose, within or beyond your project, that could be needed should your assumptions prove to be wrong, or to address identified barriers.
	Note that this does not include the critical risks inherent to the management
  of the project itself , which should be described below under
  ‘Implementation’.

(c)	 Give an indication of the scale and significance of the project’s contribution to the expected outcomes and impacts, should the project be successful.  Provide quantified estimates where possible and meaningful.
	‘Scale’ refers to how widespread the outcomes and impacts are likely to be. For example, in terms of the size of the target group, or the proportion of that group, that should benefit over time; ‘Significance’ refers to the importance, or value, of those benefits. For example, number of additional healthy life years; efficiency savings in energy supply.
	Explain your baselines, benchmarks and assumptions used for those estimates. Wherever possible, quantify your estimation of the effects that you expect from your project. Explain assumptions that you make, referring for example to any relevant studies or statistics. Where appropriate, try to use only one methodology for calculating your estimates: not different methodologies for each partner, region or country (the extrapolation should preferably be prepared by one partner).
	Your estimate must relate to this project only - the effect of other initiatives should not be taken into account.
}
\end{draft}


Almost all researchers perform some kind of computation in the course of their work,
from production of simple charts and figures to large-scale simulations.
Researchers today are faced with a great variety of useful
and increasingly open tools to perform their research tasks.
However, this variety presents its own challenge.
Because we are building and extending \textbf{generic tools for computational reproducibility},
we help researchers of all kinds perform their work reproducibly without dictating
how they do the research part of their work.
By focusing on (\textbf{reproducible computational environments}),
and offering a generic and interoperable solution,
we \emph{are} solving a big piece of the reproducibility problem
and act as a bridge between any solutions to other parts of the reproducibility problem,
for instance those classically addressed by workflow management systems.
"Do one thing and do it well" is key to success and will accelerate progress
by allowing others to concentrate on different challenges in the reproducibility problem.

Helping researchers navigate this diverse landscape of research software
with both education and practical tools
will enable a \textbf{large fraction of all researchers to perform research in a more efficient,
more productive, more reproducible, and more \emph{useful} way}.

\subsubsection{Contributions towards outcomes and impact}
\label{sec:countributions-towards-outcome-and-impact}

The expected outputs and impact of \TheProject with respect to the
work program is detailed in Tables~\ref{table:output-comparison}
and~\ref{table:impact-comparison}, respectively.

%\begin{table}[p]
\begin{table}[h!]
  \begin{center}
    \begin{tabular}{>{\raggedright}m{.30\textwidth}|m{.65\textwidth}}
      \toprule
      \textbf{Expected outputs from call}
      & \textbf{Expected outputs from the \TheProject project}\\\midrule
        %
        %
      Structured understanding of the underlying drivers, of concrete and effective
      interventions -- funding, community-based, technical and policy -- to increase
      reproducibility of the results of R\&I; and of their benefits;
      &
      %
        Developing a structured understanding of the technological challenges of practical
        reproducibility of software environments is a prerequisite for our work on the
        community-effort-based Binder tools (\WPref{reproducibility}, \WPref{impact},  \WPref{applications}).

        We know that the technologies such as the Jupyter notebook and their reproducibility
        with Binder find some acceptance in the scientist community (see Section~\ref{sec:opensource}).
        One of the outcomes of this project is an improved structured understanding of the underlying drivers
        for the wider research community to adopt or not adopt these technologies as a concrete and effective action to improve the reproducibility of their results~(\WPref{education}). We expect an interplay of technical, social, community-specific and policy drivers affecting this.
        %
      \\\midrule
      Effective solutions, policy-, technical- and practice-based, to increase the
      reproducibility of R\&I results in funding programmes, in communities and in
      the dissemination of scientific results;
      &
        The Binder tools are effective and practice-based technical solutions to increase the
        reproducibility of R\&I results in funding programmes and in scientific research more widely~\cite{Beg2021}.
        The combination of Jupyter notebooks and Binder are also effective solutions for the dissemination through
        workshops~\cite{binder-workshops} and education activities~\cite{Zeller2022} involving computation as participants can use complex
        software environments through their browser, and do not need to install any additional software locally.
      \\\midrule
      Greater collaboration, alignment of practices and joint action by stakeholders
      to increase reproducibility, including but not limited to training,
      specialised careers and guidelines for best practice.
      &
        \TheProject is developing tools that enable \textbf{automatic computational reproducibility} across many domains by aligning to the existing software specification practices. Having one tool that respects existing practices means it can be adopted across many domains, thus \textbf{facilitating greater collaboration}. We will develop \textbf{best practice guidelines} for reproducibility and open science, and disseminate them through training activities and materials. The staff hired for the project will be highly specialised research software engineers. We expect that voluntary contributors attracted by the project will also receive additional training for such specialised careers.
      \\\bottomrule
    \end{tabular}
  \end{center}
  \caption{Relating \TheProject outputs to the outputs expected by the call \label{table:output-comparison}}
\end{table}


\begin{table}[h!]
  \begin{center}
      \begin{tabular}{>{\raggedright}m{.2\textwidth}|m{.75\textwidth}}
      \toprule
      \textbf{Expected impacts from call}
      & \textbf{Expected impacts from the \TheProject project}\\\midrule
        %
        %
      Increased proportion of reproducible results from publicly funded R\&I
      &
      %
        The \TheProject project will improve the ease of use tools for reproducibility,
        thereby \textbf{increasing the number of reproducible research outputs}.
        The project will also \textbf{educate researchers} to provide the motivation, required
        technical skills, and an understanding of best practice to achieve
        reproducible science outcomes.
      \\\midrule

        %
        %
      Increased re-use of scientific results by research and innovation
      &
        \textbf{Reproducibility enables reusability}, in particular where the reproduction
        of the results can be carried out automatically (for example using the
        tools developed and improved in the \TheProject project).

        From anecdotal evidence, it is known that
        a new PhD student may need many months (sometimes exceeding 12 months) to reproduce
        a published study which is to form the basis of their new research task.
        An immediately reproducible research output made available together with the publication
        can reduce this time immensely: a binder-enabled reproducible repository may be able
        to recompute the results within an hour or a much shorter time because the process is automatic.

        The availability of an immediately reproducible research output -- as fostered by \TheProject{} -- plays a very
        important role in modern research, where generally advances are built based on prior results.\\\midrule

      Greater quality of the scientific production.
      &
        %Reproducibility is an important concept underpinning the scientific method.
        %Re-usability of results -- often enabled through their reproducibility -- has the potential to accelerate the research significantly.

        The availability of practical reproducibility as advanced by \TheProject has the potential to
        improve the quality of scientific research through new opportunities for collaboration and interdisciplinary research:

        We have already seen a good example of the Jupyter ecosystem facilitating an
        interdisciplinary collaboration: the Nobel Prize-winning LIGO scientific collaboration shared
        notebooks detailing the data processing steps which led to the discovery of
        gravitational waves, using the Binder service to allow anyone to re-compute
        the published plots~\cite{ligo-open-science}. Scientists with no background in gravitational waves
        studied these notebooks and improved the signal processing.

        In this proposal, we want to provide this ability to a wider audience through
        improved tools and documentation,
        including for disciplines which rely on processing much larger volumes of
        data.\\
      \bottomrule
    \end{tabular}
  \end{center}
  \caption{Relating \TheProject impacts to the impacts expected by the call \label{table:impact-comparison}}
\end{table}

%\newpage
\subsubsection{Measuring impact}\label{sec:KPIs}

As we are building tools for open and reproducible science, the best measure of our impact
is in the adoption and use of these tools and services based on them. This can be observed
qualitatively (anecdotal feedback and case studies) and quantitatively (counting workshop
attendees, for example).
%
% Much of our work will be in the form of contributions to existing
% public projects, such as Binder and Jupyter, which can be measured in our participation in
% those projects, such as code and documentation contributions, bug reports, and roadmap
% contributions.
%
We measure progress toward our objectives (Table~\ref{tab:objectives-tasks})
via the following Key Performance Indicators (KPIs):
\begin{compactenum}[\textbf{KPI} 1:]
\item \label{kpi:reproducibility} Fraction of published repositories for
  which Binder tools can build the computational environment -- measuring both
  improvements in Binder tools and impact from our training efforts. We will
  implement the required methodology in
  \taskref{reproducibility}{repo2docker-checker}. (Objective 1)
    % Open publications for which the authors have made a reproducible repository available
    % through \TheProject services.
\item \label{kpi:broaden} Number of researchers who benefit from using
  \TheProject{} results in their work. Due to the open source nature of our
  outcomes, it is difficult to track this number accurately. To solicit this
  information, we can request feedback from users on forums, mailing lists and
  social media, and count the users we know directly working with
  them through \WPref{applications} and \WPref{education} for example. We expect
  to under count beneficiaries for this metric, but believe the KPI is still useful
  to measure our direct impact for better reproducibility in science. (Objective 2)
\item \label{kpi:demonstrators} Number of demonstrators enabled that make
  results from scientific research reproducible. (Objective 3)
\item \label{kpi:education} Metrics to capture the number of people that have
  made use of the training and education activities of \TheProject{}, such as
  workshop attendees, viewers of training videos, access to online documentation
  of best practice. (Objective 4)% can we track people accessing a documentation web site?
\end{compactenum}


\subsubsection{Target groups and scale of impact}\label{sec:target-groups-and-scale-of-impact}

We have already discussed in Section~\ref{sec:motivation-why} that
\textbf{computational reproducibility affects the majority of researchers},
including those who carry out work that is not predominantly computational.

The impact of this project will be realised through (i) the training we develop
and disseminate, and (ii) the improved Binder tools. The improved tools will
directly impact the community of researchers using Jupyter notebooks. However,
the \myemph{improved functionality and applicability of the Binder tools outside Project
Jupyter} will benefit all researchers who need computational reproducibility, and
may be -- and in the long run -- more important.

\medskip

We can give some indication of the size and activity of the Jupyter notebook
community and use of the existing Binder tools: Jupyter notebooks are used to
support research in numerous communities, including
%\begin{compactitem}
%\item
\noemph{Journalists} and practitioners of \noemph{data-driven
journalism} at the LA Times, BuzzFeed News, Columbia Journalism School
\cite{latimes-datadesk} \cite{columbia-nytimes} \cite{data-journalism};
\noemph{Data scientists} in academia, industry and services \cite{Perkel2018};
% \item
\noemph{Research institutions} such as CERN, EuXFEL, JRC, and many more,
operating institution-wide Jupyter deployment;
% \item
\noemph{Universities} using Jupyter as a teaching platform;
% \item
\noemph{Large cloud providers} building commercial products on the
top of Jupyter (Google DataLab and Colaboratory, AWS Sagemaker, OVH AI Notebook);
% \item
within the \noemph{European Open Science Cloud}, Jupyter is used in many EOSC
projects, and a JupyterHub service is provided by the EGI foundation.
% \item
This impact was recognised by the \emph{ACM Software System Award} that was
awarded to the Jupyter team to honour \emph{"developing a software system that
had a lasting influence"} in 2017. (Prior recipients include \emph{Unix},
\emph{TCP/IP}, and the \emph{World Wide Web} \cite{acm-award}.)

There are \textbf{at least 8 million notebooks} deposited on GitHub \cite{notebookcount}, and
the size of the \textbf{notebook user base was estimated} to be of the order of
\textbf{millions in 2015} \cite{jupyter-grant}. We know that the public Binder service \textbf{\mybinder{}
was used to create over 10,000,000 computational environments}
from at least 50,000 unique repositories in 2021~(Section~\ref{sec:opensource}).


\subsubsection{Potential barriers}

A central barrier towards impact would be if the research community would not
accept or embrace the tools and practical guidance developed here.
To minimise this risk, we'll stay in close touch with all our stakeholders
(\WPref{management}, \WPref{applications}, \WPref{education}), and use
the experience of the team members who are active researchers themselves.

A key strategy in mitigating barriers to adoption is the approach of \textbf{automating existing practices}.
All of the \textbf{practices promoted} and developed by \TheProject \textbf{are beneficial} to
producers and consumers of reproducible research alike,
\textbf{independent of the specific Binder tools} in which we choose to implement the automation.
This avoids lock-in and ensures positive impact even among researchers who choose not to use any of the specific tools we develop.

\subsection{Measures to maximise impact -- dissemination, exploitation and communication}

\eucommentary{
e.g. 5 pages, including 2.3
}

\begin{draft}
  \textcolor{red}
Describe the planned measures to maximise the impact of your project by providing a first version of your ‘plan for the dissemination and exploitation including communication activities’. Describe the dissemination, exploitation and communication measures that are planned, and the target group(s) addressed (e.g. scientific community, end users, financial actors, public at large).

Please remember that this plan is an admissibility condition, unless the work programme topic explicitly states otherwise. In case your proposal is selected for funding, a more detailed ‘plan for dissemination and exploitation including communication activities’ will need to be provided as a mandatory project deliverable within 6 months after signature date. This plan shall be periodically updated in alignment with the project’s progress.

Communication measures should promote the project throughout the full lifespan of the project. The aim is to inform and reach out to society and show the activities performed, and the use and the benefits the project will have for citizens. Activities must be strategically planned, with clear objectives, start at the outset and continue through the lifetime of the project. The description of the communication activities needs to state the main messages as well as the tools and channels that will be used to reach out to each of the chosen target groups.

All measures should be proportionate to the scale of the project, and should contain concrete actions to be implemented both during and after the end of the project, e.g. standardisation activities. Your plan should give due consideration to the possible follow-up of your project, once it is finished. In the justification, explain why each measure chosen is best suited to reach the target group addressed. Where relevant, and for innovation actions, in particular, describe the measures for a plausible path to commercialise the innovations.

If exploitation is expected primarily in non-associated third countries, justify by explaining how that exploitation is still in the Union’s interest.

Describe possible feedback to policy measures generated by the project that will
contribute to designing, monitoring, reviewing and rectifying (if necessary)
existing policy and programmatic measures or shaping and supporting the
implementation of new policy initiatives and decisions.
\begin{compactitem}
\item	Outline your strategy for the management of intellectual property, foreseen protection measures, such as patents, design rights, copyright, trade secrets, etc., and how these would be used to support exploitation.
	If your project is selected, you will need an appropriate consortium agreement
  to manage (amongst other things) the ownership and access to key knowledge
  (IPR, research data etc.). Where relevant, these will allow you, collectively
  and individually, to pursue market opportunities arising from the project.
\item If your project is selected, you must indicate the owner(s) of the results (results ownership list) in the final periodic report.
\end{compactitem}
\end{draft}


\TheProject is contributing to tools for open and reproducible science. It is
essential that we disseminate our work in order to \textbf{reach and support user
communities} (such as researchers and research infrastructure providers), enable
them to best exploit our software and services, and achieve impact.
This section outlines how the project will establish and organise the
dissemination, communication, and exploitation actions to promote the project
and the adoption of its outcomes beyond the project's lifetime.


% The dissemination and communication plan is outlined in the following sub-sections.
% Therein we distinguish:
% \begin{itemize}
% \item Dissemination as the public disclosure of the results of the project through
% a process of promotion and awareness-raising right from the beginning of a project.
% It makes research results known to various stakeholder groups (like research peers, industry
% and other commercial actors, professional organisations, policymakers) in a targeted
% way, to enable them to use the results in their own work.
% \item Communication as the strategic and targeted measures for promoting the project
% and its results to a multitude of audiences, including the media and the public, and possibly
% engaging in a two-way exchange. The aim is to reach out to society as a whole and
% in particular to some specific audiences while demonstrating how EU funding contributes to tackling
% societal challenges.
% \end{itemize}

\subsubsection{Dissemination of results}

\TheProject endeavours to make open and reproducible science practices both more \textbf{understandable and actionable}
to practitioners through the development and use of open and freely available tools.
Most of \TheProject software will be in the form of contributions to existing projects,
which will be governed by the licenses of those projects.
All Jupyter and Binder software is released under the permissive BSD license,
which specifically allows commercial exploitation,
as has proven successful in enabling collaborations with industrial partners
such as Google, Microsoft, IBM, OVHCloud, and more.
Any other developments will be made publicly and freely available under open source licenses, and
hosted on public code hosting sites such as GitHub.
This means that all \textbf{\TheProject software will be available and accessible to all} who find it,
at no cost to \TheProject,
enabling long-term access beyond the funding of \TheProject.
Similarly, non-code products such as dissemination works
(workshop materials, etc.) will be made freely available under open Creative Commons licenses.

All the partners will be involved in the dissemination of \TheProject results (see draft plan of the
dissemination, communication and exploitation plan in the text box in Section~\ref{box:draft-C-D-E-plan}) and
\WPref{applications} and \WPref{education} will play a central role.

As a result, the primary dissemination effort is to:
\begin{enumerate}
  \item make sure that \textbf{prospective users are aware of the work} through
    {show cases, science demonstrators, demos, best practice
      documentation, project communication}, and
  \item \textbf{enable them to use and exploit the tools} through learning resources, training, and services.
\end{enumerate}

Our focus for dissemination will be on
\taskref{education}{workshops}
operating workshops, training various communities in the availability,
purpose, development, and use of \TheProject software and services.
We will make a particular effort to use these workshops as an opportunity
to \textbf{support diversity and inclusion in the open science community},
by running (online, hybrid, or in-person) workshops for under-served and under-represented groups in the academic and
open source communities. \textbf{Free and online training} will be streamed (e.g. Twitch),
made available to the wider community for instance on YouTube, and archived on Zenodo for long-term availability and
findability to the wider community who may not be able to attend workshops.


These resources will be hosted on free, public hosting services,
such as GitHub Pages, ReadTheDocs, or YouTube channels, and as much as possible co-developed and co-hosted with existing and
well-established organisations
(The Carpentries,\footnote{\url{https://carpentries.org}} CodeRefinery,\footnote{\url{https://coderefinery.org}}
Galaxy Training Network,\footnote{\url{https://training.galaxyproject.org}} Pangeo,\footnote{\url{http://gallery.pangeo.io}} etc.).

We will also disseminate our results through \textbf{publications and conferences}.
All publications funded by \TheProject will be \textbf{open access},
and sites expecting publications have budgeted funds for paying open access fees.
We will identify and attend appropriate conferences for disseminating our work,
including running tutorials at conferences in historically interested communities such as PyData and SciPy.
Also, we will identify and attend conferences from complementary communities such as ROpenSci,
Mozilla Science, and Julia,
as well as domain specific conferences to maximise the impact of \TheProject and to broaden its
audience outside the
traditionally included communities.

The \textbf{operation of prototype services} in \WPref{applications} is also a dissemination activity,
as services like Binder not only enable open and reproducible science by facilitating interactive publications,
they also enable \textbf{interactive demonstration of tools and functionality}
developed in \TheProject.

\TheProject will also actively seek \textbf{collaboration} (through the Community
Engagement Panel and
\TheProject partners) with existing \textbf{EOSC projects} (many already use/deploy
Jupyter notebook services) to inform and support them in exploiting \TheProject
developments and adapting their service offerings.
\subsubsection{Exploitation}

Our primary outputs are in the form of software tools, prototype services, and information resources,
all of which will be freely available to all under appropriate permissive open
licenses (such as BSD/MIT).
This means that exploitation generally has the form of:

\begin{enumerate}
  \item Use of Binder tools for \textbf{producing reproducible environments} in research, as enabled by \WPref{reproducibility}
  \item Use of Binder tools for \textbf{evaluating reproducible environments} in research or publication \textbf{policy}, as enabled by \WPref{reproducibility}
  \item Use of Binder tools in the \textbf{construction and operation of services}, as enabled by \WPref{impact}
  \item Deploying \textbf{new services} derived from our science demonstrators in \WPref{applications}
  \item \textbf{Application of effective practices for reproducibility}, developed and disseminated in \WPref{education}
  \item \textbf{Commercial exploitation} via product development and consulting.
\end{enumerate}

A key aspect of our exploitation strategy is to focus our work on existing,
active projects in Binder and the wider Jupyter ecosystem.
This means that our contributions are of \textbf{immediate practical benefit to between
  thousands and millions of users} worldwide (Section~\ref{sec:target-groups-and-scale-of-impact}).

\TheProject contributions to Binder tools will be immediately (\emph{i.e.
  already during the duration of the project}) \textbf{exploited via the public mybinder.org service}
and its thousands of daily users (Section~\ref{sec:mybinder}), as well as \textbf{many BinderHub instances worldwide}.

The science demonstrators in \WPref{applications} are themselves exploitations of the outputs
in \WPref{reproducibility} and \WPref{impact},
building services not possible before the project.
\textbf{Continued operation and adoption of} these \textbf{reproducibility services} by project partners and the researchers they serve beyond
the duration of the project is anticipated for a majority of the science applications, and will indicate
successful exploitation of the project.

%\TOWRITE{ALL}{Specific per-site exploitation?}  Not needed I think - we have
%the demonstrators clearly mentioned.

The \site{QS} SME provides \textbf{commercial support and development services around the
open-source} scientific computing ecosystem. The team has specialised in the Jupyter ecosystem
and package management solutions, which are both at the core of this proposal. The consolidation
of Binder's technical foundations will help QuantStack provide robust solutions to its clients.

Industrial use cases met by QuantStack also show that the reproducibility issues addressed in this proposal go
beyond the scientific use case:

\begin{compactitem}
\item For example, \textbf{financial institutions} using numerical libraries to price and hedge derivatives must be able
   to reproduce results obtained with past versions of their software, especially in cases of mispricing or
   mishedging. Rolling back to a state of the software environment as of several years earlier may prove
   extremely difficult. Adopting the approach of Binder to favour reproducibility has proven to be a viable way
   to address this issue.
\item Similarly, in the field of \textbf{industrial robotics}, QuantStack has developed a conda-based distribution of the
   ROS open-source ecosystem called RoboStack, and a tool akin to repo2docker to produce container images providing
   all required packages for a ROS node, that will directly benefit from this work.
\end {compactitem}

This shows that techniques and software tools that arise from the computational
research communities to address reproducibility
will help address the same issue for a much broader audience, including commercial applications.

\subsubsection{Communication activities}

The main goals of the communication activities is making sure that researchers
and potential users are aware of the outputs, tools and services developed by \TheProject
on an ongoing basis, and to demonstrate to the public the clear benefits of the work they have funded.

In order to maximise this impact, it is vital to address the audience as one project
and ensure the immediate recognition of information stemming from it.
Together with all partners involved, \TheProject will therefore build a \textbf{strong project identity}
(see draft of the communication, dissemination, and exploitation plan in Section~\ref{box:draft-C-D-E-plan}) to strengthen the project
identity and to deliver clear messages to our audiences.

We will operate a \textbf{website}
(\taskref{management}{website}) for collecting and sharing information about \TheProject and its progress.
It will provide a centralised way to access the various publicly available deliverables, publications
and articles related to the project. The site will be regularly updated over the lifetime of the project
with the project publications and public materials, such as flyers, posters and
public deliverables, organised workshops, available services, news, etc.
Site analytics will be associated with the project website, in order
to provide useful insight on how to improve its impact. In addition, the project intends to
develop its presence on \textbf{the social and content
networks}. The channels will be used for interaction
with the professional community as well as the general public
(differentiation on the content per channel based on the target group wishing to address).
As part of the project's communication plan, \TheProject will develop a social media strategy
in order to increase outreach and social impact, which can be summarised as follows: (a) identifying target
audience and key stakeholders, (b) updating social media content and sparking
discussion in social media/tweeting, (c) measuring social impact and reassessing
social media strategy as required.

\subsubsection{Communication, dissemination, and exploitation plan}

We will continually refine our Communication, Dissemination, and Exploitation plans, starting with the draft below:
\label{box:draft-C-D-E-plan}
\begin{framed}

  \centerline{\textbf{ \TheProject DRAFT COMMUNICATION, DISSEMINATION, and EXPLOITATION}}
  {
  \begin{itemize}

\item Logo creation, standard document templates for deliverables, reports, letters, presentations, project posters/leaflets, etc. and creation and/or use of relevant social media accounts such as Linkedin, Twitter, YouTube channel (80k Twitter followers for @ProjectJupyter, 34k for @IPythonDev, and 132k for the PyData Youtube channel);
\item Annual plan for publication in scientific and technical peer-reviewed journals and conference proceedings;
\item Creation of \TheProject website:
\begin{itemize}
\item links to \TheProject repository of results (videos, training material, use cases, demonstrators, software and associated documentation such as repo2docker, JupyterJub and Binder);
\item links to social media accounts, e.g. on Facebook, Instagram, LinkedIn and Twitter and public communication channel (Zulip);
\item live “Infoboard” to highlight news and outcomes, event calendar, public resources, specific social media posts; This infoBoard will be relayed by the News (and Newsletter) of partnered projects (Jupyter, Pangeo, Ifremer, NeIC, Sigma2, etc.).
\item links to the EOSC catalogues of services and national/institutional
  services using and supporting \TheProject{} tools and outcomes;
\end{itemize}
\item Use the Community Engagement Panel to review the target audience for dissemination (workshops, training, hackathon, etc.), create a live list of communication partners and associated channels and tools;
\item Coordinate (annually) with other initiatives (The Carpentries, CodeRefinery, Jupyter, Pangeo) for the collaborative development of training material and the delivery of demos, training/workshops/hackathons;
\item Refine measurable goals (based on inputs/feedback from the Community
  Engagement Panel), quantitative KPIs for dissemination and monitoring procedures;
\item Develop and maintain a live collaboration plan for links and interactions with other projects (EC-funded, Nordic and/or national projects), research institutions and industries to find opportunities to show-case \TheProject results;
\item Develop a consolidated exploitation plan to ensure the long term and sustainable exploitation of the project results beyond its lifetime, in particular the deployment of EOSC services.
    \end{itemize}
}
\end{framed}

\clearpage
\subsection{Summary}

\eucommentary{
Provide a summary of this section by presenting in the canvas below the
key elements of your project impact pathway and of the measures to
maximise its impact.
KEY ELEMENT OF THE IMPACT SECTION
}

% measured original color as 0, 0.75, 0.95
% but it's a bit low-contrast

\definecolor{summaryblue}{rgb}{0, 0.7, 0.9}
\newtcolorbox{summarybox}[1]{
    colback=white,
    colframe=summaryblue,
    fonttitle=\bfseries,
    title={#1}
}

\begin{multicols}{2}

\begin{summarybox}{SPECIFIC NEEDS}
\eucommentary{What are the specific needs that triggered this project?}

Computational reproducibility is a widely recognised challenge. \textbf{Most
researchers need computational reproducibility} to achieve reproducible science
outcomes, such as publications.

There are many tools that solve \emph{part} of the problem,
or aim to solve the whole problem while requiring wholesale adoption of a specific tool.
This may not be practical or desirable across a variety of domains or
communities.

Reliably \textbf{reproducing software environments} without requiring adoption of any
single tool chain -- as fostered by \TheProject{} --
allows for modular adoption, integrating into policies, practices, etc.
We will \textbf{lower the practical barrier to reproducibility},
and thereby increase the adoption of reproducible practices.
\end{summarybox}

\begin{summarybox}{D \& E \& C MEASURES}
\eucommentary{What dissemination, exploitation and communication measures will you apply to the results?}

\textbf{Exploitation:} All services based on BinderHub will immediately benefit from the project, improving tools
already used by hundreds of thousands of users. They will also enable the application of public policies that advocate
or require reproducibility of scientific outputs. Beyond Binder, underlying tools such as \repotodocker{}
are used in other parts of the ecosystem to enable the creation of software environments. This is of particular
interest for several of the science applications in the project, as well as other Jupyter-based platforms
currently deployed for research and education.
Such projects will also directly benefit from the improvements to \repotodocker{}.

The QuantStack SME will commercially exploit the improvements to Binder and its technological foundations in areas ranging from
industrial robotics to quantitative finance.

\textbf{Dissemination:} Results, insights, tools, and guidelines will
  be disseminated to potential users and stakeholders through conventional
  academic channels, in-person and remote workshops, websites, video tutorials
  and social media.

\textbf{Communication:} The media and general public will be informed
  about the project through news releases, videos, a website, social media,
  further media engagement, flyers and interviews, with a consistent visual
  identity.
\end{summarybox}

\begin{summarybox}{EXPECTED RESULTS}
\eucommentary{What do you expect to generate by the end of the project? }

\textbf{Improved software tools for reproducing computational environments}:
by improving the Binder tools for reproducible environments,
it will be easy for researchers to produce and share results openly and reproducibly.

\textbf{Improved understanding of good practices for reproducibility}: study results, documentation, and workshops will aid researchers and policy makers in understanding the most appropriate practices to adopt in their pursuit of reproducible research.

\end{summarybox}

\begin{summarybox}{TARGET GROUPS}
\eucommentary{Who will use or further up-take the results of the project? Who will benefit from the results of the project?}

\textbf{Computational science practitioners}: researchers with an interest in reproducibility of their own work.

\textbf{Research infrastructure providers}: organisations supporting researchers
through experimental and computational services.

\textbf{Research institutions}: institutions facilitating or enforcing the reproducibility of their researchers.

\textbf{Policy makers and publishers}: Funders and institutions requiring their subjects to follow reproducible practices.
\end{summarybox}

\begin{summarybox}{OUTCOMES}
\eucommentary{What change do you expect to see after successful dissemination and exploitation of project results to the target group(s)?}

\textbf{Adoption of Binder tools for reproducibility}:
practitioners will have access to Binder tools for reproducibility,
having improved their usability and robustness.

\textbf{Resources for reproducible practices}:
practitioners will have access to resources to learn about effective reproducibility practices.

\textbf{Facilitating practical policies for reproducible publications}:
policies will have access to tools for validating reproducible practices that they require.
\end{summarybox}

\begin{summarybox}{IMPACTS}
\eucommentary{
What are the expected wider scientific, economic and societal effects of the project contributing to the expected impacts outlined in the respective destination in the work programme?
}

\textbf{Improved reproducibility of scientific results}: Improving the \emph{ease of use} tools for reproducibility
lowers the barrier for adoption, and thereby increases the number, quality, and
access of reproducible research outputs.

\textbf{Increased re-use of scientific results}: Practical reproducibility
enables rapid re-use of published results leading to more effective research activities.
\end{summarybox}
\end{multicols}
\clearpage

% \noindent
% \fbox{
% \begin{minipage}{
% % \dimexpr\linewidth-2\fboxrule-2\fboxsep
% 2in
% }
% \textbf{SPECIFIC NEEDS 2}
%
% xyz
% \begin{compactenum}
% \item x
% \item y
% \end{compactenum}
% \end{minipage}
% }
% {
% \hfill\makebox[0pt]{\fbox{
% \textbf{SPECIFIC NEEDS}}}
% \hfill
% }

% \textbf{SPECIFIC NEEDS}
%
% xyz
% \begin{compactenum}
% \item x
% \item y
% \end{compactenum}




%%% Local Variables:
%%% mode: latex
%%% TeX-master: "proposal"
%%% End:
