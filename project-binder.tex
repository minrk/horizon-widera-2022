\paragraph*{Binder}\label{seq:project-binder}

The \textbf{Binder tools} were designed to execute
Jupyter notebooks in tailored computational environments which can be created
automatically and on demand (Step 4 in figure~\ref{fig:use-cases-binder}).
The \textbf{fully-automated} creation of the correct \textbf{software environment} is an
essential aspect of reproducibility that is not widely addressed yet.
The \TheProject{} project will address this need,
improve the \textbf{automatic software environment creation} support for computational research,
making such tools \textbf{practical} for uses cases with and without notebooks.

The Binder Project \cite{binder} (\url{https://jupyter.org/binder}) is
a subproject of the Jupyter project.
The Binder project is most popular with existing Jupyter notebook users,
but not restricted to be useful only for notebooks.

The key components of the \textbf{Binder tools} are \repotodocker{}
and \binderhub{}.
The \repotodocker{} tool creates a \textbf{software environment} inside a Docker image from a \textbf{software
specification} in a \textbf{repository}.
\binderhub{} starts a \textbf{Jupyter server}
within this container from which the user can execute the notebooks from the
repository.

The focus for this proposal is to improve \repotodocker{}. In particular,
\repotodocker{} solves the \textbf{software environment} challenge (see
Section~\ref{sec:reproducibility}) in a generic way and is independent
from Jupyter notebooks.

\TODO{Add image that depicts this Jupyter/Binder relationship.}

\paragraph*{Basic functionality of Binder}
\label{binder-how-does-it-work}

The currently most common reproducibility use case -- with the current state of the Binder
tools -- is the one we introduced in
Section~\ref{sec:reproducibility-example}. We will use this to
describe the role of the individual components of Binder:

\begin{compactitem}
\item The \binderhub{} service is given a URL that encodes the location of the data
  repository\footnote{For example the
    {\url{https://mybinder.org/v2/gh/fangohr/reproducibility-repository-example/HEAD?labpath=figure1.ipynb}}
    refers to the GitHub repository ``reproducibility-repository-example'' of the
    github user ``fangohr'', asking to open the ``figure1.ipynb'' file.}
  (see Figure~\ref{fig:mybinder-homepage})
\item \binderhub{} will use
  \repotodocker{} to create a \textbf{software environment} (Docker image)
  in which the notebooks or scripts from the repository can be executed.
\item \repotodocker{} searches the repository for \textbf{software specifications}.
\item \repotodocker{} constructs the \textbf{software environment}
  from the \textbf{software specification(s)} (a Docker image).
\item \binderhub{} asks \JupyterHub to start
  a \textbf{notebook session} in this \textbf{software environment}.
\item \binderhub{} forwards the user who requested this environment to
  the URL at which the repository (or a particular notebook) can be explored
  from within the Jupyter application.
\end{compactitem}

\paragraph*{The \mybinder{} service}
\label{sec:mybinder}
\mybinder{} is a service run by the \emph{BinderHub
federation}\footnote{\url{https://mybinder.readthedocs.io/en/latest/about/federation.html}}
of organisations. Collectively, they host a service running the \binderhub{} software
which can be reached at \url{https://mybinder.org}.
The service is actively used, with approximately 200,000 sessions being
requested and delivered by \mybinder{} every week in 2021. The number
of sessions is growing from approximately 10,000 per day in November 2018
(beginning of the available records) to 30,000 per day in 2022. We have
identified 55,000 unique repositories which used the \mybinder{} service in 2021 alone.
The data is open and available~\cite{mybinder-archive}.
The improvements made to \textbf{Binder tools} by \TheProject will be immediately
and freely available to all operators of BinderHubs, including \mybinder{}.
Examples of reproducible repositories that make use of the \mybinder service
include reproducible research repositories
\cite{GitHubRepoExampleAlbert2016,Beg2021}, interactive textbooks
\cite{Fangohr2022,Zeller2022} and Nobel Prize-wining citizen science and outreach activities
\cite{ligo-open-science,OSCOVIDA2022}.

\begin{terminology}{(Binder)}
\begin{description}
\item[Project Binder] The Project Binder is described in Section
\ref{seq:project-binder}. It is part of the Jupyter ecosystem of tools, and
allows to convert a repository with Jupyter notebooks into an browser-hosted
environment, in which the notebooks can be executed interactively (and thus
results can be reproduced).

\item[Binder tools] The Binder tools consist of \binderhub{} and \repotodocker{}.

\item[\repotodocker] \repotodocker{} is a tool to fetch a \textbf{repository} and build a \textbf{software
environment} for this repository. Currently, \repotodocker{} can retrieve
repositories from many services and formats, including GitHub, Zenodo, Dataverse, and others.
For the automatic building of \textbf{reproducible computational environments},
\repotodocker{} understands commonly used conventions for \textbf{software specifications} and
community standard tools such as Docker, conda, mamba, Julia, R, nix, and pip.

\item[\BinderHub] software for hosting a web service built on \repotodocker and \JupyterHub where individuals can share reproducible environments for
immediate and free interaction by readers in their browser,
  with no need to install anything themselves.
\item[mybinder.org] A flagship instance of \BinderHub,
  serving over 10 million sessions in 2021.
\end{description}
\end{terminology}


% \subsubsection{Binder for reproducibility}\label{sec:binder-for-reproducibility}
% \TOWRITE{}{Hmm -- perhaps we don't need this section, as we explain in the
%   Methodology section what we want to do with Binder?}




%%% Local Variables:
%%% mode: latex
%%% TeX-master: "proposal"
%%% End:
