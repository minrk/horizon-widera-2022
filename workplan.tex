
\eucommentary{Please provide the following:\\
\begin{compactitem}
\item
brief presentation of the overall structure of the work plan;
\item
timing of the different work packages and their components (Gantt chart or similar);
\item
detailed work description, i.e.:
\begin{compactitem}
\item
a description of each work package (table 3.1a);
\item
a list of work packages (table 3.1b);
\item
a list of major deliverables (table 3.1c);
\end{compactitem}
\item
graphical presentation of the components showing how they inter-relate (Pert chart or similar).
\end{compactitem}
}

\subsubsection{Quality and efficiency of the implementation}\label{sec:workplan-structure}

\ifgrantagreement
The
\else
As shown in Table~\ref{fig:wplist}, the
\fi
work plan is broken down into
six work packages:
\WPref{core} about maintaining the core Jupyter software infrastructure,
\WPref{ecosystem} for developing the ecosystem of software surrounding Jupyter,
\WPref{applications} for building applications to demonstrate the efficacy and guide the development
of core infrastructure,
\WPref{eosc} for operating services built on these
components and collaborating with existing EOSC stakeholders,
\WPref{education} for educating the public on Open Science best practices with the project's tools
and fostering diversity in the research and software communities.
This is complemented by
the usual management  work package
(\WPref{management}).
\TODO{Gantt chart link: Remove the next sentence, if we remove the Ganttchart.}
The Gantt chart on
Page~\pageref{fig:gantt} illustrates the timeline for the various
tasks for these work packages.
%, including inter-task dependencies.

\ifgrantagreement\else
%\makeatletter\wp@total@RM{management}\makeatother
\wpfigstyle{\footnotesize\def\tabcolsep{3.5pt}}
%\wpfig[pages,type,start,end]
{\wpfig}
\fi

\begin{figure}[htb]
  \centering
  \includegraphics[width=0.9\textwidth]{images/WP_low.png}
  \caption{
    \label{fig:workpackages}
    The relationships and interactions of the work packages,
    broken up into four main categories: Management (WP1),
    Development of new functionality surrounding Jupyter (WP2, WP3),
    Demonstrators and Services (WP4),
    and Education and Dissemination (WP5).
    Ultimately, all work packages benefit from and feed back to
    all other work packages.
  }
  \TODO{This Figure needs updating.}
\end{figure}

% \subsubsection{How the Work Packages will Achieve the Project Objectives}
% \label{sssec:how_the_work_packages_will_achieve}

\ganttchart[draft,xscale=.33,milestones]
\TODO{Gantt chart: HF: seems useless as it is. Can we fiddle with options? Otherwise
remove.}

\ifgrantagreement\else
\newpage
\subsubsection{Deliverables}\label{sec:deliverables}
\inputdelivs{9.3cm}
\TODO{Strage vertical lines at the left of the bottom of table~\ref{sec:deliverables}?}
\fi


\newpage
\subsubsection{Milestones}\label{sec:milestones}
\eucommentary{Milestones means control points in the project that help to chart progress. Milestones may
correspond to the completion of a key deliverable, allowing the next phase of the work to begin.
They may also be needed at intermediary points so that, if problems have arisen, corrective
measures can be taken. A milestone may be a critical decision point in the project where, for
example, the consortium must decide which of several technologies to adopt for further
development.
}
\begin{draft}
\begin{verbatim}
* TODO
- [ ] we refer to \TheProject SERVICES a lot
- [ ] would it be better to refer to TOOLS instead?
- [ ] we should probably list the services / tools somewhere for clarity
\end{verbatim}
\end{draft}


% \milestonetable



\begin{milestones}
  \milestone[
    id=startup,
    month=12,
    verif={
      Completed all corresponding deliverables
      and preparation for deployment of prototype services is underway
      }
    ]
  {Project startup, requirements gathering, and exploratory prototypes}
  {

  By milestone 1, we will have established the infrastructure for the project
  and begun exploratory prototyping development and deployment of services,
  engaging with the existing communities. We will have a network of advisors
  through the Community Engagement Panel and are coordinating our plans for
  \TheProject with those of wider open science communities and the Jupyter
  project. We will have preliminary study results to guide and evaluate
  improvements to reproducibility with \TheProject tools. }

  \milestone[
    id=prototype,
    month=24,
    verif={
      Developed functional prototypes for all important topics.
      Early users are able to access and test prototype services
    }
    ]
  {Prototype demonstrator services}
  {

  By this point, prototype demonstrator services will be useful and accessible
  to a broad range of users, and we will have begun to experiment with early-adopter
  users and local demonstrators to guide further development of \TheProject,
  ensuring that development serves the reproducibility needs of the global science community.

  There will be improvements to reproducibility.
  }

  \milestone[
    id=final,
    month=36,
    verif={Refactored and stabilised prototypes.}
    ]
  {Project conclusion}
  {
  At the end of the project,
  we will have engaged with the science communities to evaluate demonstrator services, and
  identified which services and tools shall be sustained beyond the life of the project.
  We will have made those robust and maintainable from a software engineering point of view.

  We will have training materials and have run workshops to train users in Reproducibility and Open Science,
  making use of tools and services developed through \TheProject.

  We will have developed a sustainability plan for how future maintenance and
  development of \TheProject tools be achieved under community support and
  leadership.

  \TheProject tools are more reliable and useful to a broader audience than at
  the start of the project.

  }


\end{milestones}


% ---------------------------------------------------------------------------
% Include Work package descriptions
% ---------------------------------------------------------------------------

\draftpage
\subsubsection{Work package descriptions}\label{sec:workpackages}
%%% work package style may be broken -- fix this!!

\ifgrantagreement
\begingroup
% Note: in the grant agreement, The workpackage description must not appear.
% Yet we want to compile them to get all the metadata right
% Current hack: compile them anyway, reset the page number
% appropriately, and remove them a posteriori with pdftk. We set the
% color to red to make it more visible in case we forget to remove
% them.
% See grantagreement rule in the Makefile
\newcounter{savepage}
\setcounter{savepage}{\value{page}}
\color{red} % To make sure we indeed remove the pages
\fi

%\enlargethispage{1cm}

%% Local WP number counter - should possibly be global and hidden?
\TOWRITE{ALL}{Proofread WP 1 Management pass 1}
\begin{draft}
\TOWRITE{PS (Work Package Lead)}{For WP leaders, please check the following (remove items
  once completed)}
\begin{verbatim}
* TODO items (27 March 2022)
- [ ] check links to local deliverables
- [ ] Need data management plan
- [X] do we need an IPR management plan?
      - I don't think so (HF). The IPR management needs to be part of the
        consortium agreement.
- [ ] do we need a Innovation Management Task? It is commented out for now. As
      we are open source all the way through, we don't need to worry about IPR so
      much, and we also don't need to establish a common understanding.
- [ ] have only one data management plan: it should not be hard to decide to
make everything open source. (i.e. get rid of 'draft' plan at M1)
- [ ] update mile stone names so this WP can be completed
- [ ] should we have a \Binder command?
- [ ] Is 'Binder' the right name to refer to? (Should be consistent throughout.)
- [ ] We mention here that all outputs will be open source (technical
      management). This should maybe go elsewhere?
- [ ] should also check with collaborators that they are happy with this.
\end{verbatim}
\end{draft}

% (12+4)/36 = 0.4444444444444444 -> !.44
\begin{workpackage}[id=management,type=MGT,wphases=0-36!.25,
  title=Project management,
  short=Management,
  lead=SRL,
  MPRM=1,
  QSRM=1,
  IFRRM=1,
  UIORM=1,
  SRLRM=12,
  swsites
]
\begin{wpobjectives}
  The main objective of WP1 is to establish and maintain an effective contract,
  project, and operational management approach ensuring:

 \begin{compactitem}
 \item Timely and successful implementation of the project; including
   administrative and legal coordination
    \item Technical management and quality assurance
    \item Risk and innovation management of the project as a whole; including
      data and IPR management
    \item Smooth communication and interaction with the EC and other interested
      parties
 \end{compactitem}
\end{wpobjectives}

\begin{wpdescription}
  The project will be managed by Simula, which has extensive experience in
  administering and leading EU funded and national projects. The coordinator
  together with the WP leaders, will be responsible for monitoring WP status,
  coordination of work plan updates and annual internal progress reports.
  \TOWRITE{}{Management details section has been removed}
  % The project management structure and roles of partners in the consortium are
  % presented in \ref{sect:mgt}.

\end{wpdescription}

\begin{tasklist}

\begin{task}[
  title=Administrative Management,
  id=admin,
  lead=SRL,
  PM=6,
  wphases={0-36!.166},
  partners={MP,QS,UIO,IFR},
]
The task includes the following activities:
\begin{compactenum}
\item Preparation, distribution and maintenance of all contractual documents
  (Consortium Agreement, Grant Agreement and all other legal frameworks)
\item Establishment of appropriate communication and collaborative environment
  for the consortium, as well as the EC and other relevant academic and industry
  stakeholders (the project website, intranet and communication procedures) to
  organise transfer of knowledge, present and promote project results
  (\localdelivref{infrastructure});
\item Organisation of project review and progress meetings;
\item Performing qualitative and quantitative risk analysis, planning risk
  mitigation and control
\item Progress and Financial Reporting to the EC;
\item Data and IPR Management will be managed in accordance with agreed rules
  stated in the Consortium Agreement and in accordance with the Data Management
  Plan (\localdelivref{data-management-plan}.
\end{compactenum}
\end{task}

\begin{task}[
  title=Technical Project Management,
  id=project-management,
  lead=SRL,
  PM=6,
  wphases={0-36!.166},
  partners={MP,QS,UIO,IFR}
  ]
  The task includes the following activities:
  \begin{compactenum}
\item The scientific and technical management to ensure coherent quality and
  soundness of the work and results.
\item Applying quality assurance measures across all partners for all tasks and
  deliverables.
\item Reporting of outcomes and quality assurance activities in technical
  reports and reviews.
\item The project coordinator, with the help of the work package leads, will
regularly review technological risks and recommend mitigation plans to minimise
or remove them. This will be reported on at each reporting period in the
project's technical report.
\item Set up and maintenance of technical management infrastructure required for
  a software project of this type, such as a web site, open source hosting of code and
  documentation, mailing lists, task trackers, automatic tests and continuous
  integration. We will feed back into existing open source repositories projects
  where they exist already, and make use of commonly used tools and services
  such as GitHub. All outputs will be published under an open source license.
\end{compactenum}
\end{task}

\begin{task}[
  title=Management of dissemination and communication activities,
  id=website,
  lead=SRL,
  PM=2,
  % >>> 2/36 = 0.05555555555555555
  wphases={0-36!.056},
  partners={}
]

This task comprises the management and administrative aspects of all forms of
direct dissemination and public communication activities such as press releases,
scientific and technical publications, seminars, talks, promotion through social
media, creation of advertisement materials such as flyers, posters, and
electronic feeds as well as their distribution. We will use standard community
building technology such as mailing lists, wikis and forums, to ensure
dissemination to and engagement with the user community.
\end{task}


% \begin{task}[
%   title=Innovation Management,
%   id=innovation-management,
%   lead=SRL,
%   PM=4,
%   wphases={0-36!.111},
%   partners={MP,QS,UIO,IFR}
% ]
% One of the most important criteria for success for the \TheProject project is to
% bring the project results into use. Therefore, exploitation routes will be
% sought whenever possible. In order to create a common understanding within the
% Consortium of how we can best shepherd an idea all the way from conception to
% realisation and exploitation, the Coordinator will be responsible for the
% preparation and realisation of an Innovation Plan. This plan will assure that
% research activities meet the required milestones and produce outputs fully
% aligned with the project objectives. All research activities will go through an
% initial process where the exploitation opportunity is identified along with the
% main stakeholders for the exploitation opportunity and an IP owner
% (\localdelivref{innovation-management-plan}).
% \end{task}

\begin{task}[
  title=Community Engagement Panel,
  id=community-engagement-panel,
  lead=SRL,
  PM=2,
  wphases={0-36!.056},
  partners={MP,QS,UIO,IFR},
  ]

The task includes the following activities:
\begin{compactenum}
\item Form the community engagement panel by inviting representative of relevant
  communities. Ensure that representatives from stakeholder communities include
  current and potential future Binder users.
\item Organise regular (online) community engagement panel meetings, soon after the
  beginning of the project, and subsequently at the end of years 1, 2, and 3.
\item Ensure that input and feedback from community engagement panel members are
  considered to direct the project to improve the usefulness of Binder tools
  and broaden the range of their applicability to maximise overall impact.
\item Encourage and foster voluntary collaboration and direct contributions to
  the project from the communities represented in the community engagement
  panel that go beyond the advisory role of the panel itself.
\end{compactenum}

\end{task}
\end{tasklist}


\begin{wpdelivs}

\begin{wpdeliv}[due=2,id=infrastructure,dissem=PU,nature=DEC,lead=SRL]
  {Basic project infrastructure (web site, mailing lists, issue trackers, mailing lists, repositories)}
\end{wpdeliv}


  \begin{wpdeliv}[due=3,id=dissemination-plan,dissem=PU,nature=R,lead=SRL]
    {Detailed dissemination, communication, and exploitation plan.}
  \end{wpdeliv}


% “Proposals selected for funding under Horizon Europe will need to develop a
% detailed data management plan (DMP) for making their data/research outputs
% findable, accessible, interoperable and reusable (FAIR) as a deliverable by
% month 6 and revised towards the end of a project’s lifetime.”
\begin{wpdeliv}[due=6,id=data-management-plan,dissem=PU,nature=DMP,lead=SRL]
  {Data Management Plan}
\end{wpdeliv}

\begin{wpdeliv}[due=32,id=data-management-plan-revised,dissem=PU,nature=DMP,lead=SRL]
  {Revised Data Management Plan}
\end{wpdeliv}

% \begin{wpdeliv}[due=9,id=innovation-management-plan,dissem=CO,nature=R,lead=SRL]
%   {Innovation Management Plan}
% \end{wpdeliv}

\end{wpdelivs}
\end{workpackage}
%%% Local Variables:
%%% mode: latex
%%% TeX-master: "../proposal"
%%% End:

%  LocalWords:  workpackage wphases wpobjectives wpdescription pageref wpdelivs wpdeliv
%  LocalWords:  dissem mailinglists swrepository final-mgt-rep compactitem swsites ipr
%  LocalWords:  TOWRITE tasklist delivref

\draftpage
\begin{draft}
\begin{verbatim}
- [ ] distribute tasks for repo2docker to other workpackages
\end{verbatim}
\end{draft}

\begin{workpackage}[
  id=reproducibility,
  wphases=0-36!1.03,
  title=Improving robustness of reproducibility tools,
  short=Robustness,
  lead=QS,
  SRLRM=23,
  UIORM=0,
  MPRM=2,
  QSRM=12,
  swsites,
]
\begin{wpobjectives}
  \begin{compactitem}
    \item to better understand and evaluate successful reproduction of computational environments
    \item to improve the practical reproducibility of environments constructed
      with \TheProject tools
    \item to support and maintain core Binder software infrastructure in order to keep it healthy
         and useful for open science and reproducibility
 \end{compactitem}
\end{wpobjectives}

\begin{wpdescription}
\TOWRITE{repurposed task from BOSSEE}

This Work package is focused on making \repotodocker{} do the things it does
already \emph{better}, \emph{more robustly} and \emph{more sustainably}.
(Orthogonal to those improvements, we plan to significantly extend the use cases
for \repotodocker{} in \WPref{impact}.)

To be able to asses the impact of our planned improvements, we need to have a
metric. Task \localtaskref{repo2docker-checker} will create this
for us. In addition to the evaluation of the improvements in this proposal, this
can be used more generally as an indicator for reproducibility of software
environments.

One major improvement to the existing capabilities of \repotodocker is the
\emph{time-machine} functionality, and this is implemented in \localtaskref{repo2docker-timemachine}.

In task \localtaskref{performance-optimisation}, we will speed-up the execution time of
\repotodocker{} to improve the user experience when reproducing or re-using
existing software and data.

Open source software needs ongoing maintenance to adapt to changing requirements
and dependencies. We schedule a certain amount of time for this in task
\localtaskref{maintenance}.

All changes to the software will be made available already during development
(i.e. throughout the whole project), and new features will be made available
through software releases of the Binder tools. A final release will be made and
reported through the deliverable \delivref{impact}{binder-tools-software}.

% the existing functionality of repodocker.
% Community-led open source software is critical to a sustainable future for open science.
% Commonly used tools make up a shared infrastructure,
% where investment in core components benefits the widest user community.
% \TheProject is centred around the Jupyter project,
% which is a collection of projects for interactive computing and
% communicating computational ideas.
%
% This work package is focused on developing and maintaining
% the core of Jupyter.
% In particular, we will help maintain these projects to meet the needs of the
% Jupyter community, with a focus on needs for open science.
% To serve the needs of \TheProject,
% Jupyter core infrastructure will need improvements
% to security, performance, and scalability,
% which will be provided in \localtaskref{maintenance}.
% In addition, we will develop new features in the core of Jupyter
% to bring it to a wider audience,
% and to improve its usefulness to those working toward open science practices,
% including via collaboration features (\localtaskref{collaboration})
% and accessibility (\localtaskref{accessibility}).

\end{wpdescription}

\begin{tasklist}

% template for a task
% each task should be added to exactly one workpackage
% in the workpackage task list
\begin{task}[
  title=Sample Task,
  % task id for references
  id=task-id,
  % lead institution ID
  lead=SRL,
  PM=1,
  wphases={0-36},
  % partner institution ID(s)
  % don't include lead here
  partners={XXX}
]
  The task includes the following activities
  \begin{compactitem}
  \item ...
     % deliverable will be defined in the appropriate WorkPackage.tex
    % (\localdelivref{deliv-id})
  \end{compactitem}
\end{task}

\begin{task}[
  title=repo2docker development,
  id=repo2docker-timemachine,
  lead=SRL,
  PM=38,
  wphases={0-36!1.055},
  partners={QS,MP}
]

Often, a repository of scientific results may
specify which software library is required (such as the Python library
\softwarename{pandas}), but not which version. A software environment creation
tool -- such as \repotodocker -- can then attempt to install the most recent
version of \softwarename{pandas}. This is usually the intention of the authors,
and was correct at the time the repository was created.  However, as time moves
on, the interface, behaviour and dependence on other packages of
\softwarename{pandas} will change, and at some point an automatic build of the
software for the whole repository may fail because of conflicting dependencies.

We have found through anecdotal evidence that these problems can be overcome if a
\softwarename{pandas} version can be chosen that was the most recent at the time
when the repository was created.

In this task, we will teach \repotodocker to establish the data of publication
(or last modification) of the repository, to determine the appropriate version
of software libraries from that time, and to select libraries with those
versions if no specific version is specified.

XXX


\TOWRITE{There is probably more to be done?}

\TOWRITE{HF: The following is good - but should probably go elsewhere.}
  Running someone else's analyses is a particularly difficult problem.

  There are differences between operating systems, versions of installed
  software and access to the required data sets.
  
  These challenges mean that is currently considered to be beyond the scope of
  an expert peer reviewer to verify data science analysis codes before
  publication.
  
  BinderHub, part of Project Jupyter, enables one-click running of git repositories.

  BinderHub provides a web interface to the repo2docker tool.
  \TOWRITE{End of section to move elsewhere.}


  \TOWRITE{HF: The following items are all good. I think some of the should be
    scattered across the other workpackages.}
  
  The task includes the following activities
  \begin{compactitem}
  \item extend repo2docker with support for execution on cloud resources
  \item extend repo2docker with support for execution on HPC resources with Docker support
  \item improved "first use" experience of running repo2docker locally
  \item add support for using archives such as Zenodo as source for repo2docker and BinderHub
  \item define procedures and recommendations for long term reproducibility and sustainability of repo2docker compatible repositories
  \item create educational material describing repo2docker and its benefits to researchers
  \item Enable Openshift based deployments of BinderHub
  \item User surveys about pain points using BinderHub
  \item User authentication in BinderHub
  \end{compactitem}
\end{task}

% template for a task
% each task should be added to exactly one workpackage
% in the workpackage task list
\begin{task}[
  title=Performance optimisation,
  % task id for references
  id=performance-optimisation,
  % lead institution ID
  lead=QS,
  PM=9,
  %wphases={0-24!0.375},
  % partner institution ID(s)
  % don't include lead here
  partners={SRL}
]
  The creation of reproducible software environments can take -- depending on
  the complexity and overall size of the destination environment -- quite some time.
  Often, an image can be built within few minutes,
  but there are environments that can take much longer.

  In this task, we will profile and optimise \repotodocker{} performance,
  both in terms of build time
  and image size (can be several gigabytes),
  which contributes to user-experienced performance as launching a large image
  can take longer than a small one
  when the image must be transferred across a network.
\end{task}

\begin{task}[
  title=Maintenance of open source reproducibility software,
  id=maintenance,
  lead=SRL,
  PM=6,
  %wphases={0-24!.25},
  partners={QS}
]

Developing software that people will use requires maintenance of that
software, not just new development. Through  this proposal we will
contribute general support to open source reproducibility software
where this is helpful for \TheProject. Such contributions are expected 
to the Jupyter (and as sub project) Binder code base. They support not
just all participants in \TheProject but also millions of people
relying on Jupyter software.


% Maintenance of core software is often an implicit and un-paid cost, or one
% hidden in over-describing the resources required to deliver proposed
% developments. In \TheProject, we make it clear and explicit that we will spend a
% significant amount of time developing and maintaining the core Jupyter and
% JupyterHub e-Infrastructure to respond to the needs of \TheProject and others,
% and contribute towards the sustainability and health of the community.
% 
% We will provide support to the Jupyter e-Infrastructure software, ensuring that
% it meets the needs (\localdelivref{jupyter-contributions}) of \TheProject, and
% aid in the release process to ensure that stable releases of Jupyter software
% can be used in mature \TheProject services (\localdelivref{jupyter-releases}).
% 
%   \TheProject will need improvements to core Jupyter functionality, including areas of:
% 
%   \begin{compactenum}
%     \item ease of deployment
%     \item security
%     \item scalability of JupyterHub
%     \item performance
%   \end{compactenum}
% 
%   We will contribute improvements in these areas,
%   meeting the needs of \TheProject and benefiting the wider Jupyter
%   community.

\end{task}


\end{tasklist}


\begin{wpdelivs}
  % \begin{wpdeliv}[due=1,miles=startup,id=infrastructure,dissem=PU,nature=DEC,lead=SRL]
  %   {Some Deliverable}
  % \end{wpdeliv}

  % (\localdelivref{deliv-id})

  \begin{wpdeliv}[due=24,miles=prototype,id=deliv-id-repo2docker-checker-software,dissem=PU,nature=OTHER,lead=SRL]
    {Release software tool for checking of reproducibility of software
      environments (\texttt{repo2docker-checker})}
  \end{wpdeliv}

  \begin{wpdeliv}[due=36,miles=final,id=repo2docker-checker-study-report,dissem=PU,nature=R,lead=SRL]
    {Summary of reproducibility improvements achieved. (\texttt{repo2docker-checker})}
  \end{wpdeliv}



\end{wpdelivs}

\end{workpackage}
%%% Local Variables:
%%% mode: latex
%%% TeX-master: "../proposal"
%%% End:

%  LocalWords:  workpackage wphases wpobjectives wpdescription pageref wpdelivs wpdeliv
%  LocalWords:  dissem mailinglists swrepository final-mgt-rep compactitem swsites ipr
%  LocalWords:  TOWRITE tasklist delivref

\draftpage
\begin{draft}
\TOWRITE{PS (Work Package Lead)}{For WP leaders, please check the following (remove items
once completed)}
\begin{verbatim}
- [ ] discuss what the build packs are and where they go
- [ ] how we structure and budget different tasks in this WP
- [ ] needs review and more technical detail
- [ ] add deliverable(s)
\end{verbatim}
\end{draft}

\begin{workpackage}[
  id=impact,
  wphases={6-24!0.89,24-30!0.86,30-36!0.48},
  swsites,
  title=Broadening impact,
  short=Broadening impact,
  lead=SRL,
  SRLRM=24,
  MPRM=12,
  QSRM=2,
  IFRRM=4,
  %UIORM=4,
]
\begin{wpobjectives}
  This work package extends the functionality of the \TheProject tools for
  reproducibility to broaden its applicability and increase impact.
  \WPref{applications} exploits these technological advances
  in real world use cases.

  The objectives of this work package are to
 \begin{compactitem}
 \item remove the dependency of \repotodocker{} on Kubernetes
 \item support container technologies other than Docker
 \item extend the range of software specification standards that are recognised
   and supported by \repotodocker{} \TOWRITE{}{Can we give an example here?}
 \item enable access to data from data sources outside the container
 \item enable \repotodocker{} to extract and save the software installations
   instructions (independent from container generation)
 \end{compactitem}
\end{wpobjectives}

\begin{wpdescription}
One of the design decision that have led to the current Binder software stack is
to constrain the supported infrastructure to a few key components. These
include:

\begin{compactitem}
\item Software environments can only be created inside Docker container
\item To run and orchestrate (multiple) Docker containers, a Kubernetes system must
be available
\item The user must interact with Binder environments through a Binderhub
  installation
\item There is no Binders-specific provision to access (substantial) data sets
  from inside te container. \TODO{HF: What prevents us from
    a data-publishing application at the moment? How to express this without
    making the impression one couldn't use \softwarename{curl/wget} to fetch
    data, etc. }
\end{compactitem}

The benefits of such a restrictive approach are that the software development
and maintenance effort is kept small: the wider the range of supported
infrastructure that the Binder tools can be deployed on, the higher the
complexity.

At the same time, these restrictions prevent the following scenarios:
\begin{compactitem}
\item To run the Binder software on systems where Kubernetes is not available (such as
  the Desktop of a scientist). A related use case is ``Binder@home''
  \TODO{Provide link to Binder at home task in
   \WPref{applications}}.
\item To use Binder on systems where Docker cannot be used. A use case for this
  is to create reproducible environments on HPC installations. The HPC
  administrators generally avoid use of Docker for security reasons, but much
  prefer to support container technologies that can be executed without
  administrative privileges). \TODO{Point to task in applications}
\item To use Binder for software environments that use \TOWRITE{}{insert buildback
    example / community / software system}
\item To access and use significant amounts of data from inside the software
  environment is not supported. An important use case that cannot be supported due to
  this restriction is that of ``data publishing'': the idea is that a
  (potentially large and/or complex) dataset is published \emph{together} with
  software that encodes the necessary knowledge to extract meaningful data from
  that data set. A Binder environment would make this data set accessible in an
  interactive environment. \TODO{Point to data publishing use case(s)}
 \end{compactitem}

% Open source software in general, and Jupyter in particular,
% is developed not as a monolithic application,
% but rather as a collection of related components,
% which can be assembled in numerous combinations to meet diverse needs.
% The Jupyter community is no different.
% Jupyter itself is composed of several projects,
% but there are even more projects that build on top of Jupyter to create
% things like cloud services or data pipelines.
% The goal of \TheProject is to facilitate open science through Jupyter,
% and this includes working with projects all around the Jupyter ecosystem.
% We will focus this work package on developing
% Jupyter ecosystem projects with an emphasis on open science.
%
% repo2docker is a project for creating
% reproducible environments in which Jupyter notebooks (and other user interfaces) can be run.
% It reads a number of common formats to list required software packages,
% and prepares a Docker container with those packages installed.
% BinderHub is software for operating a web service using repo2docker,
% which enables sharing of interactive and reproducible Jupyter (and Rstudio) environments on the web with a single link.
% We will develop repo2docker and BinderHub further to meet the needs of the open science community.
%
% In addition to the interactive aspects of Jupyter,
% notebooks can be used in a "workflows" style,
% where job systems run analyses and produce reports,
% either on a scheduled basis or triggered by events.
% There is a great deal of interest in using notebooks in this way,
% and much room for development of tools supporting workflows in data-driven open science.



\end{wpdescription}

\begin{tasklist}
% % add tasks from task directory here
% template for a task
% each task should be added to exactly one workpackage
% in the workpackage task list
\begin{task}[
  title=Support more software specification standards,
  % task id for references
  id=buildpacks,
  % lead institution ID
  lead=SRL,
  PM=12,
  % wphases={0-36},
  % partner institution ID(s)
  % don't include lead here
  partners={MP}
  ]

  There are two different aspects of software specification that \repotodocker{}
  needs to understand:
  \begin{compactitem}
  \item the specification of the software environment, for example through
    \softwarename{requirements.txt} files, etc (see
    Section~\ref{sec:repo2docker}
    for more details on the currently
    supported standards),
  \item from where to retrieve the software itself: this will be different on a
    GitHub repository or a Zenodo archive
    (see also \ref{sec:repo2docker} for supported repositories).
  \end{compactitem}

  For both aspects, there are requests from potential Binder users to extend the
  capabilities of \repotodocker{}.

  In this task, we will prioritise such requests and extend the \repotodocker{}
  functionality to support as many new standards as possible to best meet community needs.

\end{task}

% template for a task
% each task should be added to exactly one workpackage
% in the workpackage task list
\begin{task}[
  title=Reducing technical constraints for broader usage,
  % task id for references
  id=constraints,
  % lead institution ID
  lead=SRL,
  PM=1,
  wphases={0-36},
  % partner institution ID(s)
  % don't include lead here
  partners={XXX}
]

(Abstract current technological assumptions/requirements) (year 1-3)

  \begin{compactitem}
  \item Kubernetes
  \item Docker
     % deliverable will be defined in the appropriate WorkPackage.tex
    % (\localdelivref{deliv-id})
  \end{compactitem}

  The task includes the following activities
  \begin{compactitem}
  \item ...
     % deliverable will be defined in the appropriate WorkPackage.tex
    % (\localdelivref{deliv-id})
  \end{compactitem}
\end{task}

% template for a task
% each task should be added to exactly one workpackage
% in the workpackage task list
\begin{task}[
  title=Support more use patterns,
  % task id for references
  id=patterns,
  % lead institution ID
  lead=SRL,
  PM=16,
  %wphases={0-36},
  % partner institution ID(s)
  % don't include lead here
  partners={MP,IFR}
]
In this task, we provide the technical possibilities to use the Binder tools in
better and new ways. These are used and evaluated with real world applications
in~\WPref{applications}.

\begin{compactitem}
\item The \repotodocker{} tool searches a given repository for specifications of
  software dependencies, and carries on to compose instructions to install all
  of the software dependencies within a Docker container image. In this task, we
  modularise this functionality, and make it possible to extract the required
  instructions on their own (for example into a stand-alone shell script).

  Such functionality could be used to:
  \begin{compactitem}
  \item Extract the list of installation commands to carry out a local install
    of the required software, or an installation within any other environment.
  \item In particular on HPC systems, it may be necessary to install software
    directly on the host, and this functionality would simplify that.
  \item Having the installation instructions neatly summarised will also help
    the interested scientist to understand what software specifications are
    (explicitly or implicitly ) given within the repository.
    % \item Having the installation instructions extracted, these can be used to
    %   explore the effects of changing versions of a particular library or
    %   dependency. This will be helpful to track down the origins of observed
    %   non-reproducibility.
  \end{compactitem}

\item Improve and document the options to access external data resources from
  within the computational environments generated by BinderHub deployments. This
  is a prerequisite for \taskref{applications}{data-publishing}.
\item Deploy BinderHub with authenticated data access - useful for many
  institutes who want to offer reproducibility services and data hosting but
  need to limit or control access to those systems (to ensure, for example, that
  the computational resources are not abused).
\item Support using Binder features without a full BinderHub service. A use case for this
  is to create a Binder-like experience on a local Desktop (see
  \taskref{applications}{binder-at-home}).
\item Support deploying Binder for non-interactive use cases.

  Binder's original design goal was to allow interactive execution of notebooks within a
  software requirement that provides all the software required by the notebook.
  This may include compiled and highly customised software, which might produce
  output files, which are then processed and visualised in the notebook.

  Here, we will provide the foundations for reproducibility work that is
  non-interactive and may not use Jupyter notebooks. A use case for this is
  reproducibility at High Performance Computing facilities, where often the
  computational tasks cannot be carried out interactively (see
  \taskref{applications}{binder-at-hpc}),
  but the software environment creation problem solved by repo2docker is the same.

  \end{compactitem}
\end{task}


%
%   This feature will become part of the \repotodocker{} functionality
%   (\localdelivref{extract-dependencies}).
%
%

%%% Local Variables:
%%% mode: latex
%%% TeX-master: "../proposal"
%%% End:

%
\end{tasklist}


\begin{wpdelivs}
  \begin{wpdeliv}[due=12,id=extract-dependencies,dissem=PU,nature=OTHER,lead=SRL]
    {Release new \repotodocker{} feature that exposes the command to install
      identified software environments in stand-alone script
    %  (see \taskref{impact}{extract-dependencies}).
    }
  \end{wpdeliv}

\begin{wpdeliv}[due=36,id=binder-tools-software,dissem=PU,nature=OTHER,lead=SRL]
  {Final open source release of \TheProject tools, completed with automatic
    testing and documentation. }
\end{wpdeliv}

\end{wpdelivs}
\end{workpackage}
%%% Local Variables:
%%% mode: latex
%%% TeX-master: "../proposal"
%%% End:

%  LocalWords:  workpackage wphases wpobjectives wpdescription pageref wpdelivs wpdeliv
%  LocalWords:  dissem mailinglists swrepository final-mgt-rep compactitem swsites ipr
%  LocalWords:  TOWRITE tasklist delivref

\draftpage
\TOWRITE{ALL}{Proofread WP 1 Management pass 1}
\begin{draft}
\TOWRITE{PS (Work Package Lead)}{For WP leaders, please check the following (remove items
once completed)}
\begin{verbatim}
- [ ] have all the tasks in this Work Package a lead institution?
- [ ] have all deliverables in the WP a lead institution?
- [ ] do all tasks list all sites involved in them?
- [ ] does the table of sites and their PM efforts match lists of sites for each task?
      (each site from the table is listed in all relevant tasks, and no site is listed
      only in the table or only at some task)
\end{verbatim}
\end{draft}

\begin{workpackage}[
  id=applications,
  wphases=0-36,
  swsites,
  title=Applications and use cases,
  short=Applications,
  lead=MP,
  % EGIRM=7,
  % CDSRM=12,
  % INSERMRM=24,
  % QSRM=6,
  % SILRM=12,
  SRLRM=9,
  % UIORM=12,
  % UPSUDRM=20,
  % WTTRM=3,
  % XFELRM=36,
  % EPRM=3,
]

\TOWRITE{Everything we develop in WP2-3 should be validated here}

\begin{wpobjectives}
  The objectives of this work package are
 \begin{compactitem}
   \item to guide the development of core tools by simultaneously
     developing and using applications in diverse fields with active
     scientists from these fields, and
   \item to demonstrate that the tools we develop are valuable to diverse
     fields of science, thus ensuring we develop e-infrastructure and
     services which can cater for a broad European and global research community
   \end{compactitem}
\end{wpobjectives}

\begin{wpdescription}

  Whilst the components issued from work packages  \WPref{reproducibility} and \WPref{impact} will be
  made available as generic building blocks for reproducible open science services,
  this work package aims at specific real-world cases.

  We have selected a number of applications in a variety of domains
  to demonstrate the broad impact of \TheProject, in particular in the
  areas of \TOWRITE{tasks}
  % (\localtaskref{astro}), education
  % (\localtaskref{teaching}), fluid dynamics
  % (\localtaskref{application-gpu}), geosciences
  % (\localtaskref{geoscience}), health
  % (\localtaskref{opendose-analysis}), mathematics
  % (\localtaskref{math}) and photon science and imaging
  % (\localtaskref{reproducibility-xfel}).
  The context and vision for each of the demonstrators is described in
  section \ref{sec:science-demonstrators-in-concept} on page
  \pageref{sec:science-demonstrators-in-concept}.

  Working closely with the core developers of the Jupyter ecosystem will make it possible to
  go way beyond what is normally available "out-of-the-box" and to offer better solutions,
  thereby guiding further development of the core features.

  \medskip

  All demonstrators will validate the Jupyter service capabilities such as reproducibility,
  interactive widget use and visualisation, and show how these can
  enable new open science on EOSC.

  The particular workflows, data infrastructures and data policies for
  FAIR\footnote{Findable, Accessible, Interoperable and Reusable} sharing of data vary from one community and use-case to
  the other, or may not be fully defined yet. Therefore, this proposal
  does not enforce a specific way of handling data. Instead we
  will explore in the demonstrator tasks how existing data policies,
  infrastructure and workflows can be respected and integrated with
  authentication and authorisation, data management, and
  JupyterHub/Binder services on EOSC. EGI is a partner
  for all the tasks in this work package and will work with us to find the
  best integration solutions in the evolving EOSC
  infrastructure.

  For some of the demonstrators, authentication and authorization and/or
  data management are being advanced outside \TheProject.


\end{wpdescription}

\begin{tasklist}
% add tasks from task directory here
% template for a task
% each task should be added to exactly one workpackage
% in the workpackage task list
\begin{task}[
  title=Science demonstrators,
  % task id for references
  id=demos,
  % lead institution ID
  lead=MP,
  PM=8,
  %wphases={0-36},
  % partner institution ID(s)
  % don't include lead here
  partners={IFR,UIO}
]
In this task, we want to demonstrate the value and usefulness of
\WPref{reproducibility} and \WPref{impact} with real scientific use cases from
the research communities involved in \TheProject (see
Section~\ref{sec:science-applications} on
page~\pageref{sec:science-applications} for the initial set of science applications).

The demonstrators are designed 
to exploit the solutions developed within \TheProject (such as Binder@Home, Binder@HPC,
data publishing) and leverage existing institutional and/or national
e-infrastructures as well as core EOSC services. Synergies between the different
science applications and communities will be ensured through the technical tasks
(\taskref{applications}{binder-at-home} Binder@home,
\taskref{applications}{data-publishing} data publishing, and
\taskref{applications}{binder-at-hpc} Binder@HPC).


% \paragraph*{Context:}  For example, in marine research field there are reproducible research examples such as Argopy~\cite{maze2020},  Pangeo ecosystems (http://gallery.pangeo.io/repos/pangeo-gallery/physical-oceanography/). But even within the same research lab, we have number of researchers who depends on commercial software for their data analysis and does not have access to publish reproducible research workflows.   


% \paragraph*{Task activity:}
 
% Within the actual Binder capability, we will demonstrate following two reproducible research configurations.  We demonstrate these research use caseses and show barriers that has been preventing these workflow to be pulished as reproducible science.   

%These information will give first feed back to \WPref{reproducibility}, \WPref{impact} 
%ll enrich the process of propose better accesible optimised 
%research workflow that can benefit researchers themselves, but also include reproducible aspects. 


 % \begin{compactitem}
%  \item FAIR Nordic Earth System Modelling: this science demonstrator leverages Binder@HOME (model development, education, single column or very simple model configuration), Binder@HPC (operational runs at scale including on EuroHPC), data publishing (publication of simulation results from blue-sky research);
%  \item Demonstration of marine physics and fish habitats modelling and analysis using Pangeo ecosystem.  
  
%\TODO{Tina, move this to section 1? but where??
%This effort and outcome of it will connect to the Digital Twin Ocean project to allow ocean data and models relevant to biodiversity to be re-used by researchers and engineers. This will provide a concept demonstration of ingesting ocean data and model output that can be reproduced through the existing ocean research infrastructures
% should be able to add 'interdisciplinary' part in chap1  as it bridges biology and physics..}

     % deliverable will be defined in the appropriate WorkPackage.tex
    % (\localdelivref{deliv-id})
%  \end{compactitem}
\end{task}

% template for a task
% each task should be added to exactly one workpackage
% in the workpackage task list
\begin{task}[
  title=Binder@Home,
  % task id for references
  id=binder-at-home,
  % lead institution ID
  lead=SRL,
  PM=7,
  %wphases={0-36},
  % partner institution ID(s)
  % don't include lead here
  partners={MP,UIO}
]
In this task, we want to use the compute power of the Desktops of individual researchers and
users, to recreate software environments in which computational results can be
reproduced, and re-used: We want to provide an experience identical or similar
to that of using BinderHub, but using only the local computer.

\paragraph*{Context:} Reproducibility services using Binder currently rely on hosted Binder instances.
The BinderHub Federation provides such a service global service at mybinder.org
which is free at
the point of use, and delegates the building of the software environment and
re-execution of the code to a small number of computer centres that have
volunteered to contribute compute resources.

It thus possible to overload the system \TODO{Add link to general discussion
  that mentions how many thousand builds take place per months etc}.

To allow the up-scaling of good reproducibility practice, it would be
desirable not to depend on such a single (or even multiple hosted services).

The compute resources typically offered through mybinder are modest: at most 2 GB of RAM
and a single CPU core.
Most laptops and Desktops have similar or much better hardware capabilities than
the mybinder cloud-computing resources currently offer.

For some research areas, it is essential to access big data sets to reproduce the scientific computing workflow.  For such cases, it is essential to reproduce the work using infrastructure that has optimised access to the data, so that one does not consume unnessesary computing and network resources.  

\paragraph*{Task activity:} Based on the preparations in \WPref{reproducibility} and
\WPref{impact}, extend Binder so that users of the service can
carry out the building of the environment, and -- if desired -- launching of a
notebook server \emph{on their own hardware}, such as their laptop or on-premise or cloud infrastructures.   The working title for such
functionality is ``binder@home'' (as a reference to the crowd-based SETI@home search for
extraterrestrial intelligence at home.\footnote{https://setiathome.berkeley.edu}

We will design, implement, and test the 'binder@home' functionality, and make it
available as part of binder software. We will need a utility that takes on
the responsibility of BinderHub for a single-user use case, and which triggers
the local build of the software environment, start of Jupyter notebook
server and opening of the relevant local URL and port an a browser.
This effort will bridge the gap from mybinder.org to ``Home'';
by giving the freedom and digital sovereignty
to the researchers to chose where they execute their computational experiments in a simple manner,
without sacrificing the convenience of the mybinder.org service.


%  The task includes the following activities
%  \begin{compactitem}
%  \item ...
%     % deliverable will be defined in the appropriate WorkPackage.tex
%    % (\localdelivref{deliv-id})
%  \end{compactitem}
\end{task}

% template for a task
% each task should be added to exactly one workpackage
% in the workpackage task list
\begin{task}[
  title=Prototype Policy,
  % task id for references
  id=policy,
  % lead institution ID
  lead=SRL,
  PM=1,
  wphases={0-36},
  % partner institution ID(s)
  % don't include lead here
  partners={XXX}
]
  Develop a prototype policy for reproducible science using \TheProject tools.

  \begin{compactitem}
  \item ...
     % deliverable will be defined in the appropriate WorkPackage.tex
    % (\localdelivref{deliv-id})
  \end{compactitem}
\end{task}

\end{tasklist}



\begin{wpdelivs}
%\TODO{update due date and startup!}
%\TODO{update milestone!}
\begin{wpdeliv}[
    % id for linking with \delivref or \localdelivref
    id=deliv,
    % lead institution
    lead=XXX,
    % month when deliverable is due (max 36)
    due=12,
    % associated milestone id (see milestones.tex)
    miles=startup,
    % ~always PU, DEC
    dissem=PU,
    nature=DEC,
]
  {
  One-line name of deliverable
  }
\end{wpdeliv}


\end{wpdelivs}
\end{workpackage}
%%% Local Variables:
%%% mode: latex
%%% TeX-master: "../proposal"
%%% End:

%  LocalWords:  workpackage wphases wpobjectives wpdescription pageref wpdelivs wpdeliv
%  LocalWords:  dissem mailinglists swrepository final-mgt-rep compactitem swsites ipr
%  LocalWords:  TOWRITE tasklist delivref

\draftpage
\TOWRITE{ALL}{Proofread WP 5 Management pass 1}
\begin{draft}
\TOWRITE{PS (Work Package Lead)}{For WP leaders, please check the following (remove items
once completed)}
\begin{verbatim}
- [ ] have all the tasks in this Work Package a lead institution?
- [ ] have all deliverables in the WP a lead institution?
- [ ] do all tasks list all sites involved in them?
- [ ] does the table of sites and their PM efforts match lists of sites for each task?
      (each site from the table is listed in all relevant tasks, and no site is listed
      only in the table or only at some task)

- [ ] Binder / repo2docker documentation: tutorials and best practice guides -
      have we got this covered?
\end{verbatim}
\end{draft}

\begin{workpackage}[id=education,wphases=0-36!1,swsites,
  title=Education and Dissemination,
  short=Education,
  lead=IFR,
  IFRRM=10,
  MPRM=6,
  SRLRM=7,
  QSRM=3,
  UIORM=9
]


\begin{wpobjectives}
  The objective of this work package is to disseminate the results of this
  project, including the technical advances and guidance for best practice for
  reproducible science. This includes education of researchers about the value of
  open science, reproducibility and re-usability as well as the possibilities of
  integrating Binder tools in their workflows.

  Beyond this activity, which is directed from the project members to the wider community of
  scientists, we also plan to seek input from scientists to the project: both in
  terms of requirements for practical reproducibility in their domain and in
  technical contributions -- for example through merge requests for Binder
  tools, or open source documentation of best practice for reprocudible software
  environments.

  Our education and dissemination objectives includes:
 \begin{compactitem}
   \item Ensure awareness of the results of the project in the user community,
     and in particular in those groups that act as educators and multipliers of
     knowldege (such as the Carpentries and research infrastructure organisations).
   \item Educate the community on the value of open science, and in particular
   \item Train researchers in best practices for open and reproducible science.
   \item Produce training and education material to disseminate the ability to
     do reproducible computational science using the tools we develop.
   \item Address the shortage of researchers and research support staff trained in reproducibility.
   \item Provide documentation and tutorials which can serve as the technical
     components of reproducibility policies.
 \end{compactitem}
\end{wpobjectives}

% Potential sources of inspiration: ODK's WP2 work package about dissemination:
% PDF: p.36 of https://github.com/OpenDreamKit/OpenDreamKit/raw/master/Proposal/proposal-www.pdf
% Sources: https://github.com/OpenDreamKit/OpenDreamKit/blob/master/Proposal/WorkPackages/DisseminationCommunityBuilding.tex

\begin{wpdescription}

  Open science and reproducible science is entirely dependent on researchers
  adopting open practices. In \TheProject, we improve and developing tools that
  can facilitate these practices, but they only work if researchers actually adopt
  them. For researchers to adopt the practices, they need to (i) know about them
  and (ii) use them.

  We address challenge in multiple ways:
  \begin{enumerate}
    \item the philosophy of the Binder tools is to respect existing standards and
      best practice (and not to invent additional syntax or requirements). It is
      thus possible to use the Binder tools (to recreate a software environment)
      even if the repository authors did not anticipate the use of Binder, or
      knew about their existence. In the best possible scenario, a
      \emph{scientist can benefit from Binder tools with zero additional effort}.

    \item In this work package, we produce education materials and carry out
      education activities to spread the knowledge about \emph{good practice for
        reproducibility and re-usability in science}, such as for example
      automation of all analysis steps, and complete documentation of the
      required software stack. Only one aspect of this training is to show how
      Binder can help with reproducibility.

      Attendees and users following such training and advice will create more
      reproducible artifacts. If they -- or later users of their published
      artifacts -- want to use Binder to reproduce or re-use the results, they
      can. Even if they do not, we will have achieved an improvement of the
      reproducibility of scientific artifacts.
  \end{enumerate}

  % HF: I think training the wider (non-scientist) public about Binder is going
  % too far?
  %
  % Going further, it is also clear that open science is not just of value
  % to researchers: one of the largest benefits of open science is that it makes
  % science accessible to the broader public who may not be members of the
  % research community.
  %
  % To this end, in addition to training researchers, we will also train the
  % public in how to make use of the open science research and services
  % facilitated by \TheProject. This will be done through regular open
  % dissemination and training workshops, as well as by producing and maintaining
  % material for online courses and documentation.

  The \TheProject project will develop, through \WPref{applications}, a number
  of demonstrator repositories that show examples of reproducibility in
  different scientific domains. (We use
  those activities to inform and evaluate the technical improvement to the
  Binder tools in \WPref{reproducibility} and \WPref{impact}). We also use those
  studies to create tutorials and \emph{best practice guides for
    reproducibility} (\localtaskref{online-resources}) in this work package, and
  offer interactive workshops (\localtaskref{workshops}) to help disseminate the
  content more effectively.

  As with all the code, test and build infrastructure produced as part of the
  project, we will also make all documentation open source. Our documentation --
  which includes best practice guides for reproducibility -- can thus be
  modified and improved after the end of the project to react to new
  developments (\delivref{education}{education-materials2}).

  We will engage with the scientific to support them in making their
  work more reproducible with Binder tools. The project will benefit from these interactions
  as we will lean more about reproducibility requirements and usefulness of the
  Binder tools, so that we can tailor our work to support scientific communities as broad as possible
  (\localtaskref{community-support}).

  We will also participate in the well established academic dissemination
  activities, and events of the European E-Infrastructure projects and other
  relevant structures. EGI is a member of our the community engagement panel
  (see \TODO{XXX, \taskref{management}{community-engagement-panel}})
  and the interaction with them be useful to prioritise our resources in this
  very active field.

  Open access to all publications resulting from the project will be ensured.
  %\TODO{Should this sentence go into a data / IPR management plan?}
\end{wpdescription}

\begin{tasklist}
% add tasks from task directory here
\begin{task}[
  title=Management of dissemination and communication activities,
  id=website,
  lead=SRL,
  PM=2,
  % >>> 2/36 = 0.05555555555555555
  wphases={0-36!.056},
  partners={}
]

This task comprises the management and administrative aspects of all forms of
direct dissemination and public communication activities such as press releases,
scientific and technical publications, seminars, talks, promotion through social
media, creation of advertisement materials such as flyers, posters, and
electronic feeds as well as their distribution. We will use standard community
building technology such as mailing lists, wikis and forums, to ensure
dissemination to and engagement with the user community.
\end{task}

% template for a task
% each task should be added to exactly one workpackage
% in the workpackage task list
\begin{task}[
  title=Best practice guidelines for reproducible science,
  id=online-resources,
  lead=UIO,
  PM=10,
  wphases={0-36!.28},
  partners={SRL,MP,UIO}
]
  The aims of this task are to (i) provide online resources for Open Science and
  (ii) support \taskref{education}{workshops}.
  
  This task includes the following activities:
  \begin{compactitem}
  \item Collect and compose best practice guidelines for reproducible and
    re-usable science. Split the content into multiple topic areas so learners
    with different prior knowledge can skip the content they are familiar with
    already.
  \item Develop lesson materials on \emph{open science} best practices (version
    control, testing, automation of all steps, collaboration and peer review,
    documentation, software licensing and open source, use of Jupyter
    notebooks).
  \item Develop lesson materials on \emph{reproducible computational science},
    which focuses on combining the open science tools for reproducible science.
  \item Develop materials on \emph{using Binder tools to make science more
      reproducible and re-usable}. This includes addressing and describing the
    use cases from \WPref{applications}.
  \item Collaboration with the \href{https://coderefinery.org}{CodeRefinery}
    project for the development and maintainance of the
    \href{https://coderefinery.org/lessons/}{online lesson materials}. Following
    CodeRefinery's tradition, the aim will be to contribute the lessons to
    \href{https://software-carpentry.org/}{Software Carpentry} and
    \href{https://data-carpentry.org/}{Data Carpentry}.
  \item The training material will also be referenced on the Binder tools webpage.
  \end{compactitem}
  All material will be licensed under an open license such as
  \href{https://creativecommons.org/licenses/by-sa/4.0/}{CC BY-SA 4.0}
  (\delivref{education}{education-materials1}, \delivref{education}{education-materials1}).
\end{task}

% template for a task
% each task should be added to exactly one workpackage
% in the workpackage task list
\begin{task}[
  title=Training Workshops for more reproducible science,
  id=workshops,
  lead=UIO,
  PM=9,
  wphases={12-36!.25},
  partners={SRL,MP,IFR}
]
This task is focused on taking the content from the
\taskref{education}{online-resources} (Best practice for
reproducible science guidelines) and disseminating it through various channels and to different target audiences.

% \begin{compactitem}

%    \item Defining and implementing a strategy to enable a shared vision of the Jupyter ecosystem across all the actors from developers, users to every stakeholder: the current misalignment hinders the full exploitation of Open Software practices where co-design is a de facto approach.
%
% For instance, the official Jupyter documentation (https://jupyter.org/documentation) solely reflects the view of developers where the Jupyter ecosystem is defined as a set of software packages (jupyter-core, jupyter-client, kernels, widgets (ipywidgets, ipyleaflet, etc.). The user vision is relegated to examplars (blogs, newsletters, etc.) which inevitably tend to be restrictive but often become de facto standards. This can lead to misconceptions and makes it more difficult for on-boarding novices and new communities.
%

% \item Triggering a cultural change to help under-represented groups to actively participate to the development of open source project to ensure the sustainability of the \TheProject services deployed on EOSC-HUB.
%

%\item Foster Open innovation by collaborating with others from different background and activities (school, universities, industries, journalists, artists, etc.)
%  \end{compactitem}

To achieve these goals, the following actions/activities will take place:

  \begin{compactitem}

  %\item co-design hackathons: the co-design efforts between domain scientists, \TheProject developers and service providers will be carried out at any point in time of the project and will be registered in a co-design register to help for future engagement with new communities of users. To be fully effective,  co-design hackathons will be organized to set the stage, define rules for co-design interactions and more importantly align all actors into a common user-centred vision of \TheProject services and associated development towards a successful EOSC deployment.

%    \item Workshops on Findable, Accessible, Interoperable and Reusable (FAIR)
%      software and data to facilitate the adoption of Open Science and Open
%      Scholarship best practices (transparent, sharable and collaborative
%      Science): this would not be restricted to the Jupyter ecosystem and will
%      teach users how to make data, lab notes and other research processes freely
%      available, under terms that enable reuse (licensing), redistribution and
%      reproducibility of methods and/or results.

   \item Delivery of workshops on (i) open science, (ii) reproducible computational
     science, and (iii) the use of Binder tools to support this.

     The content is focused on key insights and tools need for more reproducible
     science, but will be contextualised and delivered in the wider field of
     Findable, Accessible, Interoperable and Reusable (FAIR) software and data.

   % \item Trainings on how to use \TheProject software and services to fully
   %   exploit \TheProject developments for repoducible science: develop training
   %   materials and organize training events for researchers and the public to
   %   enable Open Science and maximise the usefulness of \TheProject
   %   developments.

   \item \TheProject Admin trainings: we will offer training events for learning on how to
     deploy \TheProject services such as BinderHub. This will be relevant for a
     very small (but important) group of users, i.e. those that want to host
     their own BinderHub instance. We know from multiple research organisations
     that this desire exists.

   \item Open call for open innovation mini-projects: mentored by \TheProject
     staff and targeting SMEs, municipalities, journalists, artists, etc.

   \item Where possible, we will schedule dissemination events to take place
     during conferences and community events, such as PyData, EuroSciPy,
     Supercomputing meetings.

   \item We will archive recordings of the training events to support the
     increasing desire of learners to make use of online streaming services
     (such as YouTube) to work through a learning programme at their own time
     and pace.

   \item We will offer in-person and remote training.

   \item The work will be done in collaboration with
     \href{https://coderefinery.org}{CodeRefinery} project which strongly
     support \TheProject proposal and will make available its network of instructors and helpers
     to co-organize, advertise and run online workshops on Open Science best practices. \TODO{This
       is great - can we get a letter of support?}

  \item We will detail our executed activities through the reporting at the end of
    each reporting period.
  \end{compactitem}
\end{task}

\begin{task}[
  title=Community support and engagement,
  id=community-support,
  lead=SRL,
  PM=13,
  wphases={0-36!.36},
  partners={MP,QS,UIO,IFR}
]
A project such as \TheProject{} has the ambition to develop a small set of tools
that will \emph{impact many researchers} and have the potential to be useful
\emph{across all scientific domains that need electronic data processing} as part of their
scientific research and publication process.

As such, we expect that the demand through support queries, documentation
clarification questions, and helpful feedback will be substantial. With this
task, we explicitly reserve some time for such activities.

We have an opportunity here to address multiple aims simultaneously.

The aims of this task are
\begin{compactitem}
\item to engage with community members (and potentially their computing support
  staff) to help them make best use of the Binder tools. This can range from
  helping to configure a BinderHub installation, to address usage questions of
  tools such as \repotodocker{} in domain specific contexts;
\item to engage with community members to better understand diverse
  requirements, and use this information to make the Binder tools and
  reproducibility guidelines more useful for a wider diversity of scientific
  domains;
\item to engage with community members to train researchers and research
  software engineers in reproducibility practices and tools (to address a
  shortage of staff with such skills)
\item to engage with community members to invite them to contribute to the
  binder tools, the reproducibility guidelines and policy development, and other
  open source tools.
\end{compactitem}

We will achieve those aims through listening to feedback, queries and requests
for help from the community, and reserve time to respond. Depending on the
complexity of an issue, guidance by email, chat, video meeting or even an
in-person visit may be appropriate. (When demand exceeds the time budget, we
will need to prioritise which issues we can deal with first.)

We know from our experience with running and contributing to open source
projects that such engagement activities are effective in training interested
and often highly skilled scientists and research software engineers to become
contributors to open source projects. While they may have a primary interest in
improving an open source tool to suit their needs, this will likely benefit
others as well. Once somebody has contributed to a particular open source
software tool, they are more likely to make follow-up contributions - for
example to improve documentation.

\end{task}

\end{tasklist}


\begin{wpdelivs}
\begin{wpdeliv}[due=24,id=education-materials1,dissem=PU,miles=prototype,nature=R,lead=UIO]
  {First version of all training materials available online. }
\end{wpdeliv}
\begin{wpdeliv}[due=36,id=education-materials2,dissem=PU,miles=final,nature=R,lead=UIO]
  {All training sessions material completed, reviewed, and published online.}
\end{wpdeliv}
% \begin{wpdeliv}[due=36,id=report2,dissem=PU,miles=final,nature=R,lead=UIO]
%   {Community building: Report on impact of development workshops, dissemination and training activities.}
% \end{wpdeliv}
\end{wpdelivs}

\end{workpackage}
%%% Local Variables:
%%% mode: latex
%%% TeX-master: "../proposal"
%%% End:

%  LocalWords:  workpackage wphases wpobjectives wpdescription pageref wpdelivs wpdeliv
%  LocalWords:  dissem mailinglists swrepository final-mgt-rep compactitem swsites ipr
%  LocalWords:  TOWRITE tasklist delivref

\draftpage

\ifgrantagreement
\endgroup
\setcounter{page}{\value{savepage}}
\fi

%%% Local Variables:
%%% mode: latex
%%% TeX-master: "../proposal"
%%% End:

%  LocalWords:  newpage workpackages workplan



\subsubsection{Risks and risk management strategy}
\label{sec:risks}
% 
% The risk in the project execution as planned is carefully assessed and
% managed. We base our plans on long standing experience, and we bring
% together the world's experts in the relevant tools and techniques.
% 
% A key feature of this project is the involvement of a wide set of
% partners from multiple domains. While this ensures complementary
% coverage of a wide set of skills and provides robustness in different
% ways, we will have to ensure that all the partners work together as
% a closely knit team.
% 
% Our open source approach means that all our code and outputs
% will be open and visible to anybody at sites like GitHub and Bitbucket
% throughout the project. It is common for some users to run the latest
% development versions of computational and infrastructure software, thus
% beta-testing code between major releases.
% This reduces the risk of developing software which people won't use:
% where our design decision or technical approaches are
% controversial, this will be detected early by those users, giving the
% consortium useful feedback to consider.
% 
% As part of the Management Work Package, and with support from the
% Coordination Team, the project coordinator will maintain and regularly
% update a Risk Management Plan; at the end of each Reporting Period,
% this updated plan will be included in the project's Technical Report.
% It will identify and categorise all
% potential strategic risks (legal, financial, human resources risks, etc.)
% to the successful delivery of the project, their probability and impact.
% For each risk area, mechanisms for risk mitigation will be identified
% and contingency actions will be proposed.
% 
% Risks will be evaluated in terms of project goals and objectives,
% according to the following four steps:
% \begin{enumerate}
% \item Identification of risks using a structured and rational approach to
% ensure that all areas are addressed.
% \item Quantitative assessment and ranking of the risks.
% \item Definition of procedure to reduce (or minimize) risk.
% \item Monitoring and management of risks throughout the project life
% with milestone review and reassessment.
% \end{enumerate}
% 
% Finally, as reported above, a conflict resolution mechanism will be put in place,
% whereby decision making divergence and conflicts that cannot be solved
% at the Steering Committee (SC) level will be submitted to the Coordinator. The
% mediation and resolution process used is the following:
% \begin{itemize}
% \item Case presented by the involved parties.
% \item Development of a fact-based and neutral report by the coordinator
% to be provided to the conflicting parties and SC.
% \item Final decision to resolve the conflict made by SC.
% \end{itemize}
% 
\ifgrantagreement\else
An initial risk assessment appears as Table~\ref{risk-table}.

\begin{table}
\begin{center}
\begin{tabular}{|m{.2\textwidth}|m{.12\textwidth}|m{.58\textwidth}|}\hline
  Risk & Level without / with mitigation & Mitigation measures
  \\\hline

   \multicolumn{3}{|c|}{
    \textit{General technical / scientific risks}
   }
   \\\hline

  Implementing infrastructure that does not match the needs of end users & High/Low &
  Many of the members of the consortium are themselves end-users with
  a diverse range of needs and points of views; hence the design of
  the proposal and the governance of the project is naturally steered
  by demand; besides, because we provide a toolkit, users have the
  flexibility to adapt the infrastructure to their needs. In addition, the open source nature
  of the project facilitates and promotes the involvement of the wider community in terms of
  providing feedback and requesting additional features via platforms such as GitHub and Bitbucket
  on a regular basis.
  \\\hline

  Lack of predictability for tasks that are pursued jointly with
  the community & Medium/Low &
  The PIs have a strong experience managing community-developed
  projects where the execution of tasks depends on the availability of
  partners. Some tasks may end up requiring more manpower from
  \TheProject to be completed on time, while others may be entirely
  taken care of by the community. Reallocating tasks and redefining
  work plans is common practice needed to cater for a
  fast evolving context. Such random factors will be averaged out over
  the large number of independent tasks.\\\hline

  Reliance on external software components & Medium/Low & The non trivial
  software components \TheProject relies on are open source. Most are
  very mature
  and supported by an active community, which offers strong long run
  guarantees. The other components could be replaced by alternatives, or
  even taken over by the participants if necessary.
  \\\hline

  \\\hline

%  \multicolumn{3}{|c|}{
%    \textit{Use-case risks}
%  }
%  \\\hline
%
%  & & \TOWRITE{WP4}{Risks related to use-cases in WP4}
%  \\\hline

  \multicolumn{3}{|c|}{
    \textit{Management risks}
  }
  \\\hline

  Recruitment of highly qualified staff & High/Medium &

  Great care was taken coordinating with currently running projects to
  rehire personnel with strong track record, and identifying pool of
  candidates to hire from, notably in the developers community of
  software related to the project. This was favoured by the partners'
  long history of training and outreach activities. In addition, we
  have a critical mass of qualified staff in the project enabling us
  to train and mentor new recruits.

 \\\hline

  Different groups not forming effective team & Medium/Low & The participants have a long
  track record of working collaboratively on code across multiple
  sites. Aggressive planning of project meetings, workshops and
  one-to-one partner visits will facilitate effective teamwork,
  combining face-to-face time at one site with remote
  collaboration.\\\hline
  % this also justifies our generous travel budget.

  Partner leaves the consortium & High/Low & If the GA requires a replacement
  in order to achieve the project's objectives, the consortium will invite a new
  relevant partner in. If a replacement is not necessary, the resources and tasks
  of the departing partner will be reallocated to the alternative ones within the
  consortium.
  \\\hline

  \multicolumn{3}{|c|}{
    \textit{Dissemination risks}
  }
  \\\hline

  Impact of dissemination activities is lower than planned. & Medium/Low &

  Partners in the consortium have a proven track record at community
  building, training, dissemination, social media communication, and
  outreach, which reduces the risk. The Project Coordinator
  will monitor impact of all dissemination activities. If a deficiency is identified, the consortium
  will propose relevant corrective actions.\\\hline

  \end{tabular}
\end{center}
\caption{\label{risk-table}Initial Risk Assessment}
\end{table}
\fi
%\TOWRITE{NT/Eugenia}{Impredictability}

%\includegraphics[width=.94\textwidth]{Pictures/Impact-img1.png}

%   But: since Open Source softwares are freely accessible, security
%   and privacy issues are a concern. Anytime a resource is shared,
%   there is greater risk of unauthorised access and contaminated data.
%   Providers must demonstrate security solutions, which should include
%   physical security controlling access to the facility and protection
%   of user data from corruption and cyber attacks.}


\TOWRITE{ALL}{
  Add a paragraph about data management plan. What data will we produce, which data is available from the
  start, how do we handle it...
}

%
% a table showing number of person months required (table 3.1f);
% 	a table showing description and justification of subcontracting costs for each participant (table 3.1g);
% -	a table showing justifications for purchase costs (table 3.1h) for participants where those costs exceed 15% of the personnel costs (according to the budget table in proposal part A);
% -	if applicable, a table showing justifications for other costs categories (table 3.1i);
% -	if applicable, a table showing in-kind contributions from third parties (table 3.1j)
\draftpage
\subsubsection{Resources to be Committed}\label{sec:resources}

Tables summarising effort and costs are presented here.

\eucommentary{
Please indicate the number of person/months over the whole duration of the planned work, for each work package, for each participant. Identify the work-package leader for each WP by showing the relevant person-month figure in bold.
}

\wpfig[label=fig:staffeffort,caption=Summary of staff effort]

%%%%%%%%%%%%%%%%%%%%%%%%%%%%%%%%%%%%%%%%%%%%%%%%%%%%%%%%%%%%%%%%%%%%%%%%%%%%%%
% \paragraph{Purchase costs}

\noindent\textbf{Purchase costs}

\TOWRITE{}{How much justification necessary?}

\eucommentary{
Please complete the table below for each participant if the purchase costs (i.e.
the sum of the costs for 'travel and subsistence', 'equipment', and `other
goods, works and services') exceeds 15\% of the personnel costs for that
participant (according to the budget table in proposal part A). The record must
list cost items in order of costs and starting with the largest cost item, up to
the level that the remaining costs are below 15\% of personnel costs
}

All participants request less than 15\% of personnel costs in purchase costs.
These costs go toward:

\begin{itemize}[noitemsep]
\item travel to project meetings
\item site visits between project members to foster collaboration
\item conference attendance for dissemination
\item open access publication fees
\item equipment for carrying out the work (high performance laptop computers for each FTE)
\item CFS (at \site{SRL} only)
\item hosting workshops (at \site{SRL}) for dissemination
\item cloud costs (at \site{SRL}) for testing outputs and supporting development and workshops \TOWRITE{}{needs to be discussed earlier in work plan}
\end{itemize}

% below is commented-out a more detailed justification for Simula
% See line 151 in
% https://docs.google.com/spreadsheets/d/19xparkP93ANTecMqApNVEvl_w9-q12KE5yHS-IjN7T4

% \site{SRL} is the only site requesting more than 15\% of personnel costs in purchase costs.
% This is because \site{SRL} will host some shared project-wide costs,
% such as hosting workshops for \WPref{education}
% and project-wide cloud costs for testing and demonstration,
% used across all work packages.
%
% % Our project travel costs are estimated based on:
% %
% \begin{compactenum}
% \item 800 \euro per traveller for each of 3 project meetings
% \item 2000 \euro per FTE-researcher-year (1.75) for site visits, facilitating collaboration
% \item 3000 \euro per FTE-researcher-year (1.75) for conference attendance
% \item 4000 \euro for hosting each of two in-person workshops,
%       with an estimated cost of 400 \euro per participant (10 participants).
% \end{compactenum}
%
% \bigskip
% \begin{table}[H]
% \begin{tabular}{|r|r|p{8.5cm}|}
%   \hline
%   \textbf{\site{SRL}}
%     & \textbf{Cost (\euro)}
%     & \textbf{Justification}
%     \\
%   \hline
%   \textbf{Other goods, works, and services}
%     & 36100
%     & Average 600 \euro per month of cloud computing service costs,
%       3500 \euro for CFS, hosting two workshops of 10 attendees at 400 \euro per attendee,
%       3000 \euro for open access publication fees.
%     \\
%   \hline
%   \textbf{Remaining purchase costs}
%     & 50300
%     \\
%   \cline{1-2}
%   \textbf{Total}
%     & 86400
%     \\
%   \cline{1-2}
%   \end{tabular}
% \caption{Overview: 'Purchase costs' to be committed at Simula
% Research Laboratory
% (all in \texteuro)}\vspace*{-1em}
% \end{table}


%%% Local Variables:
%%% mode: latex
%%% TeX-master: "proposal"
%%% End:
