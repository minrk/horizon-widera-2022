
\eucommentary{Please provide the following:\\
\begin{compactitem}
\item
brief presentation of the overall structure of the work plan;
\item
timing of the different work packages and their components (Gantt chart or similar);
\item
detailed work description, i.e.:
\begin{compactitem}
\item
a description of each work package (table 3.1a);
\item
a list of work packages (table 3.1b);
\item
a list of major deliverables (table 3.1c);
\end{compactitem}
\item
graphical presentation of the components showing how they inter-relate (Pert chart or similar).
\end{compactitem}
}

\subsubsection{Quality and efficiency of the implementation}\label{sec:workplan-structure}

\ifgrantagreement The \else As shown in
Figure~\ref{fig:workpackages}, the \fi work plan is broken down into five work
packages: \WPref{reproducibility} focuses on robustness improvements of the
Binder tools; \WPref{impact} is advancing the Binder tools' feature set to
broaden their applicability and increase the impact of the tools and project; \WPref{applications} applies and
evaluates the reproducibility tools in real-world research contexts;
\WPref{education} is focused on engaging with and educating researchers and the
wider public in best practices for reproducible science. This is supported by
the usual management work package (\WPref{management}).

\begin{figure}[htb]
  \centering
  \includegraphics[width=0.63\textwidth]{images/WP.pdf}
  \caption{
    \label{fig:workpackages}
    The relationships and interactions of the work packages,
    broken up into four categories: management (WP1),
    development of new functionality surrounding Jupyter's Binder tools to improving robustness
    and broadening impact (WP2, WP3),
    applications of tools developed in real-world research context (WP4),
    and dissemination, education, and engagement (WP5: Dis., edu., eng.).
  }
\end{figure}

% \subsubsection{How the Work Packages will Achieve the Project Objectives}
% \label{sssec:how_the_work_packages_will_achieve}

% , including inter-task dependencies.
% add milestones, for milestone markers
% but these don't work if we have more than one milestone at the same point in time
\ganttchart[draft,xscale=.4,yscale=0.82,step=6,
    caption={
        Gantt chart illustrates the distribution of effort across the work packages over
        time. The height of grey bars corresponds to total person month effort per
        month for each work packages. We ramp up overall activity during year 1, then
        remain at a constant level. In the beginning of the project, the work on core
        parts of the technology dominates (WP2 and WP3). The activity in application of
        the new features (WP4) and dissemination (WP5) grows over time.
    },
]

\begin{workplan}

\draftpage
\subsubsection{Work Package Descriptions}\label{sec:workpackages}
%%% work package style may be broken -- fix this!!

\ifgrantagreement
\begingroup
% Note: in the grant agreement, The workpackage description must not appear.
% Yet we want to compile them to get all the metadata right
% Current hack: compile them anyway, reset the page number
% appropriately, and remove them a posteriori with pdftk. We set the
% color to red to make it more visible in case we forget to remove
% them.
% See grantagreement rule in the Makefile
\newcounter{savepage}
\setcounter{savepage}{\value{page}}
\color{red} % To make sure we indeed remove the pages
\fi

\enlargethispage{1cm}

%% Local WP number counter - should possibly be global and hidden?
\TOWRITE{ALL}{Proofread WP 1 Management pass 1}
\begin{draft}
\TOWRITE{PS (Work Package Lead)}{For WP leaders, please check the following (remove items
once completed)}
\begin{verbatim}
- [ ] have all the tasks in this Work Package a lead institution?
- [ ] have all deliverables in the WP a lead institution?
- [ ] do all tasks list all sites involved in them?
- [ ] does the table of sites and their PM efforts match lists of sites for each task?
      (each site from the table is listed in all relevant tasks, and no site is listed
      only in the table or only at some task)
\end{verbatim}
\end{draft}

\begin{workpackage}[id=management,type=MGT,wphases=0-36!.2,
  title=Project Management,
  short=Management,
  lead=SRL,
  MP=1,
  % QS=1,
  % IFR=1,
  % UIO=1,
  % SRLRM=12,
  swsites
]
\begin{wpobjectives}
The main objective of WP1 is to establish and maintain an effective contract, project, and operational management approach ensuring:

 \begin{compactitem}
    \item Timely and successful implementation of the project; including administrative and legal coordination
    \item Technical management and quality assurance
    \item Risk and innovation management of the project as a whole; including data and IPR management
    \item Smooth communication and interaction with the EC and other interested parties

 \end{compactitem}
\end{wpobjectives}

\begin{wpdescription}
The project will be managed by Simula, which has extensive experience in administering and leading EU funded and national projects. The coordinator together with the WP leaders, will be responsible for monitoring WP status, coordination of work plan updates and annual internal progress reports. The project management structure and roles of partners in the consortium are presented in \ref{sect:mgt}.

\end{wpdescription}

\begin{tasklist}

\begin{task}[
  title=Administrative Management,
  id=admin,
  lead=SRL,
  PM=24,
  wphases={0-36!.5},
  partners={MP,QS,UIO,IFR},
]
The task includes the following activities:
\begin{compactenum}
\item Preparation, distribution and maintenance of all contractual documents (Consortium Agreement, Grant Agreement and all other legal frameworks)
\item Establishment of appropriate communication and collaborative environment for the consortium, as well as the EC and other relevant academic and industry stakeholders (the project website, intranet and communication procedures) to organise transfer of knowledge, present and promote project results (\localdelivref{infrastructure});
\item Organisation of project review and progress meetings;
\item Performing qualitative and quantitative risk analysis, planning risk mitigation and control
\item Progress and Financial Reporting to the EC;
\item Data and IPR Management will be managed in accordance with agreed rules stated in the Consortium Agreement and in accordance with the Data Management Plans (\localdelivref{data-management-plan}, \localdelivref{innovation-management-plan}).
\end{compactenum}
\end{task}

\begin{task}[
  title=Technical Project Management,
  id=project-management,
  lead=SRL,
  PM=24,
  wphases={0-36!.5},
  partners={MP,QS,UIO,IFR}
]
The project scientific and technical management ensures coherent quality and soundness of the work and results. A quality assurance plan will be developed by \site{SRL}, involving all partners, and will be followed up regularly. It will address the reviews and approval of technical reports and deliverables. In addition, the Project Coordinator with the help of the coordination team will regularly review technological risks and recommend mitigation plans to minimise or remove them. This will be reported on at each Reporting Period in the project's Technical Report.
\end{task}

\begin{task}[
  title=Innovation Management,
  id=innovation-management,
  lead=SRL,
  PM=6,
  wphases={0-36!.2},
  partners={MP,QS,UIO,IFR}
]
One of the most important criteria for success for the \TheProject project is to bring the project results into use. Therefore, exploitation routes will be sought whenever possible. In order to create a common understanding within the Consortium of how we can best shepherd an idea all the way from conception to realisation and exploitation, the Coordinator will be responsible for the preparation and realisation of an Innovation Plan. This plan will assure that research activities meet the required milestones and produce outputs fully aligned with the project objectives. All research activities will go through an initial process where the exploitation opportunity is identified along with the main stakeholders for the exploitation opportunity and an IP owner
(\localdelivref{innovation-management-plan}).
\end{task}

\begin{task}[
  title=Community Engagement Panel,
  id=communikty-panel,
  lead=SRL,
  PM=24,
  wphases={0-36!.5},
  partners={MP,QS,UIO,IFR},
]
The task includes the following activities:
\begin{compactenum}
\item \TOWRITE{TODO} ...
\end{compactenum}
\end{task}

\end{tasklist}


\begin{wpdelivs}

% Rationale:
% - Eugenia recommended to have two deliverables about Data Management Plan, one early, and one at the complete end.
% - Having the Data Management Plan draft and Innovation plan at M9
%   gives some material in case we have an informal review at M9. Also
%   those are easy ones with no dependencies on progress; just
%   something to take care of at some point over the course of a few
%   weeks. So this spreads the load.

\begin{wpdeliv}[due=1,miles=startup,id=infrastructure,dissem=PU,nature=DEC,lead=SRL]
  {Basic project infrastructure (websites, wikis, issue trackers, mailing lists, repositories)}
\end{wpdeliv}

\begin{wpdeliv}[due=9,miles=startup,id=data-management-plan-draft,dissem=PU,nature=R,lead=SRL]
  {Data Management Plan draft}
\end{wpdeliv}

\begin{wpdeliv}[due=9,miles=startup,id=innovation-management-plan,dissem=CO,nature=R,lead=SRL]
  {Innovation Management Plan}
\end{wpdeliv}

\begin{wpdeliv}[due=36,miles=final,id=data-management-plan,dissem=PU,nature=R,lead=SRL]
  {Data Management Plan}
\end{wpdeliv}

\end{wpdelivs}
\end{workpackage}
%%% Local Variables:
%%% mode: latex
%%% TeX-master: "../proposal"
%%% End:

%  LocalWords:  workpackage wphases wpobjectives wpdescription pageref wpdelivs wpdeliv
%  LocalWords:  dissem mailinglists swrepository final-mgt-rep compactitem swsites ipr
%  LocalWords:  TOWRITE tasklist delivref

\draftpage

\begin{workpackage}[
  id=reproducibility,
  % wphases=0-36!1.03,
  wphases=0-24!1.54,
  title=Improving robustness of reproducibility tools,
  short=Improving robustness,
  lead=QS,
  SRLRM=23,
  UIORM=0,
  MPRM=2,
  QSRM=12,
  swsites,
]
\begin{wpobjectives}
  \begin{compactitem}
    \item to better understand and evaluate successful reproduction of computational environments
    \item to improve the practical reproducibility of environments constructed
      with \TheProject tools
    \item to support and maintain core Binder software infrastructure in order to keep it healthy
         and useful for open science and reproducibility
 \end{compactitem}
\end{wpobjectives}

\begin{wpdescription}

This Work package is focused on making \repotodocker{} do the things it does
already \emph{better}, \emph{more robustly} and \emph{more sustainably}.
(Orthogonal to those improvements, we plan to significantly extend the
\repotodocker{} use cases in \WPref{impact}.)

To be able to asses the impact of our planned improvements, we need to have a
metric. Task \localtaskref{repo2docker-checker} will create this
for us. In addition to the evaluation of the improvements in this proposal, this
can be used more generally as an indicator for reproducibility of software
environments.

One major improvement to the existing capabilities of \repotodocker is the
\emph{time-machine} functionality, and this is implemented in \localtaskref{repo2docker-timemachine}.

In task \localtaskref{performance-optimisation}, we will speed-up the execution time of
\repotodocker{} to improve the user experience when reproducing or re-using
existing software and data.

Open source software needs ongoing maintenance to adapt to changing requirements
and dependencies. We schedule a certain amount of time for this in task
\localtaskref{maintenance}.

All changes to the software will be made available online as open source already during development
(i.e. throughout the whole project), and new features will be made available
through software releases of the Binder tools. A final release will be made and
reported through the deliverable \localdelivref{repo2docker-release24}.

% the existing functionality of repodocker.
% Community-led open source software is critical to a sustainable future for open science.
% Commonly used tools make up a shared infrastructure,
% where investment in core components benefits the widest user community.
% \TheProject is centred around the Jupyter project,
% which is a collection of projects for interactive computing and
% communicating computational ideas.
%
% This work package is focused on developing and maintaining
% the core of Jupyter.
% In particular, we will help maintain these projects to meet the needs of the
% Jupyter community, with a focus on needs for open science.
% To serve the needs of \TheProject,
% Jupyter core infrastructure will need improvements
% to security, performance, and scalability,
% which will be provided in \localtaskref{maintenance}.
% In addition, we will develop new features in the core of Jupyter
% to bring it to a wider audience,
% and to improve its usefulness to those working toward open science practices,
% including via collaboration features (\localtaskref{collaboration})
% and accessibility (\localtaskref{accessibility}).

\end{wpdescription}

\begin{tasklist}

% template for a task
% each task should be added to exactly one workpackage
% in the workpackage task list
\begin{task}[
  title=Towards quantifiable progress for reproducible software environments,
  % task id for references
  id=repo2docker-checker,
  % lead institution ID
  lead=SRL,
  PM=10,
  % wphases={0-24!0.42},
  % partner institution ID(s)
  % don't include lead here
  partners={MP}
  ]
  The \repotodocker{} tool is a key component of the Binder software for
  reproducibility (see \ref{binder-how-does-it-work}). It can be used to create
  a software environment based on software dependency specification standards
  (see \ref{sec:repo2docker}) that are widely used.

  If the required software is specified -- for example through a
  \texttt{requirements.txt} file for Python dependencies -- then \repotodocker{}
  can create the software environment (currently limited to such environments in
  Docker images), within in which the main computation or data analysis can be
  reproduced.

  In this task, we will develop a tool -- with working name
  \softwarename{repo2docker-checker} -- that allows us to \emph{automatically}
  assess the reproducibility of software environments for software that is
  publicly available on GitHub, Bitbucket or GitLab repositories.

  For every repository, the \softwarename{repo2docker-checker} tool will report if an
  appropriate software could be compiled, or if a problem occurred. Software
  environments in repositories may be reproducible because the authors already
  use Binder to offer their repository in an interactive Binder environment. Or
  the software environment may be reproducible because the authors have followed
  standard conventions
  and \repotodocker{} understands these conventions.

The task includes the following activities:
\begin{compactitem}
  \item Through manual inspection of selected repositories, identify common
    failure modes of building of the software environment (such as for example
    not specifying the Python version to use).
  \item Design and develop the \softwarename{repo2docker-checker}. A prototype
    exists.\footnote{https://github.com/minrk/repo2docker-checker}
  \item Where possible, identify for what reason the software build has failed.
  \item Develop a strategy and heuristic to evaluate success of the build
    process.
  \item Identify suitable software repositories for the study.
  \item Automate the software reproduction process for the available
    repositories.
  \item Automate the analysis of the results, so the study can be repeated later.
  \item Carry out the study to estimate the fraction of reproducible repositories.
  \item Repeat the study after the robustness of \repotodocker{} has
    been improved (\localtaskref{repo2docker-timemachine} to evaluate
    progress. \TODO{Point to KPI}
  \end{compactitem}

  The tool will be made available as open source
  (\localdelivref{deliv-id-repo2docker-checker-software}).

  % Some of the findings
  % here will contribute to the \TODO{deliverable XXX in WP5 - best practice for
  %   reproducible repositories with Binder.} At the end of the project, we will
  % provide a summary of improvements in reproducibility we have achieved through
  % changes in the Binder tools.
\end{task}

\begin{task}[
  title=repo2docker development,
  id=repo2docker-timemachine,
  lead=SRL,
  PM=12,
  %wphases={0-24!0.5},
  partners={QS}
]

This tasks improves the robustness of \repotodocker{}. We illustrate this with one specific example: 
Often, a repository of scientific results may
specify which software library is required (such as the Python library
\softwarename{pandas}), but not which version.

A software environment creation tool -- such as \repotodocker{} -- can then
attempt to install the most recent version of \softwarename{pandas}. This is
usually the intention of the authors, and was correct at the time the repository
was created. However, as time moves on, the interface, behaviour and dependence
on other packages of \softwarename{pandas} will change, and at some point an
automatic build of the software for the whole repository may fail because of
conflicting dependencies.

We have found through anecdotal evidence that these problems can be
overcome if a \softwarename{pandas} version can be chosen that was the
most recent at the time when the repository was created. A related
issue is that the Python version itself (such as 3.8, 3.9 or 3.10) may
not be specified at all.

We will teach \repotodocker{} to establish the date of publication
(or last modification) of the repository, to determine the appropriate version
of software libraries from that time, and to select libraries with those
versions if no specific version is specified.

In the context of Python packages, we can use the
\softwarename{pypi-timemachine}
package.\footnote{\url{https://github.com/astrofrog/pypi-timemachine}}.

\TODO{@QS, @min - Comment on other software sources - conda, ... ?}
\end{task}

% template for a task
% each task should be added to exactly one workpackage
% in the workpackage task list
\begin{task}[
  title=Performance optimisation,
  % task id for references
  id=performance-optimisation,
  % lead institution ID
  lead=QS,
  PM=9,
  %wphases={0-24!0.375},
  % partner institution ID(s)
  % don't include lead here
  partners={SRL}
]
  The creation of reproducible software environments can take -- depending on
  the complexity and overall size -- some time: generally, an image can be built
  within few minutes, but there are tasks taking much longer.

  In this task, we will optimise the \repotodocker{} performance.

  \TODO{QS - add a little more detail here?}

  % \TODO{Mention in which deliverable this will be reported?}
  % \begin{compactitem}
  % \item ...
  %   % deliverable will be defined in the appropriate WorkPackage.tex
  %   % (\localdelivref{deliv-id})
  % \end{compactitem}
\end{task}

\begin{task}[
  title=Maintenance of open source reproducibility software,
  id=maintenance,
  lead=SRL,
  PM=6,
  %wphases={0-24!.25},
  partners={QS}
]

Developing software that people will use requires maintenance of that
software, not just new development. Through  this proposal we will
contribute general support to open source reproducibility software
where this is helpful for \TheProject. Such contributions are expected 
to the Jupyter (and as subproject) Binder code base. They support not
just all participants in \TheProject but also millions of people
relying on Jupyter software.


% Maintenance of core software is often an implicit and un-paid cost, or one
% hidden in over-describing the resources required to deliver proposed
% developments. In \TheProject, we make it clear and explicit that we will spend a
% significant amount of time developing and maintaining the core Jupyter and
% JupyterHub e-Infrastructure to respond to the needs of \TheProject and others,
% and contribute towards the sustainability and health of the community.
% 
% We will provide support to the Jupyter e-Infrastructure software, ensuring that
% it meets the needs (\localdelivref{jupyter-contributions}) of \TheProject, and
% aid in the release process to ensure that stable releases of Jupyter software
% can be used in mature \TheProject services (\localdelivref{jupyter-releases}).
% 
%   \TheProject will need improvements to core Jupyter functionality, including areas of:
% 
%   \begin{compactenum}
%     \item ease of deployment
%     \item security
%     \item scalability of JupyterHub
%     \item performance
%   \end{compactenum}
% 
%   We will contribute improvements in these areas,
%   meeting the needs of \TheProject and benefiting the wider Jupyter
%   community.

\end{task}


\end{tasklist}


\begin{wpdelivs}
  % \begin{wpdeliv}[due=1,miles=startup,id=infrastructure,dissem=PU,nature=DEC,lead=SRL]
  %   {Some Deliverable}
  % \end{wpdeliv}

  % (\localdelivref{deliv-id})

  \begin{wpdeliv}[due=12,id=deliv-id-repo2docker-checker-software,dissem=PU,nature=OTHER,lead=SRL]
    {Release software tool for checking of reproducibility of software
      environments (\texttt{repo2docker-checker})}
  \end{wpdeliv}

  \begin{wpdeliv}[due=24,id=repo2docker-checker-study-report,dissem=PU,nature=R,lead=SRL]
    {Summary of reproducibility improvements achieved.}
  \end{wpdeliv}

  \begin{wpdeliv}[due=24,id=repo2docker-release24,dissem=PU,nature=R,lead=SRL]
    {Release of \repotodocker{} with improve robustness features.}
  \end{wpdeliv}

\end{wpdelivs}

\end{workpackage}
%%% Local Variables:
%%% mode: latex
%%% TeX-master: "../proposal"
%%% End:

%  LocalWords:  workpackage wphases wpobjectives wpdescription pageref wpdelivs wpdeliv
%  LocalWords:  dissem mailinglists swrepository final-mgt-rep compactitem swsites ipr
%  LocalWords:  TOWRITE tasklist delivref

\draftpage
\begin{draft}
\TOWRITE{PS (Work Package Lead)}{For WP leaders, please check the following (remove items
once completed)}
\begin{verbatim}
- [ ] discuss what the build packs are and where they go
- [ ] how we structure and budget different tasks in this WP
- [ ] needs review and more technical detail
- [ ] add deliverable(s)
\end{verbatim}
\end{draft}

\begin{workpackage}[
  id=impact,
  wphases=0-36!1.17,
  swsites,
  title=Broadening impact,
  short=Broadening impact,
  lead=SRL,
  SRLRM=24,
  MPRM=12,
  QSRM=2,
  IFRRM=4,
  %UIORM=4,
]
\begin{wpobjectives}
  This work package extends the functionality of the \TheProject tools for
  reproducibility to broaden its applicability and increase impact.
  \WPref{applications} exploits these technological advances
  in real world use cases.

  The objectives of this work package are to
 \begin{compactitem}
 \item remove the dependency of \repotodocker{} on Kubernetes
 \item support container technologies other than Docker
 \item extend the range of software specification standards that are recognised
   and supported by \repotodocker{} \TOWRITE{}{Can we give an example here?}
 \item enable access to data from data sources outside the container
 \item enable \repotodocker{} to extract and save the software installations
   instructions (independent from container generation)
 \end{compactitem}
\end{wpobjectives}

\begin{wpdescription}
One of the design decision that have led to the current Binder software stack is
to constrain the supported infrastructure to a few key components. These
include:

\begin{compactitem}
\item Software environments can only be created inside Docker container
\item To run and orchestrate (multiple) Docker containers, a Kubernetes system must
be available
\item The user must interact with Binder environments through a Binderhub
  installation
\item There is no Binders-specific provision to access (substantial) data sets
  from inside te container. \TODO{HF: What prevents us from
    a data-publishing application at the moment? How to express this without
    making the impression one couldn't use \softwarename{curl/wget} to fetch
    data, etc. }
\end{compactitem}

The benefits of such a restrictive approach are that the software development
and maintenance effort is kept small: the wider the range of supported
infrastructure that the Binder tools can be deployed on, the higher the
complexity.

At the same time, these restrictions prevent the following scenarios:
\begin{compactitem}
\item To run the Binder software on systems where Kubernetes is not available (such as
  the Desktop of a scientist). A related use case is ``Binder@home''
  \TODO{Provide link to Binder at home task in
   \WPref{applications}}.
\item To use Binder on systems where Docker cannot be used. A use case for this
  is to create reproducible environments on HPC installations. The HPC
  administrators generally avoid use of Docker for security reasons, but much
  prefer to support container technologies that can be executed with
  user-only privileges). \TODO{Point to task in applications}
\item To use Binder for software environments that use \TOWRITE{}{insert buildback
    example / community / software system}
\item To access and use significant amounts of data from inside the software
  environment is not supported. An important use case that cannot be supported due to
  this restriction is that of ``data publishing'': the idea is that a
  (potentially large and/or complex) dataset is published \emph{together} with
  software that encodes the necessary knowledge to extract meaningful data from
  that data set. A Binder environment would make this data set accessible in an
  interactive environment. \TODO{Point to data publishing use case(s)}
 \end{compactitem}

% Open source software in general, and Jupyter in particular,
% is developed not as a monolithic application,
% but rather as a collection of related components,
% which can be assembled in numerous combinations to meet diverse needs.
% The Jupyter community is no different.
% Jupyter itself is composed of several projects,
% but there are even more projects that build on top of Jupyter to create
% things like cloud services or data pipelines.
% The goal of \TheProject is to facilitate open science through Jupyter,
% and this includes working with projects all around the Jupyter ecosystem.
% We will focus this work package on developing
% Jupyter ecosystem projects with an emphasis on open science.
%
% repo2docker is a project for creating
% reproducible environments in which Jupyter notebooks (and other user interfaces) can be run.
% It reads a number of common formats to list required software packages,
% and prepares a Docker container with those packages installed.
% BinderHub is software for operating a web service using repo2docker,
% which enables sharing of interactive and reproducible Jupyter (and Rstudio) environments on the web with a single link.
% We will develop repo2docker and BinderHub further to meet the needs of the open science community.
%
% In addition to the interactive aspects of Jupyter,
% notebooks can be used in a "workflows" style,
% where job systems run analyses and produce reports,
% either on a scheduled basis or triggered by events.
% There is a great deal of interest in using notebooks in this way,
% and much room for development of tools supporting workflows in data-driven open science.



\end{wpdescription}

\begin{tasklist}
% % add tasks from task directory here
% template for a task
% each task should be added to exactly one workpackage
% in the workpackage task list
\begin{task}[
  title=Support more software specification standards,
  % task id for references
  id=buildpacks,
  % lead institution ID
  lead=SRL,
  PM=12,
  % wphases={0-36},
  % partner institution ID(s)
  % don't include lead here
  partners={MP}
  ]

  There are two different aspects of software specification that \repotodocker{}
  needs to understand:
  \begin{enumerate}
  \item the specification of the software environment, for example through
    \softwarename{requirements.txt} files, etc (see Section
    \ref{sec:repo2docker} for a list of currently
    supported standards),
  \item from where to retrieve the software itself: this will be different on a
    GitHub repository or a Zenodo archive
    (see \ref{sec:repo2docker} for supported repositories).
  \end{enumerate}

  For both aspects, there are requests from potential Binder users to extend the
  capabilities of \repotodocker{}.

  In this task, we will prioritise such requests and extend the \repotodocker{}
  functionality to support as many new standards as possible.

\end{task}

% template for a task
% each task should be added to exactly one workpackage
% in the workpackage task list
\begin{task}[
  title=Reducing technical constraints to enable broader usage,
  % task id for references
  id=constraints,
  % lead institution ID
  lead=SRL,
  PM=14,
  %wphases={0-36},
  % partner institution ID(s)
  % don't include lead here
  partners={MP,QS}
]

In this task, we refactor and extend the existing code to be more flexible. In
particular, we want to address and remove the constraints that currently exist:

\begin{compactitem}
\item Dependency of existing Kubernetes system: at the moment, BinderHub can
  only start container environments within a Kubernetes installation. In this
  task, we will make it possible to start a container directly on the host
  machine. Setting up Kubernetes system is a complex task: while justified to
  exploit large computational resources through swarms of containers
  effectively, it is not necessary for single-machine execution of
  Binder-reproduced environments (such as anticipated for Binder@home in
  \taskref{applications}{binder-at-home}).
\item Support of only Docker containers to host created software environments.
  Docker is widespread and popular, and in particular has an attractive user
  interface for Windows and OSX. However, many HPC centres refuse to allow use
  of Docker containers on their systems for security reasons. In this task, we
  will make it possible to user container technologies, which are more
  acceptable to use on large HPC installations (such as singularity for
  example).
     % deliverable will be defined in the appropriate WorkPackage.tex
    % (\localdelivref{deliv-id})
  \end{compactitem}

  These steps are essential to enable the Binder@home
  (\taskref{applications}{binder-at-home}) and the Binder@HPC
  (\taskref{applications}{binder-at-hpc}) use cases.

%   The task includes the following activities
%   \begin{compactitem}
%   \item ...
%      % deliverable will be defined in the appropriate WorkPackage.tex
%     % (\localdelivref{deliv-id})
%   \end{compactitem}
\end{task}

% template for a task
% each task should be added to exactly one workpackage
% in the workpackage task list
\begin{task}[
  title=Support more use patterns,
  % task id for references
  id=patterns,
  % lead institution ID
  lead=SRL,
  PM=16,
  %wphases={0-36},
  % partner institution ID(s)
  % don't include lead here
  partners={MP,IFR}
]
In this task, we provide the technical possibilities to use the Binder tools in
better and new ways. These are used and evaluated with real world applications
in~\WPref{applications}.

\begin{compactitem}
\item The \repotodocker{} tool searches a given repository for specifications of
  software dependencies, and carries on to compose instructions to install all
  of the software dependencies within a Docker container image. In this task, we
  modularise this functionality, and make it possible to extract the required
  instructions on their own (for example into a stand-alone shell script).

  Such functionality could be used to:
  \begin{compactitem}
  \item Extract the list of installation commands to carry out a local install
    of the required software, or an installation within any other environment.
  \item In particular on HPC systems, it may be necessary to install software
    directly on the host, and this functionality would simplify that.
  \item Having the installation instructions neatly summarised will also help
    the interested scientist to understand what software specifications are
    (explicitly or implicitly ) given within the repository.
    % \item Having the installation instructions extracted, these can be used to
    %   explore the effects of changing versions of a particular library or
    %   dependency. This will be helpful to track down the origins of observed
    %   non-reproducibility.
  \end{compactitem}

\item Improve and document the options to access external data resources from
  within the computational environments generated by BinderHub deployments. This
  is a prerequisite for \taskref{applications}{data-publishing}.
\item Restrict access to BinderHub through authentication - useful for many
  institutes who want to offer reproducibility services and data hosting but
  need to limit or control access to those systems (to ensure, for example, that
  the computational resources are not abused).
\item Allow to use Binder without a BinderHub installation. A use case for this
  is to run a Binder service on a local Desktop (see
  \taskref{applications}{binder-at-home}).
\item Allow to use Binder without Jupyter notebooks.

  Binder's original design goal was to allow execution of notebooks within a
  software requirement that provides all the software required by the notebook.
  This may include compile and highly customised software, which might produce
  output files, which are then processed and visualised in the notebook.

  Here, we will provide the foundations for reproducibility work that is
  non-interactive and doesn't need Jupyter notebook. A use case for this is
  reproducibility at High Performance Computing facilities, where often the
  computational tasks cannot be carried out interactively (see
  \taskref{applications}{binder-at-hpc}).

    % Should we say more here? @min?
    %
    % This will require creation of a program that creates the required software
    % environment, and - if a notebook is used - start a notebook server to which
    % the user can connect.

  \end{compactitem}
\end{task}


%
%   This feature will become part of the \repotodocker{} functionality
%   (\localdelivref{extract-dependencies}).
%
%

%%% Local Variables:
%%% mode: latex
%%% TeX-master: "../proposal"
%%% End:

%
\end{tasklist}


\begin{wpdelivs}
  \begin{wpdeliv}[due=12,id=extract-dependencies,dissem=PU,nature=OTHER,lead=SRL]
    {Release new \repotodocker{} feature that exposes the command to install
      identified software environments in stand-alone script
    %  (see \taskref{impact}{extract-dependencies}).
    }
  \end{wpdeliv}

\begin{wpdeliv}[due=36,id=binder-tools-software,dissem=PU,nature=OTHER,lead=SRL]
  {Final open source release of \TheProject tools, completed with automatic
    testing and documentation. }
\end{wpdeliv}

\end{wpdelivs}
\end{workpackage}
%%% Local Variables:
%%% mode: latex
%%% TeX-master: "../proposal"
%%% End:

%  LocalWords:  workpackage wphases wpobjectives wpdescription pageref wpdelivs wpdeliv
%  LocalWords:  dissem mailinglists swrepository final-mgt-rep compactitem swsites ipr
%  LocalWords:  TOWRITE tasklist delivref

\draftpage
\TOWRITE{ALL}{Proofread WP 1 Management pass 1}
\begin{draft}
\TOWRITE{PS (Work Package Lead)}{For WP leaders, please check the following (remove items
once completed)}
\begin{verbatim}
- [ ] have all the tasks in this Work Package a lead institution?
- [ ] have all deliverables in the WP a lead institution?
- [ ] do all tasks list all sites involved in them?
- [ ] does the table of sites and their PM efforts match lists of sites for each task?
      (each site from the table is listed in all relevant tasks, and no site is listed
      only in the table or only at some task)
\end{verbatim}
\end{draft}

\begin{workpackage}[
  id=applications,
  wphases={6-36!.48,12-24!0.62,24-30!0.95,30-36!1.14},
  swsites,
  title=Applications and use cases,
  short=Applications,
  lead=MP,
  MPRM=15,
  SRLRM=3,
  UIORM=8,
  IFRRM=14
]

\begin{wpobjectives}
  The objectives of this work package are
 \begin{compactitem}
 \item to guide the development of core tools in \WPref{reproducibility} and \WPref{impact}
   by simultaneously applying them to real-world use cases from various scientific fields
 \item to do this together with active scientists from these fields to ensure we develop tools
   which can cater for a broad European and global research community
 \item to demonstrate how the tools we develop can support more reproducible and
   reusable science
   \end{compactitem}
\end{wpobjectives}

\begin{wpdescription}

  Whilst the components issued from work packages  \WPref{reproducibility} and \WPref{impact} will be
  made available as generic tools for reproducible open science,
  this work package is focused on using (and if required tailoring) them
  to make them suitable for specific real-world cases.

  The use-cases we anticipate are
  \begin{compactitem}
  \item \localtaskref{demos} Applications and best practice examples that
    demonstrate use of Binder for more reproducible and reusable science.
  \item \localtaskref{binder-at-home} Ability to recreate software environments
    and re-produce results using local compute resources (such as the Desktop of
    a researcher)
  \item \localtaskref{data-publishing} Use of Binder to facilitate access to
    large data sets where the published resource does not only include the data
    itself, but also software to access and read the data.
  \item \localtaskref{binder-at-hpc} Use of Binder at High Performance Computing
    facilities to re-produce computationally more demanding results
  \end{compactitem}

%   \TOWRITE{}{Edit this paragraph:}
%   The context and vision for each of the demonstrators is described in
%   section \ref{sec:science-demonstrators-in-concept} on page
%   \pageref{sec:science-demonstrators-in-concept}.

  Working collaboratively with core Binder developers and research active
  scientists, we merge state-of-the-art knowledge on what is technically
  possible with the understanding of the scientists what reproducibility
  features would significantly improve their workflow. We expect that -- in
  addition to iteratively refining the features of Binder -- we will also
  inspire each other to find out-of-the box solutions that each group on
  their own may not come to think about.

  %\medskip

  % The particular workflows, data infrastructures, data policies, and data access
  % protocols for
  % FAIR\footnote{Findable, Accessible, Interoperable and Reusable} sharing of data vary from one community and use-case to
  % the other, and may not be fully defined yet. Therefore, this proposal
  % does not enforce a specific way of handling data. Instead we
  % will explore in the demonstrator tasks how existing data policies,
  % infrastructure and workflows can be respected and integrated with
  % authentication and authorisation, data management, and
  % Binder services. EGI is represented in our Community Engagement Panel (\taskref{management}{community-engagement-panel})
  % for all the tasks in this work package and will work with us to find the
  % best integration solutions in the evolving EOSC infrastructure.

  \medskip

  The tasks in this work package are progressed throughout the runtime of the
  project, serving both as requirements capture (and so to inform and guide the
  work in \WPref{reproducibility} and \WPref{impact}) and evaluation of the
  technological advances of the Binder tools.

  We will use the regular technical reports to update on progress. An
  interim \localdelivref{report-use-cases-interim} and final
  report will summarise the results (\localdelivref{report-use-cases}).
  Documentation of best practice guideliness will be developed as an open
  document throughout the project, used in \WPref{education}, and be submitted
  as deliverable at the end of the project~(\localdelivref{best-practice-guide}).
  \TODO{Fix link right before this note!}
\end{wpdescription}

\begin{tasklist}

% template for a task
% each task should be added to exactly one workpackage
% in the workpackage task list
\begin{task}[
  title=Science demonstrators,
  % task id for references
  id=demos,
  % lead institution ID
  lead=MP,
  PM=8,
  %wphases={0-36},
  % partner institution ID(s)
  % don't include lead here
  partners={IFR,UIO}
]
In this task, we want to demonstrate the value and usefulness of
\WPref{reproducibility} and \WPref{impact} with real scientific use cases from
the research communities involved in \TheProject (see
Section~\ref{sec:science-applications} on
page~\pageref{sec:science-applications} for the initial set of science applications).

The demonstrators are designed 
to exploit the solutions developed within \TheProject (such as Binder@home, Binder@HPC,
data publishing) and leverage existing institutional and/or national
e-infrastructures as well as core EOSC services. Synergies between the different
science applications and communities will be ensured through the technical tasks
(\taskref{applications}{binder-at-home} Binder@home,
\taskref{applications}{data-publishing} data publishing, and
\taskref{applications}{binder-at-hpc} Binder@HPC).


% \paragraph*{Context:}  For example, in marine research field there are reproducible research examples such as Argopy~\cite{maze2020},  Pangeo ecosystems (http://gallery.pangeo.io/repos/pangeo-gallery/physical-oceanography/). But even within the same research lab, we have number of researchers who depends on commercial software for their data analysis and does not have access to publish reproducible research workflows.   


% \paragraph*{Task activity:}
 
% Within the actual Binder capability, we will demonstrate following two reproducible research configurations.  We demonstrate these research use caseses and show barriers that has been preventing these workflow to be pulished as reproducible science.   

%These information will give first feed back to \WPref{reproducibility}, \WPref{impact} 
%ll enrich the process of propose better accesible optimised 
%research workflow that can benefit researchers themselves, but also include reproducible aspects. 


 % \begin{compactitem}
%  \item FAIR Nordic Earth System Modelling: this science demonstrator leverages Binder@HOME (model development, education, single column or very simple model configuration), Binder@HPC (operational runs at scale including on EuroHPC), data publishing (publication of simulation results from blue-sky research);
%  \item Demonstration of marine physics and fish habitats modelling and analysis using Pangeo ecosystem.  
  
%\TODO{Tina, move this to section 1? but where??
%This effort and outcome of it will connect to the Digital Twin Ocean project to allow ocean data and models relevant to biodiversity to be re-used by researchers and engineers. This will provide a concept demonstration of ingesting ocean data and model output that can be reproduced through the existing ocean research infrastructures
% should be able to add 'interdisciplinary' part in chap1  as it bridges biology and physics..}

     % deliverable will be defined in the appropriate WorkPackage.tex
    % (\localdelivref{deliv-id})
%  \end{compactitem}
\end{task}

%\input{tasks/policy}
% template for a task
% each task should be added to exactly one workpackage
% in the workpackage task list
\begin{task}[
  title=Binder@Home,
  % task id for references
  id=binder-at-home,
  % lead institution ID
  lead=SRL,
  PM=7,
  %wphases={0-36},
  % partner institution ID(s)
  % don't include lead here
  partners={MP,UIO}
]
In this task we target the compute power of the desktops of individual researchers and
users to recreate software environments in which computational results can be
reproduced and re-used: We envision to provide an experience identical or similar
to that of using BinderHub, but only employing the local computer available to the user anyway.

\paragraph*{Context:} Reproducibility services using Binder currently rely on hosted Binder instances.
The BinderHub Federation provides such a service global service at mybinder.org
which is free at
the point of use. It delegates the building of the software environment and
re-execution of the code to a small number of computer centres that have
volunteered to contribute compute resources.
It may hence be possible to overload the system \TODO{Add link to general discussion
  that mentions how many thousand builds take place per months etc}.
To allow the up-scaling of good reproducibility practice, it is 
desirable not to depend exclusively on such a single hosted service (or even multiple similar hosted services).

Moreover, the compute resources typically offered through mybinder are modest, namely at most 2 GB of RAM
and a single CPU core.
In contrast most laptops and Desktops have similar or better hardware capabilities than
the free mybinder cloud-computing resources currently offer, making it highly desirable to leverage these local resources.
% not sure how/if the following two sentences fit here, maybe better elsewhere
For some research areas, it is essential to access big data sets to reproduce the scientific computing workflow.
For such cases, it is crucial to reproduce the work using infrastructure that has optimised access to the data,
so that one does not consume unnessesary computing and network resources.  

\paragraph*{Task activity:} Based on the preparations in \WPref{reproducibility} and
\WPref{impact}, we intend to extend Binder such that users of the service can
carry out the building of the environment, and -- if desired -- the launching of a
notebook server \emph{on their own hardware}, e.g. their laptop or on-premise or cloud infrastructure.
The working title for such
functionality is ``Binder@Home'' (as a reference to the crowd-based SETI@home search for
extraterrestrial intelligence at home.\footnote{https://setiathome.berkeley.edu})

We will design, implement, and test the 'Binder@Home' functionality, and make it
available as part of the binder software. That new 'Binder@Home' software component will
essentially complement BinderHub with an easy-to-use local single-user case, it will trigger
the local build of the software environment, the start of the Jupyter notebook
server, and the opening of the relevant local URL and port in a browser.
This effort will bridge the gap from mybinder.org to ``Home'';
by giving the freedom and digital sovereignty
to the researchers to chose where they execute their computational experiments in a simple manner,
without sacrificing the convenience of the mybinder.org service.

%  The task includes the following activities
%  \begin{compactitem}
%  \item ...
%     % deliverable will be defined in the appropriate WorkPackage.tex
%    % (\localdelivref{deliv-id})
%  \end{compactitem}
\end{task}

% template for a task
% each task should be added to exactly one workpackage
% in the workpackage task list
\begin{task}[
  title=Data publishing,
  % task id for references
  id=data-publishing,
  % lead institution ID
  lead=MP,
  PM=8,
  %wphases={0-36},
  % partner institution ID(s)
  % don't include lead here
  partners={IFR,UIO}
  ]
  The task focuses on the use of reproducible software environments within
  Binder to provide working and interactive code that provides access to large
  or complex data sets.

  \paragraph*{Context:} Scientists would like to publish their data. Such a data
  publication must include the data set itself, and metadata that explains how to
  interpret the data. In addition to such information, it can improve the
  quality of the data set if \emph{computer executable libraries or commands}
  are provided, which simplify the reading of the actual data files. Such
  routines encapsulate meta information about the data file (format and
  structure) in a machine-readable format.

  A binder-enabled repository can provide the access to such data sets by
  containing the specification of a software environment and hosting of the
  file-reading routines together. Such an approach significantly simplifies the
  re-use of the data (or reproduction of existing study) because the data
  reading routines do not need to be re-implemented.

  \paragraph*{Task activity:}
  We will design and implement functionality that allows such data publishing
  based on Binder tools.

  A major challenge is the link to the data: ideally, data sets are hosted on a
  separate infrastructure (such as archives, or files published together with a
  publication - for example on Zenodo). It will thus be necessary to reference
  the data on this data-holding archive and the data location within that resource.
  This data location will need to be used from the notebooks to access
  the data. %We are not aware of a common standard that could be adapted here.

  Some authentication for data access may be required: either because the data
  is not meant to be fully public, or because access to the data creates
  significant cost for the hosting party. Such authentication
  information/credentials from (the Binderhub) login must be passed to the point
  where the data-holding medium is mounted in a container.

  We will work very closely with the Max Planck Compute and Data Facility
  (MPCDF) to prototype such functionality. We will allow to access data sets
  that the MPCDF hosts themselves.

  An important outcome of this task is an evaluation of the chosen design and
  implementation, to propose a more generic model for the next feature extension
  of the Binder tools.
\end{task}

% template for a task
% each task should be added to exactly one workpackage
% in the workpackage task list
\begin{task}[
  title=Binder at HPC facilities,
  % task id for references
  id=binder-at-hpc,
  % lead institution ID
  lead=MP,
  PM=17,
  % wphases={0-36},
  % partner institution ID(s)
  % don't include lead here
  partners={SRL,IFR,UIO}
]
In this task, we want to broaden the applicability of the Binder tools to become
more useful in High Performance Computing (HPC) environments. In particular, this
requires parallel execution of software based on the reproduced computational environments.

\paragraph*{Context:}
Reproducibility of data created on HPC resources is difficult. In addition to
a specification and access to the actual (simulation) software and dependencies,
one may also need specialised HPC hardware to be able to execute the software.

The notebook interface -- which works well for many examples in computational and
data science -- may not be appropriate for HPC applications, where jobs of
substantial run-time are typically submitted to a queuing system, which will
trigger execution of the (parallel) job, once the required resources have become
available.

It is outside the scope of this proposal to find and implement a generic
reproducibility approach for HPC use cases. Nevertheless, we propose to explore
some aspects of HPC-reproducibility to influence the development of Binder,
and start the -- probably iterative -- process of finding a generic solution.

\paragraph*{Strategy:}
We focus on reproducible execution of an HPC application to compute data as the step of
novelty here. This may well be without using notebooks, but will consist of the
building of the software environment and the submission of a batch job making use of this
environment to the HPC system's queue.

We assume for this to work that the user (who wants to reproduce some results)
needs to login to the HPC resource of their choice, and uses repo2docker to
create a suitable software container, and starts execution of those containers
``manually'', for example through submission of a compute job to the Slurm
queuing system. We also assume that the hardware required for this reproduction
is available on the HPC system.

\paragraph*{Activity:} We will use repo2docker to automate the creation of
reproducible software environments on an HPC system. If no parallelism is
required, this is similar to the Binder@home scenario
(\taskref{applications}{binder-at-home}). Shared memory parallelisation, using for
example OpenMP, will work well with the approach chosen.

A main task here will be to explore the feasibility of using distributed
parallelisation (for example through MPI) where for the execution of an
MPI-program the processes on the nodes run in containers but communicate via MPI
as usual. We evaluate the situation with HPC software that allows
MPI-parallelisation (such as Octopus). Subsequently, we will share our experience with
reproducible software environments in HPC
contexts~(\localdelivref{report-use-cases-interim},~\localdelivref{report-use-cases}).

\paragraph*{Challenges:} We expect that we need to use a container technology
that is widely accepted at HPC sites (for example Singularity~\cite{Singularity2017} or CharlieCloud~\cite{CharlieCloud2017}), as Docker
on HPC systems is typically avoided due its strict root-user requirements~\cite{Gerhardt_2017}.

Another challenge is that of using accelerators such as GPU cards which are
installed on the host but need to be accessed and instructed efficiently from the software running
inside the container.
Similarly, for HPC systems, there are specialised drivers for high-performance
network cards: can these be used from the container environment and what
is the performance impact when doing so?  \cite{Liu2021}

These investigations will guide us in answering the following important questions:
Should hardware requirements be archived in the reproducible repository?
If so, what specification should we use?

The existing buildpacks that repo2docker supports
\ref{sec:repo2docker} may need to be extended for
HPC specific software provisioning tools such as Spack or Easybuild.

% The re-execution of parallel software may create significant costs, and thus
% authentication may be necessary.

%   The task includes the following activities
%   \begin{compactitem}
%   \item ...
%      % deliverable will be defined in the appropriate WorkPackage.tex
%     % (\localdelivref{deliv-id})
%   \end{compactitem}
\end{task}





%%% Local Variables:
%%% mode: latex
%%% TeX-master: "../proposal"
%%% End:

\end{tasklist}


\begin{wpdelivs}
  \begin{wpdeliv}[
    % id for linking with \delivref or \localdelivref
    id=report-use-cases-interim,
    % lead institution
    lead=MP,
    % month when deliverable is due (max 36)
    due=18,
    % ~always PU, DEC
    dissem=PU,
    nature=DEC,
    ]
    {
      Interim report on real world use cases of Binder for reproducible and reusable science
    }
  \end{wpdeliv}

  \begin{wpdeliv}[
    % id for linking with \delivref or \localdelivref
    id=report-use-cases,
    % lead institution
    lead=MP,
    % month when deliverable is due (max 36)
    due=34,
    % ~always PU, DEC
    dissem=PU,
    nature=DEC,
    ]
    {
      Final report on real world use cases of Binder for reproducible and reusable science
    }
  \end{wpdeliv}


% \begin{wpdeliv}[
%   % id for linking with \delivref or \localdelivref
%   id=best-practice-guide,
%   % lead institution
%   lead=UIO,
%   % month when deliverable is due (max 36)
%   due=34,
%   % ~always PU, DEC
%   dissem=PU,
%   nature=DEC,
%   ]
%   {
%     Best practice guide for reproducible science with Binder
%   }
% \end{wpdeliv}

\end{wpdelivs}
\end{workpackage}
%%% Local Variables:
%%% mode: latex
%%% TeX-master: "../proposal"
%%% End:

%  LocalWords:  workpackage wphases wpobjectives wpdescription pageref wpdelivs wpdeliv
%  LocalWords:  dissem mailinglists swrepository final-mgt-rep compactitem swsites ipr
%  LocalWords:  TOWRITE tasklist delivref

\draftpage
\TOWRITE{ALL}{Proofread WP 5 Management pass 1}
\begin{draft}
\TOWRITE{PS (Work Package Lead)}{For WP leaders, please check the following (remove items
once completed)}
\begin{verbatim}
- [ ] have all the tasks in this Work Package a lead institution?
- [ ] have all deliverables in the WP a lead institution?
- [ ] do all tasks list all sites involved in them?
- [ ] does the table of sites and their PM efforts match lists of sites for each task?
      (each site from the table is listed in all relevant tasks, and no site is listed
      only in the table or only at some task)

- [ ] Binder / repo2docker documentation: tutorials and best practice guides -
      have we got this covered?
\end{verbatim}
\end{draft}
% (41)/36 = 1.138888 -> !1.14
\begin{workpackage}[id=education,wphases=0-36!0.97,
  title={Dissemination, education and engagement},
  short={Dissemination, education, and engagement},
  lead=IFR,
  IFRRM=10,
  MPRM=6,
  SRLRM=7,
  QSRM=3,
  UIORM=9,
  swsites
]


\begin{wpobjectives}
  A key focus of this work package is to disseminate the results of this
  project, including the technical advances and guidance for best practice for
  reproducible science. This includes educating researchers about the value of
  open science, reproducibility and re-usability as well as the possibilities of
  integrating Binder tools in their workflows.

  Beyond this activity, which is directed from the project members to the wider community of
  scientists, we use this work package to engage with researchers and
  stakeholders and seek input from them to the project. Desired input includes
  requirements for practical reproducibility in the different domain as well as
  technical contributions -- for example through merge requests for Binder
  tools, or open source documentation of best practice for reprocudible software
  environments.

  Our dissemination, education and engagement objectives includes:
 \begin{compactitem}
   \item Ensure awareness of the results of the project in the user community,
     and in particular in those groups that act as educators and multipliers of
     knowldege (such as the Carpentries and research infrastructure organisations).
   \item Educate the community on the value of open science, and in particular
   \item train researchers in best practices for open and reproducible science.
   \item Produce and training and education material to disseminate the ability to
     publish reproducible computational science outputs using the tools we
     improve and develop.
   \item Address the shortage of researchers and research support staff trained
     in practical reproducibility.
   \item Provide documentation and tutorials which can serve as the technical
     components of reproducibility policies.
   \item Throughout these activities engage with users and stakeholders, to
     listen and understand their barriers or incentives towards more
     reproducible science, and the usability of the \TheProject outputs.
 \end{compactitem}
\end{wpobjectives}

% Potential sources of inspiration: ODK's WP2 work package about dissemination:
% PDF: p.36 of https://github.com/OpenDreamKit/OpenDreamKit/raw/master/Proposal/proposal-www.pdf
% Sources: https://github.com/OpenDreamKit/OpenDreamKit/blob/master/Proposal/WorkPackages/DisseminationCommunityBuilding.tex

\begin{wpdescription}

  Open science and reproducible science is entirely dependent on researchers
  adopting open practices. In \TheProject, we improve and developing tools that
  can facilitate these practices, but they only work if researchers actually adopt
  them. For researchers to adopt the practices, they need to (i) know about them
  and (ii) want to use them.

  We address this challenge in multiple ways:
  \begin{enumerate}
  \item The philosophy of the Binder tools is to respect existing standards and
    best practice (and not to invent additional syntax or requirements). It is
    thus possible to use the Binder tools (to recreate a software environment)
    even if the repository authors did not anticipate the use of Binder, or knew
    about their existence. In the best possible scenario, a scientific research
    output becomes automatically reproducible with Binder without the
    \emph{author having to know about Binder or having to invest additional
      effort} (beyond following best practice).

    \item In this work package, we produce education materials and carry out
      education activities to spread the knowledge about \emph{good practice for
        reproducibility and re-usability in science}, such as for example
      automation of all analysis steps, and complete documentation of the
      required software stack. Only one aspect of this training is to show how
      Binder can help with reproducibility.

      Attendees and users following such training and advice will create more
      reproducible artifacts. If they -- or later users of their published
      artifacts -- want to use Binder to reproduce or re-use the results, they
      can. Even if they do not, we will have achieved an improvement of the
      reproducibility of scientific artifacts.

    \item Throughout the activities of this project and the engagement with the
      wider science community, we aim to understand the underlying drivers for
      acceptance, rejection or lack-of-interest in adopting practices that lead
      to reproducible science results.
  \end{enumerate}

  % HF: I think training the wider (non-scientist) public about Binder is going
  % too far?
  %
  % Going further, it is also clear that open science is not just of value
  % to researchers: one of the largest benefits of open science is that it makes
  % science accessible to the broader public who may not be members of the
  % research community.
  %
  % To this end, in addition to training researchers, we will also train the
  % public in how to make use of the open science research and services
  % facilitated by \TheProject. This will be done through regular open
  % dissemination and training workshops, as well as by producing and maintaining
  % material for online courses and documentation.

  The \TheProject project will develop, through \WPref{applications}, a number
  of demonstrator repositories that show examples of reproducibility in
  different scientific domains. (We use
  those activities to inform and evaluate the technical improvement to the
  Binder tools in \WPref{reproducibility} and \WPref{impact}). We also use those
  studies to create tutorials and \emph{best practice guides for
    reproducibility} (\localtaskref{online-resources}) in this work package, and
  offer interactive workshops (\localtaskref{workshops}) to help disseminate the
  content more effectively.

  As with all the code, test and build infrastructure produced as part of the
  project, we will also make all documentation open source. Our documentation --
  which includes best practice guides for reproducibility -- can thus be
  modified and improved after the end of the project to react to new
  developments (\delivref{education}{education-materials2}).

  We will engage with the scientific to support them in making their
  work more reproducible with Binder tools. The project will benefit from these interactions
  as we will learn more about reproducibility requirements and usefulness of the
  Binder tools, so that we can tailor our work to support scientific communities as broad as possible
  (\localtaskref{community-support}).

  We will also participate in the well established academic dissemination
  activities, and events of the European e-infrastructure projects and other
  relevant structures. EGI is a member of our the community engagement panel
  (\taskref{management}{community-engagement-panel}) and the interaction with
  them will be useful to prioritise our resources in this very active field.

  Open access to all publications resulting from the project will be ensured.
  %\TODO{Should this sentence go into a data / IPR management plan?}
\end{wpdescription}

\begin{tasklist}
% template for a task
% each task should be added to exactly one workpackage
% in the workpackage task list
\begin{task}[
  title=Best practice guidelines for reproducible science,
  id=online-resources,
  lead=UIO,
  PM=13,
  %wphases={6-36!0.36},
  partners={SRL,MP,IFR}
]
The aims of this task are to (i) provide online resources for researchers on more reproducible
and reusable science, and (ii) support delivery of our workshops
(\taskref{education}{workshops}).

This task includes the following activities:
  \begin{compactitem}
  \item Collect and compose best practice guidelines for reproducible and
    reusable science (\delivref{education}{best-practice-guide}).
  \item Split the content into multiple topic areas and target audiences so learners
    with different prior knowledge and needs can be directed to the most relevant content.
  \item Develop lesson materials on \emph{open science} best practices (version
    control, testing, automation of all steps, collaboration and peer review,
    documentation, software licensing and open source, use of Jupyter
    notebooks).
  \item Develop lesson materials on \emph{reproducible computational science},
    which focuses on combining the open science tools for reproducible science.
  \item Develop materials on \emph{using Binder tools to make science more
      reproducible and reusable}. This includes addressing and describing the
    use cases from \WPref{applications}.
  \item Collaboration with the \href{https://coderefinery.org}{CodeRefinery}
    project and \href{https://carpentries.org/}{The Carpentries} (Carpentries incubator and Carpentries Lab)
    for the development and maintainance of the online lesson materials and delivery of workshops.
  \item The training material will also be referenced on the Binder tools webpage.
  \end{compactitem}
  All material will be licensed under an open license such as
  \href{https://creativecommons.org/licenses/by/4.0/}{CC 4.0}
  (\delivref{education}{best-practice-guide}).
\end{task}

% template for a task
% each task should be added to exactly one workpackage
% in the workpackage task list
\begin{task}[
  title=Training Workshops for more reproducible science,
  id=workshops,
  lead=UIO,
  PM=9,
  % wphases={12-36!.25},
  partners={SRL,MP,IFR}
]
This task is focused on taking the content from the
\taskref{education}{online-resources} (Best practice for
reproducible science guidelines) and disseminating it through various channels and to different target audiences.

% \begin{compactitem}

%    \item Defining and implementing a strategy to enable a shared vision of the Jupyter ecosystem across all the actors from developers, users to every stakeholder: the current misalignment hinders the full exploitation of Open Software practices where co-design is a de facto approach.
%
% For instance, the official Jupyter documentation (https://jupyter.org/documentation) solely reflects the view of developers where the Jupyter ecosystem is defined as a set of software packages (jupyter-core, jupyter-client, kernels, widgets (ipywidgets, ipyleaflet, etc.). The user vision is relegated to examplars (blogs, newsletters, etc.) which inevitably tend to be restrictive but often become de facto standards. This can lead to misconceptions and makes it more difficult for on-boarding novices and new communities.
%

% \item Triggering a cultural change to help under-represented groups to actively participate to the development of open source project to ensure the sustainability of the \TheProject services deployed on EOSC-HUB.
%

%\item Foster Open innovation by collaborating with others from different background and activities (school, universities, industries, journalists, artists, etc.)
%  \end{compactitem}

To achieve these goals, the following actions/activities will take place:

  \begin{compactitem}

  %\item co-design hackathons: the co-design efforts between domain scientists, \TheProject developers and service providers will be carried out at any point in time of the project and will be registered in a co-design register to help for future engagement with new communities of users. To be fully effective,  co-design hackathons will be organized to set the stage, define rules for co-design interactions and more importantly align all actors into a common user-centred vision of \TheProject services and associated development towards a successful EOSC deployment.

%    \item Workshops on Findable, Accessible, Interoperable and Reusable (FAIR)
%      software and data to facilitate the adoption of Open Science and Open
%      Scholarship best practices (transparent, sharable and collaborative
%      Science): this would not be restricted to the Jupyter ecosystem and will
%      teach users how to make data, lab notes and other research processes freely
%      available, under terms that enable reuse (licensing), redistribution and
%      reproducibility of methods and/or results.

   \item Delivery of workshops on (i) open science, (ii) reproducible computational
     science, and (iii) the use of Binder tools to support this.

     The content is focused on key insights and tools need for more reproducible
     science, but will be contextualised and delivered in the wider field of
     Findable, Accessible, Interoperable and Reusable (FAIR) software and data.

   % \item Trainings on how to use \TheProject software and services to fully
   %   exploit \TheProject developments for repoducible science: develop training
   %   materials and organize training events for researchers and the public to
   %   enable Open Science and maximise the usefulness of \TheProject
   %   developments.

   \item \TheProject Admin trainings: we will offer training events for learning on how to
     deploy \TheProject services such as BinderHub. This will be relevant for a
     very small (but important) group of users, i.e. those that want to host
     their own BinderHub instance. We know from multiple research organisations
     that this desire exists.

   \item Where possible (for instance after consultation of the Community Engagement Panel), 
     we will schedule dissemination events to take place
     during conferences and community events, such as PyData, EuroSciPy,
     Supercomputing meetings.

   \item We will archive recordings of the training events to support the
     increasing desire of learners to make use of online streaming services
     (such as YouTube) to work through a learning programme at their own time
     and pace.

   \item We will offer in-person, remote and hybrid training.

   \item The work will be done in collaboration with
     \href{https://coderefinery.org}{CodeRefinery} project (the University of Oslo is a partner of the CodeRefinery project) 
     and will make available its network of instructors and helpers
     to co-organize, advertise and run online workshops on Open Science best practices. 

  \item We will detail our executed activities through the reporting at the end of
    each reporting period.
  \end{compactitem}
\end{task}

\begin{task}[
  title=Community support and engagement,
  id=community-support,
  lead=SRL,
  PM=13,
  %wphases={12-36!.5},
  partners={MP,QS,UIO,IFR}
]
A project such as \TheProject{} has the ambition to develop a small set of tools
that will \emph{impact many researchers} and have the potential to be useful
\emph{across all scientific domains that need electronic data processing} as part of their
scientific research and publication process.

As such, we expect that the demand through support queries, documentation
clarification questions, and helpful feedback will be substantial. With this
task, we explicitly reserve some time for such activities.

We have an opportunity here to address multiple aims simultaneously.

This task complements the Community Engagement Panel and has several aims: 
\begin{compactitem}
\item to engage with community members (and potentially their computing support
  staff) to help them make best use of the Binder tools. This can range from
  helping to configure a BinderHub installation, to address usage questions of
  tools such as \repotodocker{} in domain-specific contexts;
\item to engage with community members to better understand diverse
  requirements, and use this information to make the Binder tools and
  reproducibility guidelines more useful for a wider diversity of scientific
  domains;
\item to engage with community members to train researchers and research
  software engineers in reproducibility practices and tools (to address a
  shortage of staff with such skills)
\item to engage with community members to invite them to contribute to the
  binder tools, the reproducibility guidelines and policy development, and other
  open source tools.
\end{compactitem}

We will achieve those aims through listening to feedback, queries and requests
for help from the community, and reserve time to respond. Depending on the
complexity of an issue, guidance by email, chat, video meeting or even an
in-person visit may be appropriate. (When demand exceeds the time budget, we
will need to prioritise which issues we can deal with first.)

We know from our experience with running and contributing to open source
projects that such engagement activities are effective in training interested
and often highly skilled scientists and research software engineers to become
contributors to open source projects. While they may have a primary interest in
improving an open source tool to suit their needs, this will likely benefit
others as well. Once somebody has contributed to a particular open source
software tool, they are more likely to make follow-up contributions - for
example to improve documentation.

\end{task}

\end{tasklist}

\begin{wpdelivs}
  \begin{wpdeliv}[due=24,id=best-practice-guide,dissem=PU,nature=R,lead=UIO]
  {Best practice guide for reproducible science with Binder.}
\end{wpdeliv}
\begin{wpdeliv}[due=36,id=education-materials2,dissem=PU,nature=R,lead=UIO]
  {All training sessions material completed, reviewed, and published online.}
\end{wpdeliv}
% \begin{wpdeliv}[due=36,id=report2,dissem=PU,nature=R,lead=UIO]
%   {Community building: Report on impact of development workshops, dissemination and training activities.}
% \end{wpdeliv}
\end{wpdelivs}

\end{workpackage}
%%% Local Variables:
%%% mode: latex
%%% TeX-master: "../proposal"
%%% End:

%  LocalWords:  workpackage wphases wpobjectives wpdescription pageref wpdelivs wpdeliv
%  LocalWords:  dissem mailinglists swrepository final-mgt-rep compactitem swsites ipr
%  LocalWords:  TOWRITE tasklist delivref

\draftpage

\ifgrantagreement
\endgroup
\setcounter{page}{\value{savepage}}
\fi

%%% Local Variables:
%%% mode: latex
%%% TeX-master: "../proposal"
%%% End:

%  LocalWords:  newpage workpackages workplan



% \TODO{Strange vertical lines at the left of the bottom of table~\ref{sec:deliverables}?}

\subsubsection{Deliverables}\label{sec:deliverables}
\inputdelivs{8.5cm}%
\subsubsection{Milestones}\label{sec:milestones}

\eucommentary{Milestones means control points in the project that help to chart progress. Milestones may
correspond to the completion of a key deliverable, allowing the next phase of the work to begin.
They may also be needed at intermediary points so that, if problems have arisen, corrective
measures can be taken. A milestone may be a critical decision point in the project where, for
example, the consortium must decide which of several technologies to adopt for further
development.
}

\begin{draft}
\begin{verbatim}
TODO:
- [ ] sort milestones
- [ ] check dates
- [ ] omit descriptions? Template doesn't have any
- [ ] involved WP: both input and output, or just input?
\end{verbatim}
\end{draft}

\begin{milestones}
  \milestone[
    id=study,
    month=12,
    wps={reproducibility},
    verif={Report produced},
    ]
  {Reproducibility study}
  {
  We will have preliminary study results to guide and evaluate
  improvements to reproducibility with \TheProject tools,
  and inform best practices.
  }
  \milestone[
    id=conda-time,
    month=6,
    wps={reproducibility},
    verif={Feature available in conda/mamba software},
    ]
  {Conda time machine}
  {
  The conda/mamba package manager shall be able to
  select packages for installation based on a given date
  }

  \milestone[
    id=repo2docker-time,
    month=18,
    wps={reproducibility},
    verif={Feature present in repo2docker software},
    ]
  {repo2docker takes publication time into account}
  {
  }

  \milestone[
    id=repo2docker-improved,
    month=24,
    wps={reproducibility},
    verif={Delivered in repo2docker software; Repeat study, comparing baseline results form start of project},
    ]
  {repo2docker produces robust computational environments}
  {
  Taking input from earlier study and tests,
  improvements to repo2docker are made.
  }

  \milestone[
    id=rm-kubernetes,
    month=9,
    wps={impact,applications},
    verif={Feature available in BinderHub software},
    ]
  {BinderHub can be deployed without Kubernetes}
  {
  }

  \milestone[
    id=rm-docker,
    month=15,
    wps={impact,applications},
    verif={Feature available in repo2docker software},
    ]
  {Support for alternative container technologies in repo2docker for suitability in HPC}
  {
  }

  \milestone[
    id=data-publishing,
    month=18,
    wps={impact,applications},
    verif={Demonstrated example deployment},
    ]
  {Support for authenticated data publishing}
  {
  }

  \milestone[
    id=prototype,
    month=12,
    wps={applications},
    verif={
      Deployed first functional prototypes of science demonstrators.
      Early users are able to access and test prototype services
    }
    ]
  {Prototype demonstrator services}
  {
  By this point, prototype demonstrator services will be useful and accessible
  to a broad range of users, and we will have begun to experiment with early-adopter
  users and local demonstrators to guide further development of \TheProject,
  ensuring that development serves the reproducibility needs of the global science community.
  }

  \milestone[
    id=docs-online,
    month=12,
    wps={education},
    verif={Resources available from project website},
    ]
  {Draft best practices documentation}
  {
  Draft version of documentation for best practices is online
  }
\end{milestones}

\milestonetable


\end{workplan}

\subsubsection{Risks and risk management strategy}
\label{sec:risks}

\ifgrantagreement\else
An initial risk assessment appears as Table~\ref{risk-table}.

\begin{table}
\begin{center}
\begin{tabular}{|m{.2\textwidth}|m{.12\textwidth}|m{.58\textwidth}|}\hline
  Risk & Level without / with mitigation & Mitigation measures
  \\\hline

   \multicolumn{3}{|c|}{
    \textit{General technical / scientific risks}
   }
   \\\hline

  Implementing infrastructure that does not match the needs of end users & High/Low &
  Many of the members of the consortium are themselves end-users with
  a diverse range of needs and points of view; hence the design of
  the proposal and the governance of the project is naturally steered
  by demand; besides, because we are building tools, users have the
  flexibility to adapt the infrastructure to their needs. In addition, the open source nature
  of the project facilitates and promotes the involvement of the wider community in terms of
  providing feedback and requesting additional features via platforms such as GitHub and Bitbucket
  on a regular basis.
  \\\hline

  Lack of predictability for tasks that are pursued jointly with
  the community & Medium/Low &
  The PIs have a strong experience managing community-developed
  projects where the execution of tasks depends on the availability of
  partners. Some tasks may end up requiring more effort from
  \TheProject to be completed on time, while others may be entirely
  taken care of by the community. Reallocating tasks and redefining
  work plans is common practice needed to cater to a
  fast evolving context. Such random factors will be averaged out over
  the large number of independent tasks.\\\hline

  Reliance on external software components & Medium/Low & The non-trivial
  software components \TheProject relies on are open source. Most are
  very mature
  and supported by an active community, which offers strong long run
  guarantees. The other components could be replaced by alternatives, or
  even taken over by the participants if necessary.
  \\\hline
  %\\\hline

%  \multicolumn{3}{|c|}{
%    \textit{Use-case risks}
%  }
%  \\\hline
%
%  & & \TOWRITE{WP4}{Risks related to use-cases in WP4}
%  \\\hline

  \multicolumn{3}{|c|}{
    \textit{Management risks}
  }
  \\\hline

  Recruitment of highly qualified staff & High/Medium &

  The majority of positions funded by \TheProject are already hired.
  Only two positions are to be filled, both full-time research software engineers,
  and partners have much experience hiring excellent staff at attractive sites.
  In addition, we
  have a critical mass of qualified staff in the project enabling us
  to train and mentor new recruits.
 \\\hline

  Different groups not forming effective team & Medium/Low & The participants have a long
  track record of working collaboratively across multiple
  sites. Thorough planning of project meetings, workshops and
  one-to-one partner visits will facilitate effective teamwork,
  combining in-person and remote collaboration.\\\hline
  % this also justifies our generous travel budget.

  Partner leaves the consortium & High/Low & If the Grant Agreement requires a replacement
  in order to achieve the project's objectives, the consortium will invite a new
  relevant partner in. If a replacement is not necessary, the resources and tasks
  of the departing partner will be reallocated to the alternative ones within the
  consortium.
  \\\hline

  \multicolumn{3}{|c|}{
    \textit{Dissemination risks}
  }
  \\\hline

  Impact of dissemination activities is lower than planned. & Medium/Low &

  Partners in the consortium have a proven track record at community
  building, training, dissemination, social media communication, and
  outreach, which reduce the risk. The Project Coordinator
  will monitor impact of all dissemination activities. If a deficiency is identified, the consortium
  will propose relevant corrective actions.\\\hline

  \end{tabular}
\end{center}
\caption{\label{risk-table}Initial Risk Assessment}
\end{table}
\fi

\clearpage
%
% a table showing number of person months required (table 3.1f);
% 	a table showing description and justification of subcontracting costs for each participant (table 3.1g);
% -	a table showing justifications for purchase costs (table 3.1h) for participants where those costs exceed 15% of the personnel costs (according to the budget table in proposal part A);
% -	if applicable, a table showing justifications for other costs categories (table 3.1i);
% -	if applicable, a table showing in-kind contributions from third parties (table 3.1j)
\draftpage
\wpfig[label=fig:staffeffort,caption=Summary of staff effort]
\subsubsection{Resources to be committed}\label{sec:resources}

\eucommentary{
Please indicate the number of person/months over the whole duration of the planned work, for each work package, for each participant. Identify the work-package leader for each WP by showing the relevant person-month figure in bold.
}



Table~\ref{fig:staffeffort} shows the distribution of person months across participants and work packages. 


%%%%%%%%%%%%%%%%%%%%%%%%%%%%%%%%%%%%%%%%%%%%%%%%%%%%%%%%%%%%%%%%%%%%%%%%%%%%%%
% \paragraph{Purchase costs}

%\noindent\textbf{Purchase costs}

\TOWRITE{}{How much justification necessary?}

\eucommentary{
Please complete the table below for each participant if the purchase costs (i.e.
the sum of the costs for 'travel and subsistence', 'equipment', and `other
goods, works and services') exceeds 15\% of the personnel costs for that
participant (according to the budget table in proposal part A). The record must
list cost items in order of costs and starting with the largest cost item, up to
the level that the remaining costs are below 15\% of personnel costs
}

\paragraph*{Purchase costs} All participants request less than 15\% of personnel costs in purchase costs.
These costs enable our work plan through supporting: 
\begin{itemize}[noitemsep]
\item project meetings
\item site visits between project members to foster collaboration
\item conference attendance for dissemination
\item open access publication fees
\item communication activities
\item equipment for carrying out the work (high performance laptop computers for each FTE)
\item hosting workshops for dissemination (in budget of lead site \site{SRL})
\item cloud computing costs for testing outputs and supporting development and workshops (\site{SRL} budget)
\item CFS (\site{SRL} only)
\TOWRITE{}{needs to be discussed earlier in work plan}
\end{itemize}

% below is commented-out a more detailed justification for Simula
% See line 151 in
% https://docs.google.com/spreadsheets/d/19xparkP93ANTecMqApNVEvl_w9-q12KE5yHS-IjN7T4

% \site{SRL} is the only site requesting more than 15\% of personnel costs in purchase costs.
% This is because \site{SRL} will host some shared project-wide costs,
% such as hosting workshops for \WPref{education}
% and project-wide cloud costs for testing and demonstration,
% used across all work packages.
%
% % Our project travel costs are estimated based on:
% %
% \begin{compactenum}
% \item 800 \euro per traveller for each of 3 project meetings
% \item 2000 \euro per FTE-researcher-year (1.75) for site visits, facilitating collaboration
% \item 3000 \euro per FTE-researcher-year (1.75) for conference attendance
% \item 4000 \euro for hosting each of two in-person workshops,
%       with an estimated cost of 400 \euro per participant (10 participants).
% \end{compactenum}
%
% \bigskip
% \begin{table}[H]
% \begin{tabular}{|r|r|p{8.5cm}|}
%   \hline
%   \textbf{\site{SRL}}
%     & \textbf{Cost (\euro)}
%     & \textbf{Justification}
%     \\
%   \hline
%   \textbf{Other goods, works, and services}
%     & 36100
%     & Average 600 \euro per month of cloud computing service costs,
%       3500 \euro for CFS, hosting two workshops of 10 attendees at 400 \euro per attendee,
%       3000 \euro for open access publication fees.
%     \\
%   \hline
%   \textbf{Remaining purchase costs}
%     & 50300
%     \\
%   \cline{1-2}
%   \textbf{Total}
%     & 86400
%     \\
%   \cline{1-2}
%   \end{tabular}
% \caption{Overview: 'Purchase costs' to be committed at Simula
% Research Laboratory
% (all in \texteuro)}\vspace*{-1em}
% \end{table}


%%% Local Variables:
%%% mode: latex
%%% TeX-master: "proposal"
%%% End:
