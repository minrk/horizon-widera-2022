\providecommand{\classoptions}{keys}
%% The next two lines are suggested at
%% to work around the following error:
%%
%%  ----------------------------
%%  /usr/local/texlive/2018/texmf-dist/tex/latex/chngcntr/chngcntr.sty:42: LaTeX Error: Command \counterwithout already defined.
%%                 Or name \end... illegal, see p.192 of the manual.
%%
%%  See the LaTeX manual or LaTeX Companion for explanation.
%%  Type  H <return>  for immediate help.
%%   ...
%%
%%   l.42 ...thout}{\@ifstar{\c@t@soutstar}{\c@t@sout}}
%%   ----------------------------
%%
%% The two lines:
\let\counterwithout\relax
\let\counterwithin\relax
%% Suggested fix above taken from
%% https://tex.stackexchange.com/questions/425600/latex-error-command-counterwithout-already-defined
%%

\documentclass[
  11pt,
  deliverables,
  longtasklabels,
  numericcites,
  noworkareas,
  svgnames,
  \classoptions
]{euproposal}       % for writing
%\documentclass[submit,noworkareas,deliverables]{euproposal}        % for submission
%\documentclass[submit,public,noworkareas,deliverables]{euproposal} % for public version

\usepackage[utf8]{inputenc}
\usepackage{hyperref}
\usepackage{enumitem}
\usepackage{booktabs}

% \usepackage{minitoc}
%\usepackage{varioref}

\usepackage{float}  % used to suppress floating of tables in Resources section.
\usetikzlibrary{calc,fit,positioning,shapes,arrows,snakes}
\graphicspath{{tasks/}}

\addbibresource{bibliography.bib}
% temporary fix due to http://tex.stackexchange.com/questions/311426/bibliography-error-use-of-blxbblverbaddi-doesnt-match-its-definition-ve
\makeatletter\def\blx@maxline{77}\makeatother

% %%% institutions
\WAinstitution[id=SRL,
        countryshort=NO,
        acronym=Simula]
        {Simula Research Laboratory}

\WAinstitution[id=UPSUD,
        countryshort=XX,
        acronym=XXX]
        {XXX - Replaced}

\WAinstitution[id=MP,
        countryshort=DE,
        acronym=MaxPlanck]
        {Max Planck Institute}

\WAinstitution[id=XFEL,
        countryshort=XX,
        acronym=XXX]
        {XXX - Replaced}

\WAinstitution[id=QS,
        countryshort=FR,
        acronym=QuantStack]
        {QuantStack}

\WAinstitution[id=INSERM,
        countryshort=XX,
        acronym=XXX]
        {XXX - Replaced}

\WAinstitution[id=SIL,
        countryshort=XX,
        acronym=XXX]
        {XXX - Replaced}

\WAinstitution[id=WTT,
        countryshort=XX,
        acronym=XXX]
        {XXX - Replaced}

\WAinstitution[id=UIO,
        countryshort=NO,
        acronym=UiO]
        {University of Oslo}

\WAinstitution[id=EGI,
        countryshort=XX,
        acronym=XXX]
        {XXX - Replaced}

\WAinstitution[id=CDS,
        countryshort=XX,
        acronym=XXX]
        {XXX - Replaced}

\WAinstitution[id=EP,
        countryshort=XX,
        acronym=XXX]
        {XXX - Replaced}



% \WAinstitution[id=PD,
%         countryshort=CH,
%         acronym=PersonalData]
%         {PersonalData.io}

\WAperson[id=minrk,
           personaltitle=Dr. ,
           birthdate=9 Oct. 1984,
           academictitle=Senior Research Engineer,
           affiliation=SRL,
           department=Numerical Analysis and Scientific Computing,
           privaddress=None of your business,
           privtel=that neither,
           email=benjaminrk@simula.no,
           workaddress={TODO: Simula, Oslo},
           %worktel=+33 6 77 90 32 79,
           % worktelfax=+33 6 77 90 32 79,
           %workfax=N/A
           ]
           {Benjamin Ragan-Kelley}

%%% Local Variables:
%%% mode: latex
%%% TeX-master: "proposal"
%%% End:

% LocalWords:  WAperson miko personaltitle academictitle privaddress privtel Sud
% LocalWords:  workaddress worktel workfax gc worktelfax pcg pcsa WAinstitution
% LocalWords:  shortname partof streetaddress townzip countryshort efo 3kd89
% LocalWords:  jacobs-logo.png Seefahrtstrasse Kruislann Montparnasse Universit
% LocalWords:  baz Westerfield
 % Some sections of the included files depend on this.
\input{preamble}
\usepackage{framed}
\usepackage{multicol}
\usepackage{lipsum}

\newcommand{\allparticipants}{{SRL,MP,QS,UIO,IFR}}

\newcommand{\softwarename}[1]{\texttt{#1}}
\newcommand{\repotodocker}{\softwarename{repo2docker}}
\newcommand{\binderhub}{\softwarename{BinderHub}}
% \newcommand{\mybinder}{\softwarename{mybinder.org}} % mybinder.org in monospaced font
\newcommand{\mybinder}{mybinder.org}   % mybinder.org in normal proportional
% font
% \newcommand{\myemph}[1]{\emph{#1}}%    to try bf or emph in ambition section
\newcommand{\myemph}[1]{\textbf{#1}}%    to try bf or emph in ambition section

\newcommand{\noemph}[1]{#1}%    to switch off emphasis but keep the option to
                           %    de-activate it again later

%\newcommand*{\fullref}[1]{\ref{#1} \nameref*{#1}} % One

\newcommand*{\fullref}[1]{\hyperref[{#1}]{\ref{#1} \nameref*{#1}}}
% single clickable link of the style: 1.1 Concept
% source: https://tex.stackexchange.com/questions/121865/nameref-how-to-display-section-name-and-its-number



% longtaskref: T1.2: Title of task
\newcommand\longtaskref[2]{\csname task@#1@#2@label\endcsname: ``\csname task@#1@#2@title\endcsname''}



\begin{document}

% satisfy fancyhdr with 11pt
\setlength{\headheight}{13.6pt}

\begin{draft}

\section*{Guidelines for proposal co-authors}

\begin{verbatim}
- Consistency.
  - [ ] Section or section or Sect or Sec? Use 'Section'
  - [ ] Figure or figure or Fig or Fig.? Use 'Figure'
  - [ ] Binder or binder - Binder? Use 'Binder'
  - [ ] MyBinder or Mybinder or mybinder? Use `mybinder.org` for the public
        instance of the service -> \mybinder

- we have a command \repotodocker to insert 'repo2docker' in \softwarename{}
  font.
- we have a command \binderhub to insert 'BinderHub'.
- we have a command \mybinder to insert 'mybinder.org'.
- Best use {} after the commands to enforce a space, i.e. ``\repotodocker{} is the focus''


- to discuss
  - ONHOLD [ ] names of work packages. Section 3.1.1
    - [X] Management
    - [ ] Core  -> enhancement? Robustness?
    - [ ] Impact -> new features? Increase impact?
    - [ ] Applications -> Use cases?
    - [ ] Education -> Dissemination? Education and Outreach?

    Hm, the long names see okay. Need to investigate where I got the short names
    from.



- This nice section could be included in our approach (1.2.10):

  Facilitating Open and Reproducible Science by automating existing practices &
  A key to our philosophy and success thus far has been automating
  what scientists already are (or should be) doing.
  The approaches and environment specifications used in Binder are
  not specific to Binder, and are already widely adopted.
  We only seek to automate this process,
  and implement and document as many standards as we can find in use by the community.
  By implementing what is already in use,
  we minimise "lock-in" and meet users,
  lowering the barrier to adoption relative to "bespoke" tools,
  which require a large change in tooling, and significant disruption to researchers'
  work.

- [ ] Check size limit for pdf to upload (was it 10MB? Can be seen at upload
      menu in portal)
  - [ ] check size of final.pdf
  - [ ] if size is a problem, we can downsample the 'spectrogram.png' image.

- [ ] mention reference (discretisedfield?) for interactive documentation
      in text.


- [ ] fix XXX-XX-XXX in telefon fax

- [ ] update Keywords on title page of part B
\end{verbatim}

\section*{Todo items}
- [ ] describe our relation to the Binder team

\end{draft}
\draftpage

\begin{proposal}[
  % participants
  PI=mrk,
  mrkname=Benjamin Ragan-Kelley,
  mrkaffiliation=Simula Research Laboratory,
  mrkdept=Numerical Analysis and Scientific Computing,
  mrktitle=Dr.,
  % site descriptions
  site=SRL, % Simula
  SRLacronym=Simula,
  SRLshortname=Simula Research Laboratory,
  SRLcountryshort=NO,
  SRLcountry=Norway,
  site=MP, % Max Planck
  MPacronym=MPG,
  MPshortname=Max Planck Gesellschaft,
  MPcountryshort=DE,
  MPcountry=Germany,
  site=QS, % QuantStack
  QSacronym=QuantStack,
  QSshortname=QuantStack,
  QScountryshort=FR,
  QScountry=France,
  site=IFR, % Ifremer
  IFRacronym=Ifremer,
  IFRshortname=Ifremer,
  IFRcountryshort=FR,
  IFRcountry=France,
  site=UIO, % U Oslo
  UIOacronym=UiO,
  UIOshortname=University of Oslo,
  UIOcountryshort=NO,
  UIOcountry=Norway,
  % site=XXX, % template example
  % alternative: (can be combined)
  coordinator=Simula Research Laboratory,
  % Cemail=benjaminrk@simula.no,
  % Ctelfax=(47) XXX-XX-XXX,
  %coordinatorsite=SRL,
  acronym={SOURCE},
  acrolong={SOURCE},
  proposalnumber={SEP-210850361},
  title={Supporting Open, Useful, and Reproducible Computational Environments},
  % callname=Increasing the reproducibility of scientific results,
  % callid=WIDERA-2022-ERA-01-41,
  % TODO: consistency with provided template
  % CALL: H2020-EINFRA-2015-1
  % TOPIC: e-Infrastructures for Virtual Research Environments (VRE)
  % Instrument: e-Infrastructures
  keywords={
  Open Science,
  reproducibility,
  reusability,
  education,
  accessibility,
  Jupyter,
  Binder,
  notebooks,
  cloud,
  EOSC,
  FAIR data,
  geosciences,
  health sciences,
  photon science
  },
  % computational mathematics,
  % GAP, Linbox, PARI, Sage, Singular, IPython, Jupyter, SageMathCloud, LMFDB, MathHub
  % Virtual research environments, MPIR, /GP
  % open source, free software, number theory, abstract algebra, notebooks
  % instrument= Call: HORIZON-WIDERA-2022-ERA-01-41, %Call: H2020-EINFRA-2015-1, 3 Topic 9-2015
  % challengeid = TODO,
  %challenge = {N/A},
  %objectiveid={N/A},
  %objective = TODO,
  %outcomeid = N/A,
  %outcomet = N/A,
  months=36,
  compactht]
\newcommand{\TheProject}{\pn}% \pn is defined automatically
% \input{grantagreement-history}

\ifsubmit
\else
% only abstract in draft
\draftpage

\TODO{In this abstract we want to summarize the vision of the project;
  - Context: Open Science; what is Jupyter; Jupyter is big
  - Who are we
  - What is our goal
  - What is our strategy / concept
Points to hit:
- Open Science should be practical, not just available
- Jupyter is part of the solution
- *brief* highlight of how/why Jupyter and Binder make sense:
  - Jupyter is widely adopted
  - notebook encapsulate computation
  - Binder builds on Jupyter to enable shareable reproducible environments
  - Jupyter is web-based, enabling building services
- What we plan to do
  - improve Jupyter/Binder toward open science
  - operate Jupyter-based services on EOSC
  - Open Science training (skip?)
- Who we are
  - Core Jupyter experts
  - Domain experts motivating/validating Jupyter improvements
}


% \begin{verbatim}
% Call: Increasing the service offer of the EOSC Portal

% Title: Open Science Publication of Research Environments (OSPRE)

% Points to hit:

% - Open Science should be practical, not just available
% - Jupyter is part of the solution
% - *brief* highlight of how/why Jupyter and Binder make sense:
%   - Jupyter is widely adopted
%   - notebook encapsulate computation
%   - Binder builds on Jupyter to enable shareable reproducible environments
%   - Jupyter is web-based, enabling building services
% - What we plan to do
%   - improve Binder toward open science
%   - operate Jupyter-based services on EOSC
%   - Open Science training (skip?)
% - Who we are
%   - Core Jupyter experts
%   - Domain experts motivating/validating Jupyter improvements
% \end{verbatim}



% % (a3) prompt: Services supporting scholarly communication and open access (4M): based on existing initiatives across Europe (institutional and thematic repositories, aggregators, etc.), the services should empower researchers and research communities and initiatives with the necessary tools and functionalities for systematic publishing, analysing and re-using of scientific results beyond publications (data, software and other artefacts), as well as supporting long-term preservation and curation. The services should also enable scientific workflows with adequate metrics and monitoring mechanisms supporting career development and the monitoring of funding and research impact. Support to a catch-all repository for open research should be provided.
% 
% %  To truly achieve the societal goals of Open Science,
% %  we must make progress beyond the `mere availability' of scientific results,
% %  to the practical usability and exploitation of such data once it is made available,
% %  an area where there is much room for improvement.
% %  The Jupyter project and its ecosystem show great promise
% %  as tools for bridging this gap; for making Open Science
% %  useful and accessible to all,
% %  from researchers to educators to public citizens.
% %  The Jupyter Notebook and Jupyter ecosystem are of increasing
% %  importance in computational science, data science, academia,
% %  industry, governments, and service providers,
% %  used by millions worldwide.
% %  Jupyter notebooks have great potential to push Open Science
% %  forward because they provide a complete description of a
% %  computational study that can be turned into a publication
% %  or produce part of a publication, such as a figure,
% %  making complex tasks reproducible.
% %  The Jupyter-based Binder project adds a means to execute notebooks
% %  in specified computational environments, an aspect of reproducibility
% %  not yet widely supported.
% %  In \TheProject, we will extend the capabilities of the Jupyter
% %  tools and ecosystem to add functionality that we view as having great
% %  importance for EOSC and Open Science more
% %  widely and operate services on EOSC as a demonstration.
% %
% %  Many \TheProject partners have longstanding experience and
% %  leadership roles in the Jupyter ecosystem,
% %  and in deploying services built on Jupyter to many users across the globe.
% %  Complementary to this core expertise,
% %  we integrate partners focussing on the application of these tools from a wide range of disciplines,
% %  both to demonstrate and ensure that our developments serve
% %  real-world Open Science use cases.
% %
% 
%   To truly achieve the societal goals of Open Science, we must make progress
%   beyond the `mere availability' of scientific results as Open Access, to the
%   practical usability and exploitation of such artefacts once they are made
%   available, an area where there is much room for improvement. The Jupyter
%   ecosystem shows great promise as a collection of tools for bridging this gap;
%   for making Open Science useful and accessible to all, empowering researchers,
%   educators, and public citizens. Jupyter is of increasing importance in
%   computational science, data science, academia, industry, governments, and
%   service providers, and used by millions worldwide. Jupyter notebooks have
%   great potential to push Open Science beyond publications because they
%   encapsulate a computational study that may be part of a publication, such as
%   the creation of a figure, a major part of making complex tasks reproducible.
%   The Jupyter-based Binder project adds a means to execute notebooks in
%   specified computational environments, an aspect of reproducibility not yet
%   widely supported, and of great falue to re-using scientific results.
% 
%   Tools such as Jupyter and Binder increase the value of all existing Open
%   Access initiatives by adding the axis of interactive computability, empowering
%   researchers to produce derivative and validating (or refuting) work. Services
%   such as Binder also xpose public metrics and monitoring, supporting the
%   monitoring of research impact and career development for any users of the
%   system.
% 
%   We will (i) extend the capabilities of the Jupyter tools and ecosystem to add
%   functionality that we view as essential to and providing great value for EOSC
%   and Open Science, focused on accessibility, interactive publications, and
%   reproducibility. Based on this framework of improved Jupyter tools, it will be
%   possible to build Open Science Publication of Research Environments
%   (\TheProject), and (ii) build a range of diverse innovative open services on
%   EOSC as part of this project, both to demonstrate and ensure that our
%   developments serve real-world Open Science use cases.
% 
%   Many \TheProject partners have longstanding experience and leadership roles in
%   the Jupyter ecosystem, and in deploying services built on Jupyter to many
%   users across the globe. Complementary to this core expertise, we integrate
%   partners focusing on the application of these tools to a wide range of
%   scientific disciplines and communities, for which EOSC-hosted demonstrator
%   services are developed.

\begin{abstract}
  \TODO{Comment on Binder being a generic tool -- not limited to Jupyter}

  \TODO{Comment on Best practice guidelines and Training we deliver}
  
To increase the reproducibility of scientific results, we need find technical
solutions that enable reproducibility, and which at the same time are practical
and acceptable to the researchers who (could) use them.

The Jupyter notebook and ecosystem offers high potential for reproducibility:
(i) it is used widely, (ii) the notebook is a document that includes the
computation, and (iii) the notebook is machine executable.

Reproducibility enabled through Jupyter notebooks can have high impact due to
the wide spread use of notebooks: the notebook provides an environment for
scientific data exploration and thinking that is embraced and adopted by many.

To be able to reproduce results from notebooks, the software environment in
which the notebook should be executed needs to be created. This project is
about automating and improving ways to do this by improving and extending the
Binder project.

The Binder project automates the creation of software environments in which
notebooks can be executed, based on software specification standards that
already exist.

In this project, we will (i) investigate which fraction of existing Jupyter
notebooks are reproducible. Such reproducibility either comes from the efforts
of pioneers in the field to make their repositories 'Binder-enabled' using the
already existing Binder software. Or the reproducibility of the software environment is
provided because the authors of the notebook deposited software specifications
in a standard way with the notebook, which the binder tool correctly interprets.

We will (ii) extend the robustness of the applicability of the
Binder tools. For example, authors may not specify which version of
software they have used. Using the commit dates in the repository, an
improved Binder can attempt to identify a working version. We will
(iii) extend Binder's capabilities to broaden the applicability and
increase the positive impact on reproducibility in science. 

We will evaluate our efforts with selected and new use cases, such as
data publishing and use of Binder in HPC contexts, and invite the
community of researchers to contribute to the project. 

Many \TheProject individuals have longstanding experience and leadership roles in the
development of the Jupyter ecosystem, and in deploying services built on
Jupyter to many users across the globe. Complementary to this core expertise, we
integrate partners focusing on the application of these tools to a range of
scientific disciplines and communities.

All outcomes from this project -- including improved and new software, best
practice guides, application examples and training materials -- will be made
available as open source. 
\end{abstract}

%%% Local Variables:
%%% mode: latex
%%% TeX-master: "proposal"
%%% End:


% detailed toc in draft
\setcounter{tocdepth}{4}
\fi

\tableofcontents

% ---------------------------------------------------------------------------
%  Section 1: Excellence
% ---------------------------------------------------------------------------

\section{Excellence}
\eucommentary{4 pages}
\eucommentary{\emph{
\begin{itemize}
\item Briefly describe the objectives of your proposed work. Why are they pertinent to the work programme topic? Are they measurable and verifiable? Are they realistically achievable?
\item	Describe how your project goes beyond the state-of-the-art, and the extent the proposed work is ambitious. Indicate any exceptional ground-breaking R-and-I, novel concepts and approaches, new products, services or business and organisational models. Where relevant, illustrate the advance by referring to products and services already available on the market. Refer to any patent or publication search carried out.
\item	Describe where the proposed work is positioned in terms of R-and-I maturity (i.e. where it is situated in the spectrum from idea to application, or from lab to market). Where applicable, provide an indication of the Technology Readiness Level, if possible distinguishing the start and by the end of the project.
\end{itemize}
Please bear in mind that advances beyond the state of the art must be interpreted in the light of the positioning of the project. Expectations will not be the same for RIAs at lower TRL, compared with Innovation Actions at high TRLs.
}
}
\medskip

\subsection{Objectives and ambition}

\subsubsection{Ambition}

\TheProject's ambition is to \myemph{improve the global reproducibility of
  scientific results} with focus on those aspects of the research process that
are supported by computation and software, such as computer simulation, data
processing, data analysis and creation of figures and tables in publications.

We want to achieve this ambition through
\begin{compactitem}
\item educating researchers about good reproducible practices, and
\item make it easier to perform computational research in a reproducible way
  through improving and developing relevant software tools.
\end{compactitem}

\subsubsection{State of the Art}

To make computational research reproducible, we generally need to archive
(i)~required data, (ii)~the required software, (iii)~the protocol that explains
how to process the data to obtain the result that is to be reproduced. Using
services such as Zenodo, it is possible to deposit such archives with a DOI and
make reference to them in publications.

In order to reproduce the results, it should be possible for anybody to take such an archive of a research
output, and to carry out the two necessary steps:
\begin{compactitem}
\item Step 1 to install the required software environment, and
\item Step 2 to follow the protocol to reproduce the results from the archived data.
\end{compactitem}

It is of particular value if Steps 1 \emph{and} 2 can be \emph{carried out
  automatically} by executing some kind of script or program that is part of the
archive. First, if the automatic execution is possible, we know that there is a
complete description of the protocol included in the archive, and that no
mistake is made in trying to follow the protocol. Second, the automatic
execution saves time.

There are two approaches emerging to achieve this reproducibility: (a) to use a
workflow tool or environment that caters to a given use case (for example
~\cite{reana2019} \TODO{cite other workflow systems}). Or (b) to use standard
(software engineering) computing tools and conventions (git, make, python, perl,
bash, \ldots) to specify the compute environment and reproduce steps piecemeal.

The workflow tool approach (a) is more robust, but requires `all-in' adoption by authors and reproducers alike,
and may not suit all use cases. It can also have associated `vendor lock-in' effects
- once a tool is adopted for one piece, it must be used for all associated work.
The standard computing tools approach (b) is generic, but not accessible to all researchers
 as it requires sufficient training or experience to be effective.
 The loose coupling of tools and modular choices make it more flexible (covering more use cases),
 but more difficult to follow robustly.

\medskip Researchers who use the Jupyter Notebook to orchestrate their
computational research (see Figure~\ref{fig:jeodpp}) can achieve this automatic
reproducibility with little additional effort~\cite{Beg2021}: they use the Notebook document as
the protocol of their analysis (Step 2), which can be executed automatically.
They can make use of the Binder tool (Section~\ref{seq:project-binder}) and the
associated mybinder.org service that has
been designed by the Jupyter team to automatically create the appropriate
software environment (step 1) in which the notebook can be executed.

\begin{figure}[tb]
  \centering\includegraphics[height=0.2\textheight]{images/jeodpp.png}
  \centering\includegraphics[height=0.2\textheight]{images/jeodpp-demo.jpg}
  \caption{\emph{Left}: The Joint Research Centre (JRC) Earth Observation
    Data and Processing Platform (JEODPP) is a user of the
    Jupyter Notebook (source:
    \url{https://cidportal.jrc.ec.europa.eu/home/}), where it features
    at the top of the pyramid to help users orchestrate layers of data
    analysis software and hardware. \emph{Right}: An example
    service in which an interactive visualisation is provided through
    the Jupyter Notebook rendering of the density map of the ships
    detected from Sentinel-1 images over the Mediterranean sea during
    the period October 2014 to September 2016. \cite[Figure
    6]{Soille2018}. \label{fig:jeodpp}}
  \TODO{Do we want to keep the figure? The left side is useful, and it is good
    to have some images.}
\end{figure}

\subsubsection{Beyond state of the art}

In this project, we will focus on the \emph{reproduction of the
  software environment} (Step 1) which is a prerequisite for any attempt to
reproduce the actual research outputs. In particular, we want to make the
creation of this computational environment \emph{automatic}.

We will go beyond the current state of the art by
\begin{compactitem}
\item enhancing the existing Binder software, already widely used by authors of Jupyter notebooks,
  directly improving the reproducibility of research created by the substantial
  community of researchers using such notebooks,
\item extending the capabilities of the Binder tools so that its capability to
  create arbitrary software environments automatically can be realised more broadly beyond the
  Jupyter user community,
\item adding new capabilities to the Binder tools that enable new reproducibility
  use cases, such as those that need access to large data sets, access to restricted sets,
  and reproducibility for High Performance Computing,
\item enabling the Binder tools to run on the desktop computer of individual
  researchers (rather than having to rely on central or institutional
  installations such as mybinder.org).
\end{compactitem}

\subsubsection{Motivation - Why?}\label{sec:motivation-why}

We focus on the computational reproducibility because it is a real
obstacle for practical reproducibility.

First, it affects the majority of all researchers: there are estimates that over 92\%
of all researchers work with \emph{research software} and over 50\% develop
their own~\cite{Hettrick2014}. Where experiments drive the research, this is
often data processing, analysis, and plotting. Each of those computational
research cases needs a software environment in which the actual processing can
be carried out. The software environment may consist of somewhat standard packages (for example
use of a Python, R, or Julia plotting library), or it may include tailored
programs that have been developed especially for the study.

Second, software packaging and management is a technically challenging topic,
and we cannot expect 92\% of all researchers to master it -- so we believe there
is a clear need to support this with appropriate tooling.

The complexity arises in parts from the increasing age of archived studies, and
also in the often unusual combinations of research software and libraries that
need to be combined for a particular study. Other difficulties include that a
reproduction typically needs to be done on a different computer, perhaps even on
a different operating system. If, say, a plotting library is used, then it may
have changed its interface or behaviour over time, so it is important to install
exactly the right version of the plotting library, before a reproduction of
results is attempted using it.

Third, being able to re-create the appropriate software environment is a
pre-requisite before any actual reproduction of results can be attempted: it
would be inefficient to educate researchers what data and programs to archive,
if in the future nobody (or only very few highly trained people) will be able to
execute those scripts.

Finally, we think that there are low hanging fruits: the work proposed here will
make it possible create computational environments automatically for
\emph{existing data archives}: the Binder philosophy is to understand existing standards for software
specification, and to build a software environment based on those standards.
Where researchers have used the existing software specification already, Binder tools
will work immediately on their archived files. This means that (i)~a researcher
putting together a well organised archive does not need to know about Binder tools,
yet the researcher who wants to reproduce the results later can use Binder tools to
automate the recreation of the software environment. This also means that
(ii)~improvements we propose in this work, will make some existing archives (that
have been created in the past) more easily reproducible.

%(It is part of our training programme to educate about the importance and methods
%for software specification.)

\subsubsection{Objectives}\label{sect:objectives}
\begin{compactenum}[\myemph{Objective} 1:]

\item \label{obj:reproducibility} \myemph{Facilitate better computational
    reproducibility and FAIR data} by
  improving the \myemph{reproducibility of computational environments}
  used for science, and facilitating \myemph{FAIR data practices}.
  We will contribute to the recording and automatic reproducibility
  of environments with Binder tools,
  and extend capabilities to better support FAIR
  data requirements. In particular, the archival of execution
  environments to support \myemph{reusability} of notebooks in the future
  needs attention. Such notebooks may, for example, be published alongside
  traditional publications to detail the computation of published data
  and figures, and address the Reusablity requirement of FAIR data.
  \TODO{This needs editing. It is too long in comparison to the other
    objectives. It is not clear to me how the FAIR data comes in here.}

\item \label{obj:broaden} \myemph{To broaden the impact} of existing Binder tools
    for reproducibility by expanding the feature set, applicable domains and use
    cases. In particular, improve its utility for creating computational environments
    outside the Jupyter ecosystem.

  \item \label{obj:demonstrators} \myemph{Provide demonstrator use cases for
      reproducible science} to validate and demonstrate the value of the work. We will apply
    improved tools and guidelines to a number of reproducible use cases in
    academic research, education, research infrastructures, and SMEs.

\item \label{obj:education} \myemph{Outreach and engagement} --- Develop best
  practice guidelines for reproducible science, and disseminate this by
  educating the research communities about reproducible practices and available
  tools for reproducible publications and policies. Reach out to scientists, and
  the wider research communities and reproducibility stakeholders to encourage
  engagement with this project.
  % and exploitation of existing tools for Reproducible and Open Science
  % for their research domains and interests.
  % Engaging a larger community will help \myemph{ensure the sustainability} of
  % the services and underlying infrastructure by distributing its
  % development, hosting, and maintenance over stakeholders from a
  % variety of institutions and backgrounds,
  % from the private sector to public research, education
  % and open government.



\end{compactenum}

\begin{table}
  \label{tab:objectives-tasks}
  \caption{
  Each objective and the tasks which further that goal.}
  \begin{tabular}{|m{.3\textwidth}|m{.7\textwidth}|}

    \hline

    \myemph{Objective} & \myemph{Tasks}
    \\\hline

    \ref{obj:reproducibility} &

    % \longtaskref{core}{maintenance}
    % \longtaskref{core}{jh-bh-conv},

    \\\hline

    \ref{obj:broaden} &

    % \longtaskref{core}{accessibility},
    % \longtaskref{core}{collaboration},
    % \longtaskref{ecosystem}{xeus-cpp},
    % \longtaskref{ecosystem}{jupyter-widgets},
    % \longtaskref{ecosystem}{teaching-tools}

    \\\hline

    \ref{obj:demonstrators} &
    % \longtaskref{applications}{astro},
    % \longtaskref{applications}{teaching},
    % \longtaskref{applications}{application-gpu},
    % \longtaskref{applications}{geoscience},
    % \longtaskref{applications}{opendose-analysis},
    % \longtaskref{applications}{math},
    % \longtaskref{applications}{reproducibility-xfel}

    \\\hline

    \ref{obj:education} &

    % \longtaskref{education}{workshops},
    % \longtaskref{education}{online-resources},
    % \longtaskref{education}{helpdesk}

    \\\hline

  \end{tabular}
  \TODO{Complete table, or delete.}
\end{table}




\subsubsection{Excellence}

We have a diverse and interdisciplinary team driving this project, in which we
bring together research software developers, researchers, research support staff,
and educators -- each world class in their domain -- with the common vision to
work towards better open source tools for better reproducibility in science.

Through our multiple and interdisciplinary roles, we see the same process of
generating reproducible research outputs from the perspective of different
stakeholders, and can propose and develop solutions that are \emph{useful and
  practical} in real-world research environments.

We extend our own experience through our wide network of collaborators and
colleagues and will -- as part of the execution of this project -- seek
constant exchange with and feedback from different additional stakeholders in
our \emph{Community Engagegment Panel} to shape the work of the
\TheProject project. As a team experienced in developing open source software,
we expect to be able to go beyond this and attract development contributions
from volunteers to support this project.

The existing Binder tools -- which are the baseline for this project --
originate from Project Jupyter. We have core Jupyter and Binder developers in
our team, and thus direct access to developer expertise and experience.

\subsubsection{Impact}

We have already discussed above that computational reproducibility affects the
majority of researchers, including those who carry out work that is not
predominantly computational.

The impact of this project will be realised through (i) the training we develop
and disseminate, and (ii) the improved Binder tools. The improved tools will
directly impact the community of researchers using Jupyter notebooks. However,
the \emph{improved functionality and applicability of the Binder tools outside Project
Jupyter} will benefit all researchers who need computational reproducibility, and
may be -- and in the long run -- more important.

\medskip

We can give some indication of the size and activity of the Jupyter notebook
community and use of the existing Binder tools: Jupyter Notebooks are used to
support research in numerous communities, including
%\begin{compactitem}
%\item
\noemph{Journalists} and practitioners of \noemph{data-driven
journalism} at the LA Times, BuzzFeed News, Columbia Journalism School
\cite{latimes-datadesk} \cite{columbia-nytimes} \cite{data-journalism};
\noemph{Data scientists} in academia, industry and services \cite{Perkel2018};
% \item
\noemph{Research institutions} such as CERN, EuXFEL, JRC, and many more,
operating institution-wide Jupyter deployment;
% \item
\noemph{Universities} using Jupyter as a teaching platform;
% \item
\noemph{Large cloud providers} building commercial products on the
top of Jupyter (Google DataLab and Colaboratory, Amazon Sagemaker, Microsoft Azure
Notebooks);
% \item
within the \noemph{European Open Science Cloud} Jupyter is used in many EOSC
projects, and a JupyterHub service is provided by the EGI foundation.
% \item
This impact was recognised by the \emph{ACM Software System Award} that was
awarded to the Jupyter team to honour \emph{"developing a software system that
had a lasting influence"} in 2017. (Prior recipients include \emph{Unix},
\emph{TCP/IP}, and the \emph{World Wide Web} \cite{acm-award}.)

There are 8 million notebooks deposited on GitHub \cite{notebookcount}, and
the size of the Notebook user base was estimated to be of the order of
millions in 2015 \cite{jupyter-grant}. We know that the Binder service for
Jupyter notebooks is used to create about 30,000 computational environments
every day (Section~\ref{sec:mybinder}) to enable execution of notebooks within
that computational environment.


%  research effectiveness for many of their millions of users.

%   Our aim is to re-use the work that has gone into the development of the Binder
%   tools, and to make this functionality for automatic creation of computational
%   environments available to researchers outside the Jupyter user community to
%   improve reproducibility at a wider scale.




\subsubsection{Technical Readiness Level (TRL)}

The Binder software and service prototype at mybinder.org is TRL 6. We will
bring Binder to at least TRL 8 during the course of the project.

\TOWRITE{}{Check TRL requirements for this call. I think this is leftover from BOSSEE}

  %%% Local Variables:
  %%% mode: latex
  %%% TeX-master: "proposal"
  %%% End:

% % \subsection{Context and motivation}


\begin{draft}

In many scientific disciplines, it is common for researchers to rely on
heterogeneous computational tools and technologies to collect data, explore the
input datasets, run simulations, visualise the outcome, and share their result
with peers or with a larger audience. Often, such data analysis cycles are
iteratively refined.

For simple datasets, processes may remain manageable. However, when dealing with
larger and more complex use cases, including big data from research facilities
or High Performance Computing resources, the complexity makes iteration cycles
slower for the researchers. A complex iteration cycle also makes research
results more difficult to reproduce. Results that cannot be reproduced make
research ineffective: they create barriers towards re-using the results in
future research work, a critical aspect of open science. This situation is
exacerbated by the current and accelerating increase of the amount of scientific
data being available, including the data becoming accessible through the
EOSC-Hub. But this growing availability of data also provides a massive
opportunity for open science.

Project Jupyter has developed as one piece of various solutions to the data
deluge, by enabling the construction of computational services accessible from
anywhere, any web-browser-enabled device, with access to any data. Jupyter-based
tools such as \href{https://mybinder.org}{Binder} and repo2docker show great
promise for enabling researchers to better perform \textbf{reproducible and open
  science}. Jupyter was recognised for its contribution to data analysis in
research with the prestigious 2017 \emph{ACM Software System Award}, of which
previous winners include TCP/IP, UNIX, and the World Wide Web. It is widely used
today in research, education, and industry. We will build on these tools, both
improving their capabilities and expanding their accessibility to new
communities, both academic and demographic, in order to \textbf{further the
  mission of open science}.

In this proposal, core team members of Jupyter projects -- including
recipients of the \emph{ACM Software System Award} -- and key contributors to
the open source scientific computing ecosystem, detail improvements to the
capabilities of Project Jupyter-related tools to \textbf{facilitate reproducible science}.
By collaborating with a wide variety
of stakeholders from diverse scientific and educational domains, we aim to
demonstrate and ensure that such innovative open source tools -- built on Project
Jupyter -- are feasible, valuable, and effective in furthering reproducible and open science. The
goal is to \textbf{improve the accessibility, interactivity,
  reproducibility, and re-usability of computational research and open science.}

\end{draft}

\draftpage

%%% Local Variables:
%%% mode: latex
%%% TeX-master: "proposal"
%%% End:

% \subsection{Objectives and ambition}
% \input{ambition.tex}
% \draftpage
% 

\subsubsection{Objectives}\label{sect:objectives}

% \noindent The aims of \TheProject are to:

% \TOWRITE{remove aims}
% \begin{compactenum}[\myemph{Aim} 1:]
% \item Enable a
%   \myemph{sustainable}, \myemph{community-developed}, general purpose, \myemph{interoperable} toolbox for
%   interactive computing, data processing, and visualization
%   \myemph{that facilitates the entire life-cycle of open science},
%   from initial exploration to \myemph{reproducible publication}, research and development in
%   industry, teaching, and outreach.
%   This is by supporting and steering the Jupyter software ecosystem,
%   which exists to develop open source software,
%   open standards, and services for interactive computing across dozens of programming languages.
%
% \item Leverage this technology for all scientists, across borders,
%   domains, disciplines, and demographics, through
%   \myemph{free public distributed collaborative services} tightly integrated
%   into the European Open Science Cloud (EOSC),
%   in collaboration with a federation of related services
%   operated by the wider community.
%
% \item Demonstrate the value and versatility of such services through
%   \myemph{innovative co-designed tailored applications} in a variety of disciplines and
%   contexts.
%
% \item \myemph{Support open science} and maximize impact through development and
%   dissemination of best practices,
%   \myemph{training}, and \myemph{community building}
%   around the usage and development of the above toolbox,
%   with a focus on \myemph{interoperability}, \myemph{reproducibility}
%   and \myemph{reusability}.
% \end{compactenum}
% develop and support the Jupyter ecosystem in a direction that benefits and facilitates open science
%     make these tools accessible to as many people as possible via operation of free, public services
%     demonstrate and ensure that these developments are useful to real scientists and the public
%     foster open science through training of students and researchers in best practices using these tools


% \item \label{aim:facilitation}
%   Facilitate open science through the development
%   of tools enabling reproducibility, sharing, and collaboration.

% \item \label{aim:accessibility}
%   Maximise accessibility and interoperability of open science services and tools,
%   across domains, disciplines, and demographics.

% \item \label{aim:sustainability}
%   Maximise sustainability of software tools for open science
%   by developing the community and contributing
%   to and supporting community-led software efforts.



%   Support open source software for open science, and notably the
%   Jupyter ecosystem,



% \end{compactenum}

\medskip
\noindent We will achieve our ambition through the following objectives:

\TOWRITE{mention TRL goals}

\begin{compactenum}[\myemph{Objective} 1:]

\item \label{obj:reproducibility} \myemph{Facilitate reproducibility and FAIR data} ---
  improving the \myemph{reproducibility of computational environments}
  used for science, and facilitating \myemph{FAIR data practices}.
  We will contribute to the recording and reproducibility
  of environments with repo2docker and Binder,
  and extend capabilities to better support FAIR
  data requirements. In particular, the archival of execution
  environments to support \myemph{reusability} of notebooks in the future
  needs attention. Such notebooks may, for example, be published alongside
  traditional publications to detail the computation of published data
  and figures, and address the Reusablity requirement of FAIR data.

\item \label{obj:broaden} broaden the impact of existing tools for reproducibility by expanding the applicable domains and use cases
\item \label{obj:demonstrators}
  \myemph{Demonstrators in science and education} ---
  We will demonstrate and ensure the versatility and value of the components and
  the services built from them,
  through applications to a number of
  domains in academic research, education, research infrastructures, SMEs, and for
  the public sector, driven through our project partners. In
  particular, we will contribute demonstrators in the following areas:
  % astronomy (\taskref{applications}{astro}), education
  % (\taskref{applications}{teaching}), fluid dynamics
  % (\taskref{applications}{application-gpu}), geosciences
  % (\taskref{applications}{geoscience}), health
  % (\taskref{applications}{opendose-analysis}), mathematics
  % (\taskref{applications}{math}),
  % and photon science (\taskref{applications}{reproducibility-xfel}),
  % involving universities, research infrastructure facilities, and SMEs.

\item \label{obj:education}
  \myemph{Outreach, engagement, and sustainability} ---
  Reach out to scientists and the wider research
  communities to encourage engagement
  and exploitation of existing tools for reproducible and open science
  for their research domains and interests.
  Educate the research communities about reproducible practices,
  and available tools for reproducible publications and policies.
  Engaging a larger community will help \myemph{ensure the sustainability} of
  the services and underlying infrastructure by distributing its
  development, hosting, and maintenance over stakeholders from a
  variety of institutions and backgrounds,
  from the private sector to public research, education
  and open government.



\end{compactenum}

\begin{table}
  \label{tab:objectives-tasks}
  \caption{
  Each objective and the tasks which further that goal.}
  \begin{tabular}{|m{.3\textwidth}|m{.7\textwidth}|}

    \hline

    \myemph{Objective} & \myemph{Tasks}
    \\\hline

    \ref{obj:reproducibility} &

    % \longtaskref{core}{maintenance}
    % \longtaskref{core}{jh-bh-conv},

    \\\hline

    \ref{obj:broaden} &

    % \longtaskref{core}{accessibility},
    % \longtaskref{core}{collaboration},
    % \longtaskref{ecosystem}{xeus-cpp},
    % \longtaskref{ecosystem}{jupyter-widgets},
    % \longtaskref{ecosystem}{teaching-tools}

    \\\hline

    \ref{obj:demonstrators} &
    % \longtaskref{applications}{astro},
    % \longtaskref{applications}{teaching},
    % \longtaskref{applications}{application-gpu},
    % \longtaskref{applications}{geoscience},
    % \longtaskref{applications}{opendose-analysis},
    % \longtaskref{applications}{math},
    % \longtaskref{applications}{reproducibility-xfel}

    \\\hline

    \ref{obj:education} &

    % \longtaskref{education}{workshops},
    % \longtaskref{education}{online-resources},
    % \longtaskref{education}{helpdesk}

    \\\hline

  \end{tabular}
  \TODO{HF: Is this table compulsory?} 
\end{table}

% \draftpage
% % Requirements to address here:
%
% Topic:
% This topic aims to fund activities to
% a) determine how increased reproducibility generates gains and savings in the R&I process and improve overall performance - alongside the demonstrated positive effects on their quality, integrity and trust-worthiness, and
% b) find, experiment and mainstream concrete solutions and best-practices to increase the reproducibility of research funded with European taxpayers money, including through the more systematic integration of sex and gender as variables whenever relevant.
% Consequently, actions should help understand and promote reproducibility by:
% 1) creating an open knowledge base of results, methodologies and interventions on the drivers and consequences of reproducibility for the R\&I system; and to fill the main gaps in such knowledge;
% 2) develop, validate, pilot and deploy practices and practical tools for funders, publishers and scientists;
% 3) promote uptake, greater collaboration, and increased alignment of the activities of stakeholders - scientific and technical communities, publishers and funders among others - to increase reproducibility.

\label{sect:workprogramme}
\subsubsection{Relation to the Work Programme}

\TOWRITE{}{old from BOSSEE}
The \TheProject project addresses the challenges of the
``Increasing the reproducibility of scientific results'' call (ID: HORIZON-WIDERA-2022-ERA-01-41).

Our strategy is based on taking the increasingly popular Jupyter
Notebook and Jupyter Ecosystem: we want to evolve and improve them
so that new innovative services based of the Jupyter tools can be developed for
the EOSC.
\medskip


There is evidence that the Jupyter Notebook is an e-infrastructure
that is useful across many domains: it is already widely adopted in
numerous communities and used by millions of researchers and educators worldwide
\cite{jupyter-grant}.

\begin{itemize}
\item \emph{Journalists} and practitioners of \emph{data-driven
    journalism} at the LA Times, BuzzFeed News, Columbia Journalism School \cite{latimes-datadesk} \cite{columbia-nytimes} \cite{data-journalism},
\item \emph{Research institutions} such as CERN, JRC, and many more,
  operating institution-wide Jupyter deployment,
\item \emph{Universities} using Jupyter as a teaching platform,
\item \emph{Large cloud providers} building commercial products on the
  top of Jupyter (Google DataLab and Colaboratory, Amazon Sagemaker, Microsoft Azure
  Notebooks),
\item \emph{Other EOSC projects}. Jupyter is already planned to become
  an important service on the European Open Science Cloud (for example
  the EOSC-04-funded PaNOSC project \cite{panosc}).
\item \emph{Data scientists}: some argue that the Jupyter Notebook is
  \emph{the} tool of choice for data scientists across domains
  \cite{Perkel2018}.
\item Over 3 million notebooks are deposited on GitHub \cite{notebookcount}.
\end{itemize}
%
\begin{figure}[tb]
  \centering\includegraphics[height=0.2\textheight]{images/jeodpp.png}
  \centering\includegraphics[height=0.2\textheight]{images/jeodpp-demo.jpg}
  \caption{\emph{Left}: The Joint Research Centre (JRC) Earth Observation
    Data and Processing Platform (JEODPP) is a heavy user of the
    Jupyter Notebook (source:
    \url{https://cidportal.jrc.ec.europa.eu/home/}), where it features
    at the top of the pyramid to help users with interactive data
    visualisation and analysis. \emph{Right}: An example
    service in which an interactive visualisation is provided through
    the Jupyter Notebook rendering of the density map of the ships
    detected from Sentinel-1 images over the Mediterranean sea during
    the period October 2014 to September 2016. \cite[Figure
    6]{Soille2018}. \label{fig:jeodpp}}
\end{figure}
%
A particular example is the Joint Research Centre Earth Observation
Data and Processing Platform (JEODPP) shown in Fig.~\ref{fig:jeodpp},
illustrating the interactive data exploration within an environment
that allows to save and communicate the data exploration
conveniently. These projects are building upon Jupyter as it is
available at the moment.
\bigskip

\eucommentary{
a) determine how increased reproducibility generates gains and savings in the R\&I process and improve overall performance - alongside the demonstrated positive effects on their quality, integrity and trust-worthiness, and
}

\eucommentary{
b) find, experiment and mainstream concrete solutions and best-practices to increase the reproducibility of research funded with European taxpayers' money, including through the more systematic integration of sex and gender as variables whenever relevant.
}

\eucommentary{
Consequently, actions should help understand and promote reproducibility by:
1) creating an open knowledge base of results, methodologies and interventions on the drivers and consequences of reproducibility for the R\&I system; and to fill the main gaps in such knowledge;
}

\eucommentary{
2) develop, validate, pilot and deploy practices and practical tools for funders, publishers and scientists;
}

\eucommentary{
3) promote uptake, greater collaboration, and increased alignment of the activities of stakeholders - scientific and technical communities, publishers and funders among others - to increase reproducibility.
}

%
% % Lots of EOSC and EU-funded projects are built upon jupyter
% %
% % Opendreamkit
% % PaNOSC
% % JEODPP https://www.sciencedirect.com/science/article/pii/S0167739X1730078X?via%3Dihub
% % EOSC-Pilot
% % EGI https://ec.europa.eu/info/funding-tenders/opportunities/portal/screen/opportunities/topic-details/infraeosc-02-2019;freeTextSearchKeyword=innovative;typeCodes=0,1;statusCodes=31094501,31094502;programCode=null;programDivisionCode=null;focusAreaCode=null;crossCuttingPriorityCode=null;callCode=Default;sortQuery=openingDate;orderBy=asc;onlyTenders=false
% %
% % Jupyter is a critical piece of European e-infrastructure; this project is important for sustainability, we need not just to build upon Jupyter but to consolidate the foundations.
% %
% % We also want to enable novel use cases to enable advances in European computational and data science activities that build on the Jupyter ecosystem.
% %
% % Who is better placed than the team who built Jupyter in the first place to move Jupyter forward?
% %
% % JEODPP Image (jeodpp_new_small_4.png)
% %
% %
% % The main challenge we need to address is “Develop an agile, fit-for-purpose and sustainable service offering accessible through the EOSC hub that can satisfy the evolving needs of the scientific community by stimulating the design and prototyping of novel innovative digital services. Innovative models of collaboration that genuinely include incentive mechanisms for a user oriented open science approach should be considered.” (from Specific challenge in: https://ec.europa.eu/info/funding-tenders/opportunities/portal/screen/opportunities/topic-details/infraeosc-02-2019;freeTextSearchKeyword=innovative;typeCodes=0,1;statusCodes=31094501,31094502;programCode=null;programDivisionCode=null;focusAreaCode=null;crossCuttingPriorityCode=null;callCode=Default;sortQuery=openingDate;orderBy=asc;onlyTenders=false)
% %
% % We need to make sure to either put this phrase in and respond to how we address it, or drop the right keywords.
% %
% %
% % \TODO{We should also go through the requirements from the call [1] and
% %   show how we address those [to provide easily accessible evidence
% %   that we are addressing the call].}
% %


% ---------------------------------------------------------------------------
%  Section 1.2: Methodology
% ---------------------------------------------------------------------------
\draftpage
\TOWRITE{NT/...}{Finalise}
\TOWRITE{ALL}{Proofread concept and approach pass 2}

\subsection{Concept and Methodology}\label{sec:concept_methodology}
\eucommentary{5-8 pages}

\TODO{Check we have addressed all mandatory topics}

% \subsubsection{Concept}%\label{sec:concept}
% 
% Open Science is the principle that science, in order to be most
% {impactful} and {socially responsible}, should be done
% {publicly}, with as much of the scientific process and products
% accessible, reviewable, reproducible and reusable by as many members of
% the global community as possible.
% 
% There are exciting opportunities for Open Science for almost all academic fields
% in the modern age of computational science. As more and more research takes the
% form of code and/or data, the opportunity to share, reproduce, and reuse
% scientific work is greater than ever, even enabling new forms of
% {interdisciplinary collaboration}, and interoperable and re-usable results and
% tools.
% 
%  Simultaneously, there are obstacles -- both technical and social -- to
% making open science and reproducible science a practical reality. The challenges
% include: If a researcher has code and/or data to publish, how is that best
% done? How do researchers learn {best practices for reproducible science} in
% their field? How do previously disconnected fields benefit from each other's
% work as the same computational challenges are faced again and again by different
% communities? How can scientists be encouraged to make their work reproducible?
% 
% These are the questions that guide the \TheProject{} project.

\subsubsection{Supporting Open, Useful, and Reproducible Computational
  Environments}\label{sec:SOURCE}

Our project is titled ``\emph{Supporting Open, Useful, and Reproducible Computational Environments.}'':
\begin{itemize}
\item The work done is this project will be \emph{Supporting} scientists in their
  endavours to make their work more reproducible and re-usable.
\item We believe in the value of \emph{Open} science and \emph{Open} source. The
  best reproducibility and re-usability of scientific results is given through
  complete transparency of the steps taken in the derivation of a result. For
  the computational aspects this means to make all simulation and/or
  post-processing and analysis steps open source. While this may not always be
  possible, we advocate such openness as the best practice for reproducible science.

  All work done, including software, training and documentation materials, will
  be open source and available through an open access license.

  (We note that the collective development of the grant proposal you are reading
  is also done as open source, and can be inspected at
  \url{http://github.com/minrk/horizon-widera-2022} for those interested.)
  
\item Measures towards better reproducibility have to be \emph{Useful} and
  practical: if a proposed approach or tool burdens the scientist with
  additional work, or requires significant additional skills, it becomes less
  likely to be widely accepted.

  The philosophy we support here is that the proposed (Binder) tools for
  reproducibility are based on existing standards which are already
  adopted by many and can be considered best practice.

\item Within the wide field of reproducibility in science, we focus in this
  project on the improvement of the automatic generation of \emph{Reproducible
    Computational Environments}.
\end{itemize}

\subsubsection{Outline of concept}

In the following we explain our concept and the technology on which this
proposed project builds in more detail.
\begin{itemize}
\item Sections \fullref{sec:reproducibility} and
  \fullref{sec:reproducibility-challenges} contextualise the proposed work
  within the wide field of reproducibility.
\item Section \fullref{sec:reproducibility-concept} summarises our concept and
  approach concisely.

\item To prepare the more detailed description of the concept and the content of
  the Work Packages, we use Section
  \fullref{sec:terminology-and-repository-example} to introduce key terminology
  together with an example for computational reproducibility.

\item Sections \fullref{sec:project-jupyter} and \fullref{seq:project-binder}
  introduce the Jupyter and Binder projects, respectively.

\item Section \fullref{sec:binder-for-reproducibility} explains the use of
  Binder tools for reproducible science.

\item The methodology for this project is summarised in section
  \fullref{sec:methodology}.

\end{itemize}

%\TODO{Do we need to keep anything from the commented out lines here?}

% 
% With so much research being done that wants to be Open and Reproducible,
% how can we make Science
% 
% \begin{enumerate}
%     \item as \textbf{easy} as possible to share and reproduce?
%     \item as \textbf{useful} as possible to other researchers and the public?
% \end{enumerate}
% 
% 
% 
% 
% 
% \noindent Our plan for \textbf{increasing the reproducibility of scientific results} can be summarised as:
% 
% \begin{enumerate}
% \item improve and maintain \textbf{common software infrastructure} used for
%   reproducing computational results,
% \item develop the Jupyter ecosystem to improve capabilities to \textbf{better
%   serve Reproducible Open Science},
% \item \textbf{guide, validate, and demonstrate} our developments through
%   collaboration with a wide variety of application domains,
% \item enable students and researchers to perform Reproducible Open Science through
%   \textbf{training and education}, and improving inclusiveness by focusing
%   these on under-served and under-represented communities
% \end{enumerate}

\medskip

\subsubsection{Reproducibility}\label{sec:concept}\label{sec:reproducibility}

Before describing the focus of the work that we propose here, we want to embed
this into the much wider context of reproducibility challenges.

We will exclude the challenges of reproducing \emph{experimental} data. Our
study starts at the point where such experimental data is available in digital
form.

We will focus on the challenge of computational reproducibility: can we carry out
the same data analysis, creation of figures and tables as they are presented in
a paper, at a later stage, and get to the same results?

Such tables and figures in a publication may be computed from the analysis of
some type of raw data which could originate an experiment, another publication,
a data base, post-processing of another data set or from executing computer
simulations.

% Where additional software, such as analysis scripts, input files
% and software for the simulation are needed, 

\subsubsection{Challenges of Reproducibility}\label{sec:reproducibility-challenges}

The challenges of such ``computational reproducibility'' include:
\begin{itemize}
\item Are the different processing steps for that data recorded? This could be
the order in which analysis scripts need to be executed -- for example to
compute intermediate results -- which will be turned into a figure in the last
step?

We will call this sequence of steps the \emph{workflow}. This workflow could be
archived -- for example -- through a \softwarename{README.txt} file, or scanned
pages of a hand-written laboratory notebook as a pdf file, or as a
machine-executable script (or a Jupyter notebook).

This is particularly challenging where software is used which can only be
controlled via a Graphical User Interface, as it may require manual recording
and description of the different clicks and steps in laboratory logbook.

\item Are all the scripts and configuration files (and more generally all
software) that is needed in this process known and archived? 

\item Where software is involved, have we recorded which version of that
software is needed (or was used)? If compilation is required, do we know which
compilers (and which version) and which additional dependencies are required?

\item Are there instructions how to obtain / compile the required dependencies,
and the software itself (in particular where this is about simulation based
science or more complex analysis and interpretation software tools)?

\item Where raw data is required, is this archived, accessible, and sufficiently
documented that the format is understandable?
\end{itemize}


\subsubsection{Reproducibility concept}\label{sec:reproducibility-concept}

We can classify the reproducibility challenges listed above into different categories:

\begin{description}
\item[1. Workflow]: Are the processing commands (and their order)
correctly recorded? Do we know which part of the data set the analysis is meant
to be applied to? This is to a significant degree a question of the organisation
and documentation of the research process.

\item[2. Software environment]: Can we recreate the software environment that is
required to execute these commands?

\item[3. Importance]: Is the researcher convinced that investing effort into making
their work more reproducible is a worth while investment? This is a wide topic,
touching on expectations, existing cultures, lack of metrics that acknowledge
reproducibility efforts, and policies.

\item[4. Other]: There are other related topics, for example the challenge of
archival of (large) research data sets, of making the data FAIR, and the (for
some domains important) bit-wise reproducibility.
\end{description}

In this proposal, we start from practices that researchers increasingly adopt,
and which we argue are \emph{good reproducibility practices}. We propose to carry
out additional work to \emph{improve the toolset enabling this practice}.

To deal with the \emph{Workflow} challenges, we recommend to automate the
workflow steps as much as possible. In particular, the use of Jupyter Notebooks
to orchestrate the execution of commands seems effective~\cite{Beg2021}.
The use of the notebook is
perceived by many as an improvement of their research effectiveness because
it supports ``Thinking with Code and Data''~\cite{Granger2021}. A such, the
practice of using Notebooks (which helps improving research effectiveness) has
the very positive side effect of making the work more reproducible.


To deal with the \emph{Software environment} challenge, we recommend to follow
standard practices to describe software requirements. The \emph{focus of this
project is to extend the capabilities of the \repotodocker{} tool} to be able to
\emph{automatically create software environments} based on such software
requirement descriptions.

We can only partially address the \emph{Importance} challenge as this needs
concerted efforts from many stakeholders (such as employers of researchers,
research funders, publishers). However, we will offer training that advocates
the value of open science and that teaches existing best practice in 
effective computational science. The step from following such best practice to
making the work reproducible is -- given the Binder tools we want to develop
further here -- relatively small, or even possible without additional effort.

The \emph{Other} challenges are mostly outside the focus of this work
(although our proposal will also assist in reproducible and FAIR data
publishing, see for example Task \taskref{applications}{data-publishing}).


\subsubsection{Terminology and repository example}\label{sec:terminology-and-repository-example}

For clarity, we define a few terms that we refer to through this proposal, and
illustrate this in the context of a education example repository\footnote{
(https://github.com/fangohr/reproducibility-repository-example/}
\cite{ReproducibilityRepositoryExample2022}).

We imagine that we have a publication that contains figure
\ref{fig:reproducibility-example-covid} as a result, and we want to archive and
make available the necessary information for others to reproduce that figure.

\begin{figure}
  \centering
  \includegraphics[width=0.8\textwidth]{images/figure1.pdf}
  \caption{Figure (\softwarename{figure1.pdf}) which can be reproduced in our explanatory repository example
    \cite{ReproducibilityRepositoryExample2022}. \label{fig:reproducibility-example-covid}}
\end{figure}

\begin{description}
\item[Repository] We refer to a \emph{repository} as a collection of files.

The \emph{purpose} of the repository in our context is to archive information to make
research results reproducible.

Such a repository could be a git repository (which could be hosted on Github,
Bitbucket, Gitlab, or other services), but it could also just be a zip file of a
collection of files.

The repository could be made publicly available, for example through Zenodo,
Figshare, as an electronic supplementary to a publication, or through Github.

It is not unusual that a larger number of files might be
organised in subdirectories within the repository.

Our example repository \cite{ReproducibilityRepositoryExample2022} contains the following files:

\begin{description}
\item[\softwarename{README.md}]: an overview of the content of the repository
\item[\softwarename{figure1.ipynb}]: notebook that creates \softwarename{figure1.pdf}gs from
  the raw data. Could also be realised through a script of some kind.
\item[\softwarename{requirements.txt}]: software specification
\item[\softwarename{time\_series\_covid19\_deaths\_global.csv}]: raw data
\end{description}

\item[Script] An machine executable file (for example a bash, Python, perl, file
or similar). Such scripts can execute data processing commands, and are often
part of a repository to make the (automatic) reproduction of results possible.

In our example, the necessary steps to create the figure from the raw data are
gathered in the notebook \texttt{figure1.ipynb}. 

\item[Notebook] A Jupyter Notebook. In short, an executable document that can
  combine text, code and computation results. A Jupyter notebook can act like a
  script. The notebook is explained in Section \fullref{sec:jupyter-notebook}.

\item[Software specification] The software specification depends on the software
  tools used. In our example, the \texttt{requirements.txt} file contains
\begin{verbatim}
pandas==1.3.4
matplotlib==3.4.3
\end{verbatim}
to indicate that we need the pandas and matplotlib package with the respective
versions 1.3.4 and 3.4.3.

\item[Software environment] All the software that needs to be available and
  installed to execute the scripts and (and if desired) Jupyter notebooks.

\item[Project Binder] The Project Binder is described in Section
\ref{seq:project-binder}. It is part of the Jupyter ecosystem of tools, and
allows to convert a repository with Jupyter notebooks into an browser-hosted
environment, in which the notebooks can be executed interactively (and thus
results can be reproduced). \TODO{Point to binder section}

\item[Binder tools] The Bindertools consist of BinderHub (\TODO{See XXX}) and
\repotodocker{}. \TODO{@min - how do we summarise the role of BinderHub in one sentence?}

\item[\repotodocker] \repotodocker{} is a tool that can automatically create a
\emph{software environment} (currently within a Dotcker container) in which the
notebooks and scripts of a repository can be executed. \TODO{Point to dedicated section.}
\end{description}


\subsubsection{Project Jupyter and the surrounding ecosystem}
\label{sec:project-jupyter}

\begin{figure}[htb]\centering
  \includegraphics[width=0.9\textwidth]{use-cases-binder-logbook-solution.png}
  \caption{A typical use case for Jupyter notebooks in research.
            Image by Juliette Belin for the OpenDreamKit project, used under
            CC-BY-SA.}\label{fig:use-cases-binder}
\end{figure}

\noindent\textbf{Jupyter ecosystem as the root of \TheProject}

\TheProject has chosen to centre its efforts on the Jupyter software
ecosystem, in particular Binder and repo2docker.
Figure~\ref{fig:use-cases-binder} summarises a typical use
case of Jupyter Notebook and Binder;
both are described in more detail below.

The Jupyter notebook and Jupyter ecosystem are of increasing
importance in computational science and data science, in academia,
industry, and services. In addition to supporting high productivity of
researchers, they have great potential to push Open and Reproducible Science forward:
the notebook provides a complete description of a computational and
data science study (Step 1 in figure~\ref{fig:use-cases-binder}), and the notebook can -- in principle -- be turned
into a publication, or can be used to provide the required computation
for a part of a publication, such as a figure
(Step 2 in figure~\ref{fig:use-cases-binder}). Once the researcher has
specified what software is required to execute the notebook (Step 3
in figure~\ref{fig:use-cases-binder}), the study is completely
reproducible by anyone (Step 4 in figure~\ref{fig:use-cases-binder}).

In this way, the notebook \textbf{enables reproducibility} of complex tasks
with minimal additional effort on the user side.
The Binder project allows to execute such notebooks in
tailored computational environments; an aspect of reproducibility that
is not widely supported yet,
and a great opportunity for improving best practices in Open Science.

Furthermore, for users wanting to connect
to a local Jupyter notebook server on their machine, or to connect to
a server somewhere else on the Internet, the users only need a
web-browser to display and use the notebook regardless of the location
of the notebook server,
allowing computation to run anywhere from a local laptop to a remote supercomputer or in the cloud.
Because of these characteristics,
the Notebook is already planned to become an
important service on the European Open Science Cloud (EOSC) (for
example in \cite{panosc}),
and is an ideal component to use when building Open Science Services.

\medskip\noindent\textbf{Project Jupyter}

\emph{Project Jupyter} \cite{Jupyter}, which has grown increasingly popular in the scientific
computing community, has become the \emph{lingua franca} of interactive
computing in both academia and industry. The main goal of Project Jupyter
is to provide a consistent set of tools to improve researchers'
workflows from the exploratory phase of the analysis to the communication
of the results \cite{Kluyver2016}.

Split in 2014 from the \emph{IPython Project} \cite{IPython}, Jupyter has grown rapidly in
popularity and adoption both in the industry and academia. We estimate the user
base of the Jupyter notebook to be in the millions \cite{jupyter-grant}. Users range from data
scientists to researchers, educators, and students from many fields,
including journalists and librarians. In 2017, the Jupyter
team was awarded the \emph{ACM Software System Award}, an annual award that
honors people or an organization \emph{"for developing a software system that had a
lasting influence"}. Prior recipients include \emph{Unix}, \emph{TCP/IP}, and
the \emph{World Wide Web} \cite{acm-award}.

A large number of discrete software components make up Project Jupyter.
While these interact with one another, many can be installed separately
to serve various use cases. For this proposal, we loosely divide the
software involved into \emph{Jupyter core} developed under the guidance
of the developers who started the project, and the broader \emph{Jupyter
ecosystem} including software developed by third parties,
which may interact or build upon core Jupyter components.
Some of the components and concepts important to \TheProject are detailed below.

\begin{figure}[ht]\centering
  \centering
  \includegraphics[width=0.9\textwidth]{spectrogram_smaller.png}
  \caption{A notebook document in the Jupyter Notebook interface.}\label{fig:notebook-screenshot}
\end{figure}

\medskip\noindent\emph{Jupyter core}
\begin{itemize}
  \item \label{sec:jupyter-notebook} The \textbf{Jupyter Notebook} is the flagship application of Project Jupyter.
  It allows the creation of notebook documents, containing a mixture of text and
  interactively executable code, along with rich output from running that code.
  Figure \ref{fig:notebook-screenshot} shows an open notebook including graphs
  from an audio processing example. Notebook documents are readily shareable,
  providing a popular way to describe and illustrate computational methods and
  tools. \TODO{We should update the notebook: (i) point to a github repo with it, and (ii) binder-enable it, and (iii) increase the pixel resolution of the screen shot.}
  \textbf{JupyterLab} is the new, modular, extensible client application
  for Jupyter notebooks, but the document format, server, and user model are the same.

  \item \textbf{Jupyter kernels} are the backend software which allow Jupyter to execute
  code in many different programming languages. The \textbf{IPython} kernel is
  the reference kernel, supporting the Python programming language, and is
  developed by the Jupyter core team. Kernels for other languages are maintained
  by third parties

  \item \textbf{JupyterHub} is a multi-user extension of the Jupyter Notebook.
  It runs on one or more notebook servers, for example at a research institution.
  Users can log in to author and run notebooks securely through their web
  browser, without needing to install any special software on their own
  computer.

\end{itemize}

\medskip\noindent\emph{Jupyter ecosystem}\label{jupyter-ecosystem}

While Jupyter is a large, distributed, coordinated project,
the wider community of Jupyter users develops a great deal of
software with Jupyter integration,
providing increased or domain-specific functionality,
building on top of Jupyter, or integrating core Jupyter components in some aspect.
We call this the \textbf{Jupyter ecosystem}.
The broader Jupyter ecosystem includes many more projects than we will describe
here, but a selection of projects which are relevant to
\TheProject includes:

\begin{itemize}
  \item \textbf{Binder} builds on JupyterHub to allow sharing executable
  environments along with data files and a description of the software components
  required to run the notebooks. When someone accesses a Binder repository,
  the service builds the computational environment on demand, allowing them to
  execute and modify a copy of the notebooks.
  \textbf{repo2docker} \cite{repo2docker} and \textbf{BinderHub} \cite{binder} are components of the Binder
  software. \TOWRITE{}{More here, as repo2docker is key}
\end{itemize}

\begin{figure}[ht]\centering
  \includegraphics[width=0.5\textwidth]{ipywidgets_example.png}
  \caption{An example of using two simple slider widgets to explore the
  parameter space of a function. The \texttt{@interact} decorator creates
  the widgets and connects them to the function.}
  \label{fig:ipywidgets-example}
\end{figure}


\subsubsection{Project Binder}\label{seq:project-binder}

\TOWRITE{}{Turn the following ideas into full sentences.}

\paragraph{Relationship to Jupyter}
\begin{compactitem}
\item Binder is part of Jupyter
\item Bindersubproject (within the Jupyter project) consists of
\begin{compactitem}
\item Repo2docker
\item BinderHub
\end{compactitem}
\item Binder project is operating within Jupyter ecosystem
\item But not confined to notebooks
\item Focus for this proposal is on \repotodocker{} part
\item \repotodocker{} solves software environment question in generic way
\end{compactitem}
\TODO{Add image that depicts this Jupyter/Binder relation ship}.

\paragraph{Basic functionality of Binder}
\label{binder-how-does-it-work}

\begin{compactitem}
\item Binder is \TOWRITE{}{what?} User entry point (for example on mybinder.org) is web interface \TODO{Add image of mybinder.org webpage?}
\item Binder is passed a URL from the user, for example a URL of a git repository with notebooks
\item Binder is asking \repotodocker{} to create a docker container in which the notebooks from the repository can be executed
\item \repotodocker{} searches the repository for specifications of software requirements (see \ref{repo2docker-supported-software-specifications}).
\item \repotodocker{} composes a Dockerfile that contains all the identified commands to install software
\item \repotodocker{} builds the Docker image based on the Dockerfile
\item \ldots \TOWRITE{}{Continue, and explain role of BinderHub and Kubernetes along the way}.
\end{compactitem}

\TODO{Explain: mybinder.org (service) is not binder (software)}

\paragraph{Supported software specification formats}
\label{repo2docker-supported-software-specifications}
\begin{compactitem}
\item Python packages (\softwarename{requirements.txt})
\item Conda packages (\softwarename{environment.yml})
\item \ldots
\end{compactitem}

\TOWRITE{}{}

\medskip
\subsubsection{Binder for reproducibility}\label{sec:binder-for-reproducibility}
\TOWRITE{}{}

\medskip
\noindent\textbf{Jupyter as a basis for web services}\\
Because the Jupyter notebook is a web-based application, it can be
deployed at computational facilities or in the cloud, and can function
as the basis for services exposing computational resources of all
kinds to researchers and the public.  Because Jupyter is
\textbf{interactive}, it enables making scientific results and
communications more interactive than static publications.  The
audience can follow their own initiative and ask their own questions
of published data without needing support from the publishing author,
greatly facilitating the \textbf{practicality of Open Science}.

\medskip
\noindent\textbf{Jupyter is generic}\\
\TheProject chose Jupyter because it is
Generic.  Jupyter makes no domain-specific or even language-specific
assumptions.  Any application where mixing description, code, and
results is valuable can make use of Jupyter.  This broad applicability
makes investment in the Jupyter ecosystem extremely effective, because
improvements to Jupyter can serve many communities simultaneously.

Jupyter is built from a collection of standard protocols and file
formats.  Jupyter is not just a single, monolithic piece of
software, but a description of how such software can be built.  The
result is the ability for a variety of communities and applications to
use components of Jupyter for their purposes, and/or reimplement pieces to
meet their needs.
%
For example:
\begin{enumerate}
\item The notebook file format is a well-specified JSON document,
  which can be interpreted by many systems.  This has facilitated the
  development of different services providing rendering of notebooks, e.g. the code
  hosting website GitHub, which renders notebooks for easy viewing by
  anyone, without Jupyter software.
\item The Jupyter protocol describes how execution is performed, which
  has enabled the development of over one hundred kernel
  implementations in dozens of languages\footnote{\url{https://github.com/jupyter/jupyter/wiki/Jupyter-kernels}}.
\item Output in the Jupyter protocol uses web-standard MIME types,
  enabling any possible format to be an output in a Jupyter notebook.
\item The JupyterLab extension system provides a system for building
  applications from Jupyter components and others.
\item The Jupyter Widgets provide a system for customizing and
  extending interactivity in Jupyter-based environments.
\end{enumerate}

The popularity of Jupyter, with millions of users and hundreds of open
source contributors, is an indicator of the value and impact of this approach.

\medskip
\noindent\textbf{Improvement to the Jupyter ecosystem}\\
The benefits of focusing our work on a mature system like Jupyter include:

\begin{itemize}
\item vibrant community ensures health and sustainability,
\item large existing user base maximises impact of contributions,
\item mature software ecosystem maintains quality software through
  industry standards such as version control, tests, continuous
  integration, stable release cycles, roadmaps, and user support.
\end{itemize}

The Jupyter community aims to be inclusive, and \TheProject fully
embraces and supports that approach.  Jupyter is inclusive across a number of axes.
By being applicable across numerous domains, Jupyter and \TheProject
encourage participation from individuals of various interests and
backgrounds, and has taken action to improve diversity in the project
by participating in ``Outreachy,'' a program of paid internships for
individuals from groups that face under-representation, systemic bias,
or discrimination.  Jupyter has also operated workshops focused on
training contributors from under-represented groups.  In being free,
public, open source software, Jupyter and \TheProject are accessible
to as many individuals as possible, and invites users and contributors
beyond origin, nationality, beliefs, orientation.  One area where
Jupyter has lacked in this regard is in the User Interface
accessibility, and we will help improve this in
% \taskref{core}{accessibility}
.  Additionally, the project will
focus some of its workshops in
% \taskref{education}{workshops}
on
under-represented communities.


\begin{figure}[ht!]\centering
  \includegraphics[width=0.6\textwidth]{images/notebook_components.png}
  \caption{The architecture of the Jupyter Notebook, kernels, and tools
        which operate on notebook files}
  \label{fig:notebook-architecture}
\end{figure}

\subsubsection{Related project: Mybinder}\label{sec:mybinder}

\TOWRITE{}{The MyBinder.org federation}

\subsubsection{Methodology}\label{sec:methodology}

\textbf{Improving the robustness of reproducible environments (\WPref{reproducibility})}\\
Finally, we are explicitly allocating time in \WPref{core} for maintaining
Jupyter software, as well as new development
 % (\taskref{core}{maintenance}).
Maintenance is crucial to creating reliable, sustainable software,
but its cost is often swept under the rug in funding applications
because of the perceived pressure to focus on novelty.
Being up front and explicit about this cost is critical to the sustainability
of open source open science.

\medskip
\noindent\textbf{Broadening  (\WPref{impact})}\\
In addition to improving how successfully and how often tools like \repotodocker{} reproduce environments,
we aim to broaden the impact of the tools by making them useful in more contexts.
The existing software makes certain assumptions about where it will run,
made for the purposes of limiting maintenance scope of the Binder project.
As the project has matured, certain expansions of scope are appropriate,
as seen in the existing demand for new features in the project.

We further propose improvements to the wider Binder ecosystem
to expand the impact of the tools in more contexts to be useful to more researchers
and more institutions.
In particular, we have identified planned improvements:

\begin{itemize}
  \item Binder and its crucial software component \emph{repo2docker}
  \item relaxing Kubernetes assumption
  \item running on HPC
  \item more buildpacks
    % (\taskref{ecosystem}{r2d-and-binder}).

\end{itemize}

\medskip\noindent\textbf{Beyond the improvement to \TheProject tools
  (\WPref{applications}, \WPref{education})}\\
Beyond the improvement to the Binder tools for reproducility, we plan on
\begin{itemize}
\item Design, implementation, application, demonstration and
  evaluation of multiple demonstrators, that cover research fields such as
  \TOWRITE{}{photon and neutron science, geosciences},
  and also interests of participating SMEs (\WPref{applications}).
\item Producing \emph{training and education material} to disseminate
  the ability to do reproducible computational science using the tools
  we develop, among others (\WPref{education}).
\end{itemize}

\medskip
\noindent
\textbf{The science
  demonstrators}\label{sec:science-demonstrators-in-concept}\\

We describe the context and challenges for each demonstrator in this
section. The particular planned activities are shown in the
corresponding tasks in \WPref{applications}.\\


\TOWRITE{}{

1 page: National or international research or innovation activities
½ page: bringing together expertise and methods from different disciplines
½ page: social science and humanities - how do we integrate them, or why do we not need them
1 page: gender dimension. How taken into account in the research content (not the research team). Or justify why not relevant.
1 page: open science practices and implementation
1 page: research data management and management of other research outputs. (Also FAIR)

}


% ---------------------------------------------------------------------------
%  Section 2: Impact
%  ---------------------------------------------------------------------------
\draftpage
\begin{draft}
\TOWRITE{PS (Work Package Lead)}{For WP leaders, please check the following (remove items
once completed)}
\begin{verbatim}
- [ ] discuss what the build packs are and where they go
- [ ] how we structure and budget different tasks in this WP
- [ ] needs review and more technical detail
- [ ] add deliverable(s)
\end{verbatim}
\end{draft}

\begin{workpackage}[
  id=impact,
  wphases={6-24!0.89,24-30!0.86,30-36!0.48},
  swsites,
  title=Broadening impact,
  short=Broadening impact,
  lead=SRL,
  SRLRM=24,
  MPRM=12,
  QSRM=2,
  IFRRM=4,
  %UIORM=4,
]
\begin{wpobjectives}
  This work package extends the functionality of the \TheProject tools for
  reproducibility to broaden its applicability and increase impact.
  \WPref{applications} exploits these technological advances
  in real world use cases.

  The objectives of this work package are to
 \begin{compactitem}
 \item remove the dependency of \repotodocker{} on Kubernetes
 \item support container technologies other than Docker
 \item extend the range of software specification standards that are recognised
   and supported by \repotodocker{} \TOWRITE{}{Can we give an example here?}
 \item enable access to data from data sources outside the container
 \item enable \repotodocker{} to extract and save the software installations
   instructions (independent from container generation)
 \end{compactitem}
\end{wpobjectives}

\begin{wpdescription}
One of the design decision that have led to the current Binder software stack is
to constrain the supported infrastructure to a few key components. These
include:

\begin{compactitem}
\item Software environments can only be created inside Docker container
\item To run and orchestrate (multiple) Docker containers, a Kubernetes system must
be available
\item The user must interact with Binder environments through a Binderhub
  installation
\item There is no Binders-specific provision to access (substantial) data sets
  from inside te container. \TODO{HF: What prevents us from
    a data-publishing application at the moment? How to express this without
    making the impression one couldn't use \softwarename{curl/wget} to fetch
    data, etc. }
\end{compactitem}

The benefits of such a restrictive approach are that the software development
and maintenance effort is kept small: the wider the range of supported
infrastructure that the Binder tools can be deployed on, the higher the
complexity.

At the same time, these restrictions prevent the following scenarios:
\begin{compactitem}
\item To run the Binder software on systems where Kubernetes is not available (such as
  the Desktop of a scientist). A related use case is ``Binder@home''
  \TODO{Provide link to Binder at home task in
   \WPref{applications}}.
\item To use Binder on systems where Docker cannot be used. A use case for this
  is to create reproducible environments on HPC installations. The HPC
  administrators generally avoid use of Docker for security reasons, but much
  prefer to support container technologies that can be executed without
  administrative privileges). \TODO{Point to task in applications}
\item To use Binder for software environments that use \TOWRITE{}{insert buildback
    example / community / software system}
\item To access and use significant amounts of data from inside the software
  environment is not supported. An important use case that cannot be supported due to
  this restriction is that of ``data publishing'': the idea is that a
  (potentially large and/or complex) dataset is published \emph{together} with
  software that encodes the necessary knowledge to extract meaningful data from
  that data set. A Binder environment would make this data set accessible in an
  interactive environment. \TODO{Point to data publishing use case(s)}
 \end{compactitem}

% Open source software in general, and Jupyter in particular,
% is developed not as a monolithic application,
% but rather as a collection of related components,
% which can be assembled in numerous combinations to meet diverse needs.
% The Jupyter community is no different.
% Jupyter itself is composed of several projects,
% but there are even more projects that build on top of Jupyter to create
% things like cloud services or data pipelines.
% The goal of \TheProject is to facilitate open science through Jupyter,
% and this includes working with projects all around the Jupyter ecosystem.
% We will focus this work package on developing
% Jupyter ecosystem projects with an emphasis on open science.
%
% repo2docker is a project for creating
% reproducible environments in which Jupyter notebooks (and other user interfaces) can be run.
% It reads a number of common formats to list required software packages,
% and prepares a Docker container with those packages installed.
% BinderHub is software for operating a web service using repo2docker,
% which enables sharing of interactive and reproducible Jupyter (and Rstudio) environments on the web with a single link.
% We will develop repo2docker and BinderHub further to meet the needs of the open science community.
%
% In addition to the interactive aspects of Jupyter,
% notebooks can be used in a "workflows" style,
% where job systems run analyses and produce reports,
% either on a scheduled basis or triggered by events.
% There is a great deal of interest in using notebooks in this way,
% and much room for development of tools supporting workflows in data-driven open science.



\end{wpdescription}

\begin{tasklist}
% % add tasks from task directory here
% template for a task
% each task should be added to exactly one workpackage
% in the workpackage task list
\begin{task}[
  title=Support more software specification standards,
  % task id for references
  id=buildpacks,
  % lead institution ID
  lead=SRL,
  PM=12,
  % wphases={0-36},
  % partner institution ID(s)
  % don't include lead here
  partners={MP}
  ]

  There are two different aspects of software specification that \repotodocker{}
  needs to understand:
  \begin{compactitem}
  \item the specification of the software environment, for example through
    \softwarename{requirements.txt} files, etc (see
    Section~\ref{sec:repo2docker}
    for more details on the currently
    supported standards),
  \item from where to retrieve the software itself: this will be different on a
    GitHub repository or a Zenodo archive
    (see also \ref{sec:repo2docker} for supported repositories).
  \end{compactitem}

  For both aspects, there are requests from potential Binder users to extend the
  capabilities of \repotodocker{}.

  In this task, we will prioritise such requests and extend the \repotodocker{}
  functionality to support as many new standards as possible to best meet community needs.

\end{task}

% template for a task
% each task should be added to exactly one workpackage
% in the workpackage task list
\begin{task}[
  title=Reducing technical constraints for broader usage,
  % task id for references
  id=constraints,
  % lead institution ID
  lead=SRL,
  PM=1,
  wphases={0-36},
  % partner institution ID(s)
  % don't include lead here
  partners={XXX}
]

(Abstract current technological assumptions/requirements) (year 1-3)

  \begin{compactitem}
  \item Kubernetes
  \item Docker
     % deliverable will be defined in the appropriate WorkPackage.tex
    % (\localdelivref{deliv-id})
  \end{compactitem}

  The task includes the following activities
  \begin{compactitem}
  \item ...
     % deliverable will be defined in the appropriate WorkPackage.tex
    % (\localdelivref{deliv-id})
  \end{compactitem}
\end{task}

% template for a task
% each task should be added to exactly one workpackage
% in the workpackage task list
\begin{task}[
  title=Support more use patterns,
  % task id for references
  id=patterns,
  % lead institution ID
  lead=SRL,
  PM=16,
  %wphases={0-36},
  % partner institution ID(s)
  % don't include lead here
  partners={MP,IFR}
]
In this task, we provide the technical possibilities to use the Binder tools in
better and new ways. These are used and evaluated with real world applications
in~\WPref{applications}.

\begin{compactitem}
\item The \repotodocker{} tool searches a given repository for specifications of
  software dependencies, and carries on to compose instructions to install all
  of the software dependencies within a Docker container image. In this task, we
  modularise this functionality, and make it possible to extract the required
  instructions on their own (for example into a stand-alone shell script).

  Such functionality could be used to:
  \begin{compactitem}
  \item Extract the list of installation commands to carry out a local install
    of the required software, or an installation within any other environment.
  \item In particular on HPC systems, it may be necessary to install software
    directly on the host, and this functionality would simplify that.
  \item Having the installation instructions neatly summarised will also help
    the interested scientist to understand what software specifications are
    (explicitly or implicitly ) given within the repository.
    % \item Having the installation instructions extracted, these can be used to
    %   explore the effects of changing versions of a particular library or
    %   dependency. This will be helpful to track down the origins of observed
    %   non-reproducibility.
  \end{compactitem}

\item Improve and document the options to access external data resources from
  within the computational environments generated by BinderHub deployments. This
  is a prerequisite for \taskref{applications}{data-publishing}.
\item Deploy BinderHub with authenticated data access - useful for many
  institutes who want to offer reproducibility services and data hosting but
  need to limit or control access to those systems (to ensure, for example, that
  the computational resources are not abused).
\item Support using Binder features without a full BinderHub service. A use case for this
  is to create a Binder-like experience on a local Desktop (see
  \taskref{applications}{binder-at-home}).
\item Support deploying Binder for non-interactive use cases.

  Binder's original design goal was to allow interactive execution of notebooks within a
  software requirement that provides all the software required by the notebook.
  This may include compiled and highly customised software, which might produce
  output files, which are then processed and visualised in the notebook.

  Here, we will provide the foundations for reproducibility work that is
  non-interactive and may not use Jupyter notebooks. A use case for this is
  reproducibility at High Performance Computing facilities, where often the
  computational tasks cannot be carried out interactively (see
  \taskref{applications}{binder-at-hpc}),
  but the software environment creation problem solved by repo2docker is the same.

  \end{compactitem}
\end{task}


%
%   This feature will become part of the \repotodocker{} functionality
%   (\localdelivref{extract-dependencies}).
%
%

%%% Local Variables:
%%% mode: latex
%%% TeX-master: "../proposal"
%%% End:

%
\end{tasklist}


\begin{wpdelivs}
  \begin{wpdeliv}[due=12,id=extract-dependencies,dissem=PU,nature=OTHER,lead=SRL]
    {Release new \repotodocker{} feature that exposes the command to install
      identified software environments in stand-alone script
    %  (see \taskref{impact}{extract-dependencies}).
    }
  \end{wpdeliv}

\begin{wpdeliv}[due=36,id=binder-tools-software,dissem=PU,nature=OTHER,lead=SRL]
  {Final open source release of \TheProject tools, completed with automatic
    testing and documentation. }
\end{wpdeliv}

\end{wpdelivs}
\end{workpackage}
%%% Local Variables:
%%% mode: latex
%%% TeX-master: "../proposal"
%%% End:

%  LocalWords:  workpackage wphases wpobjectives wpdescription pageref wpdelivs wpdeliv
%  LocalWords:  dissem mailinglists swrepository final-mgt-rep compactitem swsites ipr
%  LocalWords:  TOWRITE tasklist delivref

\draftpage

% ---------------------------------------------------------------------------
%  Section 3: Implementation
% ---------------------------------------------------------------------------

\section{Quality and efficiency of the implementation}
\COMMENT{Typical granularity: 5-8 work packages with 3-5 tasks and one
  deliverable per task; 10 milestones}

\subsection{Work Plan and resources}

\label{sect:workplan}

\eucommentary{Please provide the following:\\
\begin{compactitem}
\item
brief presentation of the overall structure of the work plan;
\item
timing of the different work packages and their components (Gantt chart or similar);
\item
detailed work description, i.e.:
\begin{compactitem}
\item
a description of each work package (table 3.1a);
\item
a list of work packages (table 3.1b);
\item
a list of major deliverables (table 3.1c);
\end{compactitem}
\item
graphical presentation of the components showing how they inter-relate (Pert chart or similar).
\end{compactitem}
}

\subsubsection{Quality and efficiency of the implementation}\label{sec:workplan-structure}

\ifgrantagreement The \else As shown in Table~\ref{fig:wplist} and
Figure~\ref{fig:workpackages}, the \fi work plan is broken down into five work
packages: \WPref{reproducibility} focuses on robustness improvements of the
Binder tools, \WPref{impact} is advancing the Binder tools feature set to
increase the impact of the tools and project, \WPref{applications} applies and
evaluates the reproducibility tools in real-world research contexts.
\WPref{education} is focused on engaging with and educating researchers and the
wider public in best practices for reproducible science. This is complemented by
the usual management work package (\WPref{management}).


\TODO{Gantt chart link:
  Remove the next sentence, if we remove the Ganttchart.} The Gantt chart on
Page~\pageref{fig:gantt} illustrates the timeline for the various tasks for
these work packages.
%, including inter-task dependencies.

\ifgrantagreement\else
%\makeatletter\wp@total@RM{management}\makeatother
\wpfigstyle{\footnotesize\def\tabcolsep{3.5pt}}
%\wpfig[pages,type,start,end]
{\wpfig}
\fi

\begin{figure}[htb]
  \centering
  \includegraphics[width=0.7\textwidth]{images/WP.pdf}
  \caption{
    \label{fig:workpackages}
    The relationships and interactions of the work packages,
    broken up into four categories: Management (WP1),
    Development of new functionality surrounding Jupyter's Binder tools to improving robustness
    and Broadening impact (WP2, WP3),
    Applications of tools developed in real world research context (WP4),
    and Dissemination, education, and engagement (WP5: Dis., edu., eng.).
  }
\end{figure}

% \subsubsection{How the Work Packages will Achieve the Project Objectives}
% \label{sssec:how_the_work_packages_will_achieve}

% ---------------------------------------------------------------------------
% Include Work package descriptions
% ---------------------------------------------------------------------------

\ganttchart[draft,xscale=.33,milestones]

\begin{workplan}

\draftpage
\subsubsection{Work package descriptions}\label{sec:workpackages}
%%% work package style may be broken -- fix this!!

\ifgrantagreement
\begingroup
% Note: in the grant agreement, The workpackage description must not appear.
% Yet we want to compile them to get all the metadata right
% Current hack: compile them anyway, reset the page number
% appropriately, and remove them a posteriori with pdftk. We set the
% color to red to make it more visible in case we forget to remove
% them.
% See grantagreement rule in the Makefile
\newcounter{savepage}
\setcounter{savepage}{\value{page}}
\color{red} % To make sure we indeed remove the pages
\fi

%\enlargethispage{1cm}

%% Local WP number counter - should possibly be global and hidden?
\TOWRITE{ALL}{Proofread WP 1 Management pass 1}
\begin{draft}
\TOWRITE{PS (Work Package Lead)}{For WP leaders, please check the following (remove items
  once completed)}
\begin{verbatim}
* TODO items (27 March 2022)
- [ ] check links to local deliverables
- [ ] Need data management plan
- [X] do we need an IPR management plan?
      - I don't think so (HF). The IPR management needs to be part of the
        consortium agreement.
- [ ] do we need a Innovation Management Task? It is commented out for now. As
      we are open source all the way through, we don't need to worry about IPR so
      much, and we also don't need to establish a common understanding.
- [ ] have only one data management plan: it should not be hard to decide to
make everything open source. (i.e. get rid of 'draft' plan at M1)
- [ ] update mile stone names so this WP can be completed
- [ ] should we have a \Binder command?
- [ ] Is 'Binder' the right name to refer to? (Should be consistent throughout.)
- [ ] We mention here that all outputs will be open source (technical
      management). This should maybe go elsewhere?
- [ ] should also check with collaborators that they are happy with this.
\end{verbatim}
\end{draft}

% (12+4)/36 = 0.4444444444444444 -> !.44
\begin{workpackage}[id=management,type=MGT,wphases=0-36!.25,
  title=Project management,
  short=Management,
  lead=SRL,
  MPRM=1,
  QSRM=1,
  IFRRM=1,
  UIORM=1,
  SRLRM=12,
  swsites
]
\begin{wpobjectives}
  The main objective of WP1 is to establish and maintain an effective contract,
  project, and operational management approach ensuring:

 \begin{compactitem}
 \item Timely and successful implementation of the project; including
   administrative and legal coordination
    \item Technical management and quality assurance
    \item Risk and innovation management of the project as a whole; including
      data and IPR management
    \item Smooth communication and interaction with the EC and other interested
      parties
 \end{compactitem}
\end{wpobjectives}

\begin{wpdescription}
  The project will be managed by Simula, which has extensive experience in
  administering and leading EU funded and national projects. The coordinator
  together with the WP leaders, will be responsible for monitoring WP status,
  coordination of work plan updates and annual internal progress reports.
  \TOWRITE{}{Management details section has been removed}
  % The project management structure and roles of partners in the consortium are
  % presented in \ref{sect:mgt}.

\end{wpdescription}

\begin{tasklist}

\begin{task}[
  title=Administrative Management,
  id=admin,
  lead=SRL,
  PM=6,
  wphases={0-36!.166},
  partners={MP,QS,UIO,IFR},
]
The task includes the following activities:
\begin{compactenum}
\item Preparation, distribution and maintenance of all contractual documents
  (Consortium Agreement, Grant Agreement and all other legal frameworks)
\item Establishment of appropriate communication and collaborative environment
  for the consortium, as well as the EC and other relevant academic and industry
  stakeholders (the project website, intranet and communication procedures) to
  organise transfer of knowledge, present and promote project results
  (\localdelivref{infrastructure});
\item Organisation of project review and progress meetings;
\item Performing qualitative and quantitative risk analysis, planning risk
  mitigation and control
\item Progress and Financial Reporting to the EC;
\item Data and IPR Management will be managed in accordance with agreed rules
  stated in the Consortium Agreement and in accordance with the Data Management
  Plan (\localdelivref{data-management-plan}.
\end{compactenum}
\end{task}

\begin{task}[
  title=Technical Project Management,
  id=project-management,
  lead=SRL,
  PM=6,
  wphases={0-36!.166},
  partners={MP,QS,UIO,IFR}
  ]
  The task includes the following activities:
  \begin{compactenum}
\item The scientific and technical management to ensure coherent quality and
  soundness of the work and results.
\item Applying quality assurance measures across all partners for all tasks and
  deliverables.
\item Reporting of outcomes and quality assurance activities in technical
  reports and reviews.
\item The project coordinator, with the help of the work package leads, will
regularly review technological risks and recommend mitigation plans to minimise
or remove them. This will be reported on at each reporting period in the
project's technical report.
\item Set up and maintenance of technical management infrastructure required for
  a software project of this type, such as a web site, open source hosting of code and
  documentation, mailing lists, task trackers, automatic tests and continuous
  integration. We will feed back into existing open source repositories projects
  where they exist already, and make use of commonly used tools and services
  such as GitHub. All outputs will be published under an open source license.
\end{compactenum}
\end{task}

\input{tasks/website-general.tex}

% \begin{task}[
%   title=Innovation Management,
%   id=innovation-management,
%   lead=SRL,
%   PM=4,
%   wphases={0-36!.111},
%   partners={MP,QS,UIO,IFR}
% ]
% One of the most important criteria for success for the \TheProject project is to
% bring the project results into use. Therefore, exploitation routes will be
% sought whenever possible. In order to create a common understanding within the
% Consortium of how we can best shepherd an idea all the way from conception to
% realisation and exploitation, the Coordinator will be responsible for the
% preparation and realisation of an Innovation Plan. This plan will assure that
% research activities meet the required milestones and produce outputs fully
% aligned with the project objectives. All research activities will go through an
% initial process where the exploitation opportunity is identified along with the
% main stakeholders for the exploitation opportunity and an IP owner
% (\localdelivref{innovation-management-plan}).
% \end{task}

\begin{task}[
  title=Community Engagement Panel,
  id=community-engagement-panel,
  lead=SRL,
  PM=2,
  wphases={0-36!.056},
  partners={MP,QS,UIO,IFR},
  ]

The task includes the following activities:
\begin{compactenum}
\item Form the community engagement panel by inviting representative of relevant
  communities. Ensure that representatives from stakeholder communities include
  current and potential future Binder users.
\item Organise regular (online) community engagement panel meetings, soon after the
  beginning of the project, and subsequently at the end of years 1, 2, and 3.
\item Ensure that input and feedback from community engagement panel members are
  considered to direct the project to improve the usefulness of Binder tools
  and broaden the range of their applicability to maximise overall impact.
\item Encourage and foster voluntary collaboration and direct contributions to
  the project from the communities represented in the community engagement
  panel that go beyond the advisory role of the panel itself.
\end{compactenum}

\end{task}
\end{tasklist}


\begin{wpdelivs}

\begin{wpdeliv}[due=2,id=infrastructure,dissem=PU,nature=DEC,lead=SRL]
  {Basic project infrastructure (web site, mailing lists, issue trackers, mailing lists, repositories)}
\end{wpdeliv}


  \begin{wpdeliv}[due=3,id=dissemination-plan,dissem=PU,nature=R,lead=SRL]
    {Detailed dissemination, communication, and exploitation plan.}
  \end{wpdeliv}


% “Proposals selected for funding under Horizon Europe will need to develop a
% detailed data management plan (DMP) for making their data/research outputs
% findable, accessible, interoperable and reusable (FAIR) as a deliverable by
% month 6 and revised towards the end of a project’s lifetime.”
\begin{wpdeliv}[due=6,id=data-management-plan,dissem=PU,nature=DMP,lead=SRL]
  {Data Management Plan}
\end{wpdeliv}

\begin{wpdeliv}[due=32,id=data-management-plan-revised,dissem=PU,nature=DMP,lead=SRL]
  {Revised Data Management Plan}
\end{wpdeliv}

% \begin{wpdeliv}[due=9,id=innovation-management-plan,dissem=CO,nature=R,lead=SRL]
%   {Innovation Management Plan}
% \end{wpdeliv}

\end{wpdelivs}
\end{workpackage}
%%% Local Variables:
%%% mode: latex
%%% TeX-master: "../proposal"
%%% End:

%  LocalWords:  workpackage wphases wpobjectives wpdescription pageref wpdelivs wpdeliv
%  LocalWords:  dissem mailinglists swrepository final-mgt-rep compactitem swsites ipr
%  LocalWords:  TOWRITE tasklist delivref

\draftpage
\begin{draft}
\begin{verbatim}
- [ ] distribute tasks for repo2docker to other workpackages
\end{verbatim}
\end{draft}

\begin{workpackage}[
  id=reproducibility,
  wphases=0-36!1.03,
  title=Improving robustness of reproducibility tools,
  short=Robustness,
  lead=QS,
  SRLRM=23,
  UIORM=0,
  MPRM=2,
  QSRM=12,
  swsites,
]
\begin{wpobjectives}
  \begin{compactitem}
    \item to better understand and evaluate successful reproduction of computational environments
    \item to improve the practical reproducibility of environments constructed
      with \TheProject tools
    \item to support and maintain core Binder software infrastructure in order to keep it healthy
         and useful for open science and reproducibility
 \end{compactitem}
\end{wpobjectives}

\begin{wpdescription}
\TOWRITE{repurposed task from BOSSEE}

This Work package is focused on making \repotodocker{} do the things it does
already \emph{better}, \emph{more robustly} and \emph{more sustainably}.
(Orthogonal to those improvements, we plan to significantly extend the use cases
for \repotodocker{} in \WPref{impact}.)

To be able to asses the impact of our planned improvements, we need to have a
metric. Task \localtaskref{repo2docker-checker} will create this
for us. In addition to the evaluation of the improvements in this proposal, this
can be used more generally as an indicator for reproducibility of software
environments.

One major improvement to the existing capabilities of \repotodocker is the
\emph{time-machine} functionality, and this is implemented in \localtaskref{repo2docker-timemachine}.

In task \localtaskref{performance-optimisation}, we will speed-up the execution time of
\repotodocker{} to improve the user experience when reproducing or re-using
existing software and data.

Open source software needs ongoing maintenance to adapt to changing requirements
and dependencies. We schedule a certain amount of time for this in task
\localtaskref{maintenance}.

All changes to the software will be made available already during development
(i.e. throughout the whole project), and new features will be made available
through software releases of the Binder tools. A final release will be made and
reported through the deliverable \delivref{impact}{binder-tools-software}.

% the existing functionality of repodocker.
% Community-led open source software is critical to a sustainable future for open science.
% Commonly used tools make up a shared infrastructure,
% where investment in core components benefits the widest user community.
% \TheProject is centred around the Jupyter project,
% which is a collection of projects for interactive computing and
% communicating computational ideas.
%
% This work package is focused on developing and maintaining
% the core of Jupyter.
% In particular, we will help maintain these projects to meet the needs of the
% Jupyter community, with a focus on needs for open science.
% To serve the needs of \TheProject,
% Jupyter core infrastructure will need improvements
% to security, performance, and scalability,
% which will be provided in \localtaskref{maintenance}.
% In addition, we will develop new features in the core of Jupyter
% to bring it to a wider audience,
% and to improve its usefulness to those working toward open science practices,
% including via collaboration features (\localtaskref{collaboration})
% and accessibility (\localtaskref{accessibility}).

\end{wpdescription}

\begin{tasklist}

\input{tasks/study}
\input{tasks/repo2docker}
\input{tasks/performance-optimisation}
\input{tasks/maintenance}

\end{tasklist}


\begin{wpdelivs}
  % \begin{wpdeliv}[due=1,miles=startup,id=infrastructure,dissem=PU,nature=DEC,lead=SRL]
  %   {Some Deliverable}
  % \end{wpdeliv}

  % (\localdelivref{deliv-id})

  \begin{wpdeliv}[due=24,miles=prototype,id=deliv-id-repo2docker-checker-software,dissem=PU,nature=OTHER,lead=SRL]
    {Release software tool for checking of reproducibility of software
      environments (\texttt{repo2docker-checker})}
  \end{wpdeliv}

  \begin{wpdeliv}[due=36,miles=final,id=repo2docker-checker-study-report,dissem=PU,nature=R,lead=SRL]
    {Summary of reproducibility improvements achieved. (\texttt{repo2docker-checker})}
  \end{wpdeliv}



\end{wpdelivs}

\end{workpackage}
%%% Local Variables:
%%% mode: latex
%%% TeX-master: "../proposal"
%%% End:

%  LocalWords:  workpackage wphases wpobjectives wpdescription pageref wpdelivs wpdeliv
%  LocalWords:  dissem mailinglists swrepository final-mgt-rep compactitem swsites ipr
%  LocalWords:  TOWRITE tasklist delivref

\draftpage
\begin{draft}
\TOWRITE{PS (Work Package Lead)}{For WP leaders, please check the following (remove items
once completed)}
\begin{verbatim}
- [ ] discuss what the build packs are and where they go
- [ ] how we structure and budget different tasks in this WP
- [ ] needs review and more technical detail
- [ ] add deliverable(s)
\end{verbatim}
\end{draft}

\begin{workpackage}[
  id=impact,
  wphases={6-24!0.89,24-30!0.86,30-36!0.48},
  swsites,
  title=Broadening impact,
  short=Broadening impact,
  lead=SRL,
  SRLRM=24,
  MPRM=12,
  QSRM=2,
  IFRRM=4,
  %UIORM=4,
]
\begin{wpobjectives}
  This work package extends the functionality of the \TheProject tools for
  reproducibility to broaden its applicability and increase impact.
  \WPref{applications} exploits these technological advances
  in real world use cases.

  The objectives of this work package are to
 \begin{compactitem}
 \item remove the dependency of \repotodocker{} on Kubernetes
 \item support container technologies other than Docker
 \item extend the range of software specification standards that are recognised
   and supported by \repotodocker{} \TOWRITE{}{Can we give an example here?}
 \item enable access to data from data sources outside the container
 \item enable \repotodocker{} to extract and save the software installations
   instructions (independent from container generation)
 \end{compactitem}
\end{wpobjectives}

\begin{wpdescription}
One of the design decision that have led to the current Binder software stack is
to constrain the supported infrastructure to a few key components. These
include:

\begin{compactitem}
\item Software environments can only be created inside Docker container
\item To run and orchestrate (multiple) Docker containers, a Kubernetes system must
be available
\item The user must interact with Binder environments through a Binderhub
  installation
\item There is no Binders-specific provision to access (substantial) data sets
  from inside te container. \TODO{HF: What prevents us from
    a data-publishing application at the moment? How to express this without
    making the impression one couldn't use \softwarename{curl/wget} to fetch
    data, etc. }
\end{compactitem}

The benefits of such a restrictive approach are that the software development
and maintenance effort is kept small: the wider the range of supported
infrastructure that the Binder tools can be deployed on, the higher the
complexity.

At the same time, these restrictions prevent the following scenarios:
\begin{compactitem}
\item To run the Binder software on systems where Kubernetes is not available (such as
  the Desktop of a scientist). A related use case is ``Binder@home''
  \TODO{Provide link to Binder at home task in
   \WPref{applications}}.
\item To use Binder on systems where Docker cannot be used. A use case for this
  is to create reproducible environments on HPC installations. The HPC
  administrators generally avoid use of Docker for security reasons, but much
  prefer to support container technologies that can be executed without
  administrative privileges). \TODO{Point to task in applications}
\item To use Binder for software environments that use \TOWRITE{}{insert buildback
    example / community / software system}
\item To access and use significant amounts of data from inside the software
  environment is not supported. An important use case that cannot be supported due to
  this restriction is that of ``data publishing'': the idea is that a
  (potentially large and/or complex) dataset is published \emph{together} with
  software that encodes the necessary knowledge to extract meaningful data from
  that data set. A Binder environment would make this data set accessible in an
  interactive environment. \TODO{Point to data publishing use case(s)}
 \end{compactitem}

% Open source software in general, and Jupyter in particular,
% is developed not as a monolithic application,
% but rather as a collection of related components,
% which can be assembled in numerous combinations to meet diverse needs.
% The Jupyter community is no different.
% Jupyter itself is composed of several projects,
% but there are even more projects that build on top of Jupyter to create
% things like cloud services or data pipelines.
% The goal of \TheProject is to facilitate open science through Jupyter,
% and this includes working with projects all around the Jupyter ecosystem.
% We will focus this work package on developing
% Jupyter ecosystem projects with an emphasis on open science.
%
% repo2docker is a project for creating
% reproducible environments in which Jupyter notebooks (and other user interfaces) can be run.
% It reads a number of common formats to list required software packages,
% and prepares a Docker container with those packages installed.
% BinderHub is software for operating a web service using repo2docker,
% which enables sharing of interactive and reproducible Jupyter (and Rstudio) environments on the web with a single link.
% We will develop repo2docker and BinderHub further to meet the needs of the open science community.
%
% In addition to the interactive aspects of Jupyter,
% notebooks can be used in a "workflows" style,
% where job systems run analyses and produce reports,
% either on a scheduled basis or triggered by events.
% There is a great deal of interest in using notebooks in this way,
% and much room for development of tools supporting workflows in data-driven open science.



\end{wpdescription}

\begin{tasklist}
% % add tasks from task directory here
\input{tasks/buildpacks}
\input{tasks/constraints}
\input{tasks/patterns}
%
\end{tasklist}


\begin{wpdelivs}
  \begin{wpdeliv}[due=12,id=extract-dependencies,dissem=PU,nature=OTHER,lead=SRL]
    {Release new \repotodocker{} feature that exposes the command to install
      identified software environments in stand-alone script
    %  (see \taskref{impact}{extract-dependencies}).
    }
  \end{wpdeliv}

\begin{wpdeliv}[due=36,id=binder-tools-software,dissem=PU,nature=OTHER,lead=SRL]
  {Final open source release of \TheProject tools, completed with automatic
    testing and documentation. }
\end{wpdeliv}

\end{wpdelivs}
\end{workpackage}
%%% Local Variables:
%%% mode: latex
%%% TeX-master: "../proposal"
%%% End:

%  LocalWords:  workpackage wphases wpobjectives wpdescription pageref wpdelivs wpdeliv
%  LocalWords:  dissem mailinglists swrepository final-mgt-rep compactitem swsites ipr
%  LocalWords:  TOWRITE tasklist delivref

\draftpage
\TOWRITE{ALL}{Proofread WP 1 Management pass 1}
\begin{draft}
\TOWRITE{PS (Work Package Lead)}{For WP leaders, please check the following (remove items
once completed)}
\begin{verbatim}
- [ ] have all the tasks in this Work Package a lead institution?
- [ ] have all deliverables in the WP a lead institution?
- [ ] do all tasks list all sites involved in them?
- [ ] does the table of sites and their PM efforts match lists of sites for each task?
      (each site from the table is listed in all relevant tasks, and no site is listed
      only in the table or only at some task)
\end{verbatim}
\end{draft}

\begin{workpackage}[
  id=applications,
  wphases=0-36,
  swsites,
  title=Applications and use cases,
  short=Applications,
  lead=MP,
  % EGIRM=7,
  % CDSRM=12,
  % INSERMRM=24,
  % QSRM=6,
  % SILRM=12,
  SRLRM=9,
  % UIORM=12,
  % UPSUDRM=20,
  % WTTRM=3,
  % XFELRM=36,
  % EPRM=3,
]

\TOWRITE{Everything we develop in WP2-3 should be validated here}

\begin{wpobjectives}
  The objectives of this work package are
 \begin{compactitem}
   \item to guide the development of core tools by simultaneously
     developing and using applications in diverse fields with active
     scientists from these fields, and
   \item to demonstrate that the tools we develop are valuable to diverse
     fields of science, thus ensuring we develop e-infrastructure and
     services which can cater for a broad European and global research community
   \end{compactitem}
\end{wpobjectives}

\begin{wpdescription}

  Whilst the components issued from work packages  \WPref{reproducibility} and \WPref{impact} will be
  made available as generic building blocks for reproducible open science services,
  this work package aims at specific real-world cases.

  We have selected a number of applications in a variety of domains
  to demonstrate the broad impact of \TheProject, in particular in the
  areas of \TOWRITE{tasks}
  % (\localtaskref{astro}), education
  % (\localtaskref{teaching}), fluid dynamics
  % (\localtaskref{application-gpu}), geosciences
  % (\localtaskref{geoscience}), health
  % (\localtaskref{opendose-analysis}), mathematics
  % (\localtaskref{math}) and photon science and imaging
  % (\localtaskref{reproducibility-xfel}).
  The context and vision for each of the demonstrators is described in
  section \ref{sec:science-demonstrators-in-concept} on page
  \pageref{sec:science-demonstrators-in-concept}.

  Working closely with the core developers of the Jupyter ecosystem will make it possible to
  go way beyond what is normally available "out-of-the-box" and to offer better solutions,
  thereby guiding further development of the core features.

  \medskip

  All demonstrators will validate the Jupyter service capabilities such as reproducibility,
  interactive widget use and visualisation, and show how these can
  enable new open science on EOSC.

  The particular workflows, data infrastructures and data policies for
  FAIR\footnote{Findable, Accessible, Interoperable and Reusable} sharing of data vary from one community and use-case to
  the other, or may not be fully defined yet. Therefore, this proposal
  does not enforce a specific way of handling data. Instead we
  will explore in the demonstrator tasks how existing data policies,
  infrastructure and workflows can be respected and integrated with
  authentication and authorisation, data management, and
  JupyterHub/Binder services on EOSC. EGI is a partner
  for all the tasks in this work package and will work with us to find the
  best integration solutions in the evolving EOSC
  infrastructure.

  For some of the demonstrators, authentication and authorization and/or
  data management are being advanced outside \TheProject.


\end{wpdescription}

\begin{tasklist}
% add tasks from task directory here
\input{tasks/demos}
\input{tasks/binder-at-home}
\input{tasks/policy}
\end{tasklist}



\begin{wpdelivs}
%\TODO{update due date and startup!}
%\TODO{update milestone!}
\begin{wpdeliv}[
    % id for linking with \delivref or \localdelivref
    id=deliv,
    % lead institution
    lead=XXX,
    % month when deliverable is due (max 36)
    due=12,
    % associated milestone id (see milestones.tex)
    miles=startup,
    % ~always PU, DEC
    dissem=PU,
    nature=DEC,
]
  {
  One-line name of deliverable
  }
\end{wpdeliv}


\end{wpdelivs}
\end{workpackage}
%%% Local Variables:
%%% mode: latex
%%% TeX-master: "../proposal"
%%% End:

%  LocalWords:  workpackage wphases wpobjectives wpdescription pageref wpdelivs wpdeliv
%  LocalWords:  dissem mailinglists swrepository final-mgt-rep compactitem swsites ipr
%  LocalWords:  TOWRITE tasklist delivref

\draftpage
\TOWRITE{ALL}{Proofread WP 5 Management pass 1}
\begin{draft}
\TOWRITE{PS (Work Package Lead)}{For WP leaders, please check the following (remove items
once completed)}
\begin{verbatim}
- [ ] have all the tasks in this Work Package a lead institution?
- [ ] have all deliverables in the WP a lead institution?
- [ ] do all tasks list all sites involved in them?
- [ ] does the table of sites and their PM efforts match lists of sites for each task?
      (each site from the table is listed in all relevant tasks, and no site is listed
      only in the table or only at some task)

- [ ] Binder / repo2docker documentation: tutorials and best practice guides -
      have we got this covered?
\end{verbatim}
\end{draft}

\begin{workpackage}[id=education,wphases=0-36!1,swsites,
  title=Education and Dissemination,
  short=Education,
  lead=IFR,
  IFRRM=10,
  MPRM=6,
  SRLRM=7,
  QSRM=3,
  UIORM=9
]


\begin{wpobjectives}
  The objective of this work package is to disseminate the results of this
  project, including the technical advances and guidance for best practice for
  reproducible science. This includes education of researchers about the value of
  open science, reproducibility and re-usability as well as the possibilities of
  integrating Binder tools in their workflows.

  Beyond this activity, which is directed from the project members to the wider community of
  scientists, we also plan to seek input from scientists to the project: both in
  terms of requirements for practical reproducibility in their domain and in
  technical contributions -- for example through merge requests for Binder
  tools, or open source documentation of best practice for reprocudible software
  environments.

  Our education and dissemination objectives includes:
 \begin{compactitem}
   \item Ensure awareness of the results of the project in the user community,
     and in particular in those groups that act as educators and multipliers of
     knowldege (such as the Carpentries and research infrastructure organisations).
   \item Educate the community on the value of open science, and in particular
   \item Train researchers in best practices for open and reproducible science.
   \item Produce training and education material to disseminate the ability to
     do reproducible computational science using the tools we develop.
   \item Address the shortage of researchers and research support staff trained in reproducibility.
   \item Provide documentation and tutorials which can serve as the technical
     components of reproducibility policies.
 \end{compactitem}
\end{wpobjectives}

% Potential sources of inspiration: ODK's WP2 work package about dissemination:
% PDF: p.36 of https://github.com/OpenDreamKit/OpenDreamKit/raw/master/Proposal/proposal-www.pdf
% Sources: https://github.com/OpenDreamKit/OpenDreamKit/blob/master/Proposal/WorkPackages/DisseminationCommunityBuilding.tex

\begin{wpdescription}

  Open science and reproducible science is entirely dependent on researchers
  adopting open practices. In \TheProject, we improve and developing tools that
  can facilitate these practices, but they only work if researchers actually adopt
  them. For researchers to adopt the practices, they need to (i) know about them
  and (ii) use them.

  We address challenge in multiple ways:
  \begin{enumerate}
    \item the philosophy of the Binder tools is to respect existing standards and
      best practice (and not to invent additional syntax or requirements). It is
      thus possible to use the Binder tools (to recreate a software environment)
      even if the repository authors did not anticipate the use of Binder, or
      knew about their existence. In the best possible scenario, a
      \emph{scientist can benefit from Binder tools with zero additional effort}.

    \item In this work package, we produce education materials and carry out
      education activities to spread the knowledge about \emph{good practice for
        reproducibility and re-usability in science}, such as for example
      automation of all analysis steps, and complete documentation of the
      required software stack. Only one aspect of this training is to show how
      Binder can help with reproducibility.

      Attendees and users following such training and advice will create more
      reproducible artifacts. If they -- or later users of their published
      artifacts -- want to use Binder to reproduce or re-use the results, they
      can. Even if they do not, we will have achieved an improvement of the
      reproducibility of scientific artifacts.
  \end{enumerate}

  % HF: I think training the wider (non-scientist) public about Binder is going
  % too far?
  %
  % Going further, it is also clear that open science is not just of value
  % to researchers: one of the largest benefits of open science is that it makes
  % science accessible to the broader public who may not be members of the
  % research community.
  %
  % To this end, in addition to training researchers, we will also train the
  % public in how to make use of the open science research and services
  % facilitated by \TheProject. This will be done through regular open
  % dissemination and training workshops, as well as by producing and maintaining
  % material for online courses and documentation.

  The \TheProject project will develop, through \WPref{applications}, a number
  of demonstrator repositories that show examples of reproducibility in
  different scientific domains. (We use
  those activities to inform and evaluate the technical improvement to the
  Binder tools in \WPref{reproducibility} and \WPref{impact}). We also use those
  studies to create tutorials and \emph{best practice guides for
    reproducibility} (\localtaskref{online-resources}) in this work package, and
  offer interactive workshops (\localtaskref{workshops}) to help disseminate the
  content more effectively.

  As with all the code, test and build infrastructure produced as part of the
  project, we will also make all documentation open source. Our documentation --
  which includes best practice guides for reproducibility -- can thus be
  modified and improved after the end of the project to react to new
  developments (\delivref{education}{education-materials2}).

  We will engage with the scientific to support them in making their
  work more reproducible with Binder tools. The project will benefit from these interactions
  as we will lean more about reproducibility requirements and usefulness of the
  Binder tools, so that we can tailor our work to support scientific communities as broad as possible
  (\localtaskref{community-support}).

  We will also participate in the well established academic dissemination
  activities, and events of the European E-Infrastructure projects and other
  relevant structures. EGI is a member of our the community engagement panel
  (see \TODO{XXX, \taskref{management}{community-engagement-panel}})
  and the interaction with them be useful to prioritise our resources in this
  very active field.

  Open access to all publications resulting from the project will be ensured.
  %\TODO{Should this sentence go into a data / IPR management plan?}
\end{wpdescription}

\begin{tasklist}
% add tasks from task directory here
\input{tasks/website-general}
\input{tasks/online-resources}
\input{tasks/training}
\input{tasks/community-support}
\end{tasklist}


\begin{wpdelivs}
\begin{wpdeliv}[due=24,id=education-materials1,dissem=PU,miles=prototype,nature=R,lead=UIO]
  {First version of all training materials available online. }
\end{wpdeliv}
\begin{wpdeliv}[due=36,id=education-materials2,dissem=PU,miles=final,nature=R,lead=UIO]
  {All training sessions material completed, reviewed, and published online.}
\end{wpdeliv}
% \begin{wpdeliv}[due=36,id=report2,dissem=PU,miles=final,nature=R,lead=UIO]
%   {Community building: Report on impact of development workshops, dissemination and training activities.}
% \end{wpdeliv}
\end{wpdelivs}

\end{workpackage}
%%% Local Variables:
%%% mode: latex
%%% TeX-master: "../proposal"
%%% End:

%  LocalWords:  workpackage wphases wpobjectives wpdescription pageref wpdelivs wpdeliv
%  LocalWords:  dissem mailinglists swrepository final-mgt-rep compactitem swsites ipr
%  LocalWords:  TOWRITE tasklist delivref

\draftpage

\ifgrantagreement
\endgroup
\setcounter{page}{\value{savepage}}
\fi

%%% Local Variables:
%%% mode: latex
%%% TeX-master: "../proposal"
%%% End:

%  LocalWords:  newpage workpackages workplan



% \TODO{Strange vertical lines at the left of the bottom of table~\ref{sec:deliverables}?}

\eucommentary{Milestones means control points in the project that help to chart progress. Milestones may
correspond to the completion of a key deliverable, allowing the next phase of the work to begin.
They may also be needed at intermediary points so that, if problems have arisen, corrective
measures can be taken. A milestone may be a critical decision point in the project where, for
example, the consortium must decide which of several technologies to adopt for further
development.
}
\begin{draft}
\begin{verbatim}
* TODO
- [ ] we refer to \TheProject SERVICES a lot
- [ ] would it be better to refer to TOOLS instead?
- [ ] we should probably list the services / tools somewhere for clarity
\end{verbatim}
\end{draft}


% \milestonetable



\begin{milestones}
  \milestone[
    id=startup,
    month=12,
    verif={
      Completed all corresponding deliverables
      and preparation for deployment of prototype services is underway
      }
    ]
  {Project startup, requirements gathering, and exploratory prototypes}
  {

  By milestone 1, we will have established the infrastructure for the project
  and begun exploratory prototyping development and deployment of services,
  engaging with the existing communities. We will have a network of advisors
  through the Community Engagement Panel and are coordinating our plans for
  \TheProject with those of wider open science communities and the Jupyter
  project. We will have preliminary study results to guide and evaluate
  improvements to reproducibility with \TheProject tools. }

  \milestone[
    id=prototype,
    month=24,
    verif={
      Developed functional prototypes for all important topics.
      Early users are able to access and test prototype services
    }
    ]
  {Prototype demonstrator services}
  {

  By this point, prototype demonstrator services will be useful and accessible
  to a broad range of users, and we will have begun to experiment with early-adopter
  users and local demonstrators to guide further development of \TheProject,
  ensuring that development serves the reproducibility needs of the global science community.

  There will be improvements to reproducibility.
  }

  \milestone[
    id=final,
    month=36,
    verif={Refactored and stabilised prototypes.}
    ]
  {Project conclusion}
  {
  At the end of the project,
  we will have engaged with the science communities to evaluate demonstrator services, and
  identified which services and tools shall be sustained beyond the life of the project.
  We will have made those robust and maintainable from a software engineering point of view.

  We will have training materials and have run workshops to train users in Reproducibility and Open Science,
  making use of tools and services developed through \TheProject.

  We will have developed a sustainability plan for how future maintenance and
  development of \TheProject tools be achieved under community support and
  leadership.

  \TheProject tools are more reliable and useful to a broader audience than at
  the start of the project.

  }


\end{milestones}


\end{workplan}

\subsubsection{Deliverables}\label{sec:deliverables}
\inputdelivs{9.3cm}%

\milestonetable

\subsubsection{Risks and risk management strategy}
\label{sec:risks}
% 
% The risk in the project execution as planned is carefully assessed and
% managed. We base our plans on long standing experience, and we bring
% together the world's experts in the relevant tools and techniques.
% 
% A key feature of this project is the involvement of a wide set of
% partners from multiple domains. While this ensures complementary
% coverage of a wide set of skills and provides robustness in different
% ways, we will have to ensure that all the partners work together as
% a closely knit team.
% 
% Our open source approach means that all our code and outputs
% will be open and visible to anybody at sites like GitHub and Bitbucket
% throughout the project. It is common for some users to run the latest
% development versions of computational and infrastructure software, thus
% beta-testing code between major releases.
% This reduces the risk of developing software which people won't use:
% where our design decision or technical approaches are
% controversial, this will be detected early by those users, giving the
% consortium useful feedback to consider.
% 
% As part of the Management Work Package, and with support from the
% Coordination Team, the project coordinator will maintain and regularly
% update a Risk Management Plan; at the end of each Reporting Period,
% this updated plan will be included in the project's Technical Report.
% It will identify and categorise all
% potential strategic risks (legal, financial, human resources risks, etc.)
% to the successful delivery of the project, their probability and impact.
% For each risk area, mechanisms for risk mitigation will be identified
% and contingency actions will be proposed.
% 
% Risks will be evaluated in terms of project goals and objectives,
% according to the following four steps:
% \begin{enumerate}
% \item Identification of risks using a structured and rational approach to
% ensure that all areas are addressed.
% \item Quantitative assessment and ranking of the risks.
% \item Definition of procedure to reduce (or minimize) risk.
% \item Monitoring and management of risks throughout the project life
% with milestone review and reassessment.
% \end{enumerate}
% 
% Finally, as reported above, a conflict resolution mechanism will be put in place,
% whereby decision making divergence and conflicts that cannot be solved
% at the Steering Committee (SC) level will be submitted to the Coordinator. The
% mediation and resolution process used is the following:
% \begin{itemize}
% \item Case presented by the involved parties.
% \item Development of a fact-based and neutral report by the coordinator
% to be provided to the conflicting parties and SC.
% \item Final decision to resolve the conflict made by SC.
% \end{itemize}
% 
\ifgrantagreement\else
An initial risk assessment appears as Table~\ref{risk-table}.

\begin{table}
\begin{center}
\begin{tabular}{|m{.2\textwidth}|m{.12\textwidth}|m{.58\textwidth}|}\hline
  Risk & Level without / with mitigation & Mitigation measures
  \\\hline

   \multicolumn{3}{|c|}{
    \textit{General technical / scientific risks}
   }
   \\\hline

  Implementing infrastructure that does not match the needs of end users & High/Low &
  Many of the members of the consortium are themselves end-users with
  a diverse range of needs and points of views; hence the design of
  the proposal and the governance of the project is naturally steered
  by demand; besides, because we provide a toolkit, users have the
  flexibility to adapt the infrastructure to their needs. In addition, the open source nature
  of the project facilitates and promotes the involvement of the wider community in terms of
  providing feedback and requesting additional features via platforms such as GitHub and Bitbucket
  on a regular basis.
  \\\hline

  Lack of predictability for tasks that are pursued jointly with
  the community & Medium/Low &
  The PIs have a strong experience managing community-developed
  projects where the execution of tasks depends on the availability of
  partners. Some tasks may end up requiring more manpower from
  \TheProject to be completed on time, while others may be entirely
  taken care of by the community. Reallocating tasks and redefining
  work plans is common practice needed to cater for a
  fast evolving context. Such random factors will be averaged out over
  the large number of independent tasks.\\\hline

  Reliance on external software components & Medium/Low & The non trivial
  software components \TheProject relies on are open source. Most are
  very mature
  and supported by an active community, which offers strong long run
  guarantees. The other components could be replaced by alternatives, or
  even taken over by the participants if necessary.
  \\\hline

  \\\hline

%  \multicolumn{3}{|c|}{
%    \textit{Use-case risks}
%  }
%  \\\hline
%
%  & & \TOWRITE{WP4}{Risks related to use-cases in WP4}
%  \\\hline

  \multicolumn{3}{|c|}{
    \textit{Management risks}
  }
  \\\hline

  Recruitment of highly qualified staff & High/Medium &

  Great care was taken coordinating with currently running projects to
  rehire personnel with strong track record, and identifying pool of
  candidates to hire from, notably in the developers community of
  software related to the project. This was favoured by the partners'
  long history of training and outreach activities. In addition, we
  have a critical mass of qualified staff in the project enabling us
  to train and mentor new recruits.

 \\\hline

  Different groups not forming effective team & Medium/Low & The participants have a long
  track record of working collaboratively on code across multiple
  sites. Aggressive planning of project meetings, workshops and
  one-to-one partner visits will facilitate effective teamwork,
  combining face-to-face time at one site with remote
  collaboration.\\\hline
  % this also justifies our generous travel budget.

  Partner leaves the consortium & High/Low & If the GA requires a replacement
  in order to achieve the project's objectives, the consortium will invite a new
  relevant partner in. If a replacement is not necessary, the resources and tasks
  of the departing partner will be reallocated to the alternative ones within the
  consortium.
  \\\hline

  \multicolumn{3}{|c|}{
    \textit{Dissemination risks}
  }
  \\\hline

  Impact of dissemination activities is lower than planned. & Medium/Low &

  Partners in the consortium have a proven track record at community
  building, training, dissemination, social media communication, and
  outreach, which reduces the risk. The Project Coordinator
  will monitor impact of all dissemination activities. If a deficiency is identified, the consortium
  will propose relevant corrective actions.\\\hline

  \end{tabular}
\end{center}
\caption{\label{risk-table}Initial Risk Assessment}
\end{table}
\fi
%\TOWRITE{NT/Eugenia}{Impredictability}

%\includegraphics[width=.94\textwidth]{Pictures/Impact-img1.png}

%   But: since Open Source softwares are freely accessible, security
%   and privacy issues are a concern. Anytime a resource is shared,
%   there is greater risk of unauthorised access and contaminated data.
%   Providers must demonstrate security solutions, which should include
%   physical security controlling access to the facility and protection
%   of user data from corruption and cyber attacks.}


\TOWRITE{ALL}{
  Add a paragraph about data management plan. What data will we produce, which data is available from the
  start, how do we handle it...
}

%
% a table showing number of person months required (table 3.1f);
% 	a table showing description and justification of subcontracting costs for each participant (table 3.1g);
% -	a table showing justifications for purchase costs (table 3.1h) for participants where those costs exceed 15% of the personnel costs (according to the budget table in proposal part A);
% -	if applicable, a table showing justifications for other costs categories (table 3.1i);
% -	if applicable, a table showing in-kind contributions from third parties (table 3.1j)
\draftpage
\subsubsection{Resources to be Committed}\label{sec:resources}

Tables summarising effort and costs are presented here.

\eucommentary{
Please indicate the number of person/months over the whole duration of the planned work, for each work package, for each participant. Identify the work-package leader for each WP by showing the relevant person-month figure in bold.
}

\wpfig[label=fig:staffeffort,caption=Summary of staff effort]

%%%%%%%%%%%%%%%%%%%%%%%%%%%%%%%%%%%%%%%%%%%%%%%%%%%%%%%%%%%%%%%%%%%%%%%%%%%%%%
% \paragraph{Purchase costs}

\noindent\textbf{Purchase costs}

\TOWRITE{}{How much justification necessary?}

\eucommentary{
Please complete the table below for each participant if the purchase costs (i.e.
the sum of the costs for 'travel and subsistence', 'equipment', and `other
goods, works and services') exceeds 15\% of the personnel costs for that
participant (according to the budget table in proposal part A). The record must
list cost items in order of costs and starting with the largest cost item, up to
the level that the remaining costs are below 15\% of personnel costs
}

All participants request less than 15\% of personnel costs in purchase costs.
These costs go toward:

\begin{itemize}[noitemsep]
\item travel to project meetings
\item site visits between project members to foster collaboration
\item conference attendance for dissemination
\item open access publication fees
\item equipment for carrying out the work (high performance laptop computers for each FTE)
\item CFS (at \site{SRL} only)
\item hosting workshops (at \site{SRL}) for dissemination
\item cloud costs (at \site{SRL}) for testing outputs and supporting development and workshops \TOWRITE{}{needs to be discussed earlier in work plan}
\end{itemize}

% below is commented-out a more detailed justification for Simula
% See line 151 in
% https://docs.google.com/spreadsheets/d/19xparkP93ANTecMqApNVEvl_w9-q12KE5yHS-IjN7T4

% \site{SRL} is the only site requesting more than 15\% of personnel costs in purchase costs.
% This is because \site{SRL} will host some shared project-wide costs,
% such as hosting workshops for \WPref{education}
% and project-wide cloud costs for testing and demonstration,
% used across all work packages.
%
% % Our project travel costs are estimated based on:
% %
% \begin{compactenum}
% \item 800 \euro per traveller for each of 3 project meetings
% \item 2000 \euro per FTE-researcher-year (1.75) for site visits, facilitating collaboration
% \item 3000 \euro per FTE-researcher-year (1.75) for conference attendance
% \item 4000 \euro for hosting each of two in-person workshops,
%       with an estimated cost of 400 \euro per participant (10 participants).
% \end{compactenum}
%
% \bigskip
% \begin{table}[H]
% \begin{tabular}{|r|r|p{8.5cm}|}
%   \hline
%   \textbf{\site{SRL}}
%     & \textbf{Cost (\euro)}
%     & \textbf{Justification}
%     \\
%   \hline
%   \textbf{Other goods, works, and services}
%     & 36100
%     & Average 600 \euro per month of cloud computing service costs,
%       3500 \euro for CFS, hosting two workshops of 10 attendees at 400 \euro per attendee,
%       3000 \euro for open access publication fees.
%     \\
%   \hline
%   \textbf{Remaining purchase costs}
%     & 50300
%     \\
%   \cline{1-2}
%   \textbf{Total}
%     & 86400
%     \\
%   \cline{1-2}
%   \end{tabular}
% \caption{Overview: 'Purchase costs' to be committed at Simula
% Research Laboratory
% (all in \texteuro)}\vspace*{-1em}
% \end{table}


%%% Local Variables:
%%% mode: latex
%%% TeX-master: "proposal"
%%% End:


\draftpage
\subsection{Capacity of participants and consortium as a whole}
\eucommentary{
The individual members of the consortium are described in a
separate section under Part A. There is no need to repeat that
information here.
\begin{itemize}
\item
  Describe the consortium. How does it match the project's objectives,
  and bring together the necessary disciplinary and inter-disciplinary
  knowledge. Show how this includes expertise in social sciences and
  humanities, open science practices, and gender aspects of R\&I, as
  appropriate. Include in the description affiliated entities and
  associated partners, if any.
\item
  Show how the partners will have access to critical infrastructure
  needed to carry out the project activities.
\item
  Describe how the members complement one another (and cover the value
  chain, where appropriate)
\item
  In what way does each of them contribute to the project? Show that
  each has a valid role, and adequate resources in the project to fulfil
  that role.
\item
  If applicable, describe the industrial/commercial involvement in the
  project to ensure exploitation of the results and explain why this is
  consistent with and will help to achieve the specific measures which
  are proposed for exploitation of the results of the project (see
  section 2.2).
\item
  \textbf{Other countries and international organisations}: If one or
  more of the participants requesting EU funding is based in a country
  or is an international organisation that is not automatically eligible
  for such funding (entities from Member States of the EU, from
  Associated Countries and from one of the countries in the exhaustive
  list included in the Work Programme General Annexes B are
  automatically eligible for EU funding), explain why the participation
  of the entity in question is essential to successfully carry out the
  project.
\end{itemize}
}

\subsubsection{Consortium composition}

\TODO{Time permitting: add map of Europe with pointers to locations of each
  site. Just to break up the pages of text without pictures.}

The \TheProject consortium spans the broad spectrum of actors required for
successfully developing and disseminating tools and infrastructure for open and
reproducible computational science, catering to the needs of the European and
global scientific community. It is composed of one academic institution
(University of Oslo), three research organisations (Max Planck Gesellschaft, Ifremer,
Simula), and one SME (QuantStack) based in three different countries (Norway,
France, Germany).

The consortium has developed through collaborations and common interests. Some
partners have been working together on different aspects of Jupyter development
(\site{QS}, \site{SRL}), software for education (\site{QS}, \site{SRL},
\site{MP}) and use of Jupyter tools for reproducible science (\site{SRL}, \site{MP})
for many years. Others contribute significant expertise in
the practice of open science and training (\site{IFR}, \site{UIO}).

Many participants (\site{MP}, \site{UIO}, \site{IFR}) have scientists involved who work on
facilitating computational and open science for scientists in their
institutions. As such, each of them has experience and a good overview of the
requirements for effective science and reproducible science from the many
research projects they are connected to. In addition, several of them are
research active in scientific domains, reproducibility and education.

The existing Binder tools -- which are the baseline for this project --
originate from Project Jupyter. We have core Jupyter and Binder developers in
our team, and thus direct access to developer expertise and experience.

Finally, we note that all project partners are long time passionate advocates of
Open and Reproducible Science; building on highly successful past experience
with OpenDreamKit, they \emph{have chosen to write this proposal fully in the
  open} on GitHub
(\href{https://github.com/minrk/horizon-widera-2022}{https://github.com/minrk/horizon-widera-2022})
for maximum transparency and engagement of the community. We have used the same
open source collaboration tools and practices as the Open Source Open Science
community.

\subsubsection{Complementarity and interdisciplinarity}

For the successful delivery of this project with its mission to enable better
reproducibility and science through better software (and services depending on
this), we need complementary expertise from researchers and research software
engineers. As we build on, improve and advance existing software tools from
Project Jupyter, it will be essential to know these well. As our approach will
provide automatic reproducible computational environments if best practice is
followed by the researchers, the education and training aspect for best practice
is also vital for this project.

The chosen consortium ensures a critical mass of scientific expertise and
excellence in key areas (such as natural sciences, education, software
engineering, Project Jupyter) with research organisations and SMEs of recognised
international reputation:
\begin{compactitem}
\item A set of use cases that cover several application domains and users, and that impose very diverse
requirements on open tools (\site{MP}, \site{IFR}, \site{UIO});
\item Lead developers in the Jupyter Ecosystem, including IPython, the Jupyter Notebook, JupyterLab,
JupyterHub, Binder, MyBinder.org, Jupyter Widgets (\site{SRL}, \site{QS})
\item Experts and major promoters of the Jupyter collaborative user interfaces
  for interactive, exploratory and reproducible computing in a variety of scientific domains (\site{MP}, \site{IFR}, \site{UIO});
\item A long experience and proven track record of success with large and complex collaborative projects,
including European E-Infrastructure projects (\site{MP}, \site{SRL}),
projects focused on large-scale infrastructures and large experimental services (\site{MP}, \site{IFR}),
as well as experience in running large scale open source projects (Jupyter
project, \site{SRL}, \site{QS});
\item Experience in educating students and experienced researchers on
  computational methods and open science (\site{SRL}, \site{MP}, \site{IFR}, \site{UIO});
\item A comprehensive range of skill sets and competencies in several relevant domains,
from applied research to standardisation to business analysis.
\end{compactitem}

\subsubsection{Capacities and roles of participants}

\paragraph{Simula Research Laboratory}

\site{SRL} is an internationally-leading Norwegian research institute in the key
ICT areas: communication systems, scientific computing and software
engineering. Dedicated to tackling scientific challenges with long-term impact and of
genuine importance to real life, Simula offers an environment that emphasises
and promotes basic research. This translates into numerous projects funded by
the EU, Norwegian government or regional institutions, that Simula was
involved in.

Benjamin Ragan-Kelley has contributed to the Jupyter Project since its
inception as a lead developer, and is now heading the Numerical Analysis and
Scientific Computing department at Simula. While continuing to contribute to the
open source software, he is also researching the effectiveness and usefulness of
such tools for education~\cite{JupyterHub-for-education-2016}, science and reproducible
science~\cite{binder,Forde2018ReproducibleRE,nbval-arxiv,repo2docker-checker2020,Beg2021}. 

Simula's role -- in addition to managing the overall project -- is to provide
technical leadership and in-depth expertise of the Jupyter and Binder project, which will be
instrumental in the execution of this project.

\paragraph{Max Planck Gesellschaft}
The Max Planck Society (\emph{Max Planck Society}, MPG) is non-profit research
organisation with 86 research institutes and nearly 24,000 staff. For this
project, we have Max Planck representatives from the Max Planck Compute and Data
Facility (MPCDF) -- the organisation's cross-institutional competence centre 
for computational and data sciences -- and staff from the Max Planck Institute
for Structure and Dynamics of matter (MPSD), who are active in condensed matter research and
reproducibility research.

% A bit odd to mention an individual here, but lots of the prior work
% was not done at MPG, so it may help reviewers to know the name.
One team member (Hans Fangohr) has a long-standing collaboration with \site{SRL}
and the Jupyter project, and a research interest in the use of open source
tools, Jupyter, and Binder for research and reproducible
research~\cite{Fangohr:ICALEPCS2017-TUCPA01,fangohr:icalepcs2019-tucpr02,nbval-arxiv,Beg2021}.
The MPCDF has experience deploying and using JupyterHub and BinderHub
installations for the many staff they support.

MPCDF and MPSD have long standing experience in delivering training and
workshops on computational methods, including best practice and reproducibility.
% not sure if we want to show case the following:
Hans Fangohr has won awards repeatedly for excellence in design and delivery of
teaching activities at different universities.
% https://fangohr.github.io/teaching/index.html#awards

In this project, the team will co-design Binder-based services for
reproducible science and data publishing. They will use the wide research
activities of scientists in the Max Planck Society -- including social sciences,
humanities and HPC-based activities -- to evaluate, improve and apply the Binder
tools for use cases such as data sharing and more reproducible HPC.

\paragraph{QuantStack}

QuantStack was founded in 2016 by a team of developers and maintainers of
key packages of the open-source scientific computing stack. QuantStack provides
support and custom development services in the Jupyter and Scientific Python
ecosystems. Clients and partners of QuantStack range from financial software
companies to robotics startups and public research institutions. The team
comprises several core developers of Jupyter subprojects and authors of popular
scientific computing and visualisation software used in both academic and
industrial contexts.

\TODO{Sylvain, are the following projects all open source? If so, please inject
  ``open source'' in the right place.}

Beyond Project Jupyter, projects developed at QuantStack include data
visualization packages for Jupyter such as bqplot, ipyvolume, ipyleaflet, and
ipysheet, as well as Jupyter language kernels such as xeus-cling and
xeus-python, and JupyterLab extensions like te draw.io and sidecar. QuantStack
is also behind the development of the xtensor framework, a high-level array
computing library and C++ dataframe.

In this project, QuantStack will provide expertise as core-Jupyter developers.

\TODO{Sylvian, -- can you add something on how the new capabilities will benefit
  QuantStack?}
  
\paragraph{Ifremer}

Ifremer, French National Institute for Ocean Science, is a French public
scientific and technological institution that works for exploring, understanding
and predicting the ocean. A pioneer in ocean science, Ifremer's cutting-edge
research is grounded in sustainable development and open science. Ifremer's
vision is to advance science, expertise and innovation by creating and sharing
ocean data, information \& knowledge. Ifremer hosts more than 1,500 personnel
spread along the French coastline in more than 20 sites.

Within this project, Ifremer will focus on testing, validating, improving and
applying the developed
tools for practical reproducibility in real-world research contexts.
\TODO{Tina, a bit more detail on the use cases, and the challenges?}

\paragraph{University of Oslo}

The University of Oslo (UiO) is Norway's oldest institution for research and
higher education, with 28,000 students and 6,000 employees. UiO aspires to be an
international hub for the research-based integration of computing into science
education and has financed a university-wide hosting service for Jupyter
notebooks through JupyterHub to introduce a computational aspect to all
curriculum programs in all science disciplines from bachelor to postdoctoral
studies.

The University of Oslo is a Silver Partner to \href{https://carpentries.org}{The
  Carpentries}, an international successful community driven project with
Instructors, Trainers, Maintainers, helpers, and supporters who share a mission
to teach foundational computational and data science skills to researchers. It
is also actively involved in the \href{https://coderefinery.org/}{CodeRefinery}
initiative that acts as a hub for FAIR (Findable, Accessible, Interoperable, and
Reusable) software practices. Twice a year, CodeRefinery organizes big online
training events with more then 300 attendees each.

The focus of UiO is to use their vast and leading experience in training 
and communication to educate researchers globally about open science and
practical reproducibility, to help translate the technical advances of this
project into wide-spread impact.

\TODO{Anne - is this an okay description?} 

\subsubsection{Connections beyond project partners}

As our ambition is to improve practical reproducibility for the \emph{global
  community of researchers}, we need to be well connected to understand
requirements and constraints from many domains. To improve our networking and
information gathering, we have started to compose our Community Engagement Panel
\taskref{management}{community-engagement-panel} with the aim to bring together
representatives from diverse research domains, research infrastructure
providers, research funders, publishers, educators, and policy makers. We expect
to also be able to use that network to support communication, dissemination and
exploitation of our results.

The project partners are research active in current topics of open science, and
are members in varies of research activities and organisations, including {Big
  data driven material science (BIGMax), Novel Materials Discovery (NOMAD), FAIR
  data infrastructure for Condensed-Matter Physics and the Chemical Physics of
  Solids (FAIRmat), Data Infrastructure Capacity for EOSC (DICE), Software
  Sustainability Institute.}

\TODO{All, Please list more. Feel free to add more relevant ones at the beginning.
  EOSC, FAIR, RDA, Carpentries, SSI, ?}

% \TOWRITE{ALL}{Add previous collaborations}

% joint software/database development
% Jupyter Project software

\jointsoft{QS,SRL}
% \jointsoft{WTT,SRL}
% \jointsoft{WTT,XFEL}
% 
% % Binder
% \jointsoft{SRL,WTT}
% 
% % k3d
% \jointsoft{SIL,SRL}
% \jointsoft{SIL,XFEL}
% \jointsoft{SRL,XFEL}
% 
% % nbval
\jointsoft{MP,SRL}
% 
% %%s
% 
% % OpenDreamKit: UPSUD, SIL, XFEL, SRL
% \jointproj{XFEL,UPSUD}
\jointproj{MP,SRL}
% \jointproj{XFEL,SIL}
% 
% \jointproj{UPSUD,SRL}
% \jointproj{UPSUD,SIL}
% 
% \jointproj{SRL,SIL}
% \jointproj{UIO,SIL}
% % Jupyter project publication ? XXX TIM
% 
% % Binder
% \jointsoft{SRL,WTT}
% 
% % research bazaar
\jointproj{SRL,UIO}
% 
% % \jointpub{A,B} % some publication
% 
% %joint supervision
% % \jointsup{A,B} %
% 
% %joint organization
% % \jointorga{A,B} % some org
% % \jointorga{SA,UJF} % PASCO'15
% 
% % joint publications
% % \jointpub{A,B} % some publication
% 
% % Jupyter publication
% \jointpub{SRL,XFEL}
% \jointpub{SRL,QS}
% \jointpub{XFEL,QS}

\coherencetable[swsites]

%%% Local Variables:
%%% mode: latex
%%% TeX-master: "proposal"
%%% End:

%%% Local Variables:
%%% mode: latex
%%% TeX-master: "proposal"
%%% End:

\draftpage

%AF\section{Technical appendix}\label{sec:appendix}
\subsection{Terminology}\label{sec:terminology}

For clarity, we define some terms that have been used
throughout this proposal. We illustrate the terms in the context of the
reproducibility example in Section~\ref{sec:reproducibility-example}.

In particular, we imagine that we have a publication that contains
figure~\ref{fig:reproducibility-example-covid} as a result, and we want to
archive and make available \cite{ReproducibilityRepositoryExample2022} the
necessary information for others to reproduce that figure.

\begin{description}
\item[Repository] We refer to a \emph{repository} as a collection of files.

The \emph{purpose} of the repository in our context is to archive information to make
research results reproducible.

Such a repository could be a git repository (which could be hosted on GitHub,
Bitbucket, GitLab, or other services), but it could also just be a zip file of a
collection of files.

The repository could be made publicly available, for example through Zenodo,
Figshare, as an electronic supplementary to a publication, or through GitHub.

It is not unusual that a larger number of files might be
organised in subdirectories within the repository.

Our example repository\footnote{
  https://github.com/fangohr/reproducibility-repository-example/} \cite{ReproducibilityRepositoryExample2022} contains the following files:

\begin{description}
\item[\softwarename{README.md}]: an overview of the content of the repository
\item[\softwarename{figure1.ipynb}]: notebook that creates
  \softwarename{figure1.pdf} from
  the raw data. Could also be realised through a script or other executable
  program of some kind.
\item[\softwarename{requirements.txt}]: software specification
\item[\softwarename{time\_series\_covid19\_deaths\_global.csv}]: raw data
\end{description}

\item[Script] A machine-executable file (for example a Bash, Python, Perl
script, Makefile or similar). Such scripts can execute data processing commands,
and are often part of a repository to make the (automatic) reproduction of
results possible.

In our example, the necessary steps to create the figure from the raw data are
gathered in the notebook \texttt{figure1.ipynb}.

\item[Notebook] A Jupyter notebook. In short, an executable document that can
  combine text, code, and computation results. A Jupyter notebook can be used like a
  script. The notebook is explained in Section \fullref{sec:jupyter-notebook}.

\item[Software specification] The software specification depends on the software
  tools used. In our example, the \texttt{requirements.txt} file contains
\begin{verbatim}
pandas==1.3.4
matplotlib==3.4.3
\end{verbatim}
to indicate that we need the pandas and matplotlib package with the respective
versions 1.3.4 and 3.4.3.

\item[Software environment] All the software that needs to be available and
  installed to execute the scripts and (and if desired) Jupyter notebooks.

\item[Project Binder] The Project Binder is described in Section
\ref{seq:project-binder}. It is part of the Jupyter ecosystem of tools, and
allows to convert a repository with Jupyter notebooks into an browser-hosted
environment, in which the notebooks can be executed interactively (and thus
results can be reproduced).

\item[Binder tools] The Binder tools consist of \binderhub{}
  (Section~\ref{sec:binderhub}) and \repotodocker{} (see
  Section~\ref{sec:repo2docker}) .

\item[\repotodocker] \repotodocker{} is a tool that can automatically create a
\emph{software environment} (currently within a Docker container) in which the
notebooks and scripts of a repository can be executed
(Section~\ref{sec:repo2docker} and \ref{binder-how-does-it-work}).
\end{description}

\subsubsection{Project Jupyter and the surrounding ecosystem}
\label{sec:project-jupyter}

\begin{figure}[htb]\centering
  \includegraphics[width=0.9\textwidth]{use-cases-binder-logbook-solution.png}
  \caption{A typical use case for Jupyter notebooks in research.
            Image by Juliette Belin for the OpenDreamKit project, used under
            CC-BY-SA.}\label{fig:use-cases-binder}
\end{figure}

\noindent\textbf{Jupyter ecosystem as the root of \TheProject}

\TheProject has chosen to centre its efforts on the Jupyter software
ecosystem, in particular Binder and repo2docker.
Figure~\ref{fig:use-cases-binder} summarises a typical use
case of Jupyter Notebook and Binder;
both are described in more detail below.

The Jupyter notebook and Jupyter ecosystem are of increasing
importance in computational science and data science, in academia,
industry, and services. In addition to supporting high productivity of
researchers, they have great potential to push Open and Reproducible Science forward:
the notebook provides a complete description of a computational and
data science study (Step 1 in figure~\ref{fig:use-cases-binder}), and the notebook can -- in principle -- be turned
into a publication, or can be used to provide the required computation
for a part of a publication, such as a figure
(Step 2 in figure~\ref{fig:use-cases-binder}). Once the researcher has
specified what software is required to execute the notebook (Step 3
in figure~\ref{fig:use-cases-binder}), the study is completely
reproducible by anyone (Step 4 in figure~\ref{fig:use-cases-binder}).

In this way, the notebook \textbf{enables reproducibility} of complex tasks
with minimal additional effort on the user side.
The Binder project allows to execute such notebooks in
tailored computational environments; an aspect of reproducibility that
is not widely supported yet,
and a great opportunity for improving best practices in Open Science.

Furthermore, for users wanting to connect
to a local Jupyter notebook server on their machine, or to connect to
a server somewhere else on the Internet, the users only need a
web-browser to display and use the notebook regardless of the location
of the notebook server,
allowing computation to run anywhere from a local laptop to a remote supercomputer or in the cloud.
Because of these characteristics,
the Notebook is already planned to become an
important service on the European Open Science Cloud (EOSC) (for
example in \cite{panosc}),
and is an ideal component to use when building Open Science Services.

\medskip\noindent\textbf{Project Jupyter}

\emph{Project Jupyter} \cite{Jupyter}, which has grown increasingly popular in the scientific
computing community, has become the \emph{lingua franca} of interactive
computing in both academia and industry. The main goal of Project Jupyter
is to provide a consistent set of tools to improve researchers'
workflows from the exploratory phase of the analysis to the communication
of the results \cite{Kluyver2016}.

Split in 2014 from the \emph{IPython Project} \cite{IPython}, Jupyter has grown rapidly in
popularity and adoption both in the industry and academia. We estimate the user
base of the Jupyter notebook to be in the millions \cite{jupyter-grant}. Users range from data
scientists to researchers, educators, and students from many fields,
including journalists and librarians. In 2017, the Jupyter
team was awarded the \emph{ACM Software System Award}, an annual award that
honors people or an organization \emph{"for developing a software system that had a
lasting influence"}. Prior recipients include \emph{Unix}, \emph{TCP/IP}, and
the \emph{World Wide Web} \cite{acm-award}.

A large number of discrete software components make up Project Jupyter.
While these interact with one another, many can be installed separately
to serve various use cases. For this proposal, we loosely divide the
software involved into \emph{Jupyter core} developed under the guidance
of the developers who started the project, and the broader \emph{Jupyter
ecosystem} including software developed by third parties,
which may interact or build upon core Jupyter components.
Some of the components and concepts important to \TheProject are detailed below.

\begin{figure}[ht]\centering
  \centering
  \includegraphics[width=0.9\textwidth]{spectrogram_smaller.png}
  \caption{A notebook document in the Jupyter Notebook interface.}\label{fig:notebook-screenshot}
\end{figure}

\medskip\noindent\emph{Jupyter core}
\begin{itemize}
  \item \label{sec:jupyter-notebook} The \textbf{Jupyter Notebook} is the flagship application of Project Jupyter.
  It allows the creation of notebook documents, containing a mixture of text and
  interactively executable code, along with rich output from running that code.
  Figure \ref{fig:notebook-screenshot} shows an open notebook including graphs
  from an audio processing example. Notebook documents are readily shareable,
  providing a popular way to describe and illustrate computational methods and
  tools. \TODO{We should update the notebook: (i) point to a github repo with
    it, and (ii) binder-enable it, and (iii) increase the pixel resolution of
    the screen shot. Madison?}
  \textbf{JupyterLab} is the new, modular, extensible client application
  for Jupyter notebooks, but the document format, server, and user model are the same.

  \item \textbf{Jupyter kernels} are the backend software which allow Jupyter to execute
  code in many different programming languages. The \textbf{IPython} kernel is
  the reference kernel, supporting the Python programming language, and is
  developed by the Jupyter core team. Kernels for other languages are maintained
  by third parties

  \item \textbf{JupyterHub} is a multi-user extension of the Jupyter Notebook.
  It runs on one or more notebook servers, for example at a research institution.
  Users can log in to author and run notebooks securely through their web
  browser, without needing to install any special software on their own
  computer.

\end{itemize}

\medskip\noindent\emph{Jupyter ecosystem}\label{jupyter-ecosystem}

While Jupyter is a large, distributed, coordinated project,
the wider community of Jupyter users develops a great deal of
software with Jupyter integration,
providing increased or domain-specific functionality,
building on top of Jupyter, or integrating core Jupyter components in some aspect.
We call this the \textbf{Jupyter ecosystem}.
The broader Jupyter ecosystem includes many more projects than we will describe
here, but a selection of projects which are relevant to
\TheProject includes:

\begin{itemize}
  \item \textbf{Binder} builds on JupyterHub to allow sharing executable
  environments along with data files and a description of the software components
  required to run the notebooks. When someone accesses a Binder repository,
  the service builds the computational environment on demand, allowing them to
  execute and modify a copy of the notebooks.
  \textbf{repo2docker} \cite{repo2docker} and \textbf{BinderHub} \cite{binder} are components of the Binder
  software. \TOWRITE{}{More here, as repo2docker is key}
\end{itemize}

\begin{figure}[ht]\centering
  \includegraphics[width=0.5\textwidth]{ipywidgets_example.png}
  \caption{An example of using two simple slider widgets to explore the
  parameter space of a function. The \texttt{@interact} decorator creates
  the widgets and connects them to the function.}
  \label{fig:ipywidgets-example}
\end{figure}


\subsubsection{Project Binder}\label{seq:project-binder}

The Binder Project \cite{binder} (\url{https://jupyter.org/binder}) is
a subproject of the Jupyter project. The Binder project is formally operating
within the Jupyter ecosystem, but not confined to be useful only for notebooks.

The key components of the Binder software are \repotodocker{}
(Section~\ref{sec:repo2docker}) and \binderhub{}. The \repotodocker{} tool
creates a software environment inside a Docker container from a software
specification in a repository. \binderhub{} starts a Jupyter notebook server
within this container from which the user can execute the notebooks from the
repository.

\emph{\mybinder{}} (see \ref{sec:mybinder}) is a service provided by the \emph{BinderHub
  federation} that collectively host a service running the Binder software
under the URL \url{https://mybinder.org}. This is the service we made use of in
our example in section~\ref{sec:reproducibility-example}.

The focus for this proposal is to improve \repotodocker{}. In particular,
\repotodocker{} solves software environment challenge (see
Section~\ref{sec:reproducibility-concept}) in generic way and is independent
from Jupyter notebooks.

\TODO{Add image that depicts this Jupyter/Binder relation ship}.

\paragraph{Basic functionality of Binder}
\label{binder-how-does-it-work}

The currently most common reproducibility use case -- with the current state of the Binder
tools -- is the one we introduced in
Section~\ref{sec:reproducibility-example}. We will use this to
Section~\ref{sec:terminology}. We will use this to
describe the role of the individual components of Binder:


\begin{figure}
  \begin{minipage}[b]{0.67\textwidth}
    \includegraphics[width=1.0\textwidth]{images/mybinder.png}
  \end{minipage}\hfill
  \begin{minipage}[b]{0.3\textwidth}
    \caption{Home page of the \mybinder{} service.}  \label{fig:mybinder-homepage}
  \end{minipage}
\end{figure}


\begin{compactitem}
\item The \mybinder{} service is called with a URL that encodes the location of the data
  repository\footnote{For example the 
    {\url{https://mybinder.org/v2/gh/fangohr/reproducibility-repository-example/HEAD?labpath=figure1.ipynb}}
    refers to the GitHub repository ``reproducibility-repository-example'' of the
    github user ``fangohr'', asking to open the ``figure1.ipynb'' file.}

  Alternatively, there is form (see Figure~\ref{fig:mybinder-homepage})
  which users can complete with repository details
  to start the build of the corresponding environment, or to obtain the URL to
  re-use that configuration later, or share it with others.
\item From the \mybinder{} entry point, the request is forwarded to one
  \binderhub{} service of one of the organisations in the BinderHub federation
  that has available compute capacities.
\item The \binderhub{} software running at the chosen location, will ask
  \repotodocker{} to create a docker container in which the notebooks from the repository can be executed.
\item \repotodocker{} searches the repository for specifications of software requirements (see \ref{repo2docker-supported-software-specifications}).
\item \repotodocker{} composes a Dockerfile that contains all the commands
  necessary to install software.
\item \repotodocker{} builds the Docker image based on the Dockerfile
\item \binderhub{} takes the Docker image and asks Kubernetes to start up
  container based on this image
\item The notebook server is started in this Docker container.
\item \binderhub{} forwards the user who requested this virtual environment to
  the URL at which the repository (or a particular notebook) can be explored
  from within the Jupyter notebook (which runs in the container).
\end{compactitem}

\paragraph{The repo2docker software tool.}\label{sec:repo2docker}

\repotodocker{} is a tool to fetch a remote repository and build a software
environment for this repository. Currently, \repotodocker{} can retrieve
repositories from the following services and formats: GitHub, Gist, Git, GitLab,
Zenodo, Hydroshare, Figshare, Dataverse.

For the automatic building of reproducible computational environments,
\repotodocker{} understands commonly used conventions for environment specifications and
community standard tools such as Docker, conda, and pip. See
Section~\ref{repo2docker-supported-software-specifications} below for a full list of
currently supported software specifications.

\paragraph{Supported software specification formats}
\label{repo2docker-supported-software-specifications}
The \repotodocker{} tool currently supports the following software specification
formats to build Docker images:
% source:
% https://repo2docker.readthedocs.io/en/latest/config_files.html#config-files
% 9 April 2022
\begin{compactitem}
\item \softwarename{requirements.txt}, \softwarename{setup.py},
  \softwarename{Pipfile}, \softwarename{Pipfile.lock}: to specify Python
  packages and environments
\item \softwarename{Project.toml}, \softwarename{JuliaProject.toml} and (legacy)
  \softwarename{REQUIRE}: to
  specify Julia version and packages
\item \softwarename{install.R}, \softwarename{DESCRIPTION}: to install R
  libraries, or install the repository as R package
\item \softwarename{apt.get}: to install Debian packages. The Docker container
  is currently based on Ubuntu, which uses the Debian package management tool \softwarename{apt}.
\item \softwarename{environment.yml}: to specify conda packages and
  environments
\item \softwarename{default.nix}: to use the nix package manager for software provision
\item \softwarename{Dockerfile}: providing a Dockerfile enables users to define
  virtually arbitrary environments, for example based on software from the
  repositories of Linux distributions.
\end{compactitem}

\paragraph{BinderHub}\label{sec:binderhub}
BinderHub is software for hosting a web service built on \repotodocker{} and
JupyterHub where individuals can share reproducible environments for
immediate and free interaction by readers in their browser.


\paragraph{The \mybinder{} service}\label{sec:mybinder}

\emph{\mybinder{}} is a service run by the \emph{BinderHub
  federation}\footnote{\url{https://mybinder.readthedocs.io/en/latest/about/federation.html}}
of organisations that collectively host a service running the Binder software
under the URL \url{https://mybinder.org}.

The service is actively used with approximately 200,000 binder sessions being
requested and delivered by the \mybinder{} service every week in 2021. The number
of sessions is growing from approximately 10,000 per day in November 2018
(beginning of the available records) to about 30,000 per day in 2022. We have
identified 60,000 unique repositories published in the last few years which have
used the binder service. The data is available~\cite{mybinder-archive}.

This project will not provide or operate a BinderHub service (such as the global
``\url{https://mybinder.org}'' instance). The improvements achieved, however, will immediately
be made available to all operators of BinderHubs, including \mybinder{}.

\subsubsection{Binder for reproducibility}\label{sec:binder-for-reproducibility}
\TOWRITE{}{Hmm -- perhaps we don't need this section, as we explain in the
  Methodology section what we want to do with Binder?}




%%% Local Variables:
%%% mode: latex
%%% TeX-master: "proposal"
%%% End:


% Not relevant here. Should remove that file later.
% % This section is probably not needed for SOURCE.

\medskip
\noindent\textbf{Jupyter as a basis for web services}\\
Because the Jupyter notebook is a web-based application, it can be
deployed at computational facilities or in the cloud, and can function
as the basis for services exposing computational resources of all
kinds to researchers and the public.  Because Jupyter is
\textbf{interactive}, it enables making scientific results and
communications more interactive than static publications.  The
audience can follow their own initiative and ask their own questions
of published data without needing support from the publishing author,
greatly facilitating the \textbf{practicality of Open Science}.

\medskip
\noindent\textbf{Jupyter is generic}\\
\TheProject chose Jupyter because it is
Generic.  Jupyter makes no domain-specific or even language-specific
assumptions.  Any application where mixing description, code, and
results is valuable can make use of Jupyter.  This broad applicability
makes investment in the Jupyter ecosystem extremely effective, because
improvements to Jupyter can serve many communities simultaneously.

Jupyter is built from a collection of standard protocols and file
formats.  Jupyter is not just a single, monolithic piece of
software, but a description of how such software can be built.  The
result is the ability for a variety of communities and applications to
use components of Jupyter for their purposes, and/or reimplement pieces to
meet their needs.
%
For example:
\begin{enumerate}
\item The notebook file format is a well-specified JSON document,
  which can be interpreted by many systems.  This has facilitated the
  development of different services providing rendering of notebooks, e.g. the code
  hosting website GitHub, which renders notebooks for easy viewing by
  anyone, without Jupyter software.
\item The Jupyter protocol describes how execution is performed, which
  has enabled the development of over one hundred kernel
  implementations in dozens of languages\footnote{\url{https://github.com/jupyter/jupyter/wiki/Jupyter-kernels}}.
\item Output in the Jupyter protocol uses web-standard MIME types,
  enabling any possible format to be an output in a Jupyter notebook.
\item The JupyterLab extension system provides a system for building
  applications from Jupyter components and others.
\item The Jupyter Widgets provide a system for customizing and
  extending interactivity in Jupyter-based environments.
\end{enumerate}

The popularity of Jupyter, with millions of users and hundreds of open
source contributors, is an indicator of the value and impact of this approach.

\medskip
\noindent\textbf{Improvement to the Jupyter ecosystem}\\
The benefits of focusing our work on a mature system like Jupyter include:

\begin{itemize}
\item vibrant community ensures health and sustainability,
\item large existing user base maximises impact of contributions,
\item mature software ecosystem maintains quality software through
  industry standards such as version control, tests, continuous
  integration, stable release cycles, roadmaps, and user support.
\end{itemize}

The Jupyter community aims to be inclusive, and \TheProject fully
embraces and supports that approach.  Jupyter is inclusive across a number of axes.
By being applicable across numerous domains, Jupyter and \TheProject
encourage participation from individuals of various interests and
backgrounds, and has taken action to improve diversity in the project
by participating in ``Outreachy,'' a program of paid internships for
individuals from groups that face under-representation, systemic bias,
or discrimination.  Jupyter has also operated workshops focused on
training contributors from under-represented groups.  In being free,
public, open source software, Jupyter and \TheProject are accessible
to as many individuals as possible, and invites users and contributors
beyond origin, nationality, beliefs, orientation.  One area where
Jupyter has lacked in this regard is in the User Interface
accessibility, and we will help improve this in
% \taskref{core}{accessibility}
.  Additionally, the project will
focus some of its workshops in
% \taskref{education}{workshops}
on
under-represented communities.


\begin{figure}[ht!]\centering
  \includegraphics[width=0.6\textwidth]{images/notebook_components.png}
  \caption{The architecture of the Jupyter Notebook, kernels, and tools
        which operate on notebook files}
  \label{fig:notebook-architecture}
\end{figure}




\end{proposal}
\end{document}

%%% Local Variables:
%%% mode: latex
%%% TeX-master: t
%%% End:

%  LocalWords:  sud logilab urich Simula thiery acrolong igital esearch nvironments pn wp
%  LocalWords:  athematics pnlong callname callid challengeid objectiveid outcomeid emph
%  LocalWords:  compactht newcommand tableofcontents Linbox IPython textbf eucommentary
%  LocalWords:  vre TOWRITE citability Cython Laboratoire Recherche Informatique devs WPs
%  LocalWords:  clearpage draftpage programme workplan subsubsection pdatacount wplist sc
%  LocalWords:  WPref dissem pageref newpage sssec hline ganttchart xscale makeatletter
%  LocalWords:  makeatother wpfigstyle footnotesize tabcolsep wpfig inputdelivs mgt smc
%  LocalWords:  mathsoftware mathdb mathknowledge Jupyter silesia pythran Pythran ldots
%  LocalWords:  Simulagora stigmatisation compactenum planetmath.org Univ botupPM Gnuplot
%  LocalWords:  boxedminipage textwidth compactitem fangohr providecommand classoptions
%  LocalWords:  ifsubmit setcounter tocdepth neighbouring incentivesed Gowers analyse hpc
%  LocalWords:  incentivised Ebay taskref structdocs taskref minimising parallelisation
%  LocalWords:  dksbases decisionmaking oommf-nb-evaluation gantttaskchart yscale Belabas
%  LocalWords:  Boussicault endeavours GitHub isocial-decisionmaking enlargethispage
