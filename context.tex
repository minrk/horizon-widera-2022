\subsubsection{Context - Jupyter and Binder}\label{sec:context}

\begin{figure}[!htb]
    \centering
    \begin{minipage}{.48\textwidth}
        \centering
        \includegraphics[width=\linewidth]{images/spectrogram.png}
  \caption{A notebook document in the Jupyter Notebook interface.}
    \label{fig:notebook-screenshot}

    \end{minipage}%
    % some margin between images
    \hspace*{.04\textwidth}%
    \begin{minipage}{0.48\textwidth}
        \centering
        \includegraphics[width=0.9\linewidth]{images/mybinder.png}
        \caption{Home page of the \mybinder{} service, an instance of \textbf{BinderHub}.}
        \label{fig:mybinder-homepage}
    \end{minipage}
\end{figure}

The implementation of our approach to reproducibility
is focused on the Binder and Jupyter projects,
which are presented here at a high level.
In particular, we focus on the \textbf{Binder tools}
for \textbf{reproducible software environments}.
In this section, we introduce key components of Project Jupyter and Binder to help
contextualise the work we plan to do,
as well as some terminology

\paragraph*{Relationship between \TheProject and Jupyter, Binder}

Several \TheProject team members have a track record as
developers and contributors to Project Jupyter, Binder and its associated
ecosystem, including two current and founding members of the Jupyter Steering Committee,
and the leader of the Binder subproject.
The proposed work has the potential to improve the reproducibility of almost all computational research,
not just those that choose to use notebooks.


\definecolor{terminologyFrame}{rgb}{0.5, 0.5, 0.5}
\definecolor{terminologyBack}{rgb}{0.9, 0.9, 0.9}

\newtcolorbox{terminology}[1]{
    breakable,
    colback=terminologyBack,
    colframe=terminologyFrame,
    fonttitle=\bfseries,
    title={Terminology #1}
}

\begin{terminology}{}
For clarity, we define some terms that have been used
throughout this proposal. We illustrate the terms in the context of the
reproducibility example in \nameref{sec:reproducibility-example}.

\begin{description}
\item[Repository] We refer to a \emph{repository} as a collection of files.

The \emph{purpose} of the repository in our context is to collect/archive/transport information to make
research results reproducible.
Such a repository could be a \softwarename{git} repository (hosted on GitHub or other services),
but it could also be a simple zip file.
A repository typically contains some of these items:

\begin{compactitem}
  \item \textbf{Documentation}
  \item \textbf{Script}
  \item \textbf{Software specification}
  \item \textbf{Data}
\end{compactitem}

The repository could be made publicly available,
for example through Zenodo
as an electronic supplementary to a publication,
or through GitHub.

\item[Script] A machine-executable file (for example a Bash, Python, Perl
script, Makefile or similar). Such scripts can execute data processing commands,
and are often part of a repository to make the (automatic) reproduction of
results possible.

In our example, the necessary steps to create the figure from the raw data are
gathered in the notebook \texttt{figure1.ipynb}.

\item[Software specification] A description of what software is needed to run the \textbf{Script}.
  Ideally, but \emph{not necessarily} machine readable.
  The exact form of the software specification depends on the software tools used.
  In our example, a \texttt{requirements.txt} file indicates the \softwarename{pip} tool is used.

\item[Software environment] All the software that needs to be available and
  installed to execute the scripts and (and if desired) Jupyter notebooks.
  This is generally produced from the \textbf{Software specification}.
\end{description}
\end{terminology}


\paragraph*{Project Jupyter}
\label{sec:project-jupyter}
Project Jupyter \cite{Jupyter}, which has grown increasingly popular in the scientific
computing community, has become the \emph{lingua franca} of interactive
computing in both academia and industry \cite{Perkel2018}. The main goal of Project Jupyter
is to provide a consistent set of tools to improve researchers'
workflows from the exploratory phase of the analysis to the communication
of the results \cite{Kluyver2016,Granger2021}.
Jupyter has grown rapidly in popularity and adoption both in the industry and academia.
We estimate the user base of the Jupyter notebook to be in the millions
\cite{jupyter-grant}. Users range from data scientists to researchers,
educators, and students from many fields, including journalists and librarians.
The number of publicly hosted Notebook documents is exceeding 8~million~\cite{notebookcount}.

The \textbf{Jupyter Notebook} is the flagship application of Project Jupyter.
It allows the creation of notebook documents, containing a mixture of text and
\textbf{interactively executable code}, along with rich output from running that code.
Figure~\ref{fig:notebook-screenshot} shows an open notebook including graphs
from an audio processing example. Notebook documents are readily shareable,
providing a popular way to describe and illustrate computational methods and
tools. \TODO{We should update the notebook: (i) point to a GitHub repo with it,
  and (ii) binder-enable it. Sources are in data/notebook-figure}


\paragraph*{Jupyter notebook and reproducibility}
While Jupyter notebooks can make computational and data-driven research more effective
\cite{Perkel2018,Fangohr2020,Granger2021}, they also have great potential to push
open and reproducible science forward \cite{Beg2021}. The notebook provides a complete
description of a computational study (Step 1 in
figure~\ref{fig:use-cases-binder}), and the notebook can -- in principle -- be
turned into a publication, or can be used to provide the required computation
for a part of a publication, such as a figure (Step 2 in
figure~\ref{fig:use-cases-binder}). Once the researcher has specified what
software is required to execute the notebook (Step 3 in
figure~\ref{fig:use-cases-binder}), the study is completely reproducible by
anyone (Step 4 in figure~\ref{fig:use-cases-binder}).

In this way, the notebook \emph{enables reproducibility} of complex workflows
with minimal additional effort on the user side. This approach is used by a
substantial number of scientists for publications already (for example
\cite{GitHubRepoExampleAlbert2016,GitHubRepoExampleCortes2018,Beg2019-blochpoint-data-repository, }
\TODO{insert publications with reproducible repositories - the LIGO paper?
  Anything else}): it is hard to prove
but it seems plausible that a significant fraction of the 30,000 sessions
triggered on the mybinder.org service every day are used for reproducible repositories
(Section~\ref{sec:mybinder}).

% HF: this is a reference to the Joel Grus criticism. Not sure if we need it.
% The paper by Beg2021 addresses that in section 9.
(We note in passing that the use of Jupyter notebooks alone does not guarantee
reproducibility: it requires some training and/or experience to correctly specify a
computational environment and to capture that information in a machine readable
way. It is also important to include all computational steps in the notebook in
order~\cite{Beg2021}.)

\begin{terminology}{(Jupyter)}
\begin{description}
\item[Project Jupyter] The over-arching term used to refer to the large collaboration of open source
  tools for developing and sharing computational ideas

\item[Notebook] A Jupyter Notebook. An executable document that can
  combine text, code, and outputs, such as figures.
  A Jupyter notebook can serve the same function as the \textbf{Script}.

\item[JupyterLab] The flagship application of Project Jupyter.
  Allows \textbf{interactive execution} and creation of
  \textbf{Notebook} documents.
  Built on web technology,
  can be run on any computer resources on a network,
  from laptops to cloud to supercomputers.

\item[JupyterHub] The multi-user extension of the Jupyter Servers, such as JupyterLab.
  JupyterHub runs one or more JupyterLab instance for each user,
  and can run on any \emph{shared} computational infrastructure,
  from home servers to the cloud to supercomputers~\cite{Fangohr2020}.

\end{description}
\end{terminology}

\subsection{Project Binder}\label{seq:project-binder}

The Binder Project \cite{binder} (\url{https://jupyter.org/binder}) is
a subproject of the Jupyter project. The Binder project is formally operating
within the Jupyter ecosystem, but not confined to be useful only for notebooks.

The key components of the Binder software are \repotodocker{}
(Section~\ref{sec:repo2docker}) and \binderhub{}. The \repotodocker{} tool
creates a software environment inside a Docker container from a software
specification in a repository. \binderhub{} starts a Jupyter notebook server
within this container from which the user can execute the notebooks from the
repository.

\emph{\mybinder{}} (see \ref{sec:mybinder}) is a service provided by the \emph{BinderHub
Federation} that collectively host a service running the Binder software
under the URL \url{https://mybinder.org}. This is the service we made use of in
our example in section~\ref{sec:reproducibility-example}. The \emph{BinderHub Federation}

The focus for this proposal is to improve \repotodocker{}. In particular,
\repotodocker{} solves the software environment challenge (see
Section~\ref{sec:reproducibility-concept}) in a generic way and is independent
from Jupyter notebooks.

\TODO{Add image that depicts this Jupyter/Binder relation ship}.

\subsubsection{Basic functionality of Binder}
\label{binder-how-does-it-work}

The currently most common reproducibility use case -- with the current state of the Binder
tools -- is the one we introduced in
Section~\ref{sec:reproducibility-example}. We will use this to
describe the role of the individual components of Binder:


\begin{figure}[ht]
  \centering
    \includegraphics[width=0.7\textwidth]{images/mybinder.png}
    \caption{Home page of the \mybinder{} service.}
    \label{fig:mybinder-homepage}
\end{figure}


\begin{compactitem}
\item The \mybinder{} service is called with a URL that encodes the location of the data
  repository\footnote{For example the
    {\url{https://mybinder.org/v2/gh/fangohr/reproducibility-repository-example/HEAD?labpath=figure1.ipynb}}
    refers to the GitHub repository ``reproducibility-repository-example'' of the
    github user ``fangohr'', asking to open the ``figure1.ipynb'' file.}

  Alternatively, there is form (see Figure~\ref{fig:mybinder-homepage})
  which users can complete with repository details
  to start the build of the corresponding environment, or to obtain the URL to
  re-use that configuration later, or share it with others.
\item From the \mybinder{} entry point, the request is forwarded to one
  \binderhub{} service of one of the organisations in the BinderHub Federation
  that has available compute resources.
\item The \binderhub{} software running at the chosen location, will ask
  \repotodocker{} to create a Docker container in which the notebooks from the repository can be executed.
\item \repotodocker{} searches the repository for specifications of software requirements (see \ref{repo2docker-supported-software-specifications}).
\item \repotodocker{} composes a Dockerfile that contains all the commands
  necessary to install software.
\item \repotodocker{} builds the Docker image based on the Dockerfile.
\item \binderhub{} takes the Docker image and asks Kubernetes to start
  a container based on this image.
\item The notebook server is started in this Docker container.
\item \binderhub{} forwards the user who requested this virtual environment to
  the URL at which the repository (or a particular notebook) can be explored
  from within the Jupyter notebook (which runs in the container).
\end{compactitem}

\subsubsection{The repo2docker software tool.}\label{sec:repo2docker}

\repotodocker{} is a tool to fetch a remote repository and build a software
environment for this repository. Currently, \repotodocker{} can retrieve
repositories from the following services and formats: GitHub, Gist, Git, GitLab,
Zenodo, Hydroshare, Figshare, Dataverse.

For the automatic building of reproducible computational environments,
\repotodocker{} understands commonly used conventions for environment specifications and
community standard tools such as Docker, conda, mamba, and pip. See
Section~\ref{repo2docker-supported-software-specifications} below for a full list of
currently supported software specifications.

\subsubsection{Supported software specification formats}
\label{repo2docker-supported-software-specifications}
The \repotodocker{} tool currently supports the following software specification
formats to build Docker images:
% source:
% https://repo2docker.readthedocs.io/en/latest/config_files.html#config-files
% 9 April 2022
\begin{compactitem}
\item \softwarename{requirements.txt}, \softwarename{setup.py},
  \softwarename{Pipfile}, \softwarename{Pipfile.lock}: to specify Python
  packages and environments
\item \softwarename{Project.toml}, \softwarename{JuliaProject.toml} and (legacy)
  \softwarename{REQUIRE}: to
  specify Julia version and packages
\item \softwarename{install.R}, \softwarename{DESCRIPTION}: to install R
  libraries, or install the repository as R package
\item \softwarename{apt.get}: to install Debian packages. The Docker container
  is currently based on Ubuntu, which uses the Debian package management tool \softwarename{apt}.
\item \softwarename{environment.yml}: to specify conda or mamba packages and
  environments
\item \softwarename{default.nix}: to use the nix package manager for software provision
\item \softwarename{Dockerfile}: providing a Dockerfile enables users to define
  virtually arbitrary environments, for example based on software from the
  repositories of Linux distributions.
\end{compactitem}

\subsubsection{BinderHub}\label{sec:binderhub}
BinderHub is software for hosting a web service built on \repotodocker{} and
JupyterHub where individuals can share reproducible environments for
immediate and free interaction by readers in their browser.


\subsubsection{The \mybinder{} service}\label{sec:mybinder}

\emph{\mybinder{}} is a service run by the \emph{BinderHub
Federation}\footnote{\url{https://mybinder.readthedocs.io/en/latest/about/federation.html}}
of organisations. Collectively, they host a service running the BinderHub software
which can be reached from \url{https://mybinder.org}.

The service is actively used with approximately 200,000 sessions being
requested and delivered by the \mybinder{} service every week in 2021. The number
of sessions is growing from approximately 10,000 per day in November 2018
(beginning of the available records) to about 30,000 per day in 2022. We have
identified 60,000 unique repositories published in the last few years which have
used the \mybinder{} service. The data is available~\cite{mybinder-archive}.

Examples of reproducible repositories that make use of the \mybinder service
include reproducible research repositories
\cite{GitHubRepoExampleAlbert2016,Beg2021}, interactive textbooks
\cite{Fangohr2022,Zeller2022} and citizen science and outreach activities
\cite{ligo-open-science,OSCOVIDA2022}.

This \TheProject{} project will not provide or operate a BinderHub service (such as the global
``\url{https://mybinder.org}'' instance). The improvements achieved, however, will immediately
be made available to all operators of BinderHubs, including \mybinder{}.


% \subsubsection{Binder for reproducibility}\label{sec:binder-for-reproducibility}
% \TOWRITE{}{Hmm -- perhaps we don't need this section, as we explain in the
%   Methodology section what we want to do with Binder?}




%%% Local Variables:
%%% mode: latex
%%% TeX-master: "proposal"
%%% End:

