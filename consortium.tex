%\TO WRITE{ALL}{Proofread 3.4 consortium pass 2 [Done by Hans]}
%remove this, as we have more pressing things left.

\eucommentary{
The individual members of the consortium are described in a
separate section under Part A. There is no need to repeat that
information here.
\begin{itemize}
\item
  Describe the consortium. How does it match the
\item
  Describe the consortium. How does it match the project's objectives,
  and bring together the necessary disciplinary and inter-disciplinary
  knowledge. Show how this includes expertise in social sciences and
  humanities, open science practices, and gender aspects of R\&I, as
  appropriate. Include in the description affiliated entities and
  associated partners, if any.
\item
  Show how the partners will have access to critical infrastructure
  needed to carry out the project activities.
\item
  Describe how the members complement one another (and cover the value
  chain, where appropriate)
\item
  In what way does each of them contribute to the project? Show that
  each has a valid role, and adequate resources in the project to fulfil
  that role.
\item
  If applicable, describe the industrial/commercial involvement in the
  project to ensure exploitation of the results and explain why this is
  consistent with and will help to achieve the specific measures which
  are proposed for exploitation of the results of the project (see
  section 2.2).
\item
  \textbf{Other countries and international organisations}: If one or
  more of the participants requesting EU funding is based in a country
  or is an international organisation that is not automatically eligible
  for such funding (entities from Member States of the EU, from
  Associated Countries and from one of the countries in the exhaustive
  list included in the Work Programme General Annexes B are
  automatically eligible for EU funding), explain why the participation
  of the entity in question is essential to successfully carry out the
  project.
\end{itemize}
}

The \TheProject consortium spans the broad spectrum of actors required
for successfully developing and dissemiating tools and infrastructure
for open and reproducible computational science,
catering to the needs of the European and global scientific community.
It is composed of one academic institution, three research centers,
and one SME based in three different countries (Norway, France, Germany).

The Consortium ensures a critical mass of scientific expertise and excellence
in key areas (geosciences, micromagnetics, education) with research organisations and SMEs of recognised
 international reputation.
 Namely, \TheProject consortium brings in:
\begin{compactitem}
\item A set of use cases that cover several application domains and users, and that impose very diverse
requirements on open tools (\site{MP}, \site{IFR}, \site{UIO});
\item Lead developers in the Jupyter Ecosystem, including IPython, the Jupyter Notebook, JupyterLab,
JupyterHub, Binder, MyBinder.org, Jupyter Widgets (\site{SRL}, \site{QS})
\item Experts and major promoters of the Jupyter collaborative user interfaces for interactive and exploratory
computing in a variety of scientific domains (\site{MP}, \site{IFR}, \site{UIO});
\item A long experience and proven track record of success with large and complex collaborative projects,
including
European E-Infrastructure projects (\site{MP}, \site{SRL}),
projects focused on large-scale infrastructures and large experimental services (\site{MP}, \site{IFR}),
as well as experience in running large scale open source projects (Jupyter project, \site{SRL}, \site{QS});
\item A comprehensive range of skill sets and competencies in several relevant domains,
from applied research to standardisation to business analysis.
\end{compactitem}

The consortium has developed through collaborations and common interests over recent years.
Some partners have been working together on different aspects of Jupyter
and software for education for many years (\site{QS}, \site{SRL}.
Meanwhile, others joined together during a previous successful
H2020 European Research Infrastructure project OpenDreamKit \#676541 (\site{MP}, \site{SRL}).
Additionally, some partners have expertise in the practice of open science and training (\site{IFR}, \site{UIO}).

Finally, we note that the project partners are long time passionate
advocates of Open and Reproducible Science;
building on highly successful past experience with OpenDreamKit, they
\emph{have chosen to write this proposal fully in the open} on GitHub
(\href{https://github.com/minrk/horizon-widera-2022}{https://github.com/minrk/horizon-widera-2022}) for maximum transparency
and engagement of the community.
We have used the same open source collaboration tools and practices
as the Open Source Open Science community.

\TOWRITE{}{More to say here}
% \TOWRITE{ALL}{Add previous collaborations}

% joint software/database development
% Jupyter Project software

\jointsoft{QS,SRL}
% \jointsoft{WTT,SRL}
% \jointsoft{WTT,XFEL}
% 
% % Binder
% \jointsoft{SRL,WTT}
% 
% % k3d
% \jointsoft{SIL,SRL}
% \jointsoft{SIL,XFEL}
% \jointsoft{SRL,XFEL}
% 
% % nbval
\jointsoft{MP,SRL}
% 
% %%s
% 
% % OpenDreamKit: UPSUD, SIL, XFEL, SRL
% \jointproj{XFEL,UPSUD}
\jointproj{MP,SRL}
% \jointproj{XFEL,SIL}
% 
% \jointproj{UPSUD,SRL}
% \jointproj{UPSUD,SIL}
% 
% \jointproj{SRL,SIL}
% \jointproj{UIO,SIL}
% % Jupyter project publication ? XXX TIM
% 
% % Binder
% \jointsoft{SRL,WTT}
% 
% % research bazaar
\jointproj{SRL,UIO}
% 
% % \jointpub{A,B} % some publication
% 
% %joint supervision
% % \jointsup{A,B} %
% 
% %joint organization
% % \jointorga{A,B} % some org
% % \jointorga{SA,UJF} % PASCO'15
% 
% % joint publications
% % \jointpub{A,B} % some publication
% 
% % Jupyter publication
% \jointpub{SRL,XFEL}
% \jointpub{SRL,QS}
% \jointpub{XFEL,QS}

\coherencetable[swsites]

%%% Local Variables:
%%% mode: latex
%%% TeX-master: "proposal"
%%% End:

%%% Local Variables:
%%% mode: latex
%%% TeX-master: "proposal"
%%% End:
