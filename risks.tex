\subsubsection{Risks and risk management strategy}
\label{sec:risks}

\ifgrantagreement\else
An initial risk assessment appears as Table~\ref{risk-table}.

\begin{table}
\begin{center}
\begin{tabular}{m{.15\textwidth}|m{.12\textwidth}|m{.63\textwidth}}\toprule
  \textbf{Risk} & \textbf{Level without / with mitigation} & \textbf{Mitigation measures}
  \\\midrule

   \multicolumn{3}{c}{
    \textit{General technical / scientific risks}
   }
   \\\midrule

  Implementing infrastructure that does not match the needs of end users & High/Low &
  Many of the members of the consortium are themselves end-users with
  a diverse range of needs and points of view; hence the design of
  the proposal and the governance of the project is naturally steered
  by demand; besides, because we are building tools, users have the
  flexibility to adapt the infrastructure to their needs. In addition, the open source nature
  of the project facilitates and promotes the involvement of the wider community in terms of
  providing feedback and requesting additional features via platforms such as GitHub and Bitbucket
  on a regular basis.
  \\\midrule

  Lack of predictability for tasks that are pursued jointly with
  the community & Medium/Low &
  The PIs have a strong experience managing community-developed
  projects where the execution of tasks depends on the availability of
  partners. Some tasks may end up requiring more effort from
  \TheProject to be completed on time, while others may be entirely
  taken care of by the community. Reallocating tasks and redefining
  work plans is common practice needed to cater to a
  fast evolving context. Such random factors will be averaged out over
  the large number of independent tasks.\\\midrule

  Reliance on external software components & Medium/Low & The non-trivial
  software components \TheProject relies on are open source. Most are
  very mature
  and supported by an active community, which offers strong long run
  guarantees. The other components could be replaced by alternatives, or
  even taken over by the participants if necessary.
  \\\midrule
  %\\\midrule

%  \multicolumn{3}{|c|}{
%    \textit{Use-case risks}
%  }
%  \\\midrule
%
%  & & \TOWRITE{WP4}{Risks related to use-cases in WP4}
%  \\\midrule

  \multicolumn{3}{c}{
    \textit{Management risks}
  }
  \\\midrule

  Recruitment of highly qualified staff & High/Medium &

  The majority of positions funded by \TheProject are already hired.
  Only two positions are to be filled, both full-time research software engineers,
  and partners have much experience hiring excellent staff at attractive sites.
  In addition, we
  have a critical mass of qualified staff in the project enabling us
  to train and mentor new recruits.
 \\\midrule

  Different groups not forming effective team & Medium/Low & The participants have a long
  track record of working collaboratively across multiple
  sites. Thorough planning of project meetings, workshops and
  one-to-one partner visits will facilitate effective teamwork,
  combining in-person and remote collaboration.\\\midrule
  % this also justifies our generous travel budget.

  Partner leaves the consortium & High/Low & If the Grant Agreement requires a replacement
  in order to achieve the project's objectives, the consortium will invite a new
  relevant partner in. If a replacement is not necessary, the resources and tasks
  of the departing partner will be reallocated to the alternative ones within the
  consortium.
  \\\midrule

  \multicolumn{3}{c}{
    \textit{Dissemination risks}
  }
  \\\midrule

  Impact of dissemination activities is lower than planned. & Medium/Low &

  Partners in the consortium have a proven track record at community
  building, training, dissemination, social media communication, and
  outreach, which reduces the risk. The Project Coordinator
  will monitor impact of all dissemination activities. If a deficiency is identified, the consortium
  will propose relevant corrective actions.\\\bottomrule
  \end{tabular}
\end{center}
\caption{\label{risk-table}Initial Risk Assessment}
\end{table}
\fi
