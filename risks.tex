\subsubsection{Risks and risk management strategy}
\label{sec:risks}
% 
% The risk in the project execution as planned is carefully assessed and
% managed. We base our plans on long standing experience, and we bring
% together the world's experts in the relevant tools and techniques.
% 
% A key feature of this project is the involvement of a wide set of
% partners from multiple domains. While this ensures complementary
% coverage of a wide set of skills and provides robustness in different
% ways, we will have to ensure that all the partners work together as
% a closely knit team.
% 
% Our open source approach means that all our code and outputs
% will be open and visible to anybody at sites like GitHub and Bitbucket
% throughout the project. It is common for some users to run the latest
% development versions of computational and infrastructure software, thus
% beta-testing code between major releases.
% This reduces the risk of developing software which people won't use:
% where our design decision or technical approaches are
% controversial, this will be detected early by those users, giving the
% consortium useful feedback to consider.
% 
% As part of the Management Work Package, and with support from the
% Coordination Team, the project coordinator will maintain and regularly
% update a Risk Management Plan; at the end of each Reporting Period,
% this updated plan will be included in the project's Technical Report.
% It will identify and categorise all
% potential strategic risks (legal, financial, human resources risks, etc.)
% to the successful delivery of the project, their probability and impact.
% For each risk area, mechanisms for risk mitigation will be identified
% and contingency actions will be proposed.
% 
% Risks will be evaluated in terms of project goals and objectives,
% according to the following four steps:
% \begin{enumerate}
% \item Identification of risks using a structured and rational approach to
% ensure that all areas are addressed.
% \item Quantitative assessment and ranking of the risks.
% \item Definition of procedure to reduce (or minimize) risk.
% \item Monitoring and management of risks throughout the project life
% with milestone review and reassessment.
% \end{enumerate}
% 
% Finally, as reported above, a conflict resolution mechanism will be put in place,
% whereby decision making divergence and conflicts that cannot be solved
% at the Steering Committee (SC) level will be submitted to the Coordinator. The
% mediation and resolution process used is the following:
% \begin{itemize}
% \item Case presented by the involved parties.
% \item Development of a fact-based and neutral report by the coordinator
% to be provided to the conflicting parties and SC.
% \item Final decision to resolve the conflict made by SC.
% \end{itemize}
% 
\ifgrantagreement\else
An initial risk assessment appears as Table~\ref{risk-table}.

\begin{table}
\begin{center}
\begin{tabular}{|m{.2\textwidth}|m{.12\textwidth}|m{.58\textwidth}|}\hline
  Risk & Level without / with mitigation & Mitigation measures
  \\\hline

   \multicolumn{3}{|c|}{
    \textit{General technical / scientific risks}
   }
   \\\hline

  Implementing infrastructure that does not match the needs of end users & High/Low &
  Many of the members of the consortium are themselves end-users with
  a diverse range of needs and points of views; hence the design of
  the proposal and the governance of the project is naturally steered
  by demand; besides, because we provide a toolkit, users have the
  flexibility to adapt the infrastructure to their needs. In addition, the open source nature
  of the project facilitates and promotes the involvement of the wider community in terms of
  providing feedback and requesting additional features via platforms such as GitHub and Bitbucket
  on a regular basis.
  \\\hline

  Lack of predictability for tasks that are pursued jointly with
  the community & Medium/Low &
  The PIs have a strong experience managing community-developed
  projects where the execution of tasks depends on the availability of
  partners. Some tasks may end up requiring more manpower from
  \TheProject to be completed on time, while others may be entirely
  taken care of by the community. Reallocating tasks and redefining
  work plans is common practice needed to cater for a
  fast evolving context. Such random factors will be averaged out over
  the large number of independent tasks.\\\hline

  Reliance on external software components & Medium/Low & The non trivial
  software components \TheProject relies on are open source. Most are
  very mature
  and supported by an active community, which offers strong long run
  guarantees. The other components could be replaced by alternatives, or
  even taken over by the participants if necessary.
  \\\hline

  \\\hline

%  \multicolumn{3}{|c|}{
%    \textit{Use-case risks}
%  }
%  \\\hline
%
%  & & \TOWRITE{WP4}{Risks related to use-cases in WP4}
%  \\\hline

  \multicolumn{3}{|c|}{
    \textit{Management risks}
  }
  \\\hline

  Recruitment of highly qualified staff & High/Medium &

  Great care was taken coordinating with currently running projects to
  rehire personnel with strong track record, and identifying pool of
  candidates to hire from, notably in the developers community of
  software related to the project. This was favoured by the partners'
  long history of training and outreach activities. In addition, we
  have a critical mass of qualified staff in the project enabling us
  to train and mentor new recruits.

 \\\hline

  Different groups not forming effective team & Medium/Low & The participants have a long
  track record of working collaboratively on code across multiple
  sites. Aggressive planning of project meetings, workshops and
  one-to-one partner visits will facilitate effective teamwork,
  combining face-to-face time at one site with remote
  collaboration.\\\hline
  % this also justifies our generous travel budget.

  Partner leaves the consortium & High/Low & If the GA requires a replacement
  in order to achieve the project's objectives, the consortium will invite a new
  relevant partner in. If a replacement is not necessary, the resources and tasks
  of the departing partner will be reallocated to the alternative ones within the
  consortium.
  \\\hline

  \multicolumn{3}{|c|}{
    \textit{Dissemination risks}
  }
  \\\hline

  Impact of dissemination activities is lower than planned. & Medium/Low &

  Partners in the consortium have a proven track record at community
  building, training, dissemination, social media communication, and
  outreach, which reduces the risk. The Project Coordinator
  will monitor impact of all dissemination activities. If a deficiency is identified, the consortium
  will propose relevant corrective actions.\\\hline

  \end{tabular}
\end{center}
\caption{\label{risk-table}Initial Risk Assessment}
\end{table}
\fi
%\TOWRITE{NT/Eugenia}{Impredictability}

%\includegraphics[width=.94\textwidth]{Pictures/Impact-img1.png}

%   But: since Open Source softwares are freely accessible, security
%   and privacy issues are a concern. Anytime a resource is shared,
%   there is greater risk of unauthorised access and contaminated data.
%   Providers must demonstrate security solutions, which should include
%   physical security controlling access to the facility and protection
%   of user data from corruption and cyber attacks.}


\TOWRITE{ALL}{
  Add a paragraph about data management plan. What data will we produce, which data is available from the
  start, how do we handle it...
}
