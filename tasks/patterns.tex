% template for a task
% each task should be added to exactly one workpackage
% in the workpackage task list
\begin{task}[
  title=Support more use patterns,
  % task id for references
  id=patterns,
  % lead institution ID
  lead=SRL,
  PM=1,
  wphases={0-36},
  % partner institution ID(s)
  % don't include lead here
  partners={MP}
]
The task includes the following activities
  \begin{compactitem}
  \item Data access (BinderHub deployment, possibly mostly docs)
  \item Restrict access to BinderHub through authentication
    - useful for many institutes who want to offer reproducibility services and
    data hosting but need to limit or control access to those systems. 
  \item Allow to use Binder without a BinderHub installation. A use case for
    this is to run a Binder service on a local Desktop (see
    \taskref{applications}{binder-at-home}).
  \item Allow to use Binder without Jupyter notebooks.

    Binder's original design goal was to allow execution of notebooks within a
    software requirement that provides all the software required by the
    notebook. This may include compile and highly customised software, which
    might produce output files, which are then processed and visualised in the
    notebook.

    Here, we will provide the foundations for reproducibility work that is
    non-interactive and doesn't need Jupyter notebook. A use case for this is
    reproducibility at High Performance Computing facilities, where often the
    computational tasks cannot be carried out interactively (see \taskref{applications}{binder-at-hpc}).

    % Should we say more here? @min?
    %
    % This will require creation of a program that creates the required software
    % environment, and - if a notebook is used - start a notebook server to which
    % the user can connect.

  \end{compactitem}
\end{task}
