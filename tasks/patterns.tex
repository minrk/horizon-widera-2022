% template for a task
% each task should be added to exactly one workpackage
% in the workpackage task list
\begin{task}[
  title=Support more use patterns,
  % task id for references
  id=patterns,
  % lead institution ID
  lead=SRL,
  PM=16,
  %wphases={0-36},
  % partner institution ID(s)
  % don't include lead here
  partners={MP,IFR}
]
In this task, we provide the technical possibilities to use the Binder tools in
better and new ways. These are used and evaluated with real world applications
in~\WPref{applications}.

\begin{compactitem}
\item The \repotodocker{} tool searches a given repository for specifications of
  software dependencies, and carries on to compose instructions to install all
  of the software dependencies within a Docker container image. In this task, we
  modularise this functionality, and make it possible to extract the required
  instructions on their own (for example into a stand-alone shell script).

  Such functionality could be used to:
  \begin{compactitem}
  \item Extract the list of installation commands to carry out a local install
    of the required software, or an installation within any other environment.
  \item In particular on HPC systems, it may be necessary to install software
    directly on the host, and this functionality would simplify that.
  \item Having the installation instructions neatly summarised will also help
    the interested scientist to understand what software specifications are
    (explicitly or implicitly ) given within the repository.
    % \item Having the installation instructions extracted, these can be used to
    %   explore the effects of changing versions of a particular library or
    %   dependency. This will be helpful to track down the origins of observed
    %   non-reproducibility.
  \end{compactitem}

\item Improve and document the options to access external data resources from
  within the computational environments generated by BinderHub deployments. This
  is a prerequisite for \taskref{applications}{data-publishing}.
\item Restrict access to BinderHub through authentication - useful for many
  institutes who want to offer reproducibility services and data hosting but
  need to limit or control access to those systems (to ensure, for example, that
  the computational resources are not abused).
\item Allow to use Binder without a BinderHub installation. A use case for this
  is to run a Binder service on a local Desktop (see
  \taskref{applications}{binder-at-home}).
\item Allow to use Binder without Jupyter notebooks.

  Binder's original design goal was to allow execution of notebooks within a
  software requirement that provides all the software required by the notebook.
  This may include compile and highly customised software, which might produce
  output files, which are then processed and visualised in the notebook.

  Here, we will provide the foundations for reproducibility work that is
  non-interactive and doesn't need Jupyter notebook. A use case for this is
  reproducibility at High Performance Computing facilities, where often the
  computational tasks cannot be carried out interactively (see
  \taskref{applications}{binder-at-hpc}).

    % Should we say more here? @min?
    %
    % This will require creation of a program that creates the required software
    % environment, and - if a notebook is used - start a notebook server to which
    % the user can connect.

  \end{compactitem}
\end{task}


%
%   This feature will become part of the \repotodocker{} functionality
%   (\localdelivref{extract-dependencies}).
%
%

%%% Local Variables:
%%% mode: latex
%%% TeX-master: "../proposal"
%%% End:
