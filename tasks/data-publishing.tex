% template for a task
% each task should be added to exactly one workpackage
% in the workpackage task list
\begin{task}[
  title=Data publishing,
  % task id for references
  id=data-publishing,
  % lead institution ID
  lead=MP,
  PM=8,
  wphases={0-36},
  % partner institution ID(s)
  % don't include lead here
  partners={IFR,UIO}
  ]
  The task focuses on the use of reproducible software environments within
  Binder to provide working and interactive code that provides access to large
  or complex data sets.

  \paragraph*{Context:} Scientists would like to publish their data. Such a data
  publication must include the data set itself, and metadata that explains how to
  interpret the data. In addition to such information, it can improve the
  quality of the data set if \emph{computer executable libraries or commands}
  are provided, which simplify the reading of the actual data files. Such
  routines encapsulate meta information about the data file (format and
  structure) in a machine-readable format.

  A binder-enabled repository can provide the access to such data sets by
  containing the specification of a software environment and hosting of the
  file-reading routines together. Such an approach significantly simplifies the
  re-use of the data (or reproduction of existing study) because the data
  reading routines do not need to be re-implemented.

  \paragraph*{Task activity:}
  We will design and implement functionality that allows such data publishing
  based on Binder tools.

  A major challenge is the link to the data: ideally, data sets are hosted on a
  separate infrastructure (such as archives, or files published together with a
  publication - for example on Zenodo). It will thus be necessary to reference
  the data on this data-holding archive and the data location within that resource.
  This data location will need to be used from the notebooks to access
  the data. %We are not aware of a common standard that could be adapted here.

  Some authentication for data access may be required: either because the data
  is not meant to be fully public, or because access to the data creates
  significant cost for the hosting party. Such authentication
  information/credentials from (the Binderhub) login must be passed to the point
  where the data-holding medium is mounted in a container.

  We will work very closely with the Max Planck Compute and Data Facility
  (MPCDF) to prototype such functionality. We will allow to access data sets
  that the MPCDF hosts themselves.

  An important outcome of this task is an evaluation of the chosen design and
  implementation, to propose a more generic model for the next feature extension
  of the Binder tools.
\end{task}
