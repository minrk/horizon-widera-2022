% template for a task
% each task should be added to exactly one workpackage
% in the workpackage task list
\begin{task}[
  title=Towards quantifiable progress for reproducible software environments,
  % task id for references
  id=repo2docker-checker,
  % lead institution ID
  lead=SRL,
  PM=12,
  wphases={0-24},
  % partner institution ID(s)
  % don't include lead here
  partners={MP}
  ]
The \repotodocker tool is a key component of the Binder software for
reproducibility \TOWRITE{Add link to relevant section in excellence/concept/...
  XXX}. It can be used to create a software environment based on software
dependency specification standards that are widely used.

If the required software is specified -- for example through a
\texttt{requirements.txt} file for Python dependencies -- then
\repotodocker{} can create the software environment (currently limited to such
environments in Docker images), within in which the main computation or data
analysis can be reproduced.

In this task, we will develop a tool that allows us to \emph{automatically} assess the
reproducibility of software environments for software that is publicly available on
Github, Bitbucket or Gitlab repositories.

For every repository, the repo2docker-checker tool will report if an appropriate
software could be compiled, or if a problem occurred. Software environments in
repositories may be reproducible because the authors already use Binder to offer
their repository in an interactive Binder environment. Or the software
environment may be reproducible because the authors have followed standard
conventions \TOWRITE{Should somewhere list these and refer to them here. XXX}
and \repotodocker{} understands these conventions.

The task includes the following activities:
  \begin{compactitem}
  \item Design and develop the tool (working name
    \softwarename{repo2docker-checker})
  \item Develop a strategy and heuristic to evaluate success of the build
    process.
  \item Identify suitable software repositories for the study.
  \item Automate the software reproduction process for the available
    repositories.
  \item Automate the analysis of the results, so the study can be repeated later.
  \end{compactitem}
  The tool will be made available as open source (\localdelivref{deliv-id-repo2docker-checker-software}).
\end{task}
