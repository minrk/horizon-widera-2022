\begin{task}[
  title=repo2docker development,
  id=repo2docker-timemachine,
  lead=SRL,
  PM=38,
  wphases={0-36!1.055},
  partners={QS,MP}
]

Often, a repository of scientific results may
specify which software library is required (such as the Python library
\softwarename{pandas}), but not which version. A software environment creation
tool -- such as \repotodocker -- can then attempt to install the most recent
version of \softwarename{pandas}. This is usually the intention of the authors,
and was correct at the time the repository was created.  However, as time moves
on, the interface, behaviour and dependence on other packages of
\softwarename{pandas} will change, and at some point an automatic build of the
software for the whole repository may fail because of conflicting dependencies.

We have found through anecdotal evidence that these problems can be overcome if a
\softwarename{pandas} version can be chosen that was the most recent at the time
when the repository was created.

In this task, we will teach \repotodocker to establish the data of publication
(or last modification) of the repository, to determine the appropriate version
of software libraries from that time, and to select libraries with those
versions if no specific version is specified.

XXX


\TOWRITE{There is probably more to be done?}

\TOWRITE{HF: The following is good - but should probably go elsewhere.}
  Running someone else's analyses is a particularly difficult problem.

  There are differences between operating systems, versions of installed
  software and access to the required data sets.
  
  These challenges mean that is currently considered to be beyond the scope of
  an expert peer reviewer to verify data science analysis codes before
  publication.
  
  BinderHub, part of Project Jupyter, enables one-click running of git repositories.

  BinderHub provides a web interface to the repo2docker tool.
  \TOWRITE{End of section to move elsewhere.}


  \TOWRITE{HF: The following items are all good. I think some of the should be
    scattered across the other workpackages.}
  
  The task includes the following activities
  \begin{compactitem}
  \item extend repo2docker with support for execution on cloud resources
  \item extend repo2docker with support for execution on HPC resources with Docker support
  \item improved "first use" experience of running repo2docker locally
  \item add support for using archives such as Zenodo as source for repo2docker and BinderHub
  \item define procedures and recommendations for long term reproducibility and sustainability of repo2docker compatible repositories
  \item create educational material describing repo2docker and its benefits to researchers
  \item Enable Openshift based deployments of BinderHub
  \item User surveys about pain points using BinderHub
  \item User authentication in BinderHub
  \end{compactitem}
\end{task}
