% This section is probably not needed for SOURCE.

\medskip
\noindent\textbf{Jupyter as a basis for web services}\\
Because the Jupyter notebook is a web-based application, it can be
deployed at computational facilities or in the cloud, and can function
as the basis for services exposing computational resources of all
kinds to researchers and the public.  Because Jupyter is
\textbf{interactive}, it enables making scientific results and
communications more interactive than static publications.  The
audience can follow their own initiative and ask their own questions
of published data without needing support from the publishing author,
greatly facilitating the \textbf{practicality of Open Science}.

\medskip
\noindent\textbf{Jupyter is generic}\\
\TheProject chose Jupyter because it is
Generic.  Jupyter makes no domain-specific or even language-specific
assumptions.  Any application where mixing description, code, and
results is valuable can make use of Jupyter.  This broad applicability
makes investment in the Jupyter ecosystem extremely effective, because
improvements to Jupyter can serve many communities simultaneously.

Jupyter is built from a collection of standard protocols and file
formats.  Jupyter is not just a single, monolithic piece of
software, but a description of how such software can be built.  The
result is the ability for a variety of communities and applications to
use components of Jupyter for their purposes, and/or reimplement pieces to
meet their needs.
%
For example:
\begin{enumerate}
\item The notebook file format is a well-specified JSON document,
  which can be interpreted by many systems.  This has facilitated the
  development of different services providing rendering of notebooks, e.g. the code
  hosting website GitHub, which renders notebooks for easy viewing by
  anyone, without Jupyter software.
\item The Jupyter protocol describes how execution is performed, which
  has enabled the development of over one hundred kernel
  implementations in dozens of languages\footnote{\url{https://github.com/jupyter/jupyter/wiki/Jupyter-kernels}}.
\item Output in the Jupyter protocol uses web-standard MIME types,
  enabling any possible format to be an output in a Jupyter notebook.
\item The JupyterLab extension system provides a system for building
  applications from Jupyter components and others.
\item The Jupyter Widgets provide a system for customizing and
  extending interactivity in Jupyter-based environments.
\end{enumerate}

The popularity of Jupyter, with millions of users and hundreds of open
source contributors, is an indicator of the value and impact of this approach.

\medskip
\noindent\textbf{Improvement to the Jupyter ecosystem}\\
The benefits of focusing our work on a mature system like Jupyter include:

\begin{itemize}
\item vibrant community ensures health and sustainability,
\item large existing user base maximises impact of contributions,
\item mature software ecosystem maintains quality software through
  industry standards such as version control, tests, continuous
  integration, stable release cycles, roadmaps, and user support.
\end{itemize}

The Jupyter community aims to be inclusive, and \TheProject fully
embraces and supports that approach.  Jupyter is inclusive across a number of axes.
By being applicable across numerous domains, Jupyter and \TheProject
encourage participation from individuals of various interests and
backgrounds, and has taken action to improve diversity in the project
by participating in ``Outreachy,'' a program of paid internships for
individuals from groups that face under-representation, systemic bias,
or discrimination.  Jupyter has also operated workshops focused on
training contributors from under-represented groups.  In being free,
public, open source software, Jupyter and \TheProject are accessible
to as many individuals as possible, and invites users and contributors
beyond origin, nationality, beliefs, orientation.  One area where
Jupyter has lacked in this regard is in the User Interface
accessibility, and we will help improve this in
% \taskref{core}{accessibility}
.  Additionally, the project will
focus some of its workshops in
% \taskref{education}{workshops}
on
under-represented communities.


\begin{figure}[ht!]\centering
  \includegraphics[width=0.6\textwidth]{images/notebook_components.png}
  \caption{The architecture of the Jupyter Notebook, kernels, and tools
        which operate on notebook files}
  \label{fig:notebook-architecture}
\end{figure}
