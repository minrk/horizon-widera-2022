\begin{sitedescription}{UIO} \label{desc:UIO}

    \begin{center}
    \includegraphics[height=3cm]{Participants/Logos/UiO.png}
    \end{center}

    The University of Oslo (UiO) is Norway's oldest institution for research and higher education, with 28,000 students and 6,000 employees. UiO has 8 faculties, 2 museums and several centres. In addition, UiO has 10 Norwegian Centres of Excellence,  is ranked as the world's 62nd university, and has had 5 Nobel prize laureates. UiO aims to become an international hub for the research-based integration of computing into science education and has financed a university-wide hosting service for Jupyter notebooks through JupyterHub  to introduce a computational aspect to all curriculum programs in all science disciplines from bachelor to postdoctoral studies.

    The University of Oslo is a Silver Partner to \href{https://carpentries.org}{The Carpentries}, an international successful community driven project with Instructors, Trainers, Maintainers, helpers, and supporters who share a mission to teach foundational computational and data science skills to researchers.
    It is also actively involved in the \href{https://coderefinery.org/}{CodeRefinery} initiative that acts as a hub for FAIR (Findable, Accessible, Interoperable, and Reusable) software practices. Twice a year, CodeRefinery organizes big online training events with more then 300 attendees each.

     The University of Oslo aims to manage research data according to international standards and is involved in several institutional (\href{https://www.usit.uio.no/english/about/news/fair.html}{FAIR@UiO}), national (\href{https://www.sigma2.no/project-new-research-data-archive}{Sigma2 new data research archive}) and international (\href{https://www.eosc-nordic.eu/}{EOSC-Nordic}, \href{https://www.reliance-project.eu/}{RELIANCE}) initiatives to support the development of a global research community in which research data is widely shared.
     Since November 2017, UiO's policy follows the "open as standard" principle in respect of access to research data \cite{datapolicy-uio}.

    \subsubsection*{Curriculum vitae}

    % Curriculum of the personnel at this institution. This includes
    % to-be-hired people for which there is a tentative candidate.

    \begin{participant}[gender=female]{Anne Fouilloux}
  % type is one of:
  % - leadPI: leader of the participating institution
  % - PI: Principal Investigator
  % - R: researcher?
  % Who is the coordinator is specified elsewhere

  % PM=YYY:
  % A fair evaluation of the number of months you will be
  % spending on this specific project along the four years.
  % Typical numbers:
  % - full time hired personnel: 48 months
  % - lead PI or proposal coordinator: 8-12 months
  % - PI: 4-5 months
  % - participant: 2-6 months

  % salary=ZZZ:
  % Approximate monthly gross salary (in term of total cost for the
  % employer). This is optional. If you are uncomfortable having this
  % information in a public file, you can alternatively send the
  % information to Eugenia Shadlova, or to your institution
  % leader/manager if he is willing to fill in himself the budget
  % forms on the eu portal.

  % The above information is used to fill in various tables in the
  % proposal file, and to evaluate the cost of the project for the
  % institutions.

  % You may remove all those comments.

  % About half a page of free text; for whatever it's worth, you may see
  % Nicolas.Thiery.tex for an example.

  \medskip PhD, is a highly experienced Research Software Engineer dedicated to supporting
  researchers towards the adoption of Open Science best practices.

  With a solid background in Computer Sciences, she worked in various application fields, including environmental sciences, Intelligent Transport Systems, High-Performance computing, bio-informatics, meteorology and Geosciences.

  She is currently working for the \href{https://neic.no}{Nordic e-Infrastructure Collaboration} (NeIC) where she is leading the second phase of the \href{https://neic.no/nicest2/}{Nordic Collaboration on e-Infrastructures for Earth System Modeling Tools} project (NICEST2) and actively supports the \href{https://pangeo.io/}{Pangeo} community platform for Big Data geoscience. In 2021, she has been selected as a member of the Strategic Advisory Board of the \href{https://stories.ecmwf.int/destination-earth/index.html}{Destination Earth initiative}. 

She is also involved in a number of projects such as MAchine learning, Surface mass balance of glaciers, Snow cover, In-situ data, Volume change, Earth observation (MASSIVE, Research Council of Norway), \href{https://www.eosc-nordic.eu/}{EOSC-Nordic} and \href{https://www.reliance-project.eu/}{Research Lifecycle Management technologies for Earth Science Communities and Copernicus users in EOSC (RELIANCE, EU-funded, Grant number 101017501).
   Since 2015, Anne Fouilloux has been very active with \href{https://carpentries.org}{The Carpentries}, a diverse and global community of volunteers and she teaches foundational coding and data science skills to students and young researchers. She is a certified \href{https://carpentries.org/instructors/}{Carpentries instructor}, \href{https://carpentries.org/trainers/}{instructor trainer} and \href{https://carpentries.org/maintainers/}{maintainer}. 
   
\end{participant}

%%% Local Variables:
%%% mode: latex
%%% TeX-master: "../proposal"
%%% End:

    \begin{participant}[gender=male]{Gard Thomassen}
  % type is one of:
  % - leadPI: leader of the participating institution
  % - PI: Principal Investigator
  % - R: researcher?
  % Who is the coordinator is specified elsewhere

  % PM=YYY:
  % A fair evaluation of the number of months you will be
  % spending on this specific project along the four years.
  % Typical numbers:
  % - full time hired personnel: 48 months
  % - lead PI or proposal coordinator: 8-12 months
  % - PI: 4-5 months
  % - participant: 2-6 months

  % salary=ZZZ:
  % Approximate monthly gross salary (in term of total cost for the
  % employer). This is optional. If you are uncomfortable having this
  % information in a public file, you can alternatively send the
  % information to Eugenia Shadlova, or to your institution
  % leader/manager if he is willing to fill in himself the budget
  % forms on the eu portal.

  % The above information is used to fill in various tables in the
  % proposal file, and to evaluate the cost of the project for the
  % institutions.

  % You may remove all those comments.

  % About half a page of free text; for whatever it's worth, you may see
  % Nicolas.Thiery.tex for an example.

  \medskip Has a PhD in Bioinformatics, and participated in the first exome and transcriptome
sequencing of tumor/normal samples in Norway in 2010. In 2012 Thomassen became project leader for
building a system for storage, analysis and collection of sensitive data (TSD) at the University of Oslo Centre for IT (USIT). TSD is now a national eInfrastructure for research on sensitive data. TSD also delivers IT infrastructure for clinical deep sequencing at the Oslo University Hospital. Today, Gard Thomassen is the Head of the Division for Research Computing and Assistant Director at the USIT at the University of Oslo, Norway.

\end{participant}

%%% Local Variables:
%%% mode: latex
%%% TeX-master: "../proposal"
%%% End:

    \begin{participant}[gender=male]{Jean Iaquinta}
  % type is one of:
  % - leadPI: leader of the participating institution
  % - PI: Principal Investigator
  % - R: researcher?
  % Who is the coordinator is specified elsewhere

  % PM=YYY:
  % A fair evaluation of the number of months you will be
  % spending on this specific project along the four years.
  % Typical numbers:
  % - full time hired personnel: 48 months
  % - lead PI or proposal coordinator: 8-12 months
  % - PI: 4-5 months
  % - participant: 2-6 months

  % salary=ZZZ:
  % Approximate monthly gross salary (in term of total cost for the
  % employer). This is optional. If you are uncomfortable having this
  % information in a public file, you can alternatively send the
  % information to Eugenia Shadlova, or to your institution
  % leader/manager if he is willing to fill in himself the budget
  % forms on the eu portal.

  % The above information is used to fill in various tables in the
  % proposal file, and to evaluate the cost of the project for the
  % institutions.

  % You may remove all those comments.

  % About half a page of free text; for whatever it's worth, you may see
  % Nicolas.Thiery.tex for an example.

  \medskip 
  
  With a solid background in maths/physics/meteorology and a PhD in satellite remote sensing Jean Iaquinta has extensive experience in data/image acquisition/processing, metrology/instrumentation, and numerical modelling. He worked for 25 years in research and development in the public and private sectors, in several countries and application areas, ranging from meteorology/climate, intelligent transport systems and infrastructures, technology development and quality assurance. Jean coordinated large projects and is very familiar with project/technical management. He also has expertise with scientific programming and best software practices, including HPC, version control, containarization, and cloud computing. 

\end{participant}

%%% Local Variables:
%%% mode: latex
%%% TeX-master: "../proposal"
%%% End:

    \begin{participant}[gender=female]{Olga Silantyeva}
  % type is one of:
  % - leadPI: leader of the participating institution
  % - PI: Principal Investigator
  % - R: researcher?
  % Who is the coordinator is specified elsewhere

  % PM=YYY:
  % A fair evaluation of the number of months you will be
  % spending on this specific project along the four years.
  % Typical numbers:
  % - full time hired personnel: 48 months
  % - lead PI or proposal coordinator: 8-12 months
  % - PI: 4-5 months
  % - participant: 2-6 months

  % salary=ZZZ:
  % Approximate monthly gross salary (in term of total cost for the
  % employer). This is optional. If you are uncomfortable having this
  % information in a public file, you can alternatively send the
  % information to Eugenia Shadlova, or to your institution
  % leader/manager if he is willing to fill in himself the budget
  % forms on the eu portal.

  % The above information is used to fill in various tables in the
  % proposal file, and to evaluate the cost of the project for the
  % institutions.

  % You may remove all those comments.

  % About half a page of free text; for whatever it's worth, you may see
  % Nicolas.Thiery.tex for an example.

  \medskip A passionate researcher with multi-disciplinary background, who started career as a software engineer, got an urge for science, and defended her Phd in theoretical and applied mechanics. After years of software development in aerospace industry, she realized that she need to do something even more relevant for society and future generations. Thus, she decided to change her field towards hydrological science, as a part of Earth System Science, particularly important in the changing climate world. Based on her industrial experience and mathematical background, she is able to work with and develop complex numerical models, improving their relevance for the scientific community. She is an experienced presenter and skilled software developer with good understanding of current cutting edge technologies and services.   
\end{participant}

%%% Local Variables:
%%% mode: latex
%%% TeX-master: "../proposal"
%%% End:

    %\input{CVs/First.Last.tex}

    % For other to-be-hired person, please include here something like:
    % \begin{participant}[type=res,PM=3,salary=5900]{NN}
    %  <a _short_ description of the qualifications of whom you want to hire>
    % \end{participant}

    \subsubsection*{Publications, products, achievements}

    \begin{compactenum}

    \item \href{https://annefou.github.io/metos_python/}{Working with Spatio-temporal data in Python}, 2017, Anne Fouilloux, \href{https://zenodo.org/badge/latestdoi/96184802}{DOI 10.5281/zenodo.1165281}
    \end{compactenum}

    \subsubsection*{Relevant projects or activities}

    \begin{compactenum}


    \item \href{https://www.dice-eosc.eu/}{DICE-EOSC} \label{desc:diceeosc} (2021-2023):
    Data Infrastructure Capacity for EOSC. The main objective is to provide via the EOSC Portal state-of-the-art data services and significant capacity.

    \item \href{https://www.reliance-project.eu/}{RELIANCE} \label{desc:reliance} (2021-2022):
    Delivering a suite of innovative and interconnected services that extend EOSC's capabilities to support the management of the research lifecycle within the Earth Science communities and Copernicus users.

    \item \href{https://www.eosc-nordic.eu/}{EOSC-Nordic} \label{desc:eoscnordic} (2019-2022):
    Coordination of the European Open Science Cloud relevant initiatives within the Nordic and Baltic countries.

    \item \href{https://www.eosc-hub.eu}{EOSC-Hub} \label{desc:eoschub} (2018-2020):
    A single contact point for European researchers and innovators to discover, access, use and reuse a broad spectrum of resources for advanced data-driven research, an EOSC project.

    \item \href{https://coderefinery.org}{CodeRefinery} \label{desc:coderefinery} (2016-2025):

    The goal of this project is to provide students and researchers with infrastructure and training in the necessary tools and techniques to create sustainable, modular, reusable, and reproducible software.
    This is a project within the Nordic e-Infrastructure Collaboration (\href{https://neic.no}{NeIC}), an organisational unit under \href{https://www.nordforsk.org/en}{NordForsk}.

    The result of this project is a set of software development e-infrastructure solutions, coupled with necessary technical expertise and extensive training and on-boarding activities, training material and best practices guides which together form a Nordic platform for research groups and institutes to develop a better collaboration on software and thereby to catalyze reproducible research and collaboration.
    \newline
    CodeRefinery training material is licensed under \href{https://creativecommons.org/licenses/by-sa/4.0/}{CC BY-SA 4.0} and code examples are \href{https://opensource.org/}{OSI}-approved \href{https://opensource.org/licenses/mit-license.html}{MIT license}.

    The University of Olso is a CodeRefinery partner and will ensure the complementarity of the two projects thus avoiding potential fragmentation. \TheProject will benefit from all this experience as well as the estbalished network in the Nordic Countries and beyond to fully realize the potential of \TheProject EOSC services.
    \newline


    \item 2nd Nordic Collaboration on e-Infrastructures for Earth System Modeling Tools (\href{https://neic.no/nicest2}{NICEST2}, 2020-2023) \label{desc:nicest2}:

    The NICEST2 project focuses on strengthening the Nordic position within climate modeling by leveraging, reinforcing and complementing ongoing initiatives. The three main on-going activities are i) Enhance the performance and optimize and homogenize workflows used, so climate models (like EC-EARTH and NorESM) can be run in an efficient way on future computing resources (like EuroHPC); ii) Widen the usage and expertise on evaluating Earth System Models and develop new diagnostic modules for the Nordic region within the ESMValTool; iii) Create a roadmap for FAIRification of Nordic climate model data.

    \end{compactenum}

    \subsubsection*{Significant infrastructure}

    \begin{compactenum}

    \item \href{https://www.uio.no/english/services/it/research/platforms/edu-research/}{Educloud Research}:
    Educloud Research is a project-oriented self-service Platform as a Service that gives easy and fast access to i) a work environment for research projects that have cross institution and cross border users;
    ii) storage that is available from anywhere, from any device and shared between project users and project platforms (Windows, Linux, HPC); iii) a low threshold HPC system that offers batch job submission (SLURM) and interactive nodes, and that has access to the same storage as the Windows and Linux resources available to the project.

    \item \href{https://www.nrec.no/}{Norwegian Research and Education Cloud}:
    the University of Oslo is part of the Norwegian Research and Education Cloud (NREC) and provides researchers with compute and storage medium-size resources. These include multi-GPUs clusters for big data analysis.

    \item \href{https://sigma2.no}{Sigma-2}: Sigma2 is the Norwegian national provider of e-infrastructures and  manages the national e-infrastructure for large-scale data- and computational science in Norway.

    It offers services in High Performance Computing (HPC) and Data Storage and data analysis (Research Platform as a Cloud Service). The services are organized into infrastructural activities, financed by the Research Council of Norway and the Sigma2 consortium partners, which are the universities in Oslo, Bergen, Trondheim and Tromsø.

    Services are freely available to individuals and groups involved in research and education at Norwegian universities and colleges, and other organizations and project funded with public money. Cost efficient development, procurement, coordination and operation of the national e-infrastructure for research and education is the main focus for Sigma2.

    \end{compactenum}

    \end{sitedescription}
    %%% Local Variables:
    %%% mode: latex
    %%% TeX-master: "../proposal"
    %%% End:
