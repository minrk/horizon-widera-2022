\TOWRITE{ALL}{Proofread WP 5 Management pass 1}
\begin{draft}
\TOWRITE{PS (Work Package Lead)}{For WP leaders, please check the following (remove items
once completed)}
\begin{verbatim}
- [ ] have all the tasks in this Work Package a lead institution?
- [ ] have all deliverables in the WP a lead institution?
- [ ] do all tasks list all sites involved in them?
- [ ] does the table of sites and their PM efforts match lists of sites for each task?
      (each site from the table is listed in all relevant tasks, and no site is listed
      only in the table or only at some task)
\end{verbatim}
\end{draft}

\begin{workpackage}[id=education,wphases=0-36,swsites,
  title=Education and Dissemination,
  short=Education,
  lead=IFR,
  IFRRM=8,
  MPRM=6,
  SRLRM=6,
  QSRM=3,
  UIORM=9
]
\begin{wpobjectives}
  The objective of this work package is to disseminate the results of this
  project, including the technical advances and guidance for best practice for
  reproducible science. This includes education of researchers about the value of
  open science, reproducibility and re-usability as well as the possibilities of
  integrating Binder tools in their workflows.

  Beyond this activity acting from the project members to the wider community of
  scientists, we also plan to seek input from scientists to the project: both in
  terms of requirements for practical reproducibility in their domain and in
  technical contributions -- for example through merge requests for Binder
  tools, or open source documentation of best practice for reprocudible software
  environments.

  This includes:
 \begin{compactitem}
   \item Ensure awareness of the results of the project in the user community,
     and in particular in those groups that act as educators and multipliers of
     knowldege (such as the Carpentries and research infrastructure organisations).
   \item Educate the community on the value of open science, and in particular
   \item Train researchers in best practices for open and reproducible science.
   \item Produce training and education material to disseminate the ability to
     do reproducible computational science using the tools we develop.
   % \item Define individual exploitation plans; ? XXX
 \end{compactitem}
\end{wpobjectives}

% Potential sources of inspiration: ODK's WP2 work package about dissemination:
% PDF: p.36 of https://github.com/OpenDreamKit/OpenDreamKit/raw/master/Proposal/proposal-www.pdf
% Sources: https://github.com/OpenDreamKit/OpenDreamKit/blob/master/Proposal/WorkPackages/DisseminationCommunityBuilding.tex

\begin{wpdescription}

  Open science and reproducible science is entirely dependent on researchers
  adopting open practices. In \TheProject, we improve and developing tools that
  can facilitate these practices, but they only work if researchers actually adopt
  them. For researchers to adopt the practices, they need to (i) know about them
  and (ii) use them.

  We address challenge in multiple ways:
  \begin{enumerate}
    \item the philosophy of the Binder tools is to respect existing standards and
      best practice (and not to invent additional syntax or requirements). It is
      thus possible to use the Binder tools (to recreate a software environment)
      even if the repository authors did not anticipate the use of Binder, or
      knew about their existence. In the best possible scenario, a
      \emph{scientist can use Binder with zero additional effort}.

    \item In this work package, we produce education materials and carry out
      education activities to spread the knowledge about \emph{good practice for
        reproducibility and re-usability in science}, such as for example
      automation of all analysis steps, and complete documentation of the
      required software stack. Only one aspect of this training is to show how
      Binder can help with reproducibility.
    
      Attendees and users following such training and advice will create more
      reproducible artifacts. If they - or later users of their published
      artifacts - want to use Binder to reproduce or re-use the results, they
      can. Even if they do not, we will have achieved an improvement of the
      reproducibility of scientific artifacts.
  \end{enumerate}

  % HF: I think training the wider (non-scientist) public about Binder is going
  % too far?
  % 
  % Going further, it is also clear that open science is not just of value
  % to researchers: one of the largest benefits of open science is that it makes
  % science accessible to the broader public who may not be members of the
  % research community.
  % 
  % To this end, in addition to training researchers, we will also train the
  % public in how to make use of the open science research and services
  % facilitated by \TheProject. This will be done through regular open
  % dissemination and training workshops, as well as by producing and maintaining
  % material for online courses and documentation.

\TheProject will develop, through \WPref{applications}, a number of demonstrator repositories that
show examples of reproducibility. We use those activities to inform and evaluate
the
technical improvement to the Binder tools in \WPref{reproducibility} and
\WPref{impact}. We also use those studies to create tutorials and \emph{best practice guides
  for reproducibility} in this work package.

We will also participate in the concertation activities, consultations and other
meetings and events of the European E-Infrastructure projects. \TODO{Do we still
want this sentence? If so, mention EGI here?}

As with all the code, test and build infrastructure produced as part of the
project, we will also make all documentation open source. Our documentation --
which includes best practice guides for reproducibility -- can thus be modified
and improved after the end of the project to react to new developments.

Open access to all publications resulting from the project will be ensured.
\TODO{Should this sentence go into a data / IPR management plan?}
\end{wpdescription}

\begin{tasklist}
% add tasks from task directory here
\begin{task}[
  title=Management of dissemination and communication activities,
  id=website,
  lead=SRL,
  PM=2,
  % >>> 2/36 = 0.05555555555555555
  wphases={0-36!.056},
  partners={}
]

This task comprises the management and administrative aspects of all forms of
direct dissemination and public communication activities such as press releases,
scientific and technical publications, seminars, talks, promotion through social
media, creation of advertisement materials such as flyers, posters, and
electronic feeds as well as their distribution. We will use standard community
building technology such as mailing lists, wikis and forums, to ensure
dissemination to and engagement with the user community.
\end{task}

% template for a task
% each task should be added to exactly one workpackage
% in the workpackage task list
\begin{task}[
  title=Training Workshops for more reproducible science,
  id=workshops,
  lead=UIO,
  PM=9,
  wphases={12-36!.25},
  partners={SRL,MP,IFR}
]
This task is focused on taking the content from the
\taskref{education}{online-resources} (Best practice for
reproducible science guidelines) and disseminating it through various channels and to different target audiences.

% \begin{compactitem}

%    \item Defining and implementing a strategy to enable a shared vision of the Jupyter ecosystem across all the actors from developers, users to every stakeholder: the current misalignment hinders the full exploitation of Open Software practices where co-design is a de facto approach.
%
% For instance, the official Jupyter documentation (https://jupyter.org/documentation) solely reflects the view of developers where the Jupyter ecosystem is defined as a set of software packages (jupyter-core, jupyter-client, kernels, widgets (ipywidgets, ipyleaflet, etc.). The user vision is relegated to examplars (blogs, newsletters, etc.) which inevitably tend to be restrictive but often become de facto standards. This can lead to misconceptions and makes it more difficult for on-boarding novices and new communities.
%

% \item Triggering a cultural change to help under-represented groups to actively participate to the development of open source project to ensure the sustainability of the \TheProject services deployed on EOSC-HUB.
%

%\item Foster Open innovation by collaborating with others from different background and activities (school, universities, industries, journalists, artists, etc.)
%  \end{compactitem}

To achieve these goals, the following actions/activities will take place:

  \begin{compactitem}

  %\item co-design hackathons: the co-design efforts between domain scientists, \TheProject developers and service providers will be carried out at any point in time of the project and will be registered in a co-design register to help for future engagement with new communities of users. To be fully effective,  co-design hackathons will be organized to set the stage, define rules for co-design interactions and more importantly align all actors into a common user-centred vision of \TheProject services and associated development towards a successful EOSC deployment.

%    \item Workshops on Findable, Accessible, Interoperable and Reusable (FAIR)
%      software and data to facilitate the adoption of Open Science and Open
%      Scholarship best practices (transparent, sharable and collaborative
%      Science): this would not be restricted to the Jupyter ecosystem and will
%      teach users how to make data, lab notes and other research processes freely
%      available, under terms that enable reuse (licensing), redistribution and
%      reproducibility of methods and/or results.

   \item Delivery of workshops on (i) open science, (ii) reproducible computational
     science, and (iii) the use of Binder tools to support this.

     The content is focused on key insights and tools need for more reproducible
     science, but will be contextualised and delivered in the wider field of
     Findable, Accessible, Interoperable and Reusable (FAIR) software and data.

   % \item Trainings on how to use \TheProject software and services to fully
   %   exploit \TheProject developments for repoducible science: develop training
   %   materials and organize training events for researchers and the public to
   %   enable Open Science and maximise the usefulness of \TheProject
   %   developments.

   \item \TheProject Admin trainings: we will offer training events for learning on how to
     deploy \TheProject services such as BinderHub. This will be relevant for a
     very small (but important) group of users, i.e. those that want to host
     their own BinderHub instance. We know from multiple research organisations
     that this desire exists.

   \item Open call for open innovation mini-projects: mentored by \TheProject
     staff and targeting SMEs, municipalities, journalists, artists, etc.

   \item Where possible, we will schedule dissemination events to take place
     during conferences and community events, such as PyData, EuroSciPy,
     Supercomputing meetings.

   \item We will archive recordings of the training events to support the
     increasing desire of learners to make use of online streaming services
     (such as YouTube) to work through a learning programme at their own time
     and pace.

   \item We will offer in-person and remote training.

   \item The work will be done in collaboration with
     \href{https://coderefinery.org}{CodeRefinery} project which strongly
     support \TheProject proposal and will make available its network of instructors and helpers
     to co-organize, advertise and run online workshops on Open Science best practices. \TODO{This
       is great - can we get a letter of support?}

  \item We will detail our executed activities through the reporting at the end of
    each reporting period.
  \end{compactitem}
\end{task}

% template for a task
% each task should be added to exactly one workpackage
% in the workpackage task list
\begin{task}[
  title=Best practice guidelines for reproducible science,
  id=online-resources,
  lead=UIO,
  PM=10,
  wphases={0-36!.28},
  partners={SRL,MP,UIO}
]
  The aims of this task are to (i) provide online resources for Open Science and
  (ii) support \taskref{education}{workshops}.
  
  This task includes the following activities:
  \begin{compactitem}
  \item Collect and compose best practice guidelines for reproducible and
    re-usable science. Split the content into multiple topic areas so learners
    with different prior knowledge can skip the content they are familiar with
    already.
  \item Develop lesson materials on \emph{open science} best practices (version
    control, testing, automation of all steps, collaboration and peer review,
    documentation, software licensing and open source, use of Jupyter
    notebooks).
  \item Develop lesson materials on \emph{reproducible computational science},
    which focuses on combining the open science tools for reproducible science.
  \item Develop materials on \emph{using Binder tools to make science more
      reproducible and re-usable}. This includes addressing and describing the
    use cases from \WPref{applications}.
  \item Collaboration with the \href{https://coderefinery.org}{CodeRefinery}
    project for the development and maintainance of the
    \href{https://coderefinery.org/lessons/}{online lesson materials}. Following
    CodeRefinery's tradition, the aim will be to contribute the lessons to
    \href{https://software-carpentry.org/}{Software Carpentry} and
    \href{https://data-carpentry.org/}{Data Carpentry}.
  \item The training material will also be referenced on the Binder tools webpage.
  \end{compactitem}
  All material will be licensed under an open license such as
  \href{https://creativecommons.org/licenses/by-sa/4.0/}{CC BY-SA 4.0}
  (\delivref{education}{education-materials1}, \delivref{education}{education-materials1}).
\end{task}

\end{tasklist}


\begin{wpdelivs}
\begin{wpdeliv}[due=18,id=report1,dissem=PU,miles=prototype,nature=R,lead=UIO]
  {Community building: Report on impact of development workshops, dissemination and training activities, reporting period 1}
\end{wpdeliv}
\begin{wpdeliv}[due=36,id=report2,dissem=PU,miles=community,nature=R,lead=UIO]
  {Community building: Report on impact of development workshops, dissemination and training activities, reporting period 2}
\end{wpdeliv}
\end{wpdelivs}

\end{workpackage}
%%% Local Variables:
%%% mode: latex
%%% TeX-master: "../proposal"
%%% End:

%  LocalWords:  workpackage wphases wpobjectives wpdescription pageref wpdelivs wpdeliv
%  LocalWords:  dissem mailinglists swrepository final-mgt-rep compactitem swsites ipr
%  LocalWords:  TOWRITE tasklist delivref
