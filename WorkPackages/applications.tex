\TOWRITE{ALL}{Proofread WP 1 Management pass 1}
\begin{draft}
\TOWRITE{PS (Work Package Lead)}{For WP leaders, please check the following (remove items
once completed)}
\begin{verbatim}
- [ ] have all the tasks in this Work Package a lead institution?
- [ ] have all deliverables in the WP a lead institution?
- [ ] do all tasks list all sites involved in them?
- [ ] does the table of sites and their PM efforts match lists of sites for each task?
      (each site from the table is listed in all relevant tasks, and no site is listed
      only in the table or only at some task)
\end{verbatim}
\end{draft}

\begin{workpackage}[
  id=applications,
  wphases=0-36,
  swsites,
  title=Applications and use cases,
  short=Applications,
  lead=MP,
  % EGIRM=7,
  % CDSRM=12,
  % INSERMRM=24,
  % QSRM=6,
  % SILRM=12,
  SRLRM=9,
  % UIORM=12,
  % UPSUDRM=20,
  % WTTRM=3,
  % XFELRM=36,
  % EPRM=3,
]

\TOWRITE{Everything we develop in WP2-3 should be validated here}

\begin{wpobjectives}
  The objectives of this work package are
 \begin{compactitem}
   \item to guide the development of core tools by simultaneously
     developing and using applications in diverse fields with active
     scientists from these fields, and
   \item to demonstrate that the tools we develop are valuable to diverse
     fields of science, thus ensuring we develop e-infrastructure and
     services which can cater for a broad European and global research community
   \end{compactitem}
\end{wpobjectives}

\begin{wpdescription}

  Whilst the components issued from work packages  \WPref{reproducibility} and \WPref{impact} will be
  made available as generic building blocks for reproducible open science services,
  this work package aims at specific real-world cases.

  We have selected a number of applications in a variety of domains
  to demonstrate the broad impact of \TheProject, in particular in the
  areas of \TOWRITE{tasks}
  % (\localtaskref{astro}), education
  % (\localtaskref{teaching}), fluid dynamics
  % (\localtaskref{application-gpu}), geosciences
  % (\localtaskref{geoscience}), health
  % (\localtaskref{opendose-analysis}), mathematics
  % (\localtaskref{math}) and photon science and imaging
  % (\localtaskref{reproducibility-xfel}).
  The context and vision for each of the demonstrators is described in
  section \ref{sec:science-demonstrators-in-concept} on page
  \pageref{sec:science-demonstrators-in-concept}.

  Working closely with the core developers of the Jupyter ecosystem will make it possible to
  go way beyond what is normally available "out-of-the-box" and to offer better solutions,
  thereby guiding further development of the core features.

  \medskip

  All demonstrators will validate the Jupyter service capabilities such as reproducibility,
  interactive widget use and visualisation, and show how these can
  enable new open science on EOSC.

  The particular workflows, data infrastructures and data policies for
  FAIR\footnote{Findable, Accessible, Interoperable and Reusable} sharing of data vary from one community and use-case to
  the other, or may not be fully defined yet. Therefore, this proposal
  does not enforce a specific way of handling data. Instead we
  will explore in the demonstrator tasks how existing data policies,
  infrastructure and workflows can be respected and integrated with
  authentication and authorisation, data management, and
  JupyterHub/Binder services on EOSC. EGI is a partner
  for all the tasks in this work package and will work with us to find the
  best integration solutions in the evolving EOSC
  infrastructure.

  For some of the demonstrators, authentication and authorization and/or
  data management are being advanced outside \TheProject.


\end{wpdescription}

\begin{tasklist}
% add tasks from task directory here
% template for a task
% each task should be added to exactly one workpackage
% in the workpackage task list
\begin{task}[
  title=Science demonstrators,
  % task id for references
  id=demos,
  % lead institution ID
  lead=MP,
  PM=8,
  %wphases={0-36},
  % partner institution ID(s)
  % don't include lead here
  partners={IFR,UIO}
]
In this task, we want to demonstrate the value and usefulness of
\WPref{reproducibility} and \WPref{impact} with real scientific use cases from
the research communities involved in \TheProject (see
Section~\ref{sec:science-applications} on
page~\pageref{sec:science-applications} for the initial set of science applications).

The demonstrators are designed 
to exploit the solutions developed within \TheProject (such as Binder@Home, Binder@HPC,
data publishing) and leverage existing institutional and/or national
e-infrastructures as well as core EOSC services. Synergies between the different
science applications and communities will be ensured through the technical tasks
(\taskref{applications}{binder-at-home} Binder@home,
\taskref{applications}{data-publishing} data publishing, and
\taskref{applications}{binder-at-hpc} Binder@HPC).


% \paragraph*{Context:}  For example, in marine research field there are reproducible research examples such as Argopy~\cite{maze2020},  Pangeo ecosystems (http://gallery.pangeo.io/repos/pangeo-gallery/physical-oceanography/). But even within the same research lab, we have number of researchers who depends on commercial software for their data analysis and does not have access to publish reproducible research workflows.   


% \paragraph*{Task activity:}
 
% Within the actual Binder capability, we will demonstrate following two reproducible research configurations.  We demonstrate these research use caseses and show barriers that has been preventing these workflow to be pulished as reproducible science.   

%These information will give first feed back to \WPref{reproducibility}, \WPref{impact} 
%ll enrich the process of propose better accesible optimised 
%research workflow that can benefit researchers themselves, but also include reproducible aspects. 


 % \begin{compactitem}
%  \item FAIR Nordic Earth System Modelling: this science demonstrator leverages Binder@HOME (model development, education, single column or very simple model configuration), Binder@HPC (operational runs at scale including on EuroHPC), data publishing (publication of simulation results from blue-sky research);
%  \item Demonstration of marine physics and fish habitats modelling and analysis using Pangeo ecosystem.  
  
%\TODO{Tina, move this to section 1? but where??
%This effort and outcome of it will connect to the Digital Twin Ocean project to allow ocean data and models relevant to biodiversity to be re-used by researchers and engineers. This will provide a concept demonstration of ingesting ocean data and model output that can be reproduced through the existing ocean research infrastructures
% should be able to add 'interdisciplinary' part in chap1  as it bridges biology and physics..}

     % deliverable will be defined in the appropriate WorkPackage.tex
    % (\localdelivref{deliv-id})
%  \end{compactitem}
\end{task}

% template for a task
% each task should be added to exactly one workpackage
% in the workpackage task list
\begin{task}[
  title=Binder@Home,
  % task id for references
  id=binder-at-home,
  % lead institution ID
  lead=SRL,
  PM=7,
  %wphases={0-36},
  % partner institution ID(s)
  % don't include lead here
  partners={MP,UIO}
]
In this task, we want to use the compute power of the Desktops of individual researchers and
users, to recreate software environments in which computational results can be
reproduced, and re-used: We want to provide an experience identical or similar
to that of using BinderHub, but using only the local computer.

\paragraph*{Context:} Reproducibility services using Binder currently rely on hosted Binder instances.
The BinderHub Federation provides such a service global service at mybinder.org
which is free at
the point of use, and delegates the building of the software environment and
re-execution of the code to a small number of computer centres that have
volunteered to contribute compute resources.

It thus possible to overload the system \TODO{Add link to general discussion
  that mentions how many thousand builds take place per months etc}.

To allow the up-scaling of good reproducibility practice, it would be
desirable not to depend on such a single (or even multiple hosted services).

The compute resources typically offered through mybinder are modest: at most 2 GB of RAM
and a single CPU core.
Most laptops and Desktops have similar or much better hardware capabilities than
the mybinder cloud-computing resources currently offer.

For some research areas, it is essential to access big data sets to reproduce the scientific computing workflow.  For such cases, it is essential to reproduce the work using infrastructure that has optimised access to the data, so that one does not consume unnessesary computing and network resources.  

\paragraph*{Task activity:} Based on the preparations in \WPref{reproducibility} and
\WPref{impact}, extend Binder so that users of the service can
carry out the building of the environment, and -- if desired -- launching of a
notebook server \emph{on their own hardware}, such as their laptop or on-premise or cloud infrastructures.   The working title for such
functionality is ``binder@home'' (as a reference to the crowd-based SETI@home search for
extraterrestrial intelligence at home.\footnote{https://setiathome.berkeley.edu}

We will design, implement, and test the 'binder@home' functionality, and make it
available as part of binder software. We will need a utility that takes on
the responsibility of BinderHub for a single-user use case, and which triggers
the local build of the software environment, start of Jupyter notebook
server and opening of the relevant local URL and port an a browser.
This effort will bridge the gap from mybinder.org to ``Home'';
by giving the freedom and digital sovereignty
to the researchers to chose where they execute their computational experiments in a simple manner,
without sacrificing the convenience of the mybinder.org service.


%  The task includes the following activities
%  \begin{compactitem}
%  \item ...
%     % deliverable will be defined in the appropriate WorkPackage.tex
%    % (\localdelivref{deliv-id})
%  \end{compactitem}
\end{task}

% template for a task
% each task should be added to exactly one workpackage
% in the workpackage task list
\begin{task}[
  title=Prototype Policy,
  % task id for references
  id=policy,
  % lead institution ID
  lead=SRL,
  PM=1,
  wphases={0-36},
  % partner institution ID(s)
  % don't include lead here
  partners={XXX}
]
  Develop a prototype policy for reproducible science using \TheProject tools.

  \begin{compactitem}
  \item ...
     % deliverable will be defined in the appropriate WorkPackage.tex
    % (\localdelivref{deliv-id})
  \end{compactitem}
\end{task}

\end{tasklist}



\begin{wpdelivs}
%\TODO{update due date and startup!}
%\TODO{update milestone!}
\begin{wpdeliv}[
    % id for linking with \delivref or \localdelivref
    id=deliv,
    % lead institution
    lead=XXX,
    % month when deliverable is due (max 36)
    due=12,
    % associated milestone id (see milestones.tex)
    miles=startup,
    % ~always PU, DEC
    dissem=PU,
    nature=DEC,
]
  {
  One-line name of deliverable
  }
\end{wpdeliv}


\end{wpdelivs}
\end{workpackage}
%%% Local Variables:
%%% mode: latex
%%% TeX-master: "../proposal"
%%% End:

%  LocalWords:  workpackage wphases wpobjectives wpdescription pageref wpdelivs wpdeliv
%  LocalWords:  dissem mailinglists swrepository final-mgt-rep compactitem swsites ipr
%  LocalWords:  TOWRITE tasklist delivref
