\begin{draft}
\TOWRITE{PS (Work Package Lead)}{For WP leaders, please check the following (remove items
once completed)}
\begin{verbatim}
- [ ] discuss what the build packs are and where they go
- [ ] how we structure and budget different tasks in this WP
- [ ] needs review and more technical detail
- [ ] add deliverable(s)
\end{verbatim}
\end{draft}

\begin{workpackage}[
  id=impact,
  wphases={6-24!0.89,24-30!0.86,30-36!0.48},
  swsites,
  title=Broadening impact,
  short=Broadening impact,
  lead=SRL,
  SRLRM=24,
  MPRM=12,
  QSRM=2,
  IFRRM=4,
  %UIORM=4,
]
\begin{wpobjectives}
  This work package extends the functionality of the \TheProject tools for
  reproducibility to broaden its applicability and increase impact.
  \WPref{applications} exploits these technological advances
  in real world use cases.

  The objectives of this work package are to
 \begin{compactitem}
 \item remove the dependency of \repotodocker{} on Kubernetes
 \item support container technologies other than Docker
 \item extend the range of software specification standards that are recognised
   and supported by \repotodocker{} \TOWRITE{}{Can we give an example here?}
 \item enable access to data from data sources outside the container
 \item enable \repotodocker{} to extract and save the software installations
   instructions (independent from container generation)
 \end{compactitem}
\end{wpobjectives}

\begin{wpdescription}
One of the design decision that have led to the current Binder software stack is
to constrain the supported infrastructure to a few key components. These
include:

\begin{compactitem}
\item Software environments can only be created inside Docker container
\item To run and orchestrate (multiple) Docker containers, a Kubernetes system must
be available
\item The user must interact with Binder environments through a Binderhub
  installation
\item There is no Binders-specific provision to access (substantial) data sets
  from inside te container. \TODO{HF: What prevents us from
    a data-publishing application at the moment? How to express this without
    making the impression one couldn't use \softwarename{curl/wget} to fetch
    data, etc. }
\end{compactitem}

The benefits of such a restrictive approach are that the software development
and maintenance effort is kept small: the wider the range of supported
infrastructure that the Binder tools can be deployed on, the higher the
complexity.

At the same time, these restrictions prevent the following scenarios:
\begin{compactitem}
\item To run the Binder software on systems where Kubernetes is not available (such as
  the Desktop of a scientist). A related use case is ``Binder@home''
  \TODO{Provide link to Binder at home task in
   \WPref{applications}}.
\item To use Binder on systems where Docker cannot be used. A use case for this
  is to create reproducible environments on HPC installations. The HPC
  administrators generally avoid use of Docker for security reasons, but much
  prefer to support container technologies that can be executed without
  administrative privileges). \TODO{Point to task in applications}
\item To use Binder for software environments that use \TOWRITE{}{insert buildback
    example / community / software system}
\item To access and use significant amounts of data from inside the software
  environment is not supported. An important use case that cannot be supported due to
  this restriction is that of ``data publishing'': the idea is that a
  (potentially large and/or complex) dataset is published \emph{together} with
  software that encodes the necessary knowledge to extract meaningful data from
  that data set. A Binder environment would make this data set accessible in an
  interactive environment. \TODO{Point to data publishing use case(s)}
 \end{compactitem}

% Open source software in general, and Jupyter in particular,
% is developed not as a monolithic application,
% but rather as a collection of related components,
% which can be assembled in numerous combinations to meet diverse needs.
% The Jupyter community is no different.
% Jupyter itself is composed of several projects,
% but there are even more projects that build on top of Jupyter to create
% things like cloud services or data pipelines.
% The goal of \TheProject is to facilitate open science through Jupyter,
% and this includes working with projects all around the Jupyter ecosystem.
% We will focus this work package on developing
% Jupyter ecosystem projects with an emphasis on open science.
%
% repo2docker is a project for creating
% reproducible environments in which Jupyter notebooks (and other user interfaces) can be run.
% It reads a number of common formats to list required software packages,
% and prepares a Docker container with those packages installed.
% BinderHub is software for operating a web service using repo2docker,
% which enables sharing of interactive and reproducible Jupyter (and Rstudio) environments on the web with a single link.
% We will develop repo2docker and BinderHub further to meet the needs of the open science community.
%
% In addition to the interactive aspects of Jupyter,
% notebooks can be used in a "workflows" style,
% where job systems run analyses and produce reports,
% either on a scheduled basis or triggered by events.
% There is a great deal of interest in using notebooks in this way,
% and much room for development of tools supporting workflows in data-driven open science.



\end{wpdescription}

\begin{tasklist}
% % add tasks from task directory here
% template for a task
% each task should be added to exactly one workpackage
% in the workpackage task list
\begin{task}[
  title=Support more software specification standards,
  % task id for references
  id=buildpacks,
  % lead institution ID
  lead=SRL,
  PM=12,
  % wphases={0-36},
  % partner institution ID(s)
  % don't include lead here
  partners={MP}
  ]

  There are two different aspects of software specification that \repotodocker{}
  needs to understand:
  \begin{enumerate}
  \item the specification of the software environment, for example through
    \softwarename{requirements.txt} files, etc (see Section
    \ref{sec:repo2docker} for a list of currently
    supported standards),
  \item from where to retrieve the software itself: this will be different on a
    GitHub repository or a Zenodo archive
    (see \ref{sec:repo2docker} for supported repositories).
  \end{enumerate}

  For both aspects, there are requests from potential Binder users to extend the
  capabilities of \repotodocker{}.

  In this task, we will prioritise such requests and extend the \repotodocker{}
  functionality to support as many new standards as possible.

\end{task}

% template for a task
% each task should be added to exactly one workpackage
% in the workpackage task list
\begin{task}[
  title=Reducing technical constraints to enable broader usage,
  % task id for references
  id=constraints,
  % lead institution ID
  lead=SRL,
  PM=14,
  %wphases={0-36},
  % partner institution ID(s)
  % don't include lead here
  partners={MP,QS}
]

In this task, we refactor and extend the existing code to be more flexible. In
particular, we want to address and remove the constraints that currently exist:

\begin{compactitem}
\item Dependency of existing Kubernetes system: at the moment, BinderHub can
  only start container environments within a Kubernetes installation. In this
  task, we will make it possible to start a container directly on the host
  machine. Setting up Kubernetes system is a complex task: while justified to
  exploit large computational resources through swarms of containers
  effectively, it is not necessary for single-machine execution of
  Binder-reproduced environments (such as anticipated for Binder@home in
  \taskref{applications}{binder-at-home}).
\item Support of only Docker containers to host created software environments.
  Docker is widespread and popular, and in particular has an attractive user
  interface for Windows and OSX. However, many HPC centres refuse to allow use
  of Docker containers on their systems for security reasons. In this task, we
  will make it possible to user container technologies, which are more
  acceptable to use on large HPC installations (such as singularity for
  example).
     % deliverable will be defined in the appropriate WorkPackage.tex
    % (\localdelivref{deliv-id})
  \end{compactitem}

  These steps are essential to enable the Binder@home
  (\taskref{applications}{binder-at-home}) and the Binder@HPC
  (\taskref{applications}{binder-at-hpc}) use cases.

%   The task includes the following activities
%   \begin{compactitem}
%   \item ...
%      % deliverable will be defined in the appropriate WorkPackage.tex
%     % (\localdelivref{deliv-id})
%   \end{compactitem}
\end{task}

% template for a task
% each task should be added to exactly one workpackage
% in the workpackage task list
\begin{task}[
  title=Support more use patterns,
  % task id for references
  id=patterns,
  % lead institution ID
  lead=SRL,
  PM=16,
  %wphases={0-36},
  % partner institution ID(s)
  % don't include lead here
  partners={MP,IFR}
]
In this task, we provide the technical possibilities to use the Binder tools in
better and new ways. These are used and evaluated with real world applications
in~\WPref{applications}.

\begin{compactitem}
\item The \repotodocker{} tool searches a given repository for specifications of
  software dependencies, and carries on to compose instructions to install all
  of the software dependencies within a Docker container image. In this task, we
  modularise this functionality, and make it possible to extract the required
  instructions on their own (for example into a stand-alone shell script).

  Such functionality could be used to:
  \begin{compactitem}
  \item Extract the list of installation commands to carry out a local install
    of the required software, or an installation within any other environment.
  \item In particular on HPC systems, it may be necessary to install software
    directly on the host, and this functionality would simplify that.
  \item Having the installation instructions neatly summarised will also help
    the interested scientist to understand what software specifications are
    (explicitly or implicitly ) given within the repository.
    % \item Having the installation instructions extracted, these can be used to
    %   explore the effects of changing versions of a particular library or
    %   dependency. This will be helpful to track down the origins of observed
    %   non-reproducibility.
  \end{compactitem}

\item Improve and document the options to access external data resources from
  within the computational environments generated by BinderHub deployments. This
  is a prerequisite for \taskref{applications}{data-publishing}.
\item Restrict access to BinderHub through authentication - useful for many
  institutes who want to offer reproducibility services and data hosting but
  need to limit or control access to those systems (to ensure, for example, that
  the computational resources are not abused).
\item Allow to use Binder without a BinderHub installation. A use case for this
  is to run a Binder service on a local Desktop (see
  \taskref{applications}{binder-at-home}).
\item Allow to use Binder without Jupyter notebooks.

  Binder's original design goal was to allow execution of notebooks within a
  software requirement that provides all the software required by the notebook.
  This may include compile and highly customised software, which might produce
  output files, which are then processed and visualised in the notebook.

  Here, we will provide the foundations for reproducibility work that is
  non-interactive and doesn't need Jupyter notebook. A use case for this is
  reproducibility at High Performance Computing facilities, where often the
  computational tasks cannot be carried out interactively (see
  \taskref{applications}{binder-at-hpc}).

    % Should we say more here? @min?
    %
    % This will require creation of a program that creates the required software
    % environment, and - if a notebook is used - start a notebook server to which
    % the user can connect.

  \end{compactitem}
\end{task}


%
%   This feature will become part of the \repotodocker{} functionality
%   (\localdelivref{extract-dependencies}).
%
%

%%% Local Variables:
%%% mode: latex
%%% TeX-master: "../proposal"
%%% End:

%
\end{tasklist}


\begin{wpdelivs}
  \begin{wpdeliv}[due=12,id=extract-dependencies,dissem=PU,nature=OTHER,lead=SRL]
    {Release new \repotodocker{} feature that exposes the command to install
      identified software environments in stand-alone script
    %  (see \taskref{impact}{extract-dependencies}).
    }
  \end{wpdeliv}

\begin{wpdeliv}[due=36,id=binder-tools-software,dissem=PU,nature=OTHER,lead=SRL]
  {Final open source release of \TheProject tools, completed with automatic
    testing and documentation. }
\end{wpdeliv}

\end{wpdelivs}
\end{workpackage}
%%% Local Variables:
%%% mode: latex
%%% TeX-master: "../proposal"
%%% End:

%  LocalWords:  workpackage wphases wpobjectives wpdescription pageref wpdelivs wpdeliv
%  LocalWords:  dissem mailinglists swrepository final-mgt-rep compactitem swsites ipr
%  LocalWords:  TOWRITE tasklist delivref
