
\begin{workpackage}[
  id=reproducibility,
  % wphases=0-36!1.03,
  wphases={0-6!1.14,6-24!0.73,24-36!0.1},
  title=Improving robustness of reproducibility tools,
  short=Improving robustness,
  lead=QS,
  SRLRM=23,
  UIORM=0,
  MPRM=2,
  QSRM=12,
  swsites,
]
\begin{wpobjectives}
  \begin{compactitem}
    \item to better understand and evaluate successful reproduction of computational environments
    \item to improve the practical reproducibility of environments constructed
      with \TheProject tools
    \item to support and maintain core Binder software infrastructure in order to keep it healthy
         and useful for open science and reproducibility
 \end{compactitem}
\end{wpobjectives}

\begin{wpdescription}

This work package is focused on making \repotodocker{} do the things it does
already \emph{better}, \emph{more robustly} and \emph{more sustainably}.
(Orthogonal to those improvements, we plan to significantly extend the
\repotodocker{} features and use cases in \WPref{impact}.)

To be able to asses the impact of our planned improvements, we need to have a
metric. We will create this in Task~\taskref{reproducibility}{repo2docker-checker}.
In addition to the evaluation of the improvements in this proposal, this
can be used more generally as an indicator for reproducibility of software
environments.

One major improvement to the existing capabilities of \repotodocker{} is the
\emph{time-machine} functionality,
implemented in \taskref{reproducibility}{repo2docker-timemachine}.

In task \taskref{reproducibility}{performance-optimisation}, we will speed up the execution time of
\repotodocker{} to improve the user experience when reproducing or re-using
existing software and data.

Open source software needs ongoing maintenance to adapt to changing requirements
and dependencies. We schedule a certain amount of time for this in task
\taskref{reproducibility}{maintenance}.

All changes to the software will be made available online as open source already during development
(i.e. throughout the whole project), and new features will be made available
through software releases of the Binder tools. A final release will be made and
reported through the deliverable \localdelivref{repo2docker-release24}.

% the existing functionality of repodocker.
% Community-led open source software is critical to a sustainable future for open science.
% Commonly used tools make up a shared infrastructure,
% where investment in core components benefits the widest user community.
% \TheProject is centred around the Jupyter project,
% which is a collection of projects for interactive computing and
% communicating computational ideas.
%
% This work package is focused on developing and maintaining
% the core of Jupyter.
% In particular, we will help maintain these projects to meet the needs of the
% Jupyter community, with a focus on needs for open science.
% To serve the needs of \TheProject,
% Jupyter core infrastructure will need improvements
% to security, performance, and scalability,
% which will be provided in \taskref{reproducibility}{maintenance}.
% In addition, we will develop new features in the core of Jupyter
% to bring it to a wider audience,
% and to improve its usefulness to those working toward open science practices,
% including via collaboration features (\taskref{reproducibility}{collaboration})
% and accessibility (\taskref{reproducibility}{accessibility}).

\end{wpdescription}

\begin{tasklist}

% template for a task
% each task should be added to exactly one workpackage
% in the workpackage task list
\begin{task}[
  title=Sample Task,
  % task id for references
  id=task-id,
  % lead institution ID
  lead=SRL,
  PM=1,
  wphases={0-36},
  % partner institution ID(s)
  % don't include lead here
  partners={XXX}
]
  The task includes the following activities
  \begin{compactitem}
  \item ...
     % deliverable will be defined in the appropriate WorkPackage.tex
    % (\localdelivref{deliv-id})
  \end{compactitem}
\end{task}

\begin{task}[
  title=repo2docker development,
  id=repo2docker-timemachine,
  lead=SRL,
  PM=38,
  wphases={0-36!1.055},
  partners={QS,MP}
]

Often, a repository of scientific results may
specify which software library is required (such as the Python library
\softwarename{pandas}), but not which version. A software environment creation
tool -- such as \repotodocker -- can then attempt to install the most recent
version of \softwarename{pandas}. This is usually the intention of the authors,
and was correct at the time the repository was created.  However, as time moves
on, the interface, behaviour and dependence on other packages of
\softwarename{pandas} will change, and at some point an automatic build of the
software for the whole repository may fail because of conflicting dependencies.

We have found through anecdotal evidence that these problems can be overcome if a
\softwarename{pandas} version can be chosen that was the most recent at the time
when the repository was created.

In this task, we will teach \repotodocker to establish the data of publication
(or last modification) of the repository, to determine the appropriate version
of software libraries from that time, and to select libraries with those
versions if no specific version is specified.

XXX


\TOWRITE{There is probably more to be done?}

\TOWRITE{HF: The following is good - but should probably go elsewhere.}
  Running someone else's analyses is a particularly difficult problem.

  There are differences between operating systems, versions of installed
  software and access to the required data sets.
  
  These challenges mean that is currently considered to be beyond the scope of
  an expert peer reviewer to verify data science analysis codes before
  publication.
  
  BinderHub, part of Project Jupyter, enables one-click running of git repositories.

  BinderHub provides a web interface to the repo2docker tool.
  \TOWRITE{End of section to move elsewhere.}


  \TOWRITE{HF: The following items are all good. I think some of the should be
    scattered across the other workpackages.}
  
  The task includes the following activities
  \begin{compactitem}
  \item extend repo2docker with support for execution on cloud resources
  \item extend repo2docker with support for execution on HPC resources with Docker support
  \item improved "first use" experience of running repo2docker locally
  \item add support for using archives such as Zenodo as source for repo2docker and BinderHub
  \item define procedures and recommendations for long term reproducibility and sustainability of repo2docker compatible repositories
  \item create educational material describing repo2docker and its benefits to researchers
  \item Enable Openshift based deployments of BinderHub
  \item User surveys about pain points using BinderHub
  \item User authentication in BinderHub
  \end{compactitem}
\end{task}

% template for a task
% each task should be added to exactly one workpackage
% in the workpackage task list
\begin{task}[
  title=Performance optimisation,
  % task id for references
  id=performance-optimisation,
  % lead institution ID
  lead=QS,
  PM=9,
  %wphases={0-24!0.375},
  % partner institution ID(s)
  % don't include lead here
  partners={SRL}
]
  The creation of reproducible software environments can take -- depending on
  the complexity and overall size of the destination environment -- quite some time.
  Often, an image can be built within few minutes,
  but there are environments that can take much longer.

  In this task, we will profile and optimise \repotodocker{} performance,
  both in terms of build time
  and image size (can be several gigabytes),
  which contributes to user-experienced performance as launching a large image
  can take longer than a small one
  when the image must be transferred across a network.
\end{task}

\begin{task}[
  title=Maintenance of open source reproducibility software,
  id=maintenance,
  lead=SRL,
  PM=6,
  %wphases={0-24!.25},
  partners={QS}
]

Developing software that people will use requires maintenance of that
software, not just new development. Through  this proposal we will
contribute general support to open source reproducibility software
where this is helpful for \TheProject. Such contributions are expected 
to the Jupyter (and as sub project) Binder code base. They support not
just all participants in \TheProject but also millions of people
relying on Jupyter software.


% Maintenance of core software is often an implicit and un-paid cost, or one
% hidden in over-describing the resources required to deliver proposed
% developments. In \TheProject, we make it clear and explicit that we will spend a
% significant amount of time developing and maintaining the core Jupyter and
% JupyterHub e-Infrastructure to respond to the needs of \TheProject and others,
% and contribute towards the sustainability and health of the community.
% 
% We will provide support to the Jupyter e-Infrastructure software, ensuring that
% it meets the needs (\localdelivref{jupyter-contributions}) of \TheProject, and
% aid in the release process to ensure that stable releases of Jupyter software
% can be used in mature \TheProject services (\localdelivref{jupyter-releases}).
% 
%   \TheProject will need improvements to core Jupyter functionality, including areas of:
% 
%   \begin{compactenum}
%     \item ease of deployment
%     \item security
%     \item scalability of JupyterHub
%     \item performance
%   \end{compactenum}
% 
%   We will contribute improvements in these areas,
%   meeting the needs of \TheProject and benefiting the wider Jupyter
%   community.

\end{task}


\end{tasklist}


\begin{wpdelivs}
  % \begin{wpdeliv}[due=1,miles=startup,id=infrastructure,dissem=PU,nature=DEC,lead=SRL]
  %   {Some Deliverable}
  % \end{wpdeliv}

  % (\localdelivref{deliv-id})

  \begin{wpdeliv}[due=12,id=deliv-id-repo2docker-checker-software,dissem=PU,nature=OTHER,lead=SRL]
    {Release software tool for checking of reproducibility of software
      environments (\texttt{repo2docker-checker})}
  \end{wpdeliv}

  \begin{wpdeliv}[due=24,id=repo2docker-checker-study-report,dissem=PU,nature=R,lead=SRL]
    {Summary of reproducibility improvements achieved.}
  \end{wpdeliv}

  \begin{wpdeliv}[due=24,id=repo2docker-release24,dissem=PU,nature=R,lead=SRL]
    {Release of \repotodocker{} with improve robustness features.}
  \end{wpdeliv}

\end{wpdelivs}

\end{workpackage}
%%% Local Variables:
%%% mode: latex
%%% TeX-master: "../proposal"
%%% End:

%  LocalWords:  workpackage wphases wpobjectives wpdescription pageref wpdelivs wpdeliv
%  LocalWords:  dissem mailinglists swrepository final-mgt-rep compactitem swsites ipr
%  LocalWords:  TOWRITE tasklist delivref
